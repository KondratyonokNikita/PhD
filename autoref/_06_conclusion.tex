\documentclass[_00_autoref.tex]{subfiles}
\begin{document}

{\let\clearpage\relax\vspace{2.2ex}
\chapter*{\MakeUppercase{ЗАКЛЮЧЕНИЕ}}\vspace{-3ex}}

\centerline{\textbf{Основные научные результаты диссертации}}

Работа посвящена исследованию свойств теоретико-числовых и криптографических алгоритмов в дедекиндовых кольцах.
В частности, получены следующие результаты.
\begin{enumerate}
    \item Доказаны критерии простоты идеалов в дедекиндовых кольцах, аналогичные критериям Эйлера и Миллера в кольце целых чисел.
    Показано, что полученные критерии можно использовать для построения эффективных вероятностных алгоритмов проверки идеала на простоту.
    Получены достаточные условия для дедекиндового кольца для того, чтобы этот алгоритм можно было модифицировать и сделать детерминированным.
    Получены оценки вычислительной сложности алгоритмов.
    Эти результаты получены в работах [\ref{source:JNT_2016}; \ref{source:NANB_2017}; \ref{source:XII_Belarussian_math_conference_2016}; \ref{source:Collection_of_articles_by_laureates_2017}; \ref{source:Collection_of_articles_by_laureates_2018}; \ref{source:CSIST_2022}] и изложены в главе~$2$.

    \item Доказаны теоремы об экстремальных свойствах алгоритма Евклида в факториальных кольцах.
    Получен класс колец, в которых верна теорема Кронекера-Валена.
    Разработан алгоритм, позволяющий проверить принадлежность кольца этому классу колец.
    Построен метод автоматического доказательства невыполнимости теоремы Кронекера-Валена в числовых кольцах.
    Доказано, что теорема Кронекера-Валена не выполняется во всех действительных квадратичных норменно-евклидовых кольцах.
    Доказан аналог теоремы Ламе о длине цепочки делений с выбором минимального по норме остатка в факториальных кольцах.
    Эти результаты получены в работах [\ref{source:Vestnik_BSU_2013}; \ref{source:JSC_2016}; \ref{source:JSC_2021}; \ref{source:Republican_Scientific_Conference_of_Students_and_Postgraduates_2013}] и изложены в главе~$3$.

    \item Доказаны теоремы накладывающие необходимые условия на параметры аналога криптосистемы RSA в дедекиндовых кольцах для обеспечения устойчивости к взлому.
    В частности теорема Винера о малой секретной экспоненте и теорема Кранакиса.
    Эти результаты получены в работах [\ref{source:NANB_2015}; \ref{source:BSU_Journal_2020}; \ref{source:Algebra_and_theory_of_algorithms}; \ref{source:XIII_Belarussian_math_conference_2021}] и изложены в главе~$4$.
\end{enumerate}

\centerline{\textbf{Рекомендации по практическому использованию результатов}}

Результаты и методы диссертации могут быть использованы при проведении исследований свойств теоретико-числовых и криптографических алгоритмов в дедекиндовых кольцах или кольцах более общего вида.
Также результаты работы могут использоваться при чтении спецкурсов для студентов математических специальностей.

\end{document}
