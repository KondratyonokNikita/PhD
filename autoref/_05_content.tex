\documentclass[_00_autoref.tex]{subfiles}
\begin{document}

{\let\clearpage\relax\vspace{2.2ex}
\chapter*{\MakeUppercase{ОСНОВНАЯ ЧАСТЬ}}\vspace{-2ex}}

\section{Предварительные сведения}

\subsection{Дедекиндовы кольца}

\begin{definition}
    \emph{Идеалом} кольца $R$ называется его подкольцо $\ideal{n}$, замкнутое относительно умножения на элементы $R$.
    А именно для любого $a \in R$ выполнено включение $a\ideal{n} \subseteq \ideal{n}$.
    Идеал $\ideal{n}$ называется \emph{тривиальным}, если он совпадает с $R$ или нулевым идеалом $0$.
    Идеал $\ideal{n}$ называется \emph{собственным}, если он не совпадает с $R$.
\end{definition}

\begin{definition}
    Собственный идеал $\ideal{n} \subset R$ называется \emph{простым}, если факторкольцо $R/\ideal{n}$ является областью целостности.
\end{definition}

\begin{definition}
    Собственный идеал $\ideal{n} \subset R$ называется \emph{максимальным}, если он не содержится ни в каком другом собственном идеале.
\end{definition}

\begin{remark}
    Любой максимальный идеал является простым.
\end{remark}

\begin{definition}
    Кольцо $R$ называется \emph{дедекиндовым}, если любой нетривиальный идеал раскладывается в произведение простых идеалов единственным способом с точностью до порядка множителей.
\end{definition}

\begin{definition}
    \emph{Нормой} $\Nm{\ideal{n}}$ идеала $\ideal{n} \subset R$ называется мощность факторкольца $R/\ideal{n}$.

    Говорят, что дедекиндово кольцо $R$ является \emph{дедекиндовым кольцом с конечной нормой} (finite norm property), если для любого собственного идеала $\ideal{n} \subseteq R$ факторкольцо $R/\ideal{n}$ конечно.
\end{definition}

Далее в работе будем рассматривать только дедекиндовы кольца с конечной нормой.

\begin{example}
    Рассмотрим примеры дедекиндовых колец с конечной нормой.
    \begin{itemize}
        \item Пусть $K$~-- числовое поле.
        Кольцо $\mathbb{Z}_K$, образованное алгебраическими целыми элементами этого поля является дедекиндовым с конечным полем остатков.
        Частными случаями этого примера являются кольцо целых чисел и гауссовых чисел.
        
        \item Пусть $f(x, y) = y - mx - b$~-- прямая.
        Тогда $K[x, y]/(f(x, y)) \cong K[x]$.
        Следовательно, это координатное кольцо является факториальным.
        
        \item Пусть $f(x, y) = y - x^2$~-- парабола.
        Тогда $K[x, y]/(f(x, y)) \cong K[x]$.
        Следовательно, это координатное кольцо является факториальным.
    
        \item Пусть $f(x, y) = x^2 + y^2 - 1$.
        Если $K = \mathbb{Q}$, то координатное кольцо $\mathbb{Q}[x, y]/(f)$ не изоморфно $K[x]$, так как первое не является факториальным кольцом.
        Это можно показать, рассмотрев элементы $y^2 = yy$ и $1-x^2 = (1-x)(1+x)$.
        Однако, если $K = \mathbb{C}$, то координатное кольцо $\mathbb{C}[x, y]/(f) \cong \mathbb{C}[x, x^{-1}]$ уже будет факториальным, так как это локализация факториального кольца.
    \end{itemize}
\end{example}

\begin{definition}
    Идеал $\ideal{n}$ называется \emph{главным}, если $\ideal{n} = aR$, где $a \in R$.
    Если все идеалы кольца $R$ являются главными, то говорят, что $R$ \emph{кольцо главных идеалов}.
\end{definition}

\begin{definition}
    \emph{Произведением} двух идеалов $\ideal{a}$ и $\ideal{b}$ называется идеал, порожденный всеми произведениями $ab$, где $a \in \ideal{a}$, $b \in \ideal{b}$.
\end{definition}

\begin{definition}
    Пусть $a, b \in R$ и $\ideal{n} \subseteq R$.
    Будем говорить, что $a$ \emph{сравнимо} с $b$ по модулю $\ideal{n}$ и писать $a \equiv b \pmod{\ideal{n}}$, если $a - b \in \ideal{n}$.
\end{definition}

\begin{definition}[\cite{source:Petukhova}]
    Функцией Эйлера нетривиального идеала $\ideal{n} \subset R$ называется функция
    \begin{equation*}
        \varphi(\ideal{n}) = \left|
            \invertible{(R/\ideal{n})}
        \right|.
    \end{equation*}
\end{definition}

\begin{statement}[Обобщенная теорема Эйлера]\cite{source:Petukhova}\label{statement:euler_function}
    Пусть $m \in R$ и $\ideal{n} \subset R$~-- идеал.
    Если $Rm + \ideal{n} = R$, то
    \begin{equation*}
        m^{\varphi(\ideal{n})}\equiv 1 \pmod{\ideal{n}}.
    \end{equation*}
\end{statement}

\begin{definition}
    \emph{Первообразным корнем по модулю идеала} $\ideal{n}$ будем называть такой элемент $g \in R$, что $g^{\varphi (\ideal{n})} \equiv 1 \pmod{\ideal{n}}$ и $g^{l} \not\equiv 1 \pmod{\ideal{n}}$ при $1 \leq l < \varphi(\ideal{n})$.
\end{definition}

\begin{statement}[Обобщенная китайская теорема об остатках]\label{statement:chinese_remainder_theorem}
    Пусть $\ideal{n}_1, \ideal{n}_2, \dots, \ideal{n}_k$~-- попарно взаимнопростые идеалы кольца $R$.
    Тогда
    \begin{equation*}
        \begin{split}
            R/(\ideal{n}_1\ideal{n}_2\dots\ideal{n}_k) \cong & (R/\ideal{n}_1) \times (R/\ideal{n}_2) \times \dots \times (R/\ideal{n}_k)\\
            \invertible{R/(\ideal{n}_1\ideal{n}_2\dots\ideal{n}_k)} \cong & \invertible{R/\ideal{n}_1} \times \invertible{R/\ideal{n}_2} \times \dots \times \invertible{R/\ideal{n}_k}
        \end{split}
    \end{equation*}
\end{statement}

\begin{definition}
    Элемент $a \in R$ будем называть \emph{квадратичным вычетом по модулю идеала} $\ideal{n}$, если существует $b \in R$, что $b^2 \equiv a \pmod{\ideal{n}}$.

    Для простого идеала $\ideal{p}$ и $a \in \invertible{R/\ideal{p}}$ определим \emph{символ Лежандра} следующим образом
    \begin{equation*}
        \jacobi{a}{\ideal{p}} = \begin{cases}
            1, \textrm{ если } a \textrm{ квадратичный вычет по модулю } \ideal{p}\\
            -1, \textrm{ иначе}.
        \end{cases}
    \end{equation*}

    Для нетривиального идеала $\ideal{n} = \ideal{p}_1  \dots \ideal{p}_k$ и $a \in \invertible{R/\ideal{n}}$ определим \emph{символ Якоби} следующим образом
    \begin{equation*}
        \jacobi{a}{\ideal{n}} = \left(\frac{a}{\ideal{p}_1}\right) \dots \left(\frac{a}{\ideal{p}_k}\right)
    \end{equation*}
\end{definition}

\begin{definition}
    \emph{Расширение поля} $K$ это такое поле $E$, которое содержит поле $K$ в качестве подполя.
    Для любого расширения $E \supset K$ поле $E$ является векторным пространством над полем $K$.
    Размерность этого векторного пространства называется \emph{степенью расширения} и обозначается $[E:K]$.
    Говорят, что $E \supset K$ \emph{конечное расширение}, если его степень конечна.
\end{definition}

\begin{definition}
    Пусть $E$ расширение поля $K$.
    Элемент $E$ называется \emph{алгебраическим} над $K$, если он является корнем ненулевого многочлена с коэффициентами в $K$.
    Элементы, не являющиеся алгебраическими, называются \emph{трансцендентными}.
    
    Если каждый элемент расширения $E \supset K$ является алгебраическим над $K$, то $E \supset K$ называется \emph{алгебраическим расширением}. 
\end{definition}

\begin{definition}
    Алгебраическое расширение $E \supset K$ называется \emph{нормальным}, если каждый неприводимый многочлен $f(x)$ над $K$, имеющий хотя бы один корень в $E$, разлагается в $E$ на линейные множители.

    Алгебраическое расширение $E \supset K$ называется \emph{сепарабельным}, если каждый элемент $E$ является сепарабельным, то есть его минимальный многочлен не имеет кратных корней.
    В частности, теорема о примитивном элементе утверждает, что любое конечное сепарабельное расширение имеет примитивный элемент.
    
    Расширение $E \supset K$ называется \emph{расширением Галуа}, если оно одновременно сепарабельное и нормальное.

    Для любого расширения $E \supset K$ можно рассмотреть группу автоморфизмов поля $E$, действующих тождественно на поле $K$.
    Когда расширение является расширением Галуа, эта группа называется \emph{группой Галуа} данного расширения и обозначается $\Gal{L/K}$.
\end{definition}

Пусть $R$ дедекиндово кольцо с полем частных $K$.
Пусть расширение $L \supset K$ конечное, сепарабельное и нормальное расширение, а $\Gal{L/K}$ является абелевой.
Положим $S$ алгебраическое замыкание $R$ в $L$.

\begin{definition}
    Пусть $\ideal{p}$ простой идеал кольца $R$.
    Рассмотрим идеал $\ideal{p}S$, который он генерирует в кольце $S$.
    Этот идеал не обязан быть простым, но существует единственная факторизация его на простые идеалы
    \begin{equation*}
        \ideal{p}S = \prod_{\ideal{q}} \ideal{q}^{e_{\ideal{q}}},
    \end{equation*}
    где произведение берется по всем простым идеалам кольца $S$ и $e_{\ideal{q}} > 0$ только для конечного количества простых $\ideal{q}$.
    Если $e_{\ideal{q}} > 0$ для некоторого $\ideal{q}$, то говорят, что $\ideal{q}$ лежит над (lie over, lie above) простым идеалом $\ideal{p}$.

    Число $e_{\ideal{q}}$ из разложения $\ideal{p}S = \prod_{\ideal{q}} \ideal{q}^{e_{\ideal{q}}}$ называется \emph{индексом разветвления} (ramification index) $\ideal{q}$.

    Если для простого идеала $\ideal{p} \subseteq R$ выполнено $e_{\ideal{q}} > 1$ для некоторого $\ideal{q} \subseteq S$, то говорят, что идеал $\ideal{p}$ \emph{ветвится} в $L$.
    Если идеал $\ideal{p}S$ простой в $S$, то говорят, что $\ideal{p}$ \emph{инертный} (inert).
    Если для всех $\ideal{q}$ выполняется или $e_{\ideal{q}} = 0$, или $e_{\ideal{q}} = 1$, то говорят, что $\ideal{p}$ \emph{полностью разлагается} (splits completely) в $L$.
\end{definition}

\begin{definition}
    Пусть $\ideal{p}$ простой идеал кольца $R$, не ветвящийся в $L$ и пусть $\ideal{P} = \ideal{p}S$ соответствующий идеал в $S$.
    Тогда существует единственный такой элемент $\sigma \in \Gal{L/K}$, что для любого $\alpha \in L$
    \begin{equation*}
        \sigma(\alpha) \equiv \alpha^{\Nm{\ideal{p}}} \pmod{\ideal{P}}.
    \end{equation*}
    Этот элемент называют \emph{символом Артина} идеала $\ideal{p}$.
\end{definition}

\begin{definition}
    Пусть $\phi: \Gal{L/K} \to \invertible{\mathbb{C}}$ гомоморфизм.
    Рассмотрим функцию
    \begin{equation*}
        \chi(\ideal{p}) = \begin{cases}
            \phi(\sigma_{\ideal{p}}), & \textrm{если } \ideal{p} \textrm{ не ветвится}\\
            0, & \textrm{иначе}
        \end{cases}
    \end{equation*}
    где $\ideal{p}$ простой и $\sigma_{\ideal{p}}$ символ Артина идеала $\ideal{p}$.
    Используя мультипликативность, эту функцию можно определить для всех идеалов $R$.
    Полученную функцию $\chi$ будем называть \emph{характером}.
    Характер, принимающий только значения $0$ и $1$, называется \emph{главным}.
\end{definition}

\begin{definition}
    Будем говорить, что характер $\chi$ \emph{задан по модулю идеала} $\ideal{f} \subset R$, если для всех идеалов $\ideal{n} \subseteq R$, из сравнения $\ideal{n} \equiv 1 \pmod{\ideal{f}}$ следует равенство $\chi(\ideal{n}) = 1$.
\end{definition}

\subsection{Факториальные кольца}

\begin{definition}
    Кольцо $R$ называется \emph{факториальным}, если каждый его ненулевой элемент $x \in R$ либо обратим, либо однозначно представляется в виде произведения неприводимых элементов с точностью до перестановки множителей.

    Факториальное кольцо является кольцом главных идеалов.
    Далее рассматривая идеалы факториальных колец будем писать элементы, порождающие эти идеалы.
\end{definition}

\begin{definition}
    Пусть $R$ факториальное кольцо.
    Функцию $\elementnorm{\cdot}: R \to \mathbb{N} \cup \{0, -\infty\}$ будем называть нормой в $R$, если
    \begin{itemize}
        \item $\elementnorm{x} = -\infty$ тогда и только тогда, когда $x = 0$;

        \item $\elementnorm{xy} \ge \elementnorm{x}$;

        \item для $x, y \in \zeroless{R}$ равенство $\elementnorm{xy} = \elementnorm{x}$ выполнено тогда и только тогда, когда $y \in \invertible{K}$.
    \end{itemize}
\end{definition}

\begin{remark}
    Для любого факториального кольца $R$ можно определить норму.
    Рассмотрим разложение элемента $x$ на простые множители
    \begin{equation*}
        x = \varepsilon p_1^{\alpha_1} \dots p_k^{\alpha_k},
    \end{equation*}
    где $\varepsilon \in \invertible{R}$, $p_1, \dots, p_k$~-- простые элементы $R$.
    Тогда функция
    \begin{equation*}
        \elementnorm{x} = \begin{cases}
            \sum_{i=1}^k \alpha_{i}, & \textrm{ если } x \neq 0\\
            -\infty, & \textrm{ если } x = 0
        \end{cases}
    \end{equation*}
    является нормой в $R$.
\end{remark}

Далее будем считать, что факториальное кольцо $R$ задано вместе с нормой $\elementnorm{\cdot}$.

\begin{definition}
    Пусть $R$ факториальное кольцо с полем частных $F$.
    Функцию $\fr{\cdot}: F \to F$ будем называть дробной частью в $F$, если
    \begin{itemize}
        \item $\fr{\alpha + q} = \fr{\alpha}$ для любых $\alpha \in F$, $q \in R$;

        \item если $m \in R$, $n \in \zeroless{R}$ и $(m, n) = 1$, то $\fr{m/n} = r/n$, где $r \in R$, $(m-r)/n \in R$ и $\elementnorm{r} = \min \{\elementnorm{s} | s \in R, (m-s)/n \in R\}$.
    \end{itemize}
    Функцию $\int{\cdot}: F \to R$ будем называть целой частью, если
    \begin{equation*}
        \int{\alpha} = \alpha - \fr{\alpha}.
    \end{equation*}
\end{definition}

\begin{definition}
    Обозначим через $F_1$ множество всех несократимых дробей $F$
    \begin{equation*}
        F_1 = \{
            \alpha \in F \big| \alpha = \fr{\alpha}
        \}.
    \end{equation*}
\end{definition}

\begin{remark}\label{remark:easy_fr}
    Для любого факториального кольца $R$ можно определить дробную и целую часть.
    Рассмотрим произвольный элемент $X \in F/R$.
    Этот элемент можно представить в виде $X = \{m/n + t | t \in R\}$, где $m \in R$, $n \in \zeroless{R}$ $(m, n) = 1$.
    Существует элемент $t_0 \in R$, минимизирующий норму $\elementnorm{m + n t_0}$.
    Тогда для любого элемента $x \in X$ положим
    \begin{equation*}
        \fr{x} = m/n + t_0.
    \end{equation*}

    Несложно заметить, что эта функция является дробной частью.
    Целую часть определим следующим образом
    \begin{equation*}
        \int{x} = x - \fr{x}.
    \end{equation*}
\end{remark}

Далее будем считать, что факториальное кольцо $R$ задано вместе дробной частью $\fr{\cdot}$ и целой частью $\int{\cdot}$.

\subsection{Числовые кольца}

Пусть $K$~-- числовое поле со степенью расширения равной $n$.
Пусть $R = \mathbb{Z}_K$~-- кольцо алгебраических целых элементов поля $K$.
И пусть $(e_i)_{1 \le i \le n}$~-- базис $\mathbb{Z}_K$.
Будем считать, что группа $\invertible{\mathbb{Z}_K}$ бесконечна и образована $r$ фундаментальными единицами и корнями единицы в $K$.
Будем обозначать их через $\{\varepsilon_1, \dots, \varepsilon_r\}$ и считать, что они известны.
Обозначим $\nu$~-- генератор корней из единицы в $K$ порядка $l$.

Обозначим через $N_{K/\mathbb{Q}}$ норму в числовом поле $K$.
Далее будем полагать, что $\mathbb{Z}_K$ евклидово по отношению к норме $N_{K/\mathbb{Q}}$.

\begin{definition}
    Для любого $a \in \zeroless{R}$ обозначим через $\overline{a} \in \zeroless{R}$ сопряжённый элемент определяемый как $\overline{a} = \elementnorm{a}/a$.
\end{definition}

Через $(\sigma_i)_{1 \le i \le n}$ обозначим вложения $K$ в $\mathbb{C}$.
Пусть $\sigma_i$ для $1 \le i \le r_1$ является действительным, а для $r_1 < i < r_1 + r_2$ мнимым и $\sigma_{i+r_2} = \overline{\sigma_{i}}$.
Определим функцию $\Phi(x): K \to \mathbb{R}^n$ следующим образом
\begin{equation*}
    \Phi(x) = \left(
        \sigma_1(x), \ldots, \sigma_{r_1}(x),
        \mathcal{R}\sigma_{r_1 + 1}(x), \ldots, \mathcal{R}\sigma_{r_1 + r_2}(x),
        \mathcal{I}\sigma_{r_1 + 1}(x), \ldots, \mathcal{I}\sigma_{r_1 + r_2}(x)
    \right)
\end{equation*}

Пусть $x, y \in \mathbb{R}^n$, $x = (x_i)_{1 \le i \le n}$ и $y = (y_i)_{1 \le i \le n}$.
Определим произведение в $\mathbb{R}^n$ следующим образом
\begin{equation*}
    (xy)_i =
    \begin{cases}
        x_i y_i                       & \textrm{ если } 1 \le i \le r_1,\\
        x_i y_i - x_{i+r_2} y_{i+r_2} & \textrm{ если } r_1 < i \le r_1+r_2,\\
        x_{i-r_2} y_i - x_i y_{i-r_2} & \textrm{ если } r_1+r_2 < i \le n.
    \end{cases}
\end{equation*}

Обозначим через $\mathcal{N}:\mathbb{R}^n \to \mathbb{R}$ норму в $\mathbb{R}^n$.
По определению считаем, что
\begin{equation*}
    \mathcal{N}(x) = \prod\limits_{i=1}^{r_1} x_i \prod\limits_{i=r_1+1}^{r_1+r_2} (x_i^2 + x_{i+r_2}^2).
\end{equation*}

Заметим, что для любых $x, y\in \mathbb{R}^n$ выполнено $\mathcal{N}(xy) = \mathcal{N}(x)\mathcal{N}(y)$.
А так же для любых $\xi\in K$ выполнено $N_{K/\mathbb{Q}}(\xi) = \mathcal{N}(\Phi(\xi))$.
Далее можно определить $\elementnorm{\cdot}: \mathbb{Z}_K \to \mathbb{N} \cup \{0, -\infty\}$ следующим образом.
\begin{equation*}
    \elementnorm{\xi} = \begin{cases}
        -\infty & \textrm{ если } x = 0,\\
        \mathcal{N}(\Phi(\xi)) & \textrm{ если } x \in \zeroless{R},\\
    \end{cases}
\end{equation*}
Заметим, что тогда $\mathbb{Z}_K$ является евклидовым относительно нормы $\elementnorm{\cdot}$.

Дробную часть $\fr{\cdot}$ в $K$ введем аналогично замечанию~\ref{remark:easy_fr}.

\begin{definition}
    Для $\xi \in K$ обозначим через $\int{\xi} \in \mathbb{Z}_K$ такой элемент, что
    \begin{equation*}
        \inf\limits_{z\in\mathbb{Z}_K} |\mathcal{N}(\Phi(\xi) - \Phi(z))| = |\mathcal{N}(\Phi(\xi) - \Phi(\int{\xi}))|.
    \end{equation*}
    Для $\xi \in K$ обозначим через $\fr{\xi} = \xi - \int{\xi}$.
\end{definition}

Из доказанного в работе~\cite{source:Lezowski} следует, что получающееся определение дробной части корректно.

\begin{statement}[\cite{source:Lezowski}]\label{statement:orbit}
    Пусть $x \in \mathbb{R}^n$.
    Обозначим
    \begin{equation*}
        m_{\overline{K}}(x) = \inf_{z\in\mathbb{Z}_K} |\mathcal{N}(x - \Phi(z))|.
    \end{equation*}
    Тогда для любого $\varepsilon \in \invertible{\mathbb{Z}_K}$, $Z \in \Phi(\mathbb{Z}_K)$ выполнено
    \begin{equation*}
        m_{\overline{K}}(\Phi(\varepsilon)x - Z) = m_{\overline{K}}(x).
    \end{equation*}
\end{statement}

\begin{definition}
    Пусть $x\in \mathbb{R}^n$.
    Обозначим
    \begin{equation*}
        \mathcal{F} = \left\{
            \sum\limits_{i=1}^n x_i\Phi(e_i) \Big| x_i \in \mathbb{Q}\cap[0, 1)
        \right\}
    \end{equation*}
    Определим \emph{орбиту элемента} $x$ следующим образом 
    \begin{equation*}
        \textrm{Orb}(x) = \left\{
        \Phi(\varepsilon)x - z \in \mathcal{F} \Big| \varepsilon \in \invertible{\mathbb{Z}_K}, z \in \mathbb{Z}_K
    \right\}.
    \end{equation*}
\end{definition}

Из утверждения \ref{statement:orbit} следует, что функция $m_{\overline{K}}$ принимает одно значение на всех элементах орбиты.

\begin{statement}[\cite{source:Lezowski}]
    Для любого элемента $x\in \mathbb{R}^n$ орбита $\textrm{Orb}(x)$ конечна тогда и только тогда, когда $x \in \Phi(K)$.
\end{statement}

\begin{definition}
    Для всех $1 \le i \le n$ обозначим
    \begin{equation*}
        \Gamma_i = \prod\limits_{j=1}^r \max\left\{
            |\sigma_i(\varepsilon_j)|, \frac{1}{|\sigma_i(\varepsilon_j)}
        \right\}.
    \end{equation*}

    А так же определим
    \begin{equation*}
        \Gamma(k) =
        \begin{cases}
            \left(
                \prod\limits_{j=1}^{n-1} \Gamma_j
            \right)^{\frac{1}{n}} k^{\frac{1}{n}}\ \textrm{если}\ K\ \textrm{действительное},\\
            \left(
                \prod\limits_{j=1}^{r_1} \Gamma_j \prod\limits_{j=1}^{r_1+r_2-1} \Gamma_j \Gamma_{j+r_2}
            \right)^{\frac{1}{n}} k^{\frac{1}{n}}\ \textrm{иначе},
        \end{cases}
    \end{equation*}
    где $k>0$.
\end{definition}

\begin{statement}[\cite{source:Lezowski}]\label{proposition:division_with_least_norm_remainder}
    Пусть $x \in \Phi(K)$ и $k > 0$.
    Для каждого $z \in \textrm{Orb}(x)$ обозначим
    \begin{equation*}
        \mathcal{I}_{z, k} = \{Z \in \Phi(\mathbb{Z}_K): |z_i-Z_i| \le \Gamma(k) \forall i, 1 \le i \le n\},
    \end{equation*}
    \begin{equation*}
        \mathcal{M}_k = \min\limits_{z \in \textrm{Orb}(x)} \min\limits_{Z \in \mathcal{I}_{z, k}} |\mathcal{N}(z-Z)|.
    \end{equation*}

    Тогда, если $\mathcal{M}_k \le k$, то $m_{\overline{K}}(x) = \mathcal{M}_k$.
\end{statement}

\begin{definition}
    Пусть $\ideal{n} \subseteq R$ произвольный идеал кольца $R$.
    Тогда существуют элементы $\alpha_1, \dots, \alpha_r \in R$ для $r \leq n$, что
    \begin{equation*}
        \ideal{n} = \{\xi_1 \alpha_1 + \dots + \xi_r \alpha_r | \xi_i \in R\}.
    \end{equation*}
    
    Такое представление будем обозначать $\ideal{n} = (\alpha_1, \dots, \alpha_r)$ и называть \emph{базисным представлением идеала}.
\end{definition}

\begin{definition}
    Пусть $\ideal{n} \subseteq R$ произвольный идеал кольца $R$.
    Тогда существуют элементы $e_1, \dots, e_r \in R$ для $r \leq n$, что
    \begin{equation*}
        \ideal{n} = \{x_1 e_1 + \dots + x_m e_m | x_i \in \mathbb{Z}\}.
    \end{equation*}

    Такое представление будем обозначать $\ideal{n} = (e_1, \dots, e_n)_{\mathbb{Z}}$ и называть \emph{$\mathbb{Z}$-представлением идеала}.
    Для оценки трудоемкости операций над идеалами $\mathbb{Z}$-представление будем записывать как матрицу $A \in \mathbb{\mathbb{Z}}^{n \times n}$, такую что её столбец под номером $i$ - это коэффициенты разложения $e_i$ в фиксированный целый базис $R$.
\end{definition}

\begin{statement}[\cite{source:Cohen, source:Post}]
    Любой идеал имеет $\mathbb{Z}$-представление.
\end{statement}

\begin{definition}
    Пусть $\ideal{n} \subseteq R$ произвольный идеал кольца $R$.
    Тогда существуют элементы $a \in \mathbb{N} \cup \{0\}$ и $\alpha \in R$, что 
    \begin{equation*}
        \ideal{n} = \{a \xi_1 + \alpha \xi_2 | \xi_1, \xi_2 \in \mathcal{O}_K\}.
    \end{equation*}

    Такое представление будем обозначать $\ideal{n} = (a, \alpha)_2$ и называть \emph{$2$-представлением идеала}.
    Для оценки трудоемкости операций над идеалами $2$-представление будем записывать как вектор из $\mathbb{Z}^n$ состоящий из коэффициентов разложения $\alpha$ в целый базис и целого неотрицательное число $a$.
    $2$-представление является частным случаем базисного представления и любое $2$-представление задаёт идеал.
\end{definition}

\begin{statement}[\cite{source:Cohen, source:Post}]
    Любой идеал имеет 2-представление.
\end{statement}

\begin{statement}[\cite{source:Post}]
    Существует полиномиальный алгоритм перехода от $2$-представления к $\mathbb{Z}$-представлению и обратно.
\end{statement}

\begin{remark}
    Несмотря на то, что алгоритм полиномиальный, он может оказаться достаточно трудоёмким.
\end{remark}

Важной операцией над идеалами является их сравнение.
Так как $\mathbb{Z}$-представление идеала можно записать как матрицу, то можно рассмотреть следующий способ записи $\mathbb{Z}$-представления идеала.

\begin{definition}
    Будем говорить, что матрица $A \in \mathbb{Z}^{n \times n}$ записана в нормальной эрмитовой форме, если выполнены следующие условия:
    \begin{enumerate}
        \item $A_{i, j} = 0$, если $i > j$.
        
        \item $A_{i,i} > 0$ для любого $i$.
        
        \item Для любого $i > j$ выполнено $0 \leq A_{i, j} \leq A_{i, i}$.
    \end{enumerate}
\end{definition}

\begin{definition}
    Будем говорить, что $\mathbb{Z}$-представление идеала $\ideal{a} = (e_1,\dots,e_n)_{\mathbb{Z}}$ находится в нормальной эрмитовой форме, если соответствующая матрица является матрицей в эрмитовой нормальной форме.
\end{definition}

\begin{statement}[\cite{source:Cohen, source:Post}]
    Любой идеал может быть записан в нормальной эрмитовой форме, причём такое представление единственно.
\end{statement}

\begin{statement}[\cite{source:Kannan}]
    Существует полиномиальный алгоритм получения представления идеала в нормальной эрмитовой форме из его $\mathbb{Z}$-представления.
\end{statement}

\begin{corollary}
    Существуют алгоритмы для получения $\mathbb{Z}$-представления, 2-представления или представления в нормальной эрмитовой форме из любого из этих представлений.
\end{corollary}

\begin{definition}
    Пусть $\ideal{n}$ идеал кольца $R$ с целым базисом $E$.
    Пусть $\ideal{n} = (e_1, \dots, e_n)_{\mathbb{Z}}$.
    Обозначим
    \begin{equation*}
        l(\ideal{n}) = \max\limits_{i = \overline{1,n}, j = \overline{1,n}} |a_{ij}|, 
    \end{equation*}
    где $A = (a_{ij}) \in \mathbb{Z}^{n \times n}$ матрица соответствующая указанному $\mathbb{Z}$-представлению.
    Значение $l(\ideal{n})$ будем называть \emph{абсолютным значением идеала}.
    Логарифм абсолютного значения идеала характеризует количество памяти необходимое для того, чтобы закодировать его $\mathbb{Z}$-представление.
\end{definition}

\begin{remark}
    Если кольцо $R$ факториальное, любой идеал является главным, а значит любой идеал может быть задан с помощью порождающего его элемента.
    В таких кольцах идеал, как и любой элемент, будет кодироваться в виде вектора $\mathbb{Z}^n$ коэффициентов разложения в целый базис кольца $R$.
\end{remark}

\begin{remark}
    В работе используется оценка сложности перемножения двух натуральных чисел, получаемая из алгоритма Шенхаге-Штрассена~\cite{source:Schonhage}.
    Для упрощения записи оценки сложности алгоритмов в работе введем следующее обозначение.
    Пусть $f(L),$ $g(L)$ две различные функции натурального аргумента $L$.
    Будем писать $f(L) = \tilde O(g(L))$, если существует положительная функция $h(L)$, такая что $f(L) \le h(L)g(L)$ для любых $L \in \mathbb{N}$, и $h(L) = O(\log g(L)\log \log g(L))$.
    
    Любое положительная действительное число $C$ будем называть эффективно вычислимой константной, если оно зависит только от инвариантов поля $K$ и существует алгоритм нахождения данного числа.
\end{remark}

Далее предполагаем, что элементы заданы с помощью коэффициентов своего разложения в целый базис $R$.

\begin{proposition}\label{proposition:operations}
    Пусть $a, b \in \zeroless{R}$ и $l(a) \leq L$, $l(b) \leq L$, тогда $a + b$, $a b,$ $b/a$ (включая проверку условия $a|b$), $\elementnorm{a}$, $\overline{a}$ могут быть  вычислены за $\tilde{O}(\log L)$ бинарных операций.
\end{proposition}

\begin{remark}
    Из доказательства предложения~\ref{proposition:operations} следует, что существует константа $D$, такая что $\elementnorm(a) \leq D (l(a))^n$ для любого $a \in R$.
    Следовательно, по правилу Крамера, существуют константы $E$ и $q$, такие что $l(\overline{a}) \leq E L^q$ для любого $a \in R$.
\end{remark}

\begin{proposition}\label{proposition:mod}
    Существует константа $M$, такая что для любых $a, m \in \zeroless{R}$, $l(a) \leq L, l(m) \leq L$ за $\tilde{O}(\log L)$ бинарных операций может быть найдено $z \in R$, удовлетворяющее условию $a \equiv z \pmod{m}$ и $l(z) \leq M l(m)$.
    Такое $z$ будем обозначать $a \pmod{m}$.
\end{proposition}

\begin{corollary}\label{corollary:mod}
    Пусть $k \in \mathbb{N}$, и для $a, b, m \in \zeroless{R}$ выполнено $l(a) \leq L$, $l(b) \leq L$, $l(m) \leq L$.
    Тогда $a + b \pmod{m}$ и $a^k \pmod{m}$ могут быть определены за $\tilde{O}(\log L)$, $\tilde{O}(\log k \log L)$  бинарных операций соответственно.
\end{corollary}

\begin{proposition}\label{proposition:equality}
    Пусть идеалы $\ideal{a}$ и $\ideal{b}$ заданы в виде нормальной эрмитовой формы и $l(\ideal{a}), l(\ideal{b}) \leq L$.
    Тогда проверка равенства $\ideal{a} = \ideal{b}$ может быть выполнена за $O(\log L)$ бинарных операций.
\end{proposition}

\begin{proposition}\label{proposition:particular_equality}
    Пусть $\ideal{p} = (\alpha, a)_2$~-- простой идеал, заданный в виде 2-представления, а $\ideal{n}$ произвольный идеал заданный в виде $\mathbb{Z}$-представления, причём $l(\ideal{p}), l(\ideal{n}) \leq L$.
    Тогда проверка равенства $\ideal{p} = \ideal{n}$ может быть выполнена за $\tilde{O}(\log L)$ операций.
\end{proposition}

\begin{proposition}\label{proposition:norm}
    Пусть идеал $\ideal{a}$ задан в виде $\mathbb{Z}$-представления и $l(\ideal{a}) \leq L$.
    Тогда норма $\Nm(\ideal{a})$ может быть вычислена за $\tilde{O}(\log L)$ бинарных операций.
\end{proposition}

\begin{proposition}\label{proposition:congruence}
    Пусть идеал $\ideal{a}$ задан в виде $\mathbb{Z}$-представления и $l(\ideal{a}), l(a), l(b) \leq L$, то проверка сравнения $a \equiv b \pmod{\ideal{a}}$ может быть выполнена за $\tilde{O}(\log L)$ бинарных операций.
\end{proposition}

\begin{proposition}\label{proposition:residue_modulo_ideal}
    Пусть нетривиальный идеал $\ideal{n}$ задан с помощью $\mathbb{Z}$-представления и $l(\ideal{n}) \leq L$.
    Пусть $a \in R$ и $l(a) \leq L$.
    Тогда за $\tilde{O}(\log L)$ бинарных операций можно найти $z \in R$, такое что $z \equiv a \pmod{\ideal{n}}$ и $l(z) \leq N (l(\ideal{n}))^n$.
    Такое $z$ будем обозначать $a \pmod{m}$.
\end{proposition}

\begin{corollary}
    Пусть $k \in \mathbb{N}$ и $a, b \in \zeroless{R}$, $\ideal{n}$~-- нетривиальный идеал заданный с помощью $\mathbb{Z}$-представления.
    Пусть выполнено $l(a) \leq L$, $l(b) \leq L$, $l(\ideal{n}) \leq L$.
    Элементы $a + b \pmod{\ideal{n}}$ и $a^k \pmod{\ideal{n}}$, $l(z_1) \leq N (l(\ideal{n}))^n$, $l(z_2) \leq N (l(\ideal{n}))^n$, могут быть определены за $\tilde{O}(\log L)$, $\tilde{O}(\log k \log L)$  бинарных операций соответственно.
\end{corollary}

\begin{remark}
    Отметим, что все указанные операции могут быть выполнены за полиномиальное время в случае, когда идеалы заданы с помощью одного из представлений: $\mathbb{Z}$-представление, 2-представление, нормальная эрмитова форма; в силу того, что из одного представления может быть получено другое за полиномиальное время.
\end{remark}

\begin{statement}[\cite{source:Wikstrom}]\label{statement:GCD_Wikstrom}
    Пусть $R$ кольцо целых алгебраических элементов числового поля.
    Тогда для любых $a, b \in R$ наибольший общий делитель $(a, b)$ можно вычислить на $O(n^2)$ арифметических операций в $\mathbb{Z}$, где $n$ длина бинарной записи $a$ и $b$.
\end{statement}

\begin{statement}[\cite{source:Dedekind}(Теорема Дедекинда)]\label{statement:dedekind}
    Пусть $f(T)$ минимальный многочлен алгебраического числа $\theta$ в $\mathbb{Z}[\theta]$.
    Для простого рационального числа $p$, не делящего индекс $[\mathcal{O}_K:\mathbb{Z}[\theta]]$, запишем
    \begin{equation*}
        f(T) \equiv (\pi_1(T))^{e_1}\dots (\pi_g(T))^{e_g} \modul p,
    \end{equation*}
    где $\pi_i(T)$~-- различные монические неприводимые многочлены в $\mathbb{F}_p[T]$.
    Тогда
    \begin{equation*}
        (p) = \mathfrak{p}_1^{e_1}\dots \mathfrak{p}_g^{e_g},
    \end{equation*}
    где $\mathfrak{p}_i = (p_i, T_i(\theta))$, $T_i(T) \equiv \pi_i(T)(\modul p)$.
\end{statement}

\begin{statement}[\cite{source:Coppersmith}(Теорема Копперсмита)]\label{statement:coppersmith}
    Пусть
    \begin{equation*}
        f(x, y) = \sum\limits_{i, j = 0}^{\delta} p_{i, j} x^i y^j
    \end{equation*}
    неприводимый многочлен от двух переменных над $\mathbb{Z}$.
    Пусть $X, Y \ge 0$ такие, что $|x_0| \le X$ и $|y_0| \le Y$, где $(x_0, y_0)$ решение уравнения $f(x, y) = 0$.
    Обозначим
    \begin{equation*}
        W = \max_{i, j} |p_{i, j}| X^i Y^j.
    \end{equation*}
    Пусть $XY < W^{\frac{3}{2\delta}}$, то существует полиномиальный относительно $\log W$ и $2^\delta$ алгоритм, который позволяет найти такую пару $(x_0, y_0)$, что $f(x_0, y_0) = 0$, $|x_0| \le X$ и $|y_0| \le Y$.
\end{statement}

\section{Аналог критерия Эйлера}

\begin{definition}
    Пусть характер $\chi$ задан на множестве идеалов кольца $R$, не является главным и определен по модулю идеала $\ideal{n} \subset R$.
    Через $\ideal{p}_{\chi}$ обозначим идеал минимальной нормы, для которого $\chi(\ideal{p}_{\chi}) \neq 0, 1$.
\end{definition}

\begin{definition}
    Пусть $R$ дедекиндово кольцо с полем частных $K$ и $L$ расширение поля $K$ степени не меньше $2$.
    Будем говорить, что кольцо $R$ \emph{удовлетворяет условию A для идеала $\ideal{n}$}, если существует многочлен $f_R$, что для любого характера $\chi$, не являющегося главным и определенного по модулю $\ideal{n}$, выполнено
    \begin{equation*}
        \Nm{\ideal{p}_{\chi}} \le f_R(\log{\Nm{\ideal{n}}}).
    \end{equation*}
\end{definition}

\begin{remark}
    Из работы Баха~\cite{source:Bach} следует, что, если расширенная гипотеза Римана выполнена, то условие A выполнено для всех колец $\mathcal{O}_K$ целых алгебраических чисел числового поля $K$ и $f_{\mathcal{O}_K}(x) = 12x^2 + 12\log^2 \Delta_{K}$.
\end{remark}

\begin{remark}
    Из работы Анкени~\cite{source:Ankeny} следует, что, если обобщенная гипотеза Римана выполнена, то условие A выполнено для кольца целых чисел и $f_{\mathbb{Z}}(x) = 2x^2$.
\end{remark}

\begin{proposition}\label{proposition:condition_A_with_any_homomorphism}
    Пусть кольцо $R$ удовлетворяет условию A.
    Пусть $\chi: \invertible{R/\ideal{n}} \to G$ нетривиальный гомоморфизм.
    Тогда существует идеал $\ideal{a}$ взаимнопростой с $\ideal{n}$ и такой, что $\chi(\ideal{a}) \neq 1$ и
    \begin{equation*}
        \Nm{\ideal{a}} \le f_R(\log{\Nm{\ideal{n}}}).
    \end{equation*}
\end{proposition}

\begin{proposition}\label{proposition:miller_criteria_character}
    Пусть идеал $\ideal{p}$ простой с нечетной нормой.
    Тогда сравнение
    \begin{equation*}
        x^{\Nm{\ideal{p}} - 1} \equiv 1 \pmod{\ideal{p}^2}
    \end{equation*}
    имеет не более $\Nm{\ideal{p}} - 1$ решений относительно $x \in \invertible{R/\ideal{p}^2}$.
\end{proposition}

\begin{theorem}\label{theorem:euler_criteria}
    Пусть $\ideal{n}$~-- нетривиальный идеал нечетной нормы дедекиндового кольца $R$.
    Тогда $\ideal{n}$~-- простой идеал тогда и только тогда, когда для любого $a \in \invertible{R/\ideal{n}}$ выполнено
    \begin{equation*}
        a^{\frac{\Nm{\ideal{n}} - 1}{2}} \equiv \jacobi{a}{\ideal{n}} \pmod{\ideal{n}}.
    \end{equation*}

    Если кольцо $R$ факториальное и удовлетворяет условию A, то $\ideal{n}$~-- простой идеал тогда и только тогда, когда для любого $a \in \invertible{R/\ideal{n}}$, $\Nm{a} \le f_R(\Nm{\ideal{n}})$ выполнено
    \begin{equation*}
        a^{\frac{\Nm{\ideal{n}} - 1}{2}} \equiv \jacobi{a}{\ideal{n}} \pmod{\ideal{n}}.
    \end{equation*}
\end{theorem}

\begin{algorithm}\label{algorithm:solovay_strassen}
    Дан нетривиальный идеал $\ideal{n} \subset R$.
    Необходимо определить является ли он простым.

    \begin{enumerate}
        \item Вычислить $\Nm{\ideal{n}}$;
        
        \item Выбрать случайное $a \in \invertible{R/\ideal{n}}$;

        \item Вычислить $r_0 \equiv a^{\frac{\Nm{\ideal{n}} - 1}{2}} \pmod{\ideal{n}}$, $r_1 = \jacobi{a}{\ideal{n}}$;

        \item Если $r_0 \equiv r_1 \pmod{\ideal{n}}$, то вернуть ''неизвестно'' и завершить алгоритм;

        \item Вернуть ''$\ideal{n}$ не простой'' и завершить алгоритм.
    \end{enumerate}
\end{algorithm}

\begin{remark}
    Алгоритм \ref{algorithm:solovay_strassen} является вероятностным.
    Если был получен ответ "неизвестно", то можно выполнить алгоритм еще раз.
\end{remark}

\begin{proposition}
    Пусть $\ideal{n}$~-- не простой идеал.
    Тогда вероятность ответа "$\ideal{n}$ не простой" у алгоритма \ref{algorithm:solovay_strassen} не менее $1/2$.
\end{proposition}

\begin{remark}
    Если $\ideal{n}$~-- составной, то при выполнении алгоритма \ref{algorithm:miller_rabin} $k$ раз вероятность получить ответ ''$\ideal{n}$ не простой'' не меньше $1 - \frac{1}{2^k}$.
\end{remark}

\section{Аналог критерия Миллера}

\begin{theorem}\label{theorem:miller_criteria}
    Пусть $\ideal{n}$~-- нетривиальный идеал нечетной нормы дедекиндового кольца $R$.
    Пусть $\Nm{\ideal{n}} - 1 = 2^t u$, $(u, 2) = 1$.
    Тогда $\ideal{n}$~-- простой идеал тогда и только тогда, когда для любого $a \in \invertible{R/\ideal{n}}$, $a^u \not\equiv 1 \pmod{\ideal{n}}$ существует $k\in \{0, \dots, t-1\}$, такое что $a^{2^{k}u} \equiv -1 \pmod{\ideal{n}}$.

    Пусть кольцо $R$ факториальное и удовлетворяет условию A.
    Тогда $\ideal{n}$~-- простой идеал тогда и только тогда, когда для любого $a \in \invertible{R/\ideal{n}}$, $\Nm{a} \le f_R(\Nm{\ideal{n}})$, $(a, \ideal{n}) = 1$, $a^u \not\equiv 1 \pmod{\ideal{n}}$ существует $k\in \{0, \dots, t-1\}$, такое что $a^{2^{k}u} \equiv -1 \pmod{\ideal{n}}$.
\end{theorem}

\begin{algorithm}\label{algorithm:miller_rabin}
    Дан идеал $\ideal{n} \subset R$.
    Необходимо определить является ли он простым.

    \begin{enumerate}
        \item Найти $u, t \in \mathbb{N}$, что $\Nm{\ideal{n}} - 1 = 2^t u$ и $(2, u) = 1$;
        
        \item Выбрать случайный $a \in \invertible{R/\ideal{n}}$;

        \item Вычислить $r_0 \equiv a^u \pmod{\ideal{n}}$;

        \item Если $r_0 = 1$, то вернуть ''неизвестно'' и завершить алгоритм;

        \item Положить $k = 0$;

        \item Пока $k < t$ выполнять:
        \begin{enumerate}
            \item Если $r_k = -1$, то вернуть ''неизвестно'' и завершить алгоритм;

            \item Увеличить $k$ на $1$;

            \item Вычислить $r_{k+1} \equiv r_k^2 \pmod{\ideal{n}}$;
        \end{enumerate}

        \item Вернуть ''$\ideal{n}$ не простой'' и завершить алгоритм.
    \end{enumerate}
\end{algorithm}

\begin{remark}
    Алгоритм \ref{algorithm:miller_rabin} является вероятностным.
    Если был получен ответ "неизвестно", то можно выполнить алгоритм еще раз.
\end{remark}

\begin{proposition}
    Пусть $\ideal{n}$~-- не простой идеал.
    Тогда вероятность ответа "$\ideal{n}$ не простой" у алгоритма \ref{algorithm:miller_rabin} не менее $1/2$.
\end{proposition}

\begin{remark}
    Если $\ideal{n}$~-- составной, то при выполнении алгоритма \ref{algorithm:miller_rabin} $k$ раз вероятность получить ответ ''$\ideal{n}$ не простой'' не меньше $1 - \frac{1}{2^k}$.
\end{remark}

\begin{remark}
    Доказанные критерии и алгоритмы можно использовать при генерации простых идеалов.
    В частности для кольца целых чисел они используются в алгоритме Гордона построения сильных простых чисел.
    Генерация простых идеалов является обязательной частью в криптосистеме RSA в дедекиндовых кольцах.
    Такая криптосистема будет рассмотрена в главе~\ref{chapter:RSA-cryptosystem}.
\end{remark}

\section{Вычислительная сложность алгоритма Миллера-Рабина}

\begin{proposition}
    Пусть $K$ числовое поле и $R$ кольцо целых алгебраических элементов числового поля $K$.
    Пусть $\ideal{n} \subseteq R$ идеал кольца $R$.
    Для того, чтобы использовать алгоритм~\ref{algorithm:solovay_strassen} для $\ideal{n}$ требуется полиномиальное относительно $\log^2 l(\ideal{n})$ количество арифметических операций в $\mathbb{Z}$.
\end{proposition}

\begin{proposition}
    Пусть $\ideal{n} \subseteq R$ идеал дедекиндового кольца $R$.
    Для того, чтобы использовать алгоритм~\ref{algorithm:miller_rabin} для $\ideal{n}$ требуется $O(\log \Nm{\ideal{n}})$ арифметических операций.
\end{proposition}

\begin{proposition}
    Пусть $K$ числовое поле и $R$ кольцо целых алгебраических элементов числового поля $K$.
    Пусть $\ideal{n} \subseteq R$ идеал кольца $R$.
    Для того, чтобы использовать алгоритм~\ref{algorithm:miller_rabin} для $\ideal{n}$ требуется $\tilde{O}(\log^2 l(\ideal{n}))$ бинарных операций.
\end{proposition}

\section*{Выводы по главе \ref{chapter:Primality}}
\addcontentsline{toc}{section}{Выводы по главе \ref{chapter:Primality}}

В данной главе диссертации приведены основные определения и известные свойства алгебраических структур.
Доказаны аналоги критериев Эйлера и Миллера в случае дедекиндовых колец.
На основе доказанных критериев, построены алгоритмы тестирования на простоту и доказана их эффективность.
Получены достаточные условия того, что алгоритмы можно модифицировать, сделав детерминированными.
Получены оценки вычислительной сложности в случае произвольных дедекиндовых колец в терминах арифметических операций, а так же в числовых кольцах в терминах бинарных операций.

\begin{definition}
    Пусть $a$ и $b$ ненулевые элементы факториального кольца $R$.
    Для любых $k \in \mathbb{N}$ и $q_1, \dots, q_k \in R$ обозначим
    \begin{equation*}
        \mathcal{D}_{a, b}(q_1, \dots, q_k) = (r_{-1}, r_0, \dots, r_{k-1}, r_k) \in R^{k+2},
    \end{equation*}
    где $r_{-1} = a$, $r_0 = b$, $r_i = r_{i-2} - q_i r_{i-1}$, для $i = 1, \dots, k$.
    Выражение $\mathcal{D}_{a, b}(q_1, \dots, q_k)$ будем называть цепочкой делений для $a, b \in R$.
    Обозначим через $\mathcal{E}_{a, b}$ следующее множество:
    \begin{equation*}
        \mathcal{E}_{a, b} = \left\{
        \mathcal{D}_{a, b}(q_1, \dots, q_k) = (r_{-1}, \dots, r_k) \big| r_1, \dots, r_{k-1} \in \zeroless{R}, r_k = 0
        \right\}.
    \end{equation*}
\end{definition}

\begin{definition}
    Пусть $a$ и $b$ ненулевые элементы факториального кольца $R$.
    Цепочкой делений с выбором минимального по норме остатка для $a, b \in R$ будем называть такую цепочку делений, что $q_i = \textrm{int}(r_{i-2}/r_{i-1})$ для любого $i = 1, \dots, k$.

    Если цепочка делений с выбором минимального по норме остатка существует, то обозначим через $\mathcal{L}_{a, b}$ ее длину.
    Если ее не существует, то обозначим $\mathcal{L}_{a, b} = \infty$.
\end{definition}

\begin{definition}
    Обозначим через $\mathpzc{l}_{a, b}$ длину кратчайшей цепочки делений для $a, b \in \zeroless{R}$.
    \begin{equation*}
        \mathpzc{l}_{a, b} = \begin{cases}
            \min_{\mathcal{D}_{a, b}(q_1, \dots, q_k) \in \mathcal{E}_{a, b}} k, & \textrm{ если } \mathcal{E}_{a, b} \neq \emptyset\\
            \infty, & \textrm{ если } \mathcal{E}_{a, b} = \emptyset
        \end{cases}.
    \end{equation*}
\end{definition}

\begin{definition}
    Через $l_n(R)$ обозначим максимальную длину цепочки делений с выбором минимального по норме остатка для $a, b \in \zeroless{R}$ с ограниченной нормой.
    \begin{equation*}
        l_n(R) = \max \left\{
            \mathcal{L}_{a, b} \big| a, b \in \zeroless{R}, \elementnorm{a} \le \elementnorm{b} \le n
        \right\}.
    \end{equation*}
\end{definition}

\begin{remark}
    Теорему Кронекера-Валена в кольце целых чисел можно сформулировать в терминах этой главы следующим образом.
    Пусть $R = \mathbb{Z}$, $\elementnorm{x} = |x|$, $\fr{\alpha} = \alpha - [\alpha + 1/2]$.
    Тогда цепочка делений с выбором минимального по норме остатка является кратчайшей, т.е. $\mathcal{L}_{a, b} = \mathpzc{l}_{a, b}$ для всех $a, b \in \zeroless{R}$.

    Далее в этой главе определяются достаточные условия на факториальное кольцо $R$ с заданной нормой и дробной частью, при которых теорема Кронекера-Валена будет выполняться в этом кольце.
\end{remark}

\section{Теорема Кронекера-Валена в специальном классе факториальных колец}

\begin{definition}
    Определим функцию $\omega: F_1 \to F_1$ следующим образом
    \begin{equation*}
        \omega(\alpha) = \begin{cases}
            \fr{\alpha^{-1}}, \textrm{ если } \alpha \neq 0\\
            0, \textrm{ если } \alpha = 0
        \end{cases}
    \end{equation*}
\end{definition}

\begin{definition}
    Будем говорить, что $(x_0, \alpha, n) \in \zeroless{R} \times \zeroless{F_1} \times \mathbb{N}$ \emph{регулярная тройка}, если существуют
    \begin{itemize}
        \item $p, l \in \mathbb{N}$, $p \le n$ и $l \le p+1$,

        \item $\varepsilon_i \in \invertible{R}$, $b_i, c_i \in R$ для $i = 1, \dots, l-1$,

        \item $\varepsilon \in \{0, 1\}$,
    \end{itemize}
    для которых выполнены следующие условия
    \begin{itemize}
        \item $\beta_1 = \omega^{(p)}\left(\fr{(\alpha - x_0)^{-1}}\right)$;

        \item $\beta_{i+1} = (\varepsilon_i \beta_i + c_i)^{-1} + b_i$, $i = 1, \dots, l-1$;

        \item $\beta_{l} = \alpha^{(-1)^{\varepsilon}}$.
    \end{itemize}
\end{definition}

\begin{definition}
    Через $\mathcal{T}$ обозначим множество всех таких факториальных колец $R$, для которых существует $D_R \in \mathbb{N}$, что выполнено
    \begin{itemize}
        \item для всех $x_0 \in \zeroless{R}$, $\alpha \in \zeroless{F_1}$ тройка $(x_0, \alpha, D_R - 1)$ регулярная;

        \item если $D_R \ge 3$, то для любого $k \in [3, D_R] \cap \mathbb{N}$ и любых $x_0 \in \zeroless{R}$, $\alpha \in \zeroless{F_1}$ из равенства $\omega^{(k-2)}(\fr{(\alpha - x_0)^{-1}}) = 0$ следует, что тройка $(x_0, \alpha, k-2)$ регулярная.
    \end{itemize}
\end{definition}

\begin{definition}
    Обозначим
    \begin{equation*}
        [x_1: x_2: \dots: x_k] = x_{1} + \cfrac{1}{
            x_{2} + \cfrac{1}{
                x_{3} + \cfrac{1}{
                    \ddots + \cfrac{1}{
                        x_{k}
                    }
                }
            }
        }.
    \end{equation*}
    Будем говорить, что для $\alpha \in F$ и $k \in \mathbb{N}$ имеет место $(\alpha, k)$-разрешимость, если разрешимо уравнение
    \begin{equation*}
        \alpha = [x_1: x_2: \dots: x_k].
    \end{equation*}
\end{definition}

\begin{lemma}\label{lemma:omega_and_euclidean_algorithm}
    Пусть кольцо $R \in \mathcal{T}$.
    Для любых $\alpha \in F_1$ и $k \in \mathbb{N}$ имеет место $(\alpha, k)$-разрешимость тогда и только тогда, когда $\omega^{(k-1)}(\alpha) = 0$.
\end{lemma}

\begin{lemma}\label{lemma:euclidean_algorithm_and_minima}
    Пусть кольцо $R$ факториальное.
    Тогда для любых двух элементов $a, b \in \zeroless{R}$ выполнено равенство
    \begin{equation*}
        \mathcal{L}_{a, b} = \min\left\{
            k \in \mathbb{N} \big| \omega^{(k-1)}\left(\frac{a}{b}\right) = 0
        \right\},
    \end{equation*}
    где $\min \emptyset = \infty$.
\end{lemma}

\begin{theorem}\label{theorem:Kroneker_Vahlen_theorem_in_UFD}
    Пусть $R \in \mathcal{T}$.
    Тогда цепочка делений с выбором минимального по норме остатка является кратчайшей, т.е. $\mathcal{L}_{a, b} = \mathpzc{l}_{a, b}$ для всех $a, b \in \zeroless{R}$.
\end{theorem}

\subsection{Упрощенный метод проверки $R \in \mathcal{T}$}

Пусть дано некоторое факториальное кольцо $R$.
Рассмотрим алгоритм проверки принадлежности этого кольца классу $T$.

\begin{definition}
    Через $\mathcal{S}$ обозначим множество всех таких факториальных колец $R$, что для всех $x \in \zeroless{R}$ и $\alpha \in \zeroless{F_1}$ выполнено одно из условий
    \begin{itemize}
        \item $\int{(\alpha - x)^{-1}} \in \invertible{R} \cup \{0\}$;
        
        \item $x \int{(\alpha - x)^{-1}} + 1 \in \invertible{R}$.
    \end{itemize}
\end{definition}

\begin{lemma}
    Множество $\mathcal{S}$ содержится в $\mathcal{T}$.
\end{lemma}

Используя доказанную выше лемму, сформулируем метод проверки включения $R \in \mathcal{S}$.

\begin{algorithm}\label{algorithm:R_in_S}
    На вход подается факториальное кольцо $R$.
    
    \begin{enumerate}
        \item Построить множество
        \begin{equation*}
            J = \left\{
                x \in \zeroless{R} \big| \forall \alpha \in \zeroless{F_1}, \int{(\alpha - x)^{-1}} \in \invertible{R} \cup \{0\}
            \right\}
        \end{equation*}
        
        \item Для каждого $x_0 \in \zeroless{R} \setminus J$ построить множество
        \begin{equation*}
            Y(x_0) = \left\{
                f_{x_0}(\alpha) = \int{(\alpha - x_0)^{-1}} | \alpha \in \zeroless{F_1}
            \right\}
        \end{equation*}
        
        \item Для каждого $x_0 \in \zeroless{R} \setminus J$ построить множество
        \begin{equation*}
            U(x_0) = \left(
                \left\{
                    \frac{\varepsilon - 1}{x_0} | \varepsilon \in \invertible{R}
                \right\} \cap R
            \right) \cup \invertible{R}
        \end{equation*}
        
        \item Если $Y(x_0) \subseteq U(x_0)$ для всех $x_0 \in \zeroless{R} \setminus J$, то ответ ''$R \in \mathcal{S}$'', иначе ответ ''неизвестно''
    \end{enumerate}
\end{algorithm}

Докажем корректность этого алгоритма.

\begin{lemma}
    Если алгоритм~\ref{algorithm:R_in_S} вернул ответ ''$R \in \mathcal{S}$'', то $R \in \mathcal{S}$.
\end{lemma}

\subsection{Примеры проверки $R \in \mathcal{T}$}

\begin{example}\label{example:Z}
    Пусть $R = \mathbb{Z}$.
    Норма задана функцией $\elementnorm{x} = |x|$.
    Дробная часть $\fr{\alpha} = \alpha - \left[\alpha + \frac{1}{2}\right]$.
    Тогда
    \begin{itemize}
        \item $J = \left\{x \in \mathbb{Z} \big| |x| > 1\right\}$
        
        \item $Y(1) = \{-2, -1\}$, $Y(-1) = \{1, 2\}$
        
        \item $U(1) = \{-2, -1, 0, 1\}$, $U(-1) = \{-1, 0, 1, 2\}$
        
        \item $Y(1) \subseteq U(1)$, $Y(-1) \subseteq U(-1)$
    \end{itemize}
    
    Следовательно, $\mathbb{Z} \in \mathcal{S}$.
\end{example}

\begin{example}\label{example:P[t]}
    Пусть $P$ поле.
    Положим $R = \mathbb{P}[t]$.
    Норма задана функцией $\upsilon(f)=\deg f$, $f \in \mathbb{P}[t]$.
    Дробная часть $\textrm{fr}(m(t)/n(t))=r(t)/n(t)$, $m(t)\equiv r(t)(\textrm{mod}\ n(t))$, $\deg r < \deg n$, $m(t)/n(t) \in \mathbb{P}(t).$

    \begin{itemize}
        \item $\mathbb{J}=\mathbb{K}_{*}$;

        \item $\mathbb{K}_{*}\backslash\mathbb{J} = \emptyset$;
    \end{itemize}

    Следовательно, $\mathbb{P}[t] \in \mathcal{S}$.
\end{example}

\begin{example}\label{example:coordinate_ring_of_circle}
    Пусть $R$ координатное кольцо $\mathbb{Q}[i][x, y]/(x^2 + y^2 + 1)$.
    Известно, что $R$ изоморфно $\mathbb{Q}[i][t,  t^{-1}]$, где отображение задано следующим образом
    \begin{equation*}
        X \to \cos \theta \to \frac{e^{ix} + e^{-ix}}{2} \to \frac{t + t^{-1}}{2}\\
        Y \to \sin \theta \to \frac{-ie^{ix} + ie^{-ix}}{2} \to \frac{-it + it^{-1}}{2}
    \end{equation*}
    Так же известно, что $Q[i][t, t^{-1}]$ является факториальным кольцом.
    
    Обратимыми элементами в этом кольце  являются одночлены вида $\alpha t^k$, где $\alpha\in\mathbb{Q}[i]$, $k\in\mathbb{Z}$.
    Введем норму в этом кольце следующим образом
    \begin{equation*}
        \upsilon(x) = \upsilon\left(
            \sum_{i=k_1}^{k_2} \alpha_i t^i
        \right) = k_2 - k_1
    \end{equation*}
    где $\alpha_i\in\mathbb{Q}[i]$, если $x \neq 0$ и $\upsilon(0) = -\infty$.
    Проверим критерии.
    Рассмотрим два элемента кольца
    \begin{equation*}
        x = \sum_{i=k_1}^{k_2} \alpha_i t^i\\
        y = \sum_{i=l_1}^{l_2} \beta_i t^i
    \end{equation*}
    При  умножении максимальная степень будет равна $k_2 + l_2$, а минимальная $k_1 + l_1$.
    Тогда $\upsilon(xy) = k_2 + l_2 - k_1 - l_1 \ge k_2 - k_1$.
    Это равенство будет выполняться только если $l_2 - l_1 = 0$.
    А это будет означать, что $y\in\mathbb{I}$.
    
    Покажем, что $\textrm{fr}(\alpha) = \varepsilon^{-1}\textrm{fr}(\varepsilon\alpha)$ для $\varepsilon\in\mathbb{I}$, $\alpha\in\mathbb{K}$.
    Предположим, что $\textrm{fr}\left(\frac{m}{n}\right) = \frac{r}{n}$.
    Это означает, что $n | (m-r)$ и $r$ имеет минимальную норму из возможных.
    Так как норма не меняется при умножении на обратимый элемент, то $\varepsilon r$ будет иметь минимальную норму, а так же $n | (\varepsilon m - \varepsilon r)$.
    Следовательно, имеем $\textrm{fr}(\alpha) = \varepsilon^{-1}\textrm{fr}(\varepsilon\alpha)$ для $\varepsilon\in\mathbb{I}$, $\alpha\in\mathbb{K}$.
    Тогда $\textrm{int}(\alpha) = \varepsilon^{-1}\textrm{int}(\varepsilon\alpha)$ для $\varepsilon\in\mathbb{I}$, $\alpha\in\mathbb{K}$.
    
    Заметим, что умножив $\textrm{int}((\alpha-x)^{-1})$ на некоторый обратимый элемент можно сделать так, что $\alpha$ принадлежит полю частных кольца многочленов, а $x$ это многочлен.
    Тогда этот пример будет аналогичным примеру с многочленами.
    Следовательно, для данного координатного кольца выполнен аналог теоремы Кронекера-Валена.
\end{example}

\begin{example}\label{example:Z[t]}
    Пусть $R = \mathbb{Z}[t]$, $\upsilon(f)=\deg f$, $f \in \mathbb{Z}[t]$.
    Определим дробную часть в $\mathbb{F}=\mathbb{Z}(t)$ следующим образом.
    Рассмотрим отображение $\mathcal{A}:\mathbb{F}/\mathbb{K}\to\mathbb{F}$:
    \begin{equation*}
        \mathcal{A}(\mathbb{A})=m(t)/n(t),
    \end{equation*}
    где $\mathbb{A}=\{m(t)/n(t)+q(t)|q(t)\in\mathbb{Z}[t]\}.$
    Для произвольных $\mathbb{A}\in\mathbb{F}/\mathbb{K},$ $\alpha\in\mathbb{A}$ положим
    \begin{equation*}
        \begin{array}{c}
            \textrm{int}(\alpha)=r(t),\\
            \textrm{fr}(\alpha)=\mathcal{A}(\mathbb{A})-r(t),
        \end{array}
    \end{equation*}
    где $r(t)\in\mathbb{Z}[t]$ и для любого $p(t)\in\mathbb{Z}[t]$
    \begin{equation*}
        \lim_{t\to+\infty}\left|\frac{\mathcal{A}(\mathbb{A})-r(t)}{\mathcal{A}(\mathbb{A})-p(t)}\right|\le 1.
    \end{equation*}

    Докажем, что
    \begin{equation*}
        \mathbb{J}\supseteq\{f\in\mathbb{Z}[t]|\deg f>0\ \textrm{или}\ |f(t)|\equiv|x_{0}|>2\}.
    \end{equation*}
    Рассмотрим произвольные $\mathbb{A}\in\mathbb{F}/\mathbb{K}$ и $\alpha=m(t)/n(t)\in\mathbb{A}$, $\alpha=\textrm{fr}(\alpha)$.

    Имеем два случая.
    Предположим, что $\deg m > \deg n$.
    Из того, что $\alpha=\textrm{fr}(\alpha)$ и определения дробной части следует, что
    \begin{equation*}
        1 \ge \lim_{t\to+\infty}\left|\frac{\frac{m(t)}{n(t)}}{\frac{m(t)}{n(t)} - p(t)}\right| = \lim_{t\to+\infty}\left|\frac{m(t)}{m(t) - n(t)p(t)}\right|.
    \end{equation*}
    Следовательно, получаем, что для любых $p(t)\in \mathbb{Z}[t]$ выполнено $\deg m(t) \le \deg(m(t) - n(t)p(t))$.
    Предположим, что $\textrm{int}((\alpha-x_{0})^{-1})=r(t)\not\equiv0$ для некоторого $x_{0}(t)\in\mathbb{Z}[t]$.
    Тогда
    \begin{equation*}
        \begin{split}
            \textrm{fr}((\alpha - x_{0})^{-1}) = (\alpha - x_0)^{-1} - r =\\
            = \left(\frac{m}{n} - x_0\right)^{-1} - r = \frac{n}{m - x_0 n} - r =\\
            = \frac{n - r(m - nx_{0})}{m - nx_{0}}.
        \end{split}
    \end{equation*}
    Из того, что $\deg n < \deg m\le \deg(m-nx_{0})$ следует, что $\deg(n-r(m-nx_{0})) > \deg n$.
    Однако выше показано, что $\deg(n-r(m-nx_{0})) < \deg n$, исходя из определения дробной части.
    Получаем противоречие.
    Следовательно, если $\deg m > \deg n$, то для любого $x_0 \in \mathbb{Z}[t]$ выполнено $\textrm{int}((\alpha-x_{0})^{-1}) = 0$.

    Теперь предположим, что $\deg m \le \deg n$.
    Предположим, что $\textrm{int}((\alpha-x_{0})^{-1})=r(t)\not\equiv0$ для некоторого $x_{0}\in\mathbb{Z}[t]$, $\deg x_{0} > 0$.
    Тогда $\deg(m-nx_{0}) > \deg n$.
    Следовательно, имеем $\deg(n-r(m-nx_{0})) > \deg n$.
    А это противоречит определению дробной части.
    Следовательно, если $\deg m \le \deg n$, то для любого $x_{0}\in\mathbb{Z}[t]$, $\deg x_{0} > 0$ выполнено $\textrm{int}((\alpha-x_{0})^{-1}) = 0$.
    
    Предположим, что $\textrm{int}((\alpha-x_{0})^{-1})=r(t)\not\equiv0$ для некоторого $x_{0} = c \in \mathbb{Z}\setminus\{0, \pm 1, \pm 2\}$.
    Если $\deg m < \deg n$, то $\deg n < \deg(n-r(m-nc))$ или $r=const$.
    Первый вариант противоречит определению дробной части.
    Если $r=const$, то $\lim_{t\to+\infty}\left|\frac{n(t)-r(m(t)-n(t)c)}{n(t)}\right|=|1+rc|\ge2$, но это противоречит определению дробной части.
    Таким образом имеем $\deg m=\deg n$.
    Учитывая то, что $\alpha=\textrm{fr}(\alpha)$ и $\deg m=\deg n$ получаем $\lim_{t\to+\infty}\left|\frac{m(t)}{n(t)}\right|\le 0.5$.
    Если $\deg r>0,$ то $ \deg n<\deg(n-r(m-nc)),$ но это противоречие.
    Таким образом $r=const$ и
    \begin{align*}
        \lim_{t\to+\infty}\left|\frac{n(t)-r(m(t)-n(t)c)}{n(t)}\right| \ge |1+rc|-|r|/2 \ge\\
        \ge |r|(|c|-1/2)-1 \ge \frac{3}{2},
    \end{align*}
    но это противоречит определению дробной части.

    Таким образом $\mathbb{J}\supseteq\{f\in\mathbb{Z}[t]|\deg f>0\ \textrm{или}\ |f(t)|\equiv|x_{0}|>2\}$.
    Если $x_{0}\equiv\pm2,$ то $\textrm{int}((\alpha-x_{0})^{-1})=r(t)\not\equiv0$ тогда и только тогда, когда $\deg m=\deg n$ и $r(t)\equiv\pm1$, откуда следует, что $\pm2\in\mathbb{J}$.
    Несложно показать, что $0,\pm1\notin\mathbb{J}$.
    
    Тогда $\mathbb{K}_{*}\backslash\mathbb{J}\subseteq\{\pm1\}$.
    Вычисляем $\mathbb{Y}(1)=\{-2,-1\}$, $\mathbb{Y}(-1)=\{1,2\}$ и $\mathbb{U}(1)=\{-2,-1,0,1\}$, $\mathbb{U}(-1)=\{-1,0,1,2\}$.

    Следовательно, $\mathbb{Z}[t] \in \mathcal{S}.$
\end{example}

\begin{example}\label{example:a|b or b|a}
    Пусть $R$ евклидово кольцо такое, что для любых $a,b\in\mathbb{K}$ или $a|b$, или $b|a$.
    В этом случае $\mathbb{F}=\mathbb{K}\cup\{1/a|a\in\mathbb{K}_{*}\backslash\mathbb{I}\}$.

    Докажем, что $\mathbb{J}=\mathbb{K}_{*}$.
    Рассмотрим произвольный $\alpha\in\mathbb{F}_{1}^{*}$, тогда существует $b\in\mathbb{K}_{*}\backslash\mathbb{I}$ такой, что $\alpha=1/b$.
    Пусть $x\in\mathbb{K}_{*}$.
    Покажем, что $\textrm{int}((\alpha-x)^{-1}) = (\alpha-x)^{-1}$.
    Имеем $(\alpha-x)^{-1}=\frac{b}{1-bx}$.
    Предположим, что $\textrm{fr}(\frac{b}{1-bx}) \neq 0$, что эквивалентно утверждению, что $\frac{b}{1 - b x} = \frac{1}{c}$ для некоторого $c\in\mathbb{K}_{*}\backslash\mathbb{I}$.
    Так как $b$ и $1 - b x$ взаимнопростые, то $b\in\mathbb{I}$.
    Следовательно, имеем $\alpha-x=b^{-1}-x\in\mathbb{K}$, это означает, что $\alpha\in\mathbb{K}$, но это противоречит условию $\alpha \in \mathbb{F}_{1}^{*}$.
    Таким образом имеем $x\ \textrm{int}((\alpha-x)^{-1})+1=\frac{1}{1-bx}$.
    Предположим, что $1 - b x \in \mathbb{K}_{*}\backslash\mathbb{I}$.
    Так как $b$ и $1-bx$ взаимнопростые, то $b|(1-bx)$.
    Следовательно, $b\in\mathbb{I}$.
    Из последнего следует, что $\alpha-x=b^{-1}-x\in\mathbb{K}$, но это противоречит условию $\alpha\in\mathbb{F}_{1}^{*}$.
    Итого получаем $1-bx\in\mathbb{I}$.
    Следовательно, $x\ \textrm{int}((\alpha-x)^{-1}) + 1 \in \mathbb{I}$.
    
    Тогда $\mathbb{K}_{*}\backslash\mathbb{J} = \emptyset$.
    Следовательно, $\mathbb{K}\in \mathcal{S}$.
\end{example}

Рассмотрим пример факториального кольца, не лежащего в $\mathcal{S}$, но лежащего в $\mathcal{T}$.

\begin{example}\label{example:Z[i]}
    Пусть $R = \mathbb{Z}[i]$.
    Покажем, что кольцо $\mathbb{Z}[i]$ не принадлежит $\mathcal{S}.$
    Выберем $\alpha=\frac{9-4i}{20}$, $x=1$.
    Тогда $\textrm{int}((\alpha-x)^{-1})=-2+i \notin \mathbb{I} \cup \{0\}$ и $x \ \textrm{int}((\alpha-x)^{-1})+1=-1+i \notin \mathbb{I}.$

    Покажем, что $\mathbb{K}\in\mathcal{T}$.
    Заметим, что $\mathbb{F}_1=\{z\in\mathbb{C}|Re(z),Im(z)\in\mathbb{Q}\cap[-1/2,1/2[\}$.
    Положим $D_{\mathbb{K}}=3$ в определении множества $\mathcal{T}$.

    Проверим первое условие из определения $\mathcal{T}$.
    Рассмотрим произвольные $x_0\in\mathbb{Z}[i]\setminus\{0\}$ и $\alpha\in\mathbb{F}^*_1$.
    Обозначим $b=\textrm{int}((\alpha-x_0)^{-1})$.
    Пусть $p$~-- число из первого условия определения $\mathcal{T}$.
    Если $\upsilon(x_0)>5$, то при $p=1$ имеем $\beta_1=\alpha$.
    Если $b\in\mathbb{I}\cup\{0\}$, то для $p=1$ имеем $\beta_1=\textrm{fr}((\alpha-x_0)/(bx_0+1-\alpha))$, тогда $\beta_2=\alpha$.
    Далее предполагаем, что $b\not\in\mathbb{I}\cup\{0\}$ и $\upsilon(x_0)\le 5$.
    Заметим, что $\upsilon(x_0)\in\{1,2\}$.
    Достаточно рассмотреть только случаи $x_0=1$ и $x_0=1+i$ (так как для любого $x_0$ таком, что $\upsilon(x_0)\in\{1,2\}$ существует элемент $\varepsilon\in\mathbb{I}$ такой, что $x_0=\varepsilon$ или $x_0=(1+i)\varepsilon$).

    Пусть $x_0=1+i$, тогда $b=-1+i$.
    Положим $p=1$, и получим
    \begin{equation*}
        \beta_1=\textrm{fr}((\alpha-(1+i))/(\alpha(1-i)-1)),\ \beta_2=\alpha^{-1}.
    \end{equation*}

    Пусть $x_0=1$, тогда $b\in\{-2,-1\pm i,-2\pm i\}$.

    Если $b=-2$, то для $p=1$ имеем
    \begin{equation*}
        \beta_1=\textrm{fr}((\alpha-1)/(2\alpha-1)),\ \beta_2=\alpha^{-1}.
    \end{equation*}

    Если $b=-1\pm i$, то для $p=1$ имеем
    \begin{equation*}
        \beta_1=\textrm{fr}((\alpha-1)/(\alpha(1\mp i)\pm i)),\ \beta_2=\alpha^{-1}.
    \end{equation*}

    Пусть $b=-2+i$. Рассмотрим элемент
    \begin{equation*}
        \beta=\omega(\textrm{fr}((\alpha-1)^{-1}))=\textrm{fr}((\alpha-1)/(\alpha(2-i)-(1-i))).
    \end{equation*}

    Заметим, что
    \begin{equation*}
        \gamma=\textrm{int}((\alpha-1)/(\alpha(2-1)-(1-i)))\in\{1+i,1+2i,2+i,2+2i\}.
    \end{equation*}

    Если $\gamma=1+i$, то
    \begin{equation*}
        \beta=(1-\alpha(2+i))/(\alpha(2-i)-(1-i)).
    \end{equation*}

    Положим $p=2$, тогда имеем
    \begin{equation*}
        \begin{array}{c}
            \beta_1=\omega^{(2)}(\textrm{fr}((\alpha-1)^{-1}))=\textrm{fr}((\alpha(2-i)-(1-i))/(1-\alpha(2+i))),\\
            \beta_2=\alpha/(1-\alpha(2+i)),\\
            \beta_3=\alpha^{-1}.
        \end{array}
    \end{equation*}

    Если $\gamma=1+2i$, то
    \begin{equation*}
        \beta=((2+i)-\alpha(3+3i))/(\alpha(2-i)-(1-i)).
    \end{equation*}

    Положим $p=2$, тогда имеем
    \begin{equation*}
        \begin{array}{c}
            \beta_1=\omega^{(2)}(\textrm{fr}((\alpha-1)^{-1}))=\textrm{fr}((\alpha(2-i)-(1-i))/((2+i)-\alpha(3+3i))),\\
            \beta_2=\alpha/(1-\alpha(2+i)),\\
            \beta_3=\alpha^{-1}.
        \end{array}
    \end{equation*}

    Если $\gamma=2+i$, то
    \begin{equation*}
        \beta=((2-i)-4\alpha)/(\alpha(2-i)-(1-i)).
    \end{equation*}

    Положим $p=2$, тогда имеем
    \begin{equation*}
        \begin{array}{c}
            \beta_1=\omega^{(2)}(\textrm{fr}((\alpha-1)^{-1}))=\textrm{fr}((\alpha(2-i)-(1-i))/((2-i)-4\alpha)),\\
            \beta_2=\alpha/(1-\alpha(2+i)),\\
            \beta_3=\alpha^{-1}.
        \end{array}
    \end{equation*}

    Если $\gamma=2+2i$, то
    \begin{equation*}
        \beta=(3-\alpha(5+2i))/(\alpha(2-i)-(1-i)).
    \end{equation*}

    Положим $p=3$, тогда имеем
    \begin{equation*}
        \begin{array}{c}
            \beta_1=\omega^{(3)}(\textrm{fr}((\alpha-1)^{-1}))=\textrm{fr}((3-\alpha(5+2i))/(2-2i-\alpha(5-2i))),\\
            \beta_2=\alpha/(1-\alpha(2_i),\\
            \beta_3=\alpha^{-1}.
        \end{array}
    \end{equation*}

    Случай $b=-2-i$ аналогичен случаю $b=-2+i$.

    Проверим второе условие определения множества $\mathcal{T}$.
    Рассмотрим произвольные $x_0\in\mathbb{Z}[i]\setminus\{0\}$ и $\alpha\in\mathbb{F}^*_1$ такие, что $\omega(\textrm{fr}((\alpha-x_0)^{-1}))=0$.
    Так как во всех случаях, кроме $x_0=1$, $b=-2\pm i$, можно взять $D_{\mathbb{K}}=2$ вместо $D_{\mathbb{K}}=3$, то необходимо рассмотреть только случай $x_0=1$, $b=-2\pm i$.
    Пусть $b=-2+i$, тогда из условия $\omega(\textrm{fr}((\alpha-x_0)^{-1}))=0$ следует, что $\alpha\in\{1/(2+i),(2+i)/(3+3i),(2-i)/4,3/(5+2i)\}$.
    Для первого, второго и третьего элемента выполнено $\omega^{(2)}(\alpha)=0$.
    Для четвертого элемента условие $\alpha=\textrm{fr}(\alpha)$ не выполняется.
    Случай $b=-2-i$ аналогичен случаю $b=-2+i$.
    Следовательно, $\mathbb{K}=\mathbb{Z}[i]$ не принадлежит классу $\mathcal{S}$, но принадлежит классу $\mathcal{T}$.
\end{example}

В работе Роллетчека~\cite{source:Rolletschek_1990} приведен пример кольца, для которого цепочка делений с выбором минимального по норме остатка не является кратчайшей.
Следовательно, это кольцо не может принадлежать классу $\mathcal{T}$.

\begin{example}[\cite{source:Rolletschek_1990}]\label{example:Z[sqrt{-11}]}
    Пусть $R = \mathbb{Z}[\sqrt{-11}]$.
    Заметим, что множество
    \begin{equation*}
        \{6,-2i\sqrt{11},6,-3+i\sqrt{11},-1-i\sqrt{11},2,0\}
    \end{equation*}
    образует цепочку делений с выбором минимального по норме остатка для пары $(a,b)=(6,-2i\sqrt{11})$.
    Тогда $\mathcal{L}_{a,b}=5$.
    С другой стороны существует цепочка делений
    \begin{equation*}
        \{6,-2i\sqrt{11},-5+i\sqrt{11},3+i\sqrt{11},2,0\}.
    \end{equation*}
    
    Из этого следует, что $\mathpzc{l}_{a,b}\le 4\le \mathcal{L}_{a,b}$.
    Следовательно, теорема \ref{theorem:Kroneker_Vahlen_theorem_in_UFD} не выполняется.
    Предположим, что $\mathbb{Z}[i\sqrt{11}]\in\mathcal{T}$, но тогда по теореме \ref{theorem:Kroneker_Vahlen_theorem_in_UFD} получаем, что для любой пары $(c,d)\in\mathbb{K}_* \times\mathbb{K}_*$ выполнено $\mathpzc{l}_{c,d}=\mathcal{L}_{c,d}$.
    Получаем противоречие с приведенным примером.
    Аналогично можно показать, что лемма \ref{lemma:omega_and_euclidean_algorithm} не выполняется для $\mathbb{Z}[\sqrt{-11}].$
\end{example}

\subsection{Общий метод проверки $R \in \mathcal{T}$}

Приведем общий метод проверки включения $R \in \mathcal{T}$.

\begin{algorithm}\label{algorithm:R_in_T}
    На вход подается факториальное кольцо $R$.
    
    \begin{enumerate}
        \item Выбрать $D_R, M \in \mathbb{N}$
        
        \item Построить множество
        \begin{equation*}
            \mathbb{J} = \left\{
                x_0 \in R \Bigg| \int{\frac{1}{\alpha-x_0}} \in \invertible{R} \cup \{0\} \forall \alpha \in \zeroless{F}
            \right\}
        \end{equation*}
        
        \item Создать список $\mathbb{L}$, в котором будут храниться элементы из $R^i$, где $i \in \{1, \ldots, D_R - 1\}$

        \item Вычислить $\mathbb{L}=\mathbb{K}^*\setminus\mathbb{J}$. Мы будем хранить элементы из $\mathbb{L}$, в которых более двух компонент в множестве $\mathbb{L}_M$

        \item\label{step:every_element_in_L} Выбрать элемент $(x_0, \ldots, x_l) \in \mathbb{L}$ и удалить его из $\mathbb{L}$

        \item Вычислить
        \begin{equation*}
            \delta = \left(
                \left(
                    \ldots\left(
                        \left(
                            \alpha - x_0
                        \right)^{-1} - x_1
                    \right)^{-1} - \ldots
                \right)^{-1} - x_l
            \right)^{-1}
        \end{equation*}

        \item Построить множество
        \begin{equation*}
            \mathbb{A} = \left\{
                b \in R \big| b = \int{\delta}
            \right\}
        \end{equation*}

        \item\label{step:every_element_in_A} Для каждого элемента $x_{l+1} \in \mathbb{A}$ выполнить
        \begin{enumerate}
            \item Вычислить
            \begin{equation*}
                \beta_1 = \omega^{(l+1)} \left(
                    \int{(\alpha-x_0)^{-1}}
                \right) = \fr{(\delta-x_{l+1})^{-1}}
            \end{equation*}

            \item Попробовать найти такие $(\varepsilon_i) \in \invertible{R}$ и $(a_i), (b_i) \in R$, что $\elementnorm{a_i},\elementnorm{b_i} \le M$ и
            \begin{equation*}
                \begin{split}
                    \beta_{i+1}=\frac{1}{\varepsilon_i \beta_i + a_i} + b_i,\\
                    \beta_{l+2}=\alpha \textrm{ или } \beta_{l+2}=\alpha^{-1}
                \end{split}
            \end{equation*}

            \item Если такие элементы не нашлись и $l+1 \ge D_R - 1$, то вернуть ''выберите большие $D_R$ и $M$'' и завершить алгоритм
            
            \item Если такие элементы не нашлись и $l+1 < D_R - 1$, то добавить в множество $\mathbb{L}$ элемент $(x_0,\ldots,x_l,x_{l+1})$
            
            \item Если такие элементы нашлись, то перейти к следующему элементу в шаге~\ref{step:every_element_in_A}
        \end{enumerate}

        \item Если множество $\mathbb{L}$ не пустое, то выбрать другой элемент на шаге~\ref{step:every_element_in_L}
        
        \item Если множество $\mathbb{L}$ пустое и $D_R < 3$, то вернуть ''$R \in \mathcal{T}$'' и завершить алгоритм
        
        \item Для всех $k \in [3, D_R]$
        \begin{enumerate}
            \item Построить множество
            \begin{equation*}
                \mathbb{B} = \left\{
                    \alpha \in \zeroless{F} \big| \omega^{(k-2)}\left(
                        \fr{(\alpha-x_0)^{-1}}
                    \right) = 0
                \right\}
            \end{equation*}

            \item Для всех $\alpha_0 \in \mathbb{B}$ попробовать найти такие $(\varepsilon_i) \in \invertible{R}$ и $(a_i), (b_i) \in R$, что $\elementnorm{a_i}, \elementnorm{b_i} \le M$ и
            \begin{equation*}
                \begin{split}
                    \beta_{i+1}=\frac{1}{\varepsilon_i \beta_i + a_i} + b_i,\\
                    \beta_{k-1}=\alpha \textrm{ или } \beta_{k-1}=\alpha^{-1},
                \end{split}
            \end{equation*}
            где $\beta_1 = 0$

            \item Если такие элементы не были найдены, то вернуть ''выберите большие $D_R$ и $M$'' и завершить алгоритм
            
            \item Если такие элементы были найдены и проверены все $x_0 \in \zeroless{\mathbb{L}}$, то вернуть ''$R \in \mathcal{T}$'' и завершить алгоритм
        \end{enumerate}
    \end{enumerate}
\end{algorithm}

\begin{lemma}
    Пусть алгоритм~\ref{algorithm:R_in_T} вернул ''$R \in \mathcal{T}$''.
    Тогда $R \in \mathcal{T}$.
\end{lemma}

\section{Теорема Кронекера-Валена в кольце алгебраических целых чисел числового поля}

\subsection{Метод деления с выбором минимального по норме остатка}

\begin{lemma}
    Пусть $\xi \in K$ и $k > 0$.
    Обозначим $x = \Phi(\xi)$.
    Предположим, что $z' \in \textrm{Orb}(x)$ и $Z'\in\mathcal{I}_{z', k}$ такие элементы, что $\mathcal{M}_k = |\mathcal{N}(z'-Z')|$.
    Если $\mathcal{M}_k \le k$, то $\int{\xi}$ можно вычислить за $O(1)$ арифметических операций в $K$.
\end{lemma}

\begin{remark}
    Для произвольного $x \in \Phi(K)$ и $z \in \textrm{Orb}(x)$ будем предполагать, что известно $\varepsilon_z \in \invertible{\mathbb{Z}_K}$ такое, что $z = \Phi(\overline{\varepsilon_z \Phi^{-1}(x)})$, и более того можно вычислить $\varepsilon_z^{-1}$ за $O(1)$ арифметических операций.
\end{remark}

Используя доказанные выше утверждения, получаем следующий алгоритм.

\begin{algorithm}\label{algorithm:least_norm_remainder}
    Дано числовое поле $K$ и два элемента $a, b \in \mathbb{Z}_K$.
    Необходимо вычислить наименьший общий остаток $r$ при делении $a$ на $b$.

    \begin{enumerate}
        \item Вычислить $x = \Phi(a/b) \in \Phi(K)$;
        
        \item Вычислить $\textrm{Orb}(x)$;

        \item Выбрать произвольное действительное $k > 0$;

        \item Вычислить $\Gamma(k)$ \label{loop:1};

        \item Объявить переменные $z'$ и $Z'$, которые будут инициализированы позже;

        \item Для всех $z \in \textrm{Orb}(x)$
        \begin{enumerate}
            \item Вычислить $\mathcal{I}_{z, k}$;

            \item Для всех $Z \in \mathcal{I}_{z, k}$, если $z'$ и $Z'$ не инициализированы или $\mathcal{N}(z' - Z') > \mathcal{N}(z - Z)$ положить $z' = z$ и $Z' = Z$;
        \end{enumerate}

        \item Вычислить $\mathcal{M}_k = \mathcal{N}(z' - Z')$
        
        \item Если $\mathcal{M}_k > k$, то положить $k = \mathcal{M}_k$ и перейти к шагу \ref{loop:1}

        \item Вычислить $\int{\frac{a}{b}} = Z'\Phi((\varepsilon_z')^{-1})$

        \item Вернуть $r = a - b \int{\frac{a}{b}}$
    \end{enumerate}
\end{algorithm}

Оценим вычислительную сложность этого алгоритма.

\begin{proposition}
    Для любых $a, b \in \zeroless{\mathbb{Z}_K}$ наименьший по норме остаток $r$ при делении $a$ на $b$ можно найти, используя алгоритм \ref{algorithm:least_norm_remainder}, за $O(1)$ арифметических операций в $K$.
\end{proposition}

\subsection{Метод доказательства невыполнимости теоремы Кронекера-Валена}

В этой части работы будет представлен метод автоматического доказательства невыполнимости теоремы Кронекера-Валена в кольце целых алгебраических чисел числового поля $K$.
Метод доказательства приведен в алгоритме \ref{algorithm:kronecker_vahlen_common}.

\begin{algorithm}\label{algorithm:kronecker_vahlen_common}
    Дано числовое поле $K$.
    Требуется доказать, что теорема Кронекера-Валена не выполняется в $\mathbb{Z}_K$.
    
    \begin{enumerate}
        \item Взять произвольные $a, b \in \mathbb{Z}_K$;

        \item Используя алгоритм~\ref{algorithm:least_norm_remainder} вычислить цепочку делений с выбором минимального по норме остатка $\mathcal{D}_{a, b}$;

        \item Найти $c \in \mathbb{Z}_K$ такое, что $a = bx + c$ для некоторого $x \in \mathbb{Z}_K$;

        \item Используя алгоритм~\ref{algorithm:least_norm_remainder} вычислить цепочку делений с выбором минимального по норме остатка $\mathcal{D}'_{b,c}$;

        \item Если $\textrm{len}(\mathcal{D}_{a, b}) > \textrm{len}(\mathcal{D}'_{b, c}) + 1$, то теорема Кронекера-Валена не выполняется в $K$.
    \end{enumerate}
\end{algorithm}

\subsection{Теорема Кронекера-Валена в действительных квадратичных норменно-евклидовых кольцах}

Применим описанный выше метод в частном случае.
Предположим, что поле $K$ такое, что $\mathbb{Z}_K$ действительное квадратичное норменно-евклидово кольцо.
В методе будем рассматривать такие целые элементы $a$ и $b$, которые являются рациональными, т.е. $a, b \in \mathbb{Z}$.
А так же будем искать $c$ используя остаток при делении $a$ на $b$ на $\mathbb{Z}$.
Модифицированный метод приведен в алгоритме~\ref{algorithm:kronecker_vahlen_special}.

\begin{algorithm}\label{algorithm:kronecker_vahlen_special}
    Дано такое числовое поле $K$, что $\mathbb{Z}_K$ действительное квадратичное норменно-евклидово кольцо
    Требуется доказать, что теорема Кронекера-Валена не выполняется в $\mathbb{Z}_K$.

    \begin{enumerate}
        \item Взять произвольные $a, b \in \mathbb{Z}$;

        \item Используя алгоритм~\ref{algorithm:least_norm_remainder} вычислить цепочку делений с выбором минимального по норме остатка $\mathcal{D}_{a,b}$;

        \item Вычислить остаток $c$ при делении $a$ на $b$;

        \item Используя алгоритм~\ref{algorithm:least_norm_remainder} вычислить цепочку делений с выбором минимального по норме остатка $\mathcal{D}'_{b, c}$;

        \item Если $\textrm{len}(\mathcal{D}_{a, b}) > \textrm{len}(\mathcal{D}'_{b, c}) + 1$, то теорема Кронекера-Валена не выполняется в $K$;

        \item Вычислить $c = a \% b - b$;

        \item Используя алгоритм~\ref{algorithm:least_norm_remainder} вычислить цепочку делений с выбором минимального по норме остатка $\mathcal{D}''_{b,c}$;

        \item Если $\textrm{len}(\mathcal{D}_{a, b}) > \textrm{len}(\mathcal{D}''_{b, c}) + 1$, то теорема Кронекера-Валена не выполняется в $K$;
    \end{enumerate}
\end{algorithm}

Реализуем этот алгоритм на R и применим его для всех действительных квадратичных норменно-евклидовых колец.

\begin{theorem}\label{theorem:kronecker}
    Пусть поле $K$ такое, что $\mathbb{Z}_K$ действительное квадратичное норменно-евклидово кольцо.
    Тогда теорема Кронекера-Валена не выполняется в $\mathbb{Z}_K$.
\end{theorem}

\begin{proposition}
    Пусть $R = \mathbb{Q}[\sqrt{d}]$~-- квадратичное норменно-евклидово кольцо.
    Теорема Кронекера-Валена выполняется в $R$ тогда и только тогда, когда $d=-1, -2, -3, -7$.
\end{proposition}

\section{Теорема Ламе в факториальных кольцах}

Ранее было показано, что при определенных условиях на кольцо цепочка делений с выбором минимального по норме остатка является кратчайшей.
Однако существуют кольца, для которых эти условия не выполняются и в которых цепочка делений с выбором минимального по норме остатка не является кратчайшей.
Например, в работе~\cite{source:Rolletschek_1990} было показано, что теорема Кронекера-Валена не выполняется в кольце $\mathbb{Z}[\sqrt{-11}]$.
В работе~\cite{source:Cooke} было показано, что для колец целых алгебраических чисел с бесконечной группой единиц длина кратчайшей цепочки делений с выбором минимального по норме остатка ограничена константой.
Следовательно, важным вопросом является исследование асимптотического поведения длины кратчайшей цепочки делений.

\begin{definition}
    Пусть $d \neq 1$~-- целое число свободное от квадратов.
    Под \emph{фундаментальной областью поля} $\mathbb{Q}[\sqrt{d}]$ понимают множество, заданное следующим образом
    \begin{equation*}
        F(d) = \begin{cases}
            \left(
                \left[0, \frac{1}{2}\right] \times \left[0, \frac{1}{2}\right]
            \right) \cap \left(
                \mathbb{Q} \times \mathbb{Q}
            \right), \textrm{ если } d \not\equiv 1 \pmod 4\\
            \left(
                \left[0, \frac{1}{2}\right] \times \left[0, \frac{1}{4}\right]
            \right) \cap \left(
                \mathbb{Q} \times \mathbb{Q}
            \right), \textrm{ если } d \equiv 1 \pmod 4
        \end{cases}
    \end{equation*}
\end{definition}

\begin{definition}
    Пусть $d \neq 1$~-- целое число свободное от квадратов.
    Под окрестностью точки $\lambda = \lambda_1 + \lambda_2 \sqrt{d} \in \mathbb{Z}[\sqrt{d}]$ радиуса $r > 0$ понимают множество заданное следующим образом
    \begin{equation*}
        U(\lambda, r) = \left\{
            q_1 + q_2 \sqrt{d} \in Q[\sqrt{d}] \big| |(q_1 - \lambda_1)^2 - d(q_2 - \lambda_2)^2| < r
        \right\}.
    \end{equation*}

    Под окрестностью множества точек $\Lambda \subseteq \mathbb{Z}[\sqrt{d}]$ радиуса $r > 0$ понимают множество заданное следующим образом
    \begin{equation*}
        U(\Lambda, r) = \bigcup_{\lambda \in \Lambda} U(\lambda, r).
    \end{equation*}
\end{definition}

В работе~\cite{source:Selfridge} доказано следующее утверждение.

\begin{statement}[\cite{source:Selfridge}]\label{proposition:fundamental_in_circle}
    Пусть $d \neq 1$~-- целое число свободное от квадратов.
    Тогда кольцо $\mathbb{Z}[\sqrt{d}]$ является евклидовым относительно нормы числового поля $\mathbb{Q}[\sqrt{d}]$ тогда и только тогда, когда существует множество $\Lambda \subseteq \mathbb{Z}[\sqrt{d}]$, что
    \begin{equation*}
        F(d) \subseteq U(\Lambda, 1).
    \end{equation*}
\end{statement}

\begin{definition}\label{definition:euclidean_lambda}
    Для $m/n \in F_1$ рассмотрим функцию
    \begin{equation*}
        |m/n| = \begin{cases}
            \frac{\elementnorm{m}}{\elementnorm{n}}, & m \neq 0, (m, n) = 1\\
            0, & m = 0
        \end{cases}.
    \end{equation*}
    Обозначим
    \begin{equation*}
        \Lambda_K = \sup_{m/n \in F_1} |m/n|.
    \end{equation*}
\end{definition}

\begin{theorem}\label{theorem:euclidean_and_lambda}
    Если $R$~-- евклидово кольцо относительно нормы $\elementnorm{\cdot}$, то $\Lambda_R \in [0, 1]$.

    Если $R$~-- факториальное кольцо с мультипликативной нормой $\elementnorm{\cdot}$ и $\Lambda_R \in [0, 1)$, то $R$~-- евклидово относительно нормы $\elementnorm{\cdot}$ и $l_n(R) \le [\log_{\Lambda_R^{-1}} n] + 2$ для всех $n \in \mathbb{N}$, где $\log_{\infty} n = 0$.
\end{theorem}

\begin{theorem}
    Пусть $d \neq 1$ целое число свободное от квадратов.
    Если кольцо $\mathbb{Z}[\sqrt{d}]$ евклидово относительно нормы числового поля $\elementnorm{\cdot}$, то $l_n(\mathbb{Z}[\sqrt{d}]) = O(\log n)$.
\end{theorem}

\section*{Выводы по главе \ref{chapter:Kronecker-Vahlen theorem}}
\addcontentsline{toc}{section}{Выводы по главе \ref{chapter:Kronecker-Vahlen theorem}}

В данной главе диссертации получены достаточные условия для выполнения аналога теоремы Кронекера-Валена в факториальном кольце.
Разработан метод проверки этого достаточного условия.
Разработан метод автоматического доказательства невыполнимости теоремы Кронекера-Валена в факториальном кольце.
Используя этот метод доказано, что теорема Кронекера-Валена не выполняется во всех действительных квадратичных норменно-евклидовых кольцах.
Получены оценки длины цепочки делений с выбором минимального по норме остатка в евклидовых кольцах.


Изложенный далее алгоритм аналога криптосистемы RSA был предложен в работе Петуховой и Тронина~\cite{source:Petukhova}.
Была показана корректность полученной криптосистемы и представлены ограничения  на кольцо для ее эффективного применения.
В этой части исследуется криптосистема RSA в дедекиндовых кольцах с конечным полем остатков.
Целью является получение доказательств теорем, связанных с ее криптостойкостью.
Например теоремы Винера, теоремы об эквивалентности факторизации и взлома криптосистемы, а так же изучение методов взлома криптосистемы.

\begin{algorithm}[\cite{source:Petukhova}]\label{algorithm:RSA_in_dedekind}
    Аналог криптосистемы RSA в дедекиндовых кольцах.

    \begin{enumerate}
        \item Выбираются максимальные идеалы $\ideal{p}$, $\ideal{q}\in R$

        \item Вычисляется $\varphi(\ideal{N}),$ где $\ideal{N} = \ideal{p} \ideal{q}$

        \item Выбирается случайное целое $e \in [1, \varphi(\ideal{N})],$ $(e, \varphi(\ideal{N}))=1$

        \item Вычисляется целое положительное $d$ такое, что $ed \equiv 1 \pmod{\varphi(\ideal{N})}$
    \end{enumerate}

    Пара $(\ideal{N}, e)$ это публичный ключ $A$, пара $(\ideal{N}, d)$ секретный ключ $A$.
    Функцией шифрования называется

    \begin{equation*}
        \begin{array}{c}
            f: R/\ideal{N} \to R/\ideal{N},\\
            f(x) \equiv x^{e} \pmod{\varphi(\ideal{N})}.
        \end{array}
    \end{equation*}

    Функцией расшифрования называется

    \begin{equation*}
        \begin{array}{c}
            f^{-1}: R/\ideal{N} \to R/\ideal{N},\\
            f^{-1}(x) \equiv x^{d}\pmod{\varphi(\ideal{N})}.
        \end{array}
    \end{equation*}
\end{algorithm}

\begin{remark}
    Корректность приведенной криптосистемы гарантируется аналогом теоремы Эйлера для дедекиндовых колец.
\end{remark}

\section{Анализ аналога криптосистемы RSA}

Нетрудно заметить, что зная разложение на множители $\ideal{N} = \ideal{p}\ideal{q}$ для модуля криптосистемы RSA можно эффективно найти секретный ключ.
В некоторых случаях можно доказать обратное утверждение.

\begin{theorem}\label{theorem:factor}\ref{source:BSU_Journal_2020}
    Пусть $K$~-- числовое поле и $\mathbb{Z}_K$ его кольцо целых алгебраических элементов.
    Пусть $\mathbb{Z}_K$~-- кольцо с единственной факторизацией, $((N), e, d)$ параметры криптосистемы RSA в $\mathbb{Z}_K$.
    Если $d$ известно, то $N$ можно эффективно разложить на множители с вероятностью не менее $\frac{1}{2}$ за полиномиальное относительно длины бинарной записи $N$ количество арифметических операций в $\mathbb{Z}$.
\end{theorem}

Следующая теорема является аналогом теоремы Винера о малой секретной экспоненте \cite{source:Wiener}.

\begin{theorem}\label{theorem:Wiener}\ref{source:BSU_Journal_2020}
    Пусть $(\ideal{N},e,d)$, $\ideal{N}=\ideal{p} \ideal{q}$~-- параметры криптосистемы RSA в дедекиндовом кольце $R$.
    Пусть $\Nm{\ideal{q}} < \Nm{\ideal{p}} < \alpha^2 \Nm{\ideal{q}},$ где $\alpha > 1.$
    Если $d<\frac{1}{\sqrt{2\alpha+2}}(\Nm{\ideal{N}})^{1/4},$ то $d$ можно эффективно вычислить за полиномиальное относительно $\log \Nm{\ideal{N}}$ число бинарных операций.
\end{theorem}

\begin{remark}
    Доказанная выше теорема является основой для атаки Винера на криптосистему RSA.
    При соблюдении определенных условий на параметры криптосистемы, можно сделать использование этой атаки невозможным.
    Однако существуют атаки, от которых невозможно полностью защититься.
    
    Метод повторного шифрования является примером такой атаки.
    Предположим, что было перехвачено некоторое зашифрованное сообщение $y = x^e \pmod{\ideal{N}}$, где $x \in \mathbb{Z}_{K} / \ideal{N}$~-- некоторое сообщение.
    Построим последовательность $y_i = y^{e^i} \pmod{\ideal{N}}$, где $i \in \{1, 2, \ldots\}$.
    Используя свойства возведения в степень и то, что $\mathbb{Z}_{K} / \ideal{N}$ конечно, получаем, что существует таое $m \in \mathbb{N}$, что $y_m = y$.
    Тогда $y_{m-1} = x$.
    
    Единственный способ защиты от этого метода взлома состоит в том, чтобы сделать $m$ достаточно большим.
\end{remark}

Обозначим через $R_{\mathfrak{m}}$ и $R_{\mathfrak{m}}^{\times}$ аддитивную и мультипликативную группы вычетов по модулю $\mathfrak{m}$.
Заметим, что если $\mathfrak{m}=\mathfrak{m}_1 \mathfrak{m}_2$, то $R_{\mathfrak{m}}^{\times} \cong R_{\mathfrak{m}_1}^{\times} \times R_{\mathfrak{m}_2}^{\times}$.

\begin{theorem}\label{theorem:iterated}
    Пусть $\ideal{N} = \ideal{p} \ideal{q}$~-- модуль криптосистемы RSA в дедекиндовом кольце $R$.
    Предположим, что существуют различные простые числа $r$, $s$ и положительные целые числа $k$, $l$ такие, что $\varphi(\ideal{p}) = rk$, $\varphi(\ideal{q}) = sl$ и числа $r - 1$, $s - 1$ имеют различные простые делители $r_1$, $s_1$ соответственно.

    Пусть $y$ и $e$~-- независимые равномерно распределенные случайные величины со значениями в $R / \ideal{N}$ и $\invertible{\mathbb{Z}_{\varphi(\ideal{N})}}$ соответственно.
    Обозначим
    \begin{equation*}
        m_{e,y} = \min \{m \in \mathbb{N} | y_m = y\}.
    \end{equation*}
    Тогда выполняется неравенство
    \begin{equation*}
        P(m_{e,y} \ge r_1s_1)\ge(1-r^{-1})(1-s^{-1})(1-r_1^{-1})(1-s_1^{-1}).
    \end{equation*}
\end{theorem}

\begin{theorem}\label{theorem:d_is_known_2}
    Пусть $(\ideal{N}, e, d)$ параметры криптосистемы RSA в дедекиндовом кольце $R$, где $\Nm{\ideal{p}}$ и $\Nm{\ideal{q}}$ имеют одинаковую битовую длину.
    Пусть $e d \le (\Nm{\ideal{N}})^2$, $\Nm{\ideal{N}} \ge 3$.
    Если $d$ известно, то существует эффективный алгоритм, который позволяет найти $\Nm{\ideal{p}}$ и $\Nm{\ideal{q}}$.
\end{theorem}

\begin{remark}
    Если в условии теоремы~\ref{theorem:d_is_known_2} заменить неравенство $ed \le (\Nm{\ideal{N}})^2$ на более строгое $e d \le (\Nm{\ideal{N}})^{3/2}$, то получим, что
    \begin{equation*}
        k - \overline{k} < 6(\Nm{\ideal{N}})^{-3/2}(ed-1) < 6.
    \end{equation*}
    
    Следовательно, вычислив $\overline{k} = \frac{ed-1}{\Nm{\ideal{N}}}$, можно перебрать все возможные $k$ и для каждого вычислить $\varphi(\ideal{N})$, $\Nm{\ideal{p}}$, $\Nm{\ideal{q}}$.
\end{remark}

\begin{theorem}
    Пусть дедекиндово кольцо $R$ является евклидовым относительно некоторой нормы $\upsilon(\cdot)$ и $\Lambda_{R} < 1$, где $\Lambda_{R}$ задано в определении~\ref{definition:euclidean_lambda}.
    Тогда это кольцо главных идеалов.
    Для простоты будем обозначать идеалы соответствующими элементами кольца.

    Пусть $(N, e_1, d_1)$ и $(N, e_2, d_2)$ параметры криптосистемы RSA в $R$ и $(e_1, e_2) = 1$.
    Пусть перехвачены сообщения $c_1 \equiv m^{e_1} \pmod{N}$ и $c_2 \equiv m^{e_2} \pmod{N}$.
    Тогда сообщение $m$ можно вычислить за полиномиальное относительно $\log \upsilon(N)$ количество арифметических операций в $R$.
\end{theorem}

\begin{remark}
    В работе \cite{source:Vaskouski_CSIST} в доказательстве предложения 1 показано, что $\Lambda_{R} < 1$ во всех квадратичных норменно-евклидовых кольцах.
\end{remark}

Приведем примеры работы криптосистемы для некоторых координатных колец.

\begin{example}
	Рассмотрим кольцо многочленов от двух переменных над $\mathbb{Z}_2$.
	Рассмотрим координатное кольцо
	\begin{equation*}
		R = \frac{\mathbb{Z}_2(x, y)}{y-x} \cong \mathbb{Z}_2[x]
	\end{equation*}
	
	Это кольцо дедекиндово.
	Рассмотрим идеалы $\ideal{p} = (x^3 + x + 1)$ и $\ideal{q} = (x^3 + x^2 + 1)$.
	И $R/\ideal{p}$, и $R/\ideal{q}$ состоит из многочленов степени не более $2$.
	Следовательно, норма обоих идеалов равна $8$.
	
	Вычислим $\ideal{N} = \ideal{p}\ideal{q} = (x^6 + x^5 + x^4 + x^3 + x^2 + x + 1)$.
	Заметим, что $R/\ideal{N}$ состоит из многочленов степени не более $5$.
	Следовательно, $\Nm{\ideal{N}} = 64$.
	Вычислим $\varphi(\ideal{N}) = 7 * 7 = 49$.

	Заметим, что
	\begin{equation*}
		\begin{array}{l}
			x^{7i} = 1\\
			x^{7i+1} = x\\
			x^{7i+2} = x^2\\
			x^{7i+3} = x^3\\
			x^{7i+4} = x^4\\
			x^{7i+5} = x^5\\
			x^{7i+6} = x^5 + x^4 + x^3 + x^2 + x + 1
		\end{array}
	\end{equation*}

	Выберем случайное $e = 11$.
	Тогда $d = 9$.
	Возьмем элемент $x^4 + x^2 + 1$.
	Зашифруем его, а затем расшифруем
	\begin{equation*}
	    \begin{split}
    		(x^4 + x^2 + 1)^{11} = x^2 + x + 1\\
    		(x^2 + x + 1)^{9} = x^4 + x^2 + 1
	    \end{split}
	\end{equation*}
\end{example}

\begin{example}
    Рассмотрим кольцо многочленов от двух переменных над $\mathbb{Z}_2$. И рассмотрим координатное кольцо
    \begin{equation*}
        R = \frac{\mathbb{Z}_2(x, y)}{\langle xy - 1\rangle}.
    \end{equation*}
    
    Так как в этом координатном кольце $xy - 1 = 0$, то оно состоит из многочленов вида $xf(x) + yg(y) + c$.
    Из этого несложно заметить, что $R \cong \mathbb{Z}_2(x, x^{-1})$.
    
    Рассмотрим идеал $\ideal{p} = (x + 1)$.
    Его норма $2$, так как фактор кольцо $R/\ideal{p}$ состоит из элементов $0$ и $1$.
    Рассмотрим идеал $\ideal{q} = (x^2 + x + 1)$.
    Кольцо $R/\ideal{q}$ состоит из $x+1$, $x$, $1$, $0$.
    Следовательно, норма этого идеала $4$.

    Вычислим $\ideal{N} = \ideal{p}\ideal{q} = x^3 + 1$.
    Несложно заметить, что $R/\ideal{N}$ состоит из многочленов степени не более $2$.
    Значит норма $\Nm{\ideal{N}} = 8$.
    Вычислим $\varphi_K(\ideal{N}) = 3$

    Заметим, что
    \begin{equation*}
    	\begin{array}{l}
    		x^{3i} = 1\\
    		x^{3i+1} = x\\
    		x^{3i+2} = x^2
    	\end{array}
    \end{equation*}

    Выберем случайное $e = 2$.
    Тогда $d = 2$.
    Возьмем элемент $x^2 + 1$.
    Зашифруем его, а затем расшифруем
    \begin{equation*}
        \begin{split}
            (x^2 + 1)^2 = x^4 + 1 = x + 1\\
            (x + 1)^2 = x^2 + 1
        \end{split}
    \end{equation*}
\end{example}

\section{Факторизация идеалов}

В работе \cite{source:Darkey-Mensah} приводится алгоритм факторизации идеалов в дедекиндовых кольцах.
Однако у этого алгоритма есть определенные ограничения.
Они описаны в работе \cite{source:Darkey-Mensah} и состоят в том, что надо уметь вычислять радикал идеала, сумму идеалов и частное.
Так же две из трех частей алгоритма факторизации приведены только для случая координатных колец.

Пусть $R$ кольцо целых алгебраических элементов числового поля $K = \mathbb{Q}(\theta)$.
Будем предполагать, что поле $K$ фиксировано и, следовательно, известен индекс $[R: \mathbb{Z}[\theta]]$.
А так же разложение на простые идеалы всех простых делителей индекса.
В этом случае, используя теорему Дедекинда~\ref{statement:dedekind} можно построить полиномиальное сведение задачи факторизации идеала к задаче факторизации целых чисел.

\begin{algorithm}
    Алгоритм факторизации идеала числового кольца.

    \begin{enumerate}
        \item Пусть дан идеал $(N)$ в форме своего $2$-представления.
        
        \item Считаем норму идеала, равную норме элемента $N$ и раскладываем норму на множители одним из известных алгоритмов для факторизации целых чисел.
        Например методом решета числового поля или алгоритмом Шора.
        Получаем разложение
        \begin{equation*}
            n = \Nm{N} = \prod_{i=1}^{k} p_i^{\alpha_i}.
        \end{equation*}
        Таким образом, мы знаем, что
        \begin{equation*}
            (\Nm{N}) = \prod_{i=1}^{k} (p_i)^{\alpha_i}.
        \end{equation*}
    
        \item Факторизуем идеал $(p_i)$ с помощью теоремы Дедекинда \ref{statement:dedekind} и получаем двухэлементные представления идеалов
        \begin{equation*}
            (p_i) = \prod_{j=1}^{l_i} (p_i, f_{i, j}(\theta))
        \end{equation*}
    
        \item Преобразуем полученные простые идеалы в $\mathbb{Z}$-представление и объединяем равные.
        Получаем представление
        \begin{equation*}
            (\Nm{N}) = \prod_{i=1}^{l} \mathfrak{p}_i^{\beta_i}
        \end{equation*}
    
        \item Используем бинарный поиск для нахождения степеней, в которых $\mathfrak{p}_i$ входит в $(N)$.
    \end{enumerate}
\end{algorithm}

\begin{proposition}
    Разложить идеал $(p)$, используя теорему Дедекинда, можно за $O((n\log n + \log p)n\log n\log\log n\log^2 p)$ бинарных операций.
\end{proposition}

\begin{remark}
    Таким образом, зная разложение $(p_i)$ на произведение простых идеалов, можно найти одинаковые идеалы и разложение $(\Nm{N})$ на произведение различных идеалов за $O(P(n)Q(\log |N|))$ бинарных операций, так как $k \le \log \Nm{N}$ и $l_i \le n$.
    Таким образом, найти разложение идеала $(\Nm{N})$ на произведение различных простых идеалов можно за полиномиальное относительно $\log\Nm{N}$ количество бинарных операций, если разложение $\Nm{N}$ на множители известно.

    Это показывает, что аналог криптосистемы RSA в некотором смысле не дает никакого выигрыша при использовании в кольцах алгебраических целых чисел числовых полей.
\end{remark}

\begin{remark}
    Задача факторизации в дедекиндовых кольцах является более сложной, так как нет доказательства аналога теоремы Дедекинда в дедекиндовых кольцах.
\end{remark}

\section*{Выводы по главе \ref{chapter:RSA-cryptosystem}}
\addcontentsline{toc}{section}{Выводы по главе \ref{chapter:RSA-cryptosystem}}

В данной главе диссертации приведен аналог криптосистемы RSA в дедекиндовых кольцах.
Доказан аналог теоремы Винера о малой секретной экспоненте и другие теоремы, связанные с криптостойкостью криптосистемы RSA.
Приведен способ защиты от атаки повторного шифрования на аналог криптосистемы RSA.
Используя теоремы Копперсмита, доказан детерминированный аналог теоремы Винера.
Исследована задача факторизации идеалов в дедекиндовых кольцах.
Приведен метод полиномиального сведения задачи факторизации идеалов к задаче факторизации целых чисел в случае числовых колец, использующий теорему Дедекинда.

\end{document}
