\documentclass[_00_autoref.tex]{subfiles}
\begin{document}

\chapter*{\MakeUppercase{Список публикаций соискателя ученой степени}}

\renewcommand{\labelenumi}{\arabic{enumi}.}
\renewcommand{\theenumi}{\arabic{enumi}}


\vspace{-4ex}
\section*{\fontsize{14}{15}\selectfont Статьи в научных изданиях в соответствии с Положением о присуждении ученых степеней и присвоении ученых званий в Республике Беларусь}
\vspace{-4ex}

\begin{enumerate}

    \item \label{source:Vestnik_BSU_2013}
    Васьковский М.М., Кондратёнок Н.В. Конечные обобщенные цепные дроби в евклидовых кольцах // Вестник БГУ Серия 1 ''Физика, Математика, Информатика''.~--- 2013.~--- \textnumero~3.~--- С.~117-123.

    \item \label{source:NANB_2015}
    Vaskouski M., Kondratyonok N. Analogue of the RSA-cryprosystem in quadratic unique factorization domains // Доклады Национальной Академии Наук Беларуси.~--- 2015.~--- Т.~59, \textnumero~5.~--- С.~18-23.

    \item \label{source:JNT_2016}
    Vaskouski M., Kondratyonok N., Prochorov N. Primes in quadratic unique factorization domains // Journal of Number Theory.~--- 2016.~--- Vol.~168.~--- P.~101-116.

    \item \label{source:JSC_2016}
    Vaskouski M., Kondratyonok N. Shortest division chains in unique factorization domains // Journal of Symbolic Computation.~--- 2016.~--- Vol.~77.~--- P.~175-188.

    \item \label{source:NANB_2017}
    Васьковский М.М., Кондратёнок Н.В., Прохоров Н.П. Аналог теста Соловея-Штрассена в квадратичных евклидовых кольцах // Доклады Национальной Академии Наук Беларуси.~--- 2017.~--- Т.~61, \textnumero~5.~--- С.~28-32.

    \item \label{source:BSU_Journal_2020}
    Кондратёнок Н.В. Анализ RSA-криптосистемы в абстрактных числовых кольцах // Журнал Белорусского государственного университета. Математика. Информатика.~--- 2020.~--- \textnumero~1.~--- С.~13-21.

    \item \label{source:JSC_2021}
    Vaskouski M., Kondratyonok N. The Kronecker-Vahlen theorem fails in real quadratic norm-Euclidean fields // Journal of Symbolic Computation.~--- 2021.~--- Vol.~104.~--- P.~134-141.
\end{enumerate}

\vspace{-4ex}
\section*{\fontsize{14}{15}\selectfont Статьи в сборниках материалов научных конференций}
\vspace{-4ex}

\begin{enumerate}
\setcounter{enumi}{7}

    \item \label{source:Republican_Scientific_Conference_of_Students_and_Postgraduates_2013}
    Кондратёнок Н.В., Васьковский М.М. Цепные дроби в евклидовых кольцах // Материалы XVI Республиканской научной конференции студентов и аспирантов.~--- 2013.~--- Ч.~1.~--- С.~63-64.

    \item \label{source:XII_Belarussian_math_conference_2016}
    Васьковский М.М., Кондратёнок Н.В., Прохоров Н.П. Тест Соловея-Штрассена в квадратичных евклидовых кольцах // Материалы международной научной конференции ''XII Белорусская математическая конференция''.~--- 2016.~--- Ч.~5.~--- С.~15-16.

    \item \label{source:Collection_of_articles_by_laureates_2017}
    Кондратёнок Н.В., Прохоров Н.П. Критерии простоты в квадратичных кольцах с единственной факторизацией // Сборник статей лауреатов и авторов научных работ, получивших первую категорию XXIV Республиканского конкурса научных работ студентов.~--- 2017.~--- С.~19-20.

    \item \label{source:Collection_of_articles_by_laureates_2018}
    Кондратёнок Н.В., Прохоров Н.П. Аналог теоремы Кронекера-Валена и полиномиальные алгоритмы тестирования на простоту в числовых полях // Сборник статей лауреатов и авторов научных работ, получивших первую категорию XXV Республиканского конкурса научных работ студентов.~--- 2018.~--- С.~25.

    \item \label{source:Algebra_and_theory_of_algorithms}
    Васьковский М.М., Кондратёнок Н.В. Построение и анализ RSA-криптосистемы в числовых полях // Сборник материалов ''Всероссийская конференция ''Алгебра и теория алгоритмов'', посвященная 100-летию факультета математики и компьютерных наук Ивановского государственного университета''.~--- 2018.~--- С.~160-162

    \item \label{source:XIII_Belarussian_math_conference_2021}
    Кондратёнок Н.В. Свойства RSA-криптосистемы в абстрактных числовых кольцах // Материалы международной научной конференции ''XIII Белорусская математическая конференция''.~--- 2021.~--- Ч.~2.~--- С.~63-64.

    \item \label{source:CSIST_2022}
    Васьковский М.М., Кондратёнок Н.В. Аналог критерия Миллера в дедекиндовых кольцах с конечной нормой // Материалы международного научного конгресса по математике ''CSIST-2022''.~--- 2022.~--- Ч.~1.~--- С.~21-27.

\end{enumerate}

\end{document}
