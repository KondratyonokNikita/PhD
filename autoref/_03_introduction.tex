\documentclass[_00_autoref.tex]{subfiles}
\begin{document}

\chapter*{\MakeUppercase{Введение}\vspace{-2ex}}

Алгоритмическая теория чисел изучает вычислительные методы для исследования и решения задач теории чисел, например алгоритмы для проверки на простоту, целочисленной факторизации, нахождения решений диофантовых уравнений.
Интерес к алгоритмической теории чисел вызван тем, что она активно используется на практике, например в криптосистемах.

Во второй половине XX века начала активно развиваться информатика и криптография, что привело к увеличению активности работы в области алгоритмической теории чисел со стороны ведущих математиков.
В частности, были получены принципиально новые критерии простоты, приводящие к эффективным алгоритмам тестирования простоты и генерации больших простых чисел.
Примерами таких критериев служат критерий Эйлера и Миллера, приводящие к алгоритму Соловея-Штрассена и Миллера-Рабина соответственно.
В предположении справедливости расширенной гипотезы Римана в работе Н.~Анкени была доказана теорема, позволяющая получить детерминированный вариант алгоритма Миллера-Рабина.

В 2002 году в работе М.~Агравала, Н.~Каяла и Н.~Саксены было конструктивно доказано, что задача проверки на простоту в кольце целых чисел решается за полиномиальное относительно размера входных данных время.
Задача проверки на простоту существенно усложняется при переходе от целых чисел к более общим алгебраическим структурам.
Был получен полиномиальный детерминированный алгоритм проверки на простоту в конечнопорожденных дедекиндовых кольцах.

В диссертации доказываются новые критерии простоты идеалов в дедекиндовых кольцах.
Эти критерии являются аналогами критериев Эйлера и Миллера и позволяют строить эффективные тесты на простоту в дедекиндовых кольцах.

Алгоритм Евклида используется во многих теоретико-числовых алгоритмах, в частности при решении различных диофантовых уравнений.
Важным является исследование экстремальных свойств этого алгоритма, а именно длины получаемых при его выполнении цепочек делений.
Для кольца целых чисел известна теорема Кронекера-Валена о том, что цепочка делений с выбором минимального по абсолютной величине остатка является кратчайшей.
Вопрос оптимальности цепочек деления с выбором минимального по норме остатка решен в кольце гауссовых чисел, кольце многочленов, мнимых квадратичных кольцах, что показано в работах Э.~Баха, Э.~Калтофена, Г.~Роллетчека и Д.~Лазара.
Задача проверки теоремы Кронекера-Валена к мнимых квадратичных кольцах существенно сложнее задачи в кольце целых чисел, так как существует мнимое квадратичное кольцо, в котором эта теорема не выполняется.
Методы доказательства для мнимых квадратичных колец связаны с представлением элементов кольца точками на плоскости и не подходят для действительных квадратичных колец.
Задача становится еще сложнее ввиду того, что в кольцах с бесконечной группой единиц длину кратчайшей цепочки делений можно ограничить сверху константой.

В диссертации разработаны новые подходы к исследованию экстремальных свойств цепочек делений в факториальных кольцах.
Первый подход позволил разработать метод автоматического доказательства теоремы Кронекера-Валена в факториальных кольцах.
Второй подход позволил решить проблему проверки теоремы Кронекера-Валена в квадратичных кольцах.
Установлено, что теорема Кронекера-Валена выполняется в квадратичном норменно-евклидовом кольце $\mathcal{O}_{\mathbb{Q}[\sqrt{d}]}$ тогда и только тогда, когда $d \in \{-1, -2, -3, -7\}$.

Особый интерес представляет криптосистема RSA, так как ее идея достаточно проста, но при этом стойкость криптосистемы RSA основана на фундаментальной задаче факторизации.
Впервые она была предложена Р.Л.~Ривестом, А.~Шамиром и Л.М.~Адлеманом в 1977 году.
При изучении криптосистемы RSA можно выделить два типа задач: использование криптосистемы RSA в кольцах более общего вида и получение необходимых условий криптостойкости.
Известны аналоги криптосистемы RSA в кольце гауссовых чисел, кольце многочленов, дедекиндовых кольцах.
В работе К.А.~Петуховой и С.Н.~Тронина предложено обобщение криптосистемы RSA на случай дедекиндовых колец.

В диссертации доказываются обобщения теорем Винера, Кранакиса, необходимых условий криптографической стойкости криптосистемы RSA в дедекиндовых кольцах.

\end{document}
