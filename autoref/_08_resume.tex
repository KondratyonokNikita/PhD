\documentclass[_00_autoref.tex]{subfiles}
\begin{document}

\newpage
\centerline{\textbf{РЕЗЮМЕ}}

\vspace{-0.3ex}
\begin{center}
Кондратёнок Никита Васильевич

\textbf{Свойства теоретико-числовых и криптографических алгоритмов в дедекиндовых кольцах}
\end{center}
\vspace{-0.3ex}

\textit{Ключевые слова}:
дедекиндово кольцо, факториальное кольцо, евклидово кольцо, простые идеалы, тест на простоту, цепная дробь, криптосистема RSA.

\textit{Цель работы}:
доказательство новых критериев простоты идеалов в дедекиндовых кольцах и исследование свойств теоретико-числовых и криптографических алгоритмов в дедекиндовых кольцах.

\textit{Методы исследования}:
методы алгебры и алгоритмической теории чисел.

\textit{Полученные результаты и их новизна}:
в диссертации доказаны новые критерии простоты идеалов в дедекиндовых кольцах, обобщающие известные критерии Миллера и Эйлера, которые являются основой для построения эффективных алгоритмов тестирования простоты; разработаны новые методы исследования экстремальных свойств алгоритма Евклида в факториальных кольцах, на основе которых доказаны аналоги теорем Кронекера-Валена и Ламе; впервые исследованы свойства аналога криптосистемы RSA в дедекиндовых кольцах и найдены необходимые условия криптостойкости аналога криптосистемы RSA в дедекиндовых кольцах.

\textit{Рекомендации по использованию}:
результаты и методы диссертации могут быть использованы при проведении исследований свойств теоретико-числовых и криптографических алгоритмов в дедекиндовых кольцах или кольцах более общего вида.
Также результаты работы могут использоваться при чтении спецкурсов для студентов математических специальностей.

\textit{Область применения}:
результаты исследования могут быть применены в алгоритмической теории чисел, при исследовании свойств теоретико-числовых и криптографических алгоритмов.

\newpage
\centerline{\textbf{РЭЗЮМЭ}}

\vspace{-0.3ex}
\begin{center}
Кандрацёнак Мікіта Васільевіч

\textbf{Уласцівасці тэарэтыка-лікавых і крыптаграфічных алгарытмаў у дэдэкіндавых кольцах}
\end{center}
\vspace{-0.3ex}

\textit{Ключавыя словы}:
дэдэкіндава кольца, факторыяльнае кольца, эўклідава кольца, простыя ідэалы, тэст на прастату, ланцужны дроб, крыптасістэма RSA

\textit{Мэта працы}:
доказ новых крытэрыяў прастаты ідэалаў у дэдэкіндавых кольцах і даследаванне ўласцівасцей тэарэтыка-лікавых і крыптаграфічных алгарытмаў у дэдэкіндавых кольцах.

\textit{Метады даследавання}:
метады алгебры і алгарытмічнай тэорыі лікаў.

\textit{Атрыманыя вынікі і іх навізна}:
у дысертацыі даказаны новыя крытэры прастаты ідэалаў у дэдэкіндавых кольцах, абагульняючыя вядомыя крытэрыі Мілера і Эйлера, якія з'яўляюцца асновай для пабудовы эфектыўных алгарытмаў тэсціравання прастаты, распрацаваны новыя метады даследавання экстрэмальных уласцівасцяў алгарытму Еўкліда ў факторыяльных кольцах Ламе, упершыню даследаваны ўласцівасці і знойдзены неабходныя ўмовы крыптаўстойлівасці аналага крыптасістэмы RSA, які выкарыстоўвае ідэалы ў дэдэкіндавых кольцах.

\textit{Рэкамендацыі па выкарыстанні}:
вынікі і метады дысертацыі могуць быць выкарыстаны пры правядзенні даследаванняў уласцівасцяў тэарэтыка-лікавых і крыптаграфічных алгарытмаў у дэдэкіндавых кольцах або кольцах больш агульнага выгляду.
Таксама вынікі працы могуць выкарыстоўвацца пры чытанні спецкурсаў для студэнтаў матэматычных спецыяльнасцей.

\textit{Вобласць ужывання}:
вынікі даследавання могуць быць ужытыя ў алгарытмічнай тэорыі лікаў, пры даследаванні ўласцівасцей тэарэтыка-лікавых і крыптаграфічных алгарытмаў.

\newpage
\centerline{\textbf{SUMMARY}}

\vspace{-0.3ex}
\begin{center}
Kondratyonok Nikita Vasilyevich

\textbf{Properties of number-theoretic and cryptographic algorithms in Dedekind domains}
\end{center}
\vspace{-0.3ex}

\textit{Keywords}:
Dedekind domain, factorial ring, Euclidean ring, prime ideals, primality test, continued fraction, RSA cryptosystem

\textit{Objective}:
proof of new ideal primality criteria in Dedekind domains and investigation of the properties of number-theoretic and cryptographic algorithms in Dedekind domains.

\textit{Research methods}:
methods of Algebra and Algorithmic Number Theory.

\textit{The results obtained and their novelty}:
new ideal primality criteria in Dedekind domains are obtained.
Obtained criteria generalize the well-known Miller and Euler criteria which are the basis for constructing effective primality testing algorithms.
New methods for studying the extremal properties of the Euclid algorithm in factorial rings are developed.
Using developed methods analogues of Kronecker-Wahlen and Lame theorems were proved.
The properties of an analogue of the RSA cryptosystem in Dedekind domains were studied and the necessary conditions for the cryptographic security of an analogue of the RSA cryptosystem in Dedekind domains were found for the first time.

\textit{Recommendations for use}:
the results and methods of the dissertation can be used to study the properties of number-theoretic and cryptographic algorithms in Dedekind domains or rings of a more general form.
Also, the results of the work can be used when reading special courses for students of mathematical specialties.

\textit{Application area}:
the results of the study can be applied in algorithmic number theory, in the study of the properties of number-theoretic and cryptographic algorithms.

\end{document}
