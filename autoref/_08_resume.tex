\documentclass[_00_autoref.tex]{subfiles}
\begin{document}

\newpage
\centerline{\textbf{РЕЗЮМЕ}}

\vspace{-0.3ex}
\begin{center}
Кондратёнок Никита Васильевич

\textbf{Свойства теоретико-числовых и криптографических алгоритмов в дедекиндовых кольцах}
\end{center}
\vspace{-0.3ex}

\textit{Ключевые слова}.
Дедекиндово кольцо, факториальное кольцо, евклидово кольцо, простые идеалы, тест на простоту, цепная дробь, криптосистема RSA.

\textit{Цель работы}.
Доказательство новых критериев простоты идеалов в дедекиндовых кольцах и исследование свойств теоретико-числовых и криптографических алгоритмов в дедекиндовых кольцах.

\textit{Методы исследования}.
Методы алгебры и алгоритмической теории чисел.

\textit{Полученные результаты и их новизна}.
В диссертации доказаны новые критерии простоты идеалов в дедекиндовых кольцах, обобщающие известные критерии Миллера и Эйлера, которые являются основой для построения эффективных алгоритмов тестирования простоты; разработаны новые методы исследования экстремальных свойств алгоритма Евклида в факториальных кольцах, на основе которых доказаны аналоги теорем Кронекера-Валена и Ламе; впервые исследованы свойства аналога криптосистемы RSA в дедекиндовых кольцах и найдены необходимые условия криптостойкости аналога криптосистемы RSA в дедекиндовых кольцах.

\textit{Рекомендации по использованию}.
Результаты и методы диссертации могут быть использованы при проведении исследований свойств теоретико-числовых и криптографических алгоритмов в дедекиндовых кольцах или кольцах более общего вида.
Также результаты работы могут использоваться при чтении спецкурсов для студентов математических специальностей.

\textit{Область применения}.
Результаты исследования могут быть применены в алгоритмической теории чисел, при исследовании свойств теоретико-числовых и криптографических алгоритмов.

\newpage
\centerline{\textbf{РЭЗЮМЭ}}

\vspace{-0.3ex}
\begin{center}
Кандрацёнак Мікіта Васільевіч

\textbf{Ўласцівасці тэарэтыка-лікавых і крыптаграфічных алгарытмаў ў дедекиндовых кольцах}
\end{center}
\vspace{-0.3ex}

\textit{Ключавыя словы}.
Дедекиндово кольца, фактарыяла кольца, эўклідавай кольца, простыя ідэалы, тэст на прастату, ланцуговая дроб, криптосистема RSA.

\textit{Мэта работы}.
Доказ новых крытэрыяў прастаты ідэалаў у дедекиндовых кольцах і даследаванне уласцівасцяў тэарэтыка-лікавых і крыптаграфічных алгарытмаў ў дедекиндовых кольцах.

\textit{Метады даследавання}.
Метады алгебры і алгарытмічнай тэорыі лікаў.

\textit{Атрыманыя вынікі і іх навізна}.
У дысертацыі даказаны новыя крытэрыі прастаты ідэалаў у дедекиндовых кольцах, абагульняючыя вядомыя крытэрыі Мілера і Эйлера, якія з'яўляюцца асновай для пабудовы эфектыўных алгарытмаў тэставання прастаты; распрацаваны новыя метады даследавання экстрэмальных уласцівасцяў алгарытму Еўкліда ў фактарыяльнай кольцах, на аснове якіх даказаны аналагі тэарэм Кронекера-Валена і Ламе; ўпершыню даследаваны ўласцівасці аналага криптосистемы RSA ў дедекиндовых кольцах і знойдзеныя неабходныя ўмовы крыптаўстойлівасці аналага криптосистемы RSA ў дедекиндовых кольцах.У дысертацыі даказаны новыя крытэрыі прастаты ідэалаў у дедекиндовых кольцах, абагульняючыя вядомыя крытэрыі Мілера і Эйлера, якія з'яўляюцца асновай для пабудовы эфектыўных алгарытмаў тэставання прастаты; распрацаваны новыя метады даследавання экстрэмальных уласцівасцяў алгарытму Еўкліда ў фактарыяльнай кольцах, на аснове якіх даказаны аналагі тэарэм Кронекера-Валена і Ламе; ўпершыню даследаваны ўласцівасці аналага криптосистемы RSA ў дедекиндовых кольцах і знойдзеныя неабходныя ўмовы крыптаўстойлівасці аналага криптосистемы RSA ў дедекиндовых кольцах.

\textit{Рэкамендацыі па выкарыстанні}.
Вынікі і метады дысертацыі могуць быць выкарыстаны пры правядзенні даследаванняў уласцівасцяў тэарэтыка-лікавых і крыптаграфічных алгарытмаў ў дедекиндовых кольцах або кольцах больш агульнага выгляду.
Таксама вынікі працы могуць выкарыстоўвацца пры чытанні спецкурсаў для студэнтаў матэматычных спецыяльнасцяў.

\textit{Галіна прымянення}.
Вынікі даследавання могуць быць ужытыя ў алгарытмічнай тэорыі лікаў, пры даследаванні уласцівасцяў тэарэтыка-лікавых і крыптаграфічных алгарытмаў.

\newpage
\centerline{\textbf{SUMMARY}}

\vspace{-0.3ex}
\begin{center}
Kondratyonok Nikita Vasilyevich

\textbf{Properties of number-theoretic and cryptographic algorithms in dedekind domains}
\end{center}
\vspace{-0.3ex}

\textit{Keywords}.
Dedekind ring, factorial ring, Euclidean ring, prime ideals, primality test, continued fraction, RSA cryptosystem.

\textit{The purpose of the research}.
Proof of new criteria for the simplicity of ideals in Dedekind rings and investigation of properties of number-theoretic and cryptographic algorithms in dedekind rings.

\textit{Methods of the research}.
Methods of algebra and algorithmic number theory.

\textit{The obtained results and their novelty}.
In the dissertation, new criteria for the primality of ideals in Dedekind rings are proved, generalizing the well-known Miller and Euler criteria, which are the basis for constructing effective algorithms for testing simplicity; new methods for investigating the extreme properties of the Euclid algorithm in factorial rings are developed, on the basis of which analogues of the Kronecker-Valen and Lame theorems are proved; the properties of the analog of the RSA cryptosystem in Dedekind rings are investigated for the first time. the necessary conditions for the cryptographic stability of the analog of the RSA cryptosystem in dedekind rings are found.

\textit{Recommendations for use}.
The results and methods of the dissertation can be used in conducting research on the properties of number-theoretic and cryptographic algorithms in dedekind rings or rings of a more general type.
Also, the results of the work can be used when reading special courses for students of mathematical specialties.

\textit{Field of applications}.
The results of the study can be applied in algorithmic number theory, in the study of the properties of number-theoretic and cryptographic algorithms.

\end{document}
