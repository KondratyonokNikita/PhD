\documentclass[8pt, xcolor=x11names]{beamer}

\usetheme[numbers, totalnumbers, compress]{Berlin}
\usetheme{Berlin}
\setbeamercovered{transparent}
\setbeamerfont{institute}{size=\normalsize}
\beamertemplatenavigationsymbolsempty

\expandafter\def\expandafter\insertshorttitle\expandafter{%
  \hfill%
  \insertframenumber\,/\,\inserttotalframenumber}

\usepackage{polyglossia}
\setmainlanguage{russian}
\setotherlanguage{english}
\setkeys{russian}{babelshorthands=true}

\setmainfont{Times New Roman}
\setromanfont{Times New Roman} 
\setsansfont{Arial} 
\setmonofont{Courier New} 

\newfontfamily{\cyrillicfont}{Times New Roman} 
\newfontfamily{\cyrillicfontrm}{Times New Roman}
\newfontfamily{\cyrillicfonttt}{Courier New}
\newfontfamily{\cyrillicfontsf}{Arial}

\usepackage{amsfonts}
\usepackage{amsmath}
\usepackage{amssymb}

\usepackage{mathrsfs}
\usepackage{graphicx}
\usepackage{colortbl}

\DeclareMathAlphabet{\mathpzc}{OT1}{pzc}{m}{it}

\usepackage{algpseudocode}
\usepackage{algorithm}

\renewcommand{\algorithmicrequire}{\textbf{Дано:}}
\renewcommand{\algorithmicensure}{\textbf{Результат:}}
\floatname{algorithm}{Алгоритм}

\makeatletter
\long\def\beamer@@ssection*#1{\beamer@section[]{}}
\makeatother

\usepackage{multicol}

\newcommand{\notitle}{\vspace*{-3ex}}

\newcommand{\Nm}[1]{\textrm{Nm}(#1)}
\newcommand{\modul}{\textrm{mod }}
\newcommand{\ideal}[1]{\mathfrak{#1}}

\newcommand{\jacobi}[2]{\left(\frac{#1}{#2}\right)}

\newcommand{\zeroless}[1]{#1^{*}}
\newcommand{\elementnorm}[1]{\upsilon(#1)}
\newcommand{\invertible}[1]{I_{#1}}
\newcommand{\fr}[1]{\textrm{fr}\left(#1\right)}
\renewcommand{\int}[1]{\textrm{int}\left(#1\right)}

\newcommand{\Gal}[1]{\textrm{Gal}\left(#1\right)}

% \newcommand{\Orb}[1]{\textrm{Orb}(#1)}


\author[Кондратёнок Н.В.]{Кондратёнок Никита Васильевич}
\title[]{Свойства теоретико-числовых и криптографических алгоритмов в дедекиндовых кольцах}
% \institute[Кафедра высшей математики]{Кафедра высшей математики}
\date{Минск, 2022 г.}

\begin{document}

\begin{frame}
    \titlepage

    \begin{center}
        Диссертация на соискание ученой степени\\
        кандидата физико-математических наук по специальности\\
        01.01.06 ''Математическая~логика,~алгебра~и~теория~чисел''
    \end{center}

    \begin{center}
        Научный руководитель: доктор физ.мат.наук, доцент М.М. Васьковский
    \end{center}    
\end{frame}

\section{Введение}

\begin{frame}{Общая структура работы}
    \begin{block}{Основные результаты}
        \begin{itemize}
            \item Глава 1 \textendash\ Обзор литературы
            
            \item Глава 2 - Тестирование идеалов на простоту в дедекиндовых кольцах

            \item Глава 3 - Теорема Кронекера-Валера в факториальных кольцах
            \begin{itemize}
                \item Метод доказательства выполнимости аналога теоремы Кронекера-Валена в факториальных кольцах

                \item Метод автоматического поиска контрпримеров к аналогу теоремы Кронекера-Валена в числовых кольцах
            \end{itemize}
            
            \item Глава 4 - Аналог RSA-криптосистемы в дедекиндовых кольцах
        \end{itemize}
    \end{block}
\end{frame}

\begin{frame}
    \begin{block}{Числовое поле}
        Поле, содержащее $\mathbb{Q}$ и являющееся конечномерным векторным пространством над ним.
    \end{block}

    \begin{block}{Кольцо целых алгебраических чисел числового поля}
        Подкольцо числового поля, состоящее из корней многочлена $f(x) \in \mathbb{Z}[x]$, где старший коэффициент $1$.
    \end{block}
    
    \begin{block}{Примеры}
        \begin{itemize}
            \item $\mathbb{Z}$, $\mathbb{Z}[i] = \{a + bi | a, b\in\mathbb{Z}\}$
            \item $\mathbb{Z}[\sqrt{-2}] = \{a + b\sqrt{-2} | a, b\in\mathbb{Z}\}$
            \item $\mathbb{Z}[\sqrt{5}] = \left\{a + b\frac{1+\sqrt{5}}{2} | a, b\in\mathbb{Z}\right\}$
        \end{itemize}
    \end{block}
\end{frame}

\begin{frame}
    \begin{block}{Дедекиндово кольцо}
        Коммутативное кольцо с единицей и без делителей нуля, в котором любой ненулевой идеал раскладывается в конечное произведение простых идеалов.
    \end{block}

    \begin{block}{Норма идеала $\mathfrak{m}$}
        $\mathcal{N}(\mathfrak{m}) = |R/\mathfrak{m}| < \infty$
    \end{block}

    \begin{block}{Примеры}
        \begin{itemize}
            \item Кольцо целых алгебраических чисел
            
            \item Координатные кольца $K[x, y]/(f)$, где $f \in K[x, y]$
            \begin{itemize}
                \item $K[x, y]/(y - mx - b) \cong K[x]$
                \item $\mathbb{C}[x, y]/(x^2 + y^2 - 1) \cong \mathbb{C}[x, x^{-1}]$
                \item $\mathbb{Q}[x, y]/(x^2 + y^2 - 1) \not\cong \mathbb{Q}[x]$
            \end{itemize}
        \end{itemize}
    \end{block}
\end{frame}

\section{Тестирование идеалов на простоту}

\subsection{Определение условия A}

\begin{frame}
    \begin{block}{Условие A}
        Пусть характер $\chi$ задан на множестве идеалов кольца $R$, не является главным и определен по модулю идеала $\ideal{n} \subset R$.
        Через $\ideal{p}_{\chi}$ обозначим идеал минимальной нормы, для которого $\chi(\ideal{p}_{\chi}) \neq 0, 1$.
    
        Пусть $R$ дедекиндово кольцо с полем частных $K$.
        Пусть $L$ расширение поля $K$ степени не меньше $2$.
        Будем говорить, что кольцо $R$ \emph{удовлетворяет условию A для идеала $\ideal{n}$}, если существует многочлен $f_R$, что для любого характера $\chi$, не являющегося главным и определенного по модулю $\ideal{n}$, выполнено
        \begin{equation*}
            \Nm{\ideal{p}_{\chi}} \le f_R(\log{\Nm{\ideal{n}}}).
        \end{equation*}
    \end{block}

    \begin{block}{Примеры колец, удовлетворяющих условию $A$ (Ankeny, Bach)}
        \begin{itemize}
            \item Из работы Баха следует, что, если расширенная гипотеза Римана выполнена, то условие A выполнено для всех колец $\mathcal{O}_K$ целых алгебраических чисел числового поля $K$ и $f_{\mathcal{O}_K}(x) = 12x^2 + 12\log^2 \Delta_{K}$.

            \item Из работы Анкени следует, что, если обобщенная гипотеза Римана выполнена, то условие A выполнено для кольца целых чисел и $f_{\mathbb{Z}}(x) = 2x^2$.
        \end{itemize}
    \end{block}
\end{frame}

\subsection{Аналог критерия Миллера}

\begin{frame}
    \begin{block}{Аналог критерия простоты Миллера}
        Пусть $\ideal{n} \subset R$ \textendash\ нетривиальный идеал нечетной нормы.
        Пусть $\Nm{\ideal{n}} - 1 = 2^t u$, $(u, 2) = 1$.
        Тогда $\ideal{n}$ \textendash\ простой идеал $\Leftrightarrow$ $\forall a \in \invertible{R/\ideal{n}}$, $a^u \not\equiv 1 \pmod{\ideal{n}}$ $\exists k\in \{0, \dots, t-1\}$, такое что
        \begin{equation}\label{eq:miller}
            a^{2^{k}u} \equiv -1 \pmod{\ideal{n}}.
        \end{equation}
    
        Пусть кольцо $R$ факториальное и удовлетворяет условию A.
        Тогда $\ideal{n}$ \textendash\ простой идеал $\Leftrightarrow$ $\forall a \in \invertible{R/\ideal{n}}$, $\Nm{a} \le f_R(\Nm{\ideal{n}})$, $(a, \ideal{n}) = 1$, $a^u \not\equiv 1 \pmod{\ideal{n}}$ $\exists k\in \{0, \dots, t-1\}$, такое что выполнено сравнение \ref{eq:miller}.
    \end{block}
    
    \begin{block}{Идея доказательства}
        \begin{itemize}
            \item $\mathfrak{p}$ \textendash\ простой идеал $\Rightarrow$ $\textrm{Nm}(\mathfrak{p})=q^r,$ где $q$ \textendash\ рациональное простое, $r \in \mathbb{N}$
            
            % Polynomial algorithms for primality testing in algebraic number fields with class number 1_JNT proposition 10
            \item $\forall \alpha \in \mathbb{N} \setminus \{1\},$ $\exists z \in \mathcal{O}_{K,\mathfrak{p}^{\alpha}}^\times:$ $\textrm{ord}(z)=q^s(\textrm{Nm}(\mathfrak{p})-1),$ где $1 \le s \le (\alpha-1)r.$
            
            \item $\exists \mathfrak{p}^\alpha|\mathfrak{n}, \alpha > 1$
            \begin{itemize}
                \item $\exists a \in \mathcal{O}_{K, N}^\times:$ $\textrm{ord}(a)=ql$
                
                \item $\exists k \in \{1,\ldots, t-1\}:$ $a^{2^{k+1}u} \equiv 1 (\textrm{mod} \ \mathfrak{n})$ $\Rightarrow$ $q|(\textrm{Nm}(\mathfrak{n})-1).$
            \end{itemize}
                
            \item $\forall \mathfrak{p}^\alpha|\mathfrak{n}, \alpha = 1$
            \begin{itemize}
                \item $\mathcal{O}_{K,\mathfrak{n}}^\times\cong \mathcal{O}_{K,\mathfrak{p}_1}^\times \times\ldots\times \mathcal{O}_{K,\mathfrak{p}_l}^\times,$ $l>1,$ $\Rightarrow$ $\exists 2^l-1\ge 3$ элементов $\mathcal{O}_{K,\mathfrak{n}}^\times$ порядка 2 
                
                \item $\exists a \in \mathcal{O}_{K,\mathfrak{n}}^\times:$ $\textrm{ord}(a)=2$
            \end{itemize}
        \end{itemize}
    \end{block}
\end{frame}

\begin{frame}
    \begin{block}{Аналог алгоритма Миллера-Рабина}
        \begin{algorithmic}[1]
            \State определить $u, t \in \mathbb{N}$, $(u,2) = 1$, такие что $\textrm{Nm}(\mathfrak{n})-1 = 2^tu$
    
            \State выбрать $a \in R$, $a \neq 0$, $r_0\equiv a^u \pmod{\mathfrak{n}}$, $r_{k + 1} \equiv r_k^2 \pmod{\mathfrak{n}}$
            
            \If {$r_0 = 1$}
                \State\Return 'неизвестно'
            \EndIf

            \For {$k = 0$, $k < t$, $k = k + 1$}
                \If {$r_k = -1$}
                    \State\Return 'неизвестно'
                \EndIf
            \EndFor
            
            \State\Return '$\mathfrak{n}$ не является простым'
        \end{algorithmic}

        Вероятность ответа '$\mathfrak{n}$ не является простым' на одной итерации алгоритма Миллера-Рабина при условии, что $\mathfrak{n}$ не является простым, не меньше $1/2$.
    \end{block}
\end{frame}

\subsection{Аналог критерия Эйлера}

\begin{frame}
    \begin{block}{Аналог критерия простоты Эйлера}
        Пусть $\ideal{n} \subset R$ \textendash\ нетривиальный идеал нечетной нормы.
        Тогда $\ideal{n}$ \textendash\ простой идеал $\Leftrightarrow$ $\forall a \in \invertible{R/\ideal{n}}$ выполнено
        \begin{equation}\label{eq:euler}
            a^{\frac{\Nm{\ideal{n}} - 1}{2}} \equiv \jacobi{a}{\ideal{n}} \pmod{\ideal{n}}.
        \end{equation}
    
        Если кольцо $R$ факториальное и удовлетворяет условию A, то $\ideal{n}$ \textendash\ простой идеал $\Leftrightarrow$ $\forall a \in \invertible{R/\ideal{n}}$, $\Nm{a} \le f_R(\Nm{\ideal{n}})$ выполнено сравнение \ref{eq:euler}.
    \end{block}

    \begin{block}{Аналог алгоритма Соловея-Штрассена}
        \begin{algorithmic}[1]
            \State выбрать $a \in R$, $a \neq 0$

            \If {$((a), \mathfrak{n}) \not\in R^{\times}$}
                \State\Return '$\mathfrak{n}$ не является простым'
            \EndIf

            \State $r_0 = a^{(\textrm{Nm}(\mathfrak{n}) - 1) / 2} \pmod{\mathfrak{n}}$, $r_1 = \left[\frac{a}{\mathfrak{n}}\right] \pmod{\mathfrak{n}}$
    		
    		\If  {$r_0 \equiv r_1 \pmod{\mathfrak{n}}$}
    		    \State\Return 'неизвестно'
    		\Else
    		    \State\Return '$\mathfrak{n}$ не является простым'
    		\EndIf
        \end{algorithmic}

        Вероятность ответа '$\mathfrak{n}$ не является простым' на одной итерации алгоритма Соловея-Штрассена при условии, что $\mathfrak{n}$ не является простым, не меньше $1/2$.
    \end{block}
\end{frame}

\section{Теорема Кронекера-Валена}

\subsection{Введение}

\begin{frame}
    \begin{block}{Алгоритм Евклида и цепочки делений в факториальных кольцах}
        \begin{itemize}
            \item $a = 8$, $b = 5$
            
            \item $8 = 1\cdot 5 + 3,$ $5 =  1\cdot 3+2,$ $3=1\cdot 2+1,$ $2=2\cdot 1+0$ $\Rightarrow$ $l_{a,b}=4$
            
            \item $8=2\cdot 5+(-2),$ $5=(-2)\cdot (-2)+1,$ $-2=(-2)\cdot 1+0$ $\Rightarrow$ $l_{a,b}=3$
        \end{itemize}
    \end{block}

    \begin{block}{Теорема Кронекера-Валена (1901)}
        Для любых $a,$ $b\in \mathbb{N}$ алгоритм Евклида с выбором наименьших по абсолютной величине остатков на каждом шаге имеет минимальную длину.
    \end{block}
    
    \begin{block}{Аналоги теоремы Кронекера-Валена}
        \begin{itemize}
            \item верна в кольце многочленов над полем (Lazard, 1977)
            
            \item верна в кольце целых алгебраических чисел $\mathbb{Q}[\sqrt{d}]$, где $d<0, d \neq -11$ (Rolletschek, 1990)
        \end{itemize}
    \end{block}
\end{frame}

\begin{frame}
    \begin{block}{\vspace*{-3ex}}
        $$
            \mathbb{F}_{1}=\{\alpha\in\mathbb{F}|\alpha=\textrm{fr}(\alpha)\}
        $$
        $$
            \omega:\mathbb{F}_{1}\to\mathbb{F}_{1}, \omega(\alpha)=
            \begin{cases}
                \textrm{fr}(\alpha^{-1}), & \alpha\neq0,\\
                0, & \alpha = 0.
            \end{cases}
        $$
    \end{block}
    
    \begin{block}{Лемма о длине цепочки делений с выбором минимального по норме остатка}
        Пусть $\mathbb{K}$ кольцо с единственной факторизацией.
        Тогда $\forall a, b \in \mathbb{K}_{*}\times\mathbb{K}_{*}$:
        \begin{eqnarray*}
            \mathcal{L}_{a,b}=\textnormal{min}\{k\in\mathbb{N}|\omega^{(k-1)}(\textrm{fr}(a/b))=0\}
        \end{eqnarray*}
    \end{block}
    
    \begin{block}{Идея доказательства}
        \begin{itemize}
            \item $\mathcal{L}_{a,b} = \infty$ $\Rightarrow$ $\forall k \in \mathbb{N}: \omega^{(k-1)}(\textrm{fr}(a/b)) \neq 0$
            
            \item $\frac{a}{b}=q_{1}+\frac{r_{1}}{b}, \frac{b}{r_{1}}=q_{2}+\frac{r_{2}}{r_{1}}, \ldots, \frac{r_{k-3}}{r_{k-2}}=q_{k-1}+\frac{r_{k-1}}{r_{k-2}}, \frac{r_{k-2}}{r_{k-1}}=q_{k}$
            
            \item
            $
                \omega\left(\textrm{fr}\left(\frac{a}{b}\right)\right)=\textrm{fr}\left(\frac{b}{r_{1}}\right),
                \ldots,
                \omega^{(k-1)}\left(\textrm{fr}\left(\frac{a}{b}\right)\right)=\textrm{fr}\left(\frac{r_{k-2}}{r_{k-1}}\right)=0.
            $
            
            \item $\omega^{(k-1)}(\textrm{fr}(a/b))=0$
        \end{itemize}
    \end{block}
\end{frame}

\begin{frame}
    \begin{block}{Регулярная тройка $(x_{0},\alpha,n) \in \mathbb{K}_{*} \times \mathbb{F}_{1}^{*} \times \mathbb{N}$}
        \begin{itemize}
            \item $\exists p, l \in \mathbb{N}$, $p\le n$ и $l\le p+1$: $\exists \varepsilon_{i} \in \mathbb{I}$, $\varepsilon \in \{0,1\}$, $b_{i}, c_{i} \in \mathbb{K}, i = \overline{1,l-1}$
            
            \item Выполнено
            \begin{eqnarray*}
                \begin{array}{c}
                    \beta_{1}=\omega^{(p)}\left(\textrm{fr}\left(\frac{1}{\alpha-x_{0}}\right)\right),\\
                    \beta_{i+1}=(\varepsilon_{i}\beta_{i}+c_{i})^{-1}+b_{i}, i=\overline{1,l-1}\\
                    \beta_{l}=\alpha^{(-1)^{\varepsilon}},
                \end{array}
            \end{eqnarray*}
            где $\omega^{(p)}$ \textendash\ $p$-кратная композиция функции $\omega$.
        \end{itemize}
    \end{block}

    \begin{block}{Множество $\mathcal{T}$}
        Множество всех факториальных колец $\mathbb{K}$ таких, что $\exists D_\mathbb{K} \in \mathbb{N}$ т.ч.:
    
        \begin{itemize}
            \item $\forall x_{0}\in\mathbb{K}_{*}, \alpha\in\mathbb{F}_{1}^{*}$, тройка $(x_{0},\alpha,D_\mathbb{K})$ регулярная.
    
            \item Если $D_\mathbb{K}\ge3$, то $\forall k\in[3,D_\mathbb{K}]$ и $x_{0}\in\mathbb{K}_{*}, \alpha\in\mathbb{F}_{1}^{*}$ тройка $(x_{0},\alpha,k-2)$ регулярная, если $\omega^{(k-2)}(\textrm{fr}((\alpha-x_{0})^{-1}))=0.$
        \end{itemize}
    \end{block}
\end{frame}

\begin{frame}
    \begin{block}{Пример проверки первого условия для $\mathbb{Z}[i]$}
        \begin{itemize}
            \item $D_{\mathbb{K}} = 3$

            \item $\textrm{Nm}(x_0) > 5$ $\Rightarrow$ $p = 1, \beta_1 = \alpha$
            
            \item $x_0 = 1 + i$ $\Rightarrow$ $\left[\frac{1}{\alpha - x_0}\right] = -1 + i$ $\Rightarrow$ $p = 1, \beta_1 = \textrm{fr}\left(\frac{\alpha - (1 + i)}{\alpha(1 - i) - 1}\right)$ $\Rightarrow$ $\beta_2 = \frac{1}{\alpha}$
            
            \item $x_0 = 1$ $\Rightarrow$ $\left[\frac{1}{\alpha - x_0}\right] \in \{-2, -1 \pm i, -2 \pm i\}$
            \begin{itemize}
                \item $\left[\frac{1}{\alpha - x_0}\right] = -2$ $\Rightarrow$ $p = 1, \beta_1 = \textrm{fr}\left(\frac{\alpha - 1}{2\alpha - 1}\right)$ $\Rightarrow$ $\beta_2 = \frac{1}{\alpha}$

                \item $\left[\frac{1}{\alpha - x_0}\right] = -1 \pm i$ $\Rightarrow$ $p = 1, \beta_1 = \textrm{fr}\left(\frac{\alpha - 1}{\alpha(1 \mp i) \pm i}\right)$ $\Rightarrow$ $\beta_2 = \frac{1}{\alpha}$

                \item $\left[\frac{1}{\alpha - x_0}\right] = -2 + i$ $\Rightarrow$ $\omega\left(\textrm{fr}\left(\frac{1}{\alpha - 1}\right)\right) = \textrm{fr}\left(\frac{\alpha - 1}{\alpha(2 - i) - 1 + i}\right)$
                
                \item $\left[\frac{\alpha - 1}{\alpha(2 - i) - 1 + i}\right] \in \{1 + i, 1 + 2i, 2 + i, 2 + 2i\}$
            \end{itemize}
            
            \item для проверки условия (2) определения класса $\mathcal{T}$ можно рассмотреть только $x_0 = 1, \left[\frac{1}{\alpha - x_0}\right] = -2 \pm i$.
            \begin{itemize}
                \item $\left[\frac{1}{\alpha - x_0}\right] = -2 + i$ $\Rightarrow$ $\alpha \in \left\{\frac{1}{2 + i}, \frac{2 + i}{3 + 3i}, \frac{2 - i}{4}, \frac{3}{5 + 2i}\right\}$
                
                \item $\textrm{fr}\left(\frac{3}{5 + 2i}\right) \neq \frac{3}{5 + 2i}$
            \end{itemize}
        \end{itemize}
    \end{block}
\end{frame}

\begin{frame}
    \begin{block}{$(\alpha, k)$-разрешимость}
        Для $\alpha\in\mathbb{F}_1$ и $k\in\mathbb{N}$ существуют $x_{1},\ldots,x_{k}\in\mathbb{K}$ такие, что
        \begin{eqnarray*}
            \alpha = [x_{1}:x_{2}:\ldots:x_{k}] = x_{1} + \frac{1}{
                x_{2} + \frac{1}{
                    \dots + \frac{1}{x_{k}}
                }
            }
        \end{eqnarray*}
    \end{block}

    \begin{block}{Лемма о $(\alpha, k)$-разрешимости}
        Пусть $\mathbb{K}\in\mathcal{T}$.
        Уравнение $\alpha = [x_{1}:x_{2}:\ldots:x_{k}]$ является $(\alpha, k)$-разрешимым тогда и только тогда, когда $\omega^{(k-1)}(\alpha)=0$.
    \end{block}
    
    \begin{block}{Идея доказательства}
        \begin{itemize}
            \item $(\alpha,k)$-разрешимость $\Longleftrightarrow$ $\exists z\in\mathbb{K}$: $((\alpha-z)^{-1},k-1)$-разрешимость
            
            \item $\omega^{(k-1)}(\alpha) = 0$ $\Longleftrightarrow$ $\exists z\in\mathbb{K}: \omega^{(k-2)}(\textrm{fr}((\alpha-z)^{-1})) = 0$
            
            \item $(z, \alpha, k-2)$ регулярная $\Rightarrow$ $\beta_{1}=\omega^{(p)}\left(\textrm{fr}\left(\frac{1}{\alpha-z}\right)\right)$, $\beta_{i+1}=(\varepsilon_{i}\beta_{i}+c_{i})^{-1}+b_{i}$, $\beta_{l}=\alpha^{(-1)^{\varepsilon}}$
            
            \item $(\beta_{1}, k-p-1)$-разрешимость $\Rightarrow$ $(\alpha^{(-1)^{\varepsilon}}, k-p+l-2)$-разрешимо
        \end{itemize}
    \end{block}
\end{frame}

\subsection{Класс колец, для которых теорема Кронекера-Валена верна}

\begin{frame}
    \begin{block}{Аналог теоремы Кронекера-Валена в факториальных кольцах}
        Пусть $\mathbb{K}\in\mathcal{T}$.
        Тогда $\mathcal{L}_{a,b}=\mathpzc{l}_{a,b}$ для любых $a,b\in\mathbb{K}_{*}$.
    \end{block}
    
    \begin{block}{Идея доказательства}
        \begin{itemize}
            \item $\mathpzc{l}_{a,b} = k < \infty$ $\Rightarrow$ $(a/b, k)$-разрешимо $\Rightarrow$ $(\textrm{fr}(a/b), k)$-разрешимо
            
            \item $\omega^{(k-1)}(\textrm{fr}(a/b)) = 0$
            
            \item $\mathcal{L}_{a,b}=\textrm{min}\{r\in\mathbb{N}|\omega^{(r-1)}(\textrm{fr}(a/b))=0\}\le k=\mathpzc{l}_{a,b}$
        \end{itemize}
    \end{block}
\end{frame}

\begin{frame}
    \begin{block}{\vspace*{-3ex}}
        $$
        \mathcal{S} = \left\{
            \mathbb{K} \Big| \forall x\in\mathbb{K}_*, \alpha\in\mathbb{F}_{1}^{*}, \left[
                \begin{array}{l}
                    \textrm{int}((\alpha-x)^{-1})\in\mathbb{I}\cup\{0\}\\
                    x \ \textrm{int}((\alpha-x)^{-1})+1\in\mathbb{I}
                \end{array}
            \right.
        \right\} \subseteq \mathcal{T}
        $$
    \end{block}

    \begin{block}{Метод проверки условия $\mathbb{K}\in\mathcal{S}$}
        \begin{algorithmic}[1]
            \State $\mathbb{J} \gets \{x\in\mathbb{K}_{*}|\textrm{int}((\alpha-x)^{-1})\in\mathbb{I}\cup\{0\} \ \forall\alpha\in\mathbb{F}_{1}^{*}\}$.
    
            \State Для $x_0\in\mathbb{K}_*\setminus\mathbb{J}$ вычислить $\mathbb{Y}(x_{0})$ \textendash\ множество значений функции $f_{x_{0}}(\alpha)=\textrm{int}((\alpha-x_{0})^{-1}), \alpha\in\mathbb{F}_{1}^{*}$.
    
            \State Для $x_0\in\mathbb{K}_*\setminus\mathbb{J}$ вычислить $\mathbb{U}(x_{0}) \gets \{\frac{\varepsilon-1}{x_{0}}\in\mathbb{K}|\varepsilon\in\mathbb{I}\}\cup\mathbb{I}$.
    
            \If {$\mathbb{Y}(x_{0})\subseteq\mathbb{U}(x_{0})$ для любого $x_{0}\in\mathbb{K}_{*}\backslash\mathbb{J}$}
                \State\Return ''Да'', $\mathbb{K}\in\mathcal{S}$
            \Else
                \State\Return ''Неизвестно''
            \EndIf
        \end{algorithmic}
    \end{block}
\end{frame}

\begin{frame}
    \begin{block}{Примеры факториальных колец $\mathbb{K} \in \mathcal{S}$}
        \begin{itemize}
            \item $\mathbb{K}=\mathbb{Z}$
            
            \item $\mathbb{K}=\mathbb{P}[t]$, где $\mathbb{P}$ \textendash\ поле
            
            \item $\mathbb{K}=\mathbb{Z}[t]$
            
            \item $\mathbb{Q}[i][x, y]/(x^2 + y^2 + 1)$
        \end{itemize}
    \end{block}
    
    \begin{block}{Примеры факториальных колец $\mathbb{K} \not\in \mathcal{S}$, но $\mathbb{K} \in \mathcal{T}$}
        \begin{itemize}
            \item $\mathbb{K}=\mathbb{Z}[i]$.
            Выберем $\alpha=\frac{9-4i}{20}$, $x=1$.
            Тогда
            
            $$
                \textrm{int}((\alpha-x)^{-1})=-2+i \notin \mathbb{I} \cup \{0\}
            $$
            $$
                x \ \textrm{int}((\alpha-x)^{-1})+1=-1+i \notin \mathbb{I}
            $$
        \end{itemize}
    \end{block}
    
    \begin{block}{Примеры факториальных колец $\mathbb{K} \not\in \mathcal{T}$}
        \begin{itemize}
            \item $\mathbb{Z}[\sqrt{-11}]$
        \end{itemize}
    \end{block}
\end{frame}

\begin{frame}
    \begin{block}{Аналог теоремы Ламе в факториальных кольцах}
        \begin{itemize}
            \item Если $\mathbb{K}$ евклидово кольцо относительно нормы $\upsilon$, то $\Lambda_{\mathbb{K}}\in[0,1]$.
    
            \item Если $\mathbb{K}$ \textendash\ кольцо с единственной факторизацией, заданной нормой $\upsilon$ и $\Lambda_{\mathbb{K}}\in[0,1)$, то $\mathbb{K}$ евклидово относительно нормы $\upsilon$ и выполнено следующее неравенство
            \begin{eqnarray*}
                l_{n}(\mathbb{K})\le[\log_{\Lambda_{\mathbb{K}}^{-1}} n]+2
            \end{eqnarray*}
            для любого $n\in\mathbb{N}$, где $\log_{\infty} n=0$ и 
            $$
                \Lambda_{\mathbb{K}} = \sup_{a/b \in \mathbb{F}^*_1} \inf_{q\in \mathbb{K}} \frac{\upsilon(a-bq)}{\upsilon(b)}.
            $$
        \end{itemize}
    \end{block}
\end{frame}

\subsection{Метод доказательства невыполнимости теоремы Кронекера-Валена}

\begin{frame}
    \begin{block}{Метод доказательства невыполнимости теоремы Кронекера-Валена}
        \begin{algorithmic}[1]
            \State взять произвольные $a, b\in\mathbb{Z}_K$
            \State вычислить цепочку делений с выбором минимального по норме остатка $\mathcal{D}_{a, b}$
            \State найти $c\in\mathbb{Z}_K$ такое, что $a = bx+c$ для некоторого $x\in\mathbb{Z}_K$ (без ограничений на $N_{K/\mathbb{Q}}(c)$)
            \State вычислить цепочку делений с выбором минимального по норме остатка $\mathcal{D}'_{b,c}$
            \If {$\textrm{len}(\mathcal{D}_{a, b}) > \textrm{len}(\mathcal{D}'_{b, c}) + 1$}
                \State теорема Кронекера-Валена не выполняется в $K$
            \Else
                \State неизвестно
            \EndIf
        \end{algorithmic}
    \end{block}

    \begin{block}{Аналог теоремы Кронекера-Валена в квадратичных кольцах}
        Пусть $K$ действительное квадратичное норменно-евклидово кольцо.
        Тогда теорема Кронекера-Валена не выполняется в $K$.
    \end{block}

    \begin{block}{Следствие}
        Пусть $K$ \textendash\ кольцо алгебраических целых чисел числового поля $\mathbb{Q}[\sqrt{d}]$.
        Пусть $K$ квадратичное и норменно-евклидово.
        Теорема Кронекера-Валена выполняется в $K$ тогда и только тогда, когда $d=-1, -2, -3, -7$.
    \end{block}
\end{frame}

\section{RSA-криптосистема}

\subsection{Введение}

\begin{frame}
    \begin{block}{RSA-криптосистема в дедекиндовых кольцах}
        \begin{itemize}
            \item \textit{Ключи}:
            \begin{enumerate}
                \item $\mathfrak{N}=\mathfrak{p} \mathfrak{q}$, где $\mathfrak{p}, \mathfrak{q}$ \textendash\ максимальные идеалы кольца $R$, $\mathfrak{p} \ne \mathfrak{q}$
    
                \item $e \in [1, \varphi(\mathfrak{N}))$, $(e, \varphi(\mathfrak{N})) = 1$
    
                \item $ed\equiv 1 (\textrm{mod } \varphi(\mathfrak{N}))$
    
                \item $(\mathfrak{N},e)$ \textendash\ открытый ключ, $(\mathfrak{N},d)$ \textendash\ секретный ключ
            \end{enumerate}
    
            \item \textit{Шифрование}: $x \in R/\mathfrak{N}$ $\Rightarrow$ $y=x^e$.
    
            \item \textit{Расшифрование}: $y \in R/\mathfrak{N}$ $\Rightarrow$ $x=y^d$.
        \end{itemize}
    \end{block}

    \begin{block}{Пример работы}
        \begin{eqnarray*}
            \begin{array}{c}
                R = \frac{\mathbb{Z}_2(x, y)}{\langle xy - 1\rangle} \cong \mathbb{Z}_2(x, x^{-1})\\
                \begin{array}{ll}
                    \mathfrak{p} = (x+1), & \mathcal{N}(\mathfrak{p}) = 2\\
                    \mathfrak{q} = (x^2 + x + 1), & \mathcal{N}(\mathfrak{p}) = 4\\
                    \mathfrak{N} = (x+1)(x^2 + x + 1) = (x^3 + 1), & \mathcal{N}(\mathfrak{N}) = 8
                \end{array}
                \Rightarrow
                \varphi_K((x+1)(x^2 + x + 1)) = 3\\
                \begin{array}{l}
                    e = 2\\
                    d = 2
                \end{array}
                \Rightarrow
                \begin{array}{l}
                    (x^2 + 1)^2 = x^4 + 1 = x + 1\\
                    (x + 1)^2 = x^2 + 1
                \end{array}
            \end{array}
        \end{eqnarray*}
    \end{block}
\end{frame}

\subsection{Криптостойкость}

\begin{frame}
    % Coron, May Deterministic polynomial time equivalence of computing the RSA secret key and factoring
    \begin{block}{Теорема об эквивалентности взлома криптосистемы и факторизации}
        \begin{itemize}
            \item $\mathfrak{N}$ \textendash\ модуль RSA-криптосистемы в дедекиндовом кольце $R$, а $(\mathfrak{N}, e)$ и $(\mathfrak{N}, d)$ \textendash\ открытый и секретный ключи RSA-криптосистемы
            
            \item $\mathcal{N}(\mathfrak{p})$ и $\mathcal{N}(\mathfrak{q})$ имеют одинаковую битовую длину
            
            \item $ed \le (\mathcal{N}(\mathfrak{N}))^2$, $\mathcal{N}(\mathfrak{N}) \ge 3$
        \end{itemize}
        Если $d$ известно, то существует эффективный алгоритм нахождения $\mathcal{N}(\mathfrak{p})$ и $\mathcal{N}(\mathfrak{q})$.
    \end{block}
    
    % Kranakis E. Primality and Cryptography. N.Y., 1983
    \begin{block}{Аналог теоремы Кранакиса}
        \begin{itemize}
            \item $R = \mathcal{O}_K$ \textendash\ кольцо с единственной факторизацией
            
            \item  $\mathfrak{N} = \mathfrak{p}\mathfrak{q}$ \textendash\ модуль RSA-криптосистемы в дедекиндовом кольце $R$
            
            \item $(\mathfrak{N}, e)$ и $(\mathfrak{N}, d)$ \textendash\ открытый и секретный ключи RSA-криптосистемы соответственно
            
            \item $d$ известно
        \end{itemize}
        Тогда $N$ можно эффективно разложить на множители с вероятностью не менее $\frac{1}{2}$.
    \end{block}
\end{frame}

\begin{frame}
    \begin{block}{Аналог теоремы Винера}
        \begin{itemize}
            \item $\mathfrak{N} = \mathfrak{p}\mathfrak{q}$ \textendash\ модуль RSA-криптосистемы в дедекиндовом кольце $R$
            
            \item $(\mathfrak{N}, e)$ и $(\mathfrak{N}, d)$ \textendash\ открытый и секретный ключи RSA-криптосистемы
            
            \item $d < \frac{1}{\sqrt{2\alpha+2}}(\mathcal{N}(\mathfrak{N}))^{1/4}$
        \end{itemize}
        Тогда $d$ можно эффективно вычислить за $O(\log^2 \mathcal{N}(\mathfrak{N}))$ число бинарных операций.
    \end{block}
    
    \begin{block}{Атака методом повторного шифрования}
        \begin{center}
            $y=x^e, y_i=y^{e^i}; \exists i \in \mathbb{N}: y_i=y \Rightarrow x=y_{i-1}$
        \end{center}
    \end{block}
    
    \begin{block}{Теорема о атаке методом повторного шифрования}
        \begin{itemize}
            \item $\varphi(\mathfrak{p})=rk,$ $\varphi(\mathfrak{q})=sl,$ где $r,$ $s$ \textendash\ различные простые числа $k,$ $l \in \mathbb{N}$
            
            \item $r-1,$ $s-1$ имеют различные простые делители $r_1,$ $s_1$ соответственно
            
            \item $y \sim U\left((\mathcal{R}/\mathfrak{m})^*\right)$ и $e \sim \left(\mathbb{Z}_{\varphi(\mathfrak{m})}^*\right)$ \textendash\ независимые случайные величины
            
            \item $m_{e,y}=\min\{m \in \mathbb{N}: y^{e^m}= y\}$
        \end{itemize}
        Тогда
        $
            \mathbb{P}\left(
                m_{e,y} \ge r_1s_1
            \right) \ge \left(
                1-r^{-1}
            \right)\left(
                1-s^{-1}
            \right)\left(
                1-r_1^{-1}
            \right)\left(
                1-s_1^{-1}
            \right).
        $
    \end{block}
\end{frame}

\section{Заключение}

\begin{frame}
    \begin{block}{Положения, выносимые на защиту}
        \begin{itemize}
            \item Аналоги критериев Миллера и Эйлера простоты идеалов в дедекиндовых кольцах с конечной нормой.
            
            \item Аналоги теорем Кронекера-Валена и Ламе в факториальных кольцах.
            
            \item Необходимые условия криптографической стойкости аналога криптосистемы RSA в дедекиндовых кольцах.
        \end{itemize}
    \end{block}
\end{frame}

\subsection{Публикации}

\begin{frame}{Статьи в рецензируемых журналах}
    \begin{enumerate}
        \item[Q2] Vaskouski M., Kondratyonok N., Prochorov N. Primes in quadratic unique factorization domains. {\it Journal of Number Theory.} 2016. V.~168. P.~101-116.
        
        \item[Q2] Vaskouski M., Kondratyonok N. Shortest division chains in unique factorization domains. {\it Journal of Symbolic Computation.} 2016. V.~77. P.~175-188.
        
        \item[Q3] Кондратёнок Н.В. Анализ RSA-криптосистемы в абстрактных числовых кольцах. {\it Журнал Белорусского государственного университета. Математика. Информатика.} 2020. \textnumero~1. С.~13-21.

        \item[Q2] Vaskouski M., Kondratyonok N. The Kronecker-Vahlen theorem fails in real quadratic norm-Euclidean fields. {\it Journal of Symbolic Computation.} 2021. V.~104, P.~134-141.

        \item[-] Васьковский М.М., Кондратёнок Н.В. Конечные обобщенные цепные дроби в евклидовых кольцах. {\it Вестник БГУ Серия 1 ''Физика, Математика, Информатика''.} 2013. \textnumero~3. С.~117-123.
        
        \item[-] Васьковский М.М., Кондратёнок Н.В. Аналог RSA-криптосистемы в квадратичных кольцах с единственной факторизацией. {\it Доклады Национальной Академии Наук Беларуси.} 2015. Т.~59, \textnumero~5. С.~18-23. (англ)

        \item[-] Васьковский М.М., Кондратёнок Н.В., Прохоров Н.П. Аналог теста Соловея-Штрассена в квадратичных евклидовых кольцах. {\it Доклады Национальной Академии Наук Беларуси.} 2017. Т.~61. \textnumero~5. С.~28-32.
    \end{enumerate}
\end{frame}

\begin{frame}{Материалы конференций}
    \begin{small}
        \begin{enumerate}
            \item Кондратёнок Н.В., Васьковский М.М. Цепные дроби в евклидовых кольцах. {\it Материалы XVI Республиканской научной конференции студентов и аспирантов.} 2013. Ч.~1. С.~63-64.
            
            \item Васьковский М.М., Кондратёнок Н.В., Прохоров Н.П. Тест Соловея-Штрассена в квадратичных евклидовых кольцах. {\it Материалы международной научной конференции ''XII Белорусская математическая конференция''.} 2016. Ч.~5. С.~15-16
            
            \item Кондратёнок Н.В., Прохоров Н.П. Критерии простоты в квадратичных кольцах с единственной факторизацией. {\it Сборник статей лауреатов и авторов научных работ, получивших первую категорию XXIV Республиканского конкурса научных работ студентов.} 2017. С.~19-20.
            
            \item Васьковский М.М., Кондратёнок Н.В. Построение и анализ RSA-криптосистемы в числовых полях. {\it ''Всероссийская конференция ''Алгебра и теория алгоритмов'', посвященная 100-летию факультета математики и компьютерных наук Ивановского государственного университета''} 2018.
            
            \item Кондратёнок Н.В., Прохоров Н.П. Аналог теоремы Кронекера-Валена и полиномиальные алгоритмы тестирования на простоту в числовых полях. {\it Сборник статей лауреатов и авторов научных работ, получивших первую категорию XXV Республиканского конкурса научных работ студентов.} 2018. \textendash\ С.~25.
    
            \item Кондратёнок Н.В. Свойства RSA-криптосистемы в абстрактных числовых кольцах. {\it Материалы международной научной конференции ''XIII Белорусская математическая конференция''} 2021. Ч.~2. С.~63-64

            \item Васьковский М.М., Кондратёнок Н.В. Аналог критерия Миллера в дедекиндовых кольцах с конечной нормой. {\it CSIST.} 2022. (в печати)
        \end{enumerate}
    \end{small}
\end{frame}

\begin{frame}{Другое}
    \begin{small}
% На настоящий момент среди журналов по алгебре и теории чисел JSC входит в Q1, а JNT в Q2
% Слайд со статьями лучше оформить иначе. В первую группу я бы выделил статьи из Scopus (с указанием квартилей).
% JSC - Q1, JNT - Q2, Журнал БГУ (только статья за 2020) - Q4. 
% Еще можно указать твой h-индекс.
% По scopus h=2, по google scholar h=3 
        {\bf Цитирования:} scopus h-индекс = 2, google scholar h-индекс = 3
        
        {\bf Преподавание:} спецкурс "Криптографические системы в открытым ключом".

        {\bf Внедрение:}
        \begin{itemize}
            \item Васьковский М.М., Кондратёнок Н.В. Новые теоремы о стойкости криптосистемы RSA в абстрактных числовых кольцах // Акт о практическом использовании результатов исследования в образовательном процессе № 24/14 от 26.01.2021 г.
        \end{itemize}
    
        {\bf НИР:}
        \begin{itemize}
            \item НИР «Анализ общих и асимптотических свойств решений стохастических дифференциальных уравнений с приложениями в криптографии и теории кредитных рисков». (№ГР 20213106)
        \end{itemize}
    
        {\bf Гранты:}
        \begin{itemize}
            \item Стипендия имени Севченко, 2019
    
            \item Грант Huawei, 2021
        \end{itemize}
        
        {\bf Конференции:}
        \begin{itemize}
            \item Intel ISEF (2014), EUCYS (2014)
            
            \item XII БМК (2016), XIII БМК (2021)
    
            \item Всероссийская конференция "Алгебра и теория алгоритмов" (2018)
    
            \item Конференция БГУ (2018, 2019)
        \end{itemize}
    \end{small}
\end{frame}

\subsection{}

\begin{frame}
    \begin{center}
        \begin{huge}
            Спасибо за внимание
        \end{huge}
    \end{center}
\end{frame}

\end{document}
