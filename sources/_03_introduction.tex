\documentclass[_00_dissertation.tex]{subfiles}
\begin{document}

\onlyinsubfile{
    \renewcommand{\contentsname}{ОГЛАВЛЕНИЕ}
    \setcounter{tocdepth}{3}
    \tableofcontents
}

\begin{center}
    \chapter*{ВВЕДЕНИЕ}
    \addcontentsline{toc}{chapter}{ВВЕДЕНИЕ}
\end{center}

Интерес к алгоритмической теории чисел вызван тем, что она активно используется на практике, например в криптосистемах.
Важной задачей в криптографии является генерация простых элементов с большой нормой.
Для этого используются различные тесты на простоту.
Простейшие тесты, основанные на пробных делениях не подходят так как не позволяют генерировать простые элементы с достаточно большой нормой.

Во второй половине XX века начала активно развиваться информатика и криптография, что привело к увеличению активности работы в области алгоритмической теории чисел со стороны ведущих математиков.
В частности, были получены принципиально новые критерии простоты, приводящие к эффективным алгоритмам тестирования простоты и генерации больших простых чисел.
Примерами таких критериев служат критерий Эйлера и Миллера, приводящие к алгоритму Соловея-Штрассена и Миллера-Рабина соответственно, что было описано в работах \cite{source:Miller, source:Rabin, source:Solovay}.
В предположении справедливости расширенной гипотезы Римана в работе Н.~Анкени \cite{source:Ankeny} была доказана теорема, позволяющая получить детерминированный вариант алгоритма Миллера-Рабина.

В 2002 году в работе М.~Агравала, Н.~Каяла и Н.~Саксены \cite{source:AKS} было конструктивно доказано, что задача проверки на простоту решается за полиномиальное относительно размера входных данных время.

Задача проверки на простоту существенно усложняется при переходе от целых чисел к более общим алгебраическим структурам.
В работе \cite{source:Huang_Prime_in_P} был получен полиномиальный детерминированный алгоритм проверки на простоту в конечнопорожденных дедекиндовых кольцах.
Тестирование идеалов на простоту в дедекиндовых кольцах представляют большой интерес не только с теоретической точки зрения, но и с практической.
Например, такие тесты могут использоваться в алгоритмах генерации простых идеалов.
А простые идеалы необходимы для генерации ключей криптосистемы RSA, криптосистемы Рабина и других.

Важной частью многих теоретико-числовых алгоритмов являются алгоритмы вычисления наибольшего общего делителя.
В кольце целых чисел таким алгоритмом является алгоритм Евклида.
Важным является исследование экстремальных свойств этого алгоритма, что было исследовано в работах Е.~Баха, Е.~Калтофена, Г.~Роллетчека и Д.~Лазара \cite{source:Bach_Algorithmic_number_theory, source:Kaltofen, source:Rolletschek_1986, source:Rolletschek_1990, source:Lazard}.
В кольцах с бесконечной группой единиц длину кратчайшей цепочки делений можно ограничить сверху константой как показано в работе \cite{source:Cooke}.

В 1844 году в работе Г.~Ламе было доказано, что количество делений, которое нужно выполнить для нахождения наибольшего общего делителя двух целых чисел, не превосходит количества цифр в меньшем числа, умноженного на $5$.

Представленные выше цепочки делений сильно связаны с цепными дробями, которые позволяют находить хорошие приближения вещественных чисел в виде рациональных дробей, что было исследовано в работах И.М.~Виноградова,  В.В.~Беняш-Кривца, В.П.~Платонова \cite{source:Vinogradov, source:Benyash-Krivets_1, source:Benyash-Krivets_2}.

Особый интерес представляет криптосистема RSA, так как ее идея достаточно проста, но при этом стойкость криптосистемы RSA основана на фундаментальной задаче факторизации.
Впервые она была предложена Р.Л.~Ривестом, А.~Шамиром и Л.М.~Адлеманом в 1977 году в журнале ''Communications of the ACM'' \cite{source:Rivest}.
При изучении криптосистемы RSA можно выделить два типа задач: использование криптосистемы RSA в кольцах более общего вида и получение необходимых условий криптостойкости.

Различные аналоги криптосистемы RSA в кольце гауссовых чисел, кольце многочленов, дедекиндовых кольцах были предложены в работах \cite{source:Li, source:El_Kassar, source:Elkamchouchi, source:Koval, source:El_Kassar}.
В работе К.А.~Петуховой и С.Н.~Тронина \cite{source:Petukhova} предложено обобщение криптосистемы RSA на случай дедекиндовых колец.
В качестве элементов криптосистемы предлагается брать не элементы кольца, а его идеалы.

Необходимые для криптостойкости криптосистемы RSA в кольце целых чисел условия были получены в работах М.Дж.~Винера, Д.~Копперсмита, Дж.С.~Корона, А.~Мэй, Ю.С.~Харина, С.В.~Агиевича, Д.В.~Васильева, Г.В.~Матвеева \cite{source:Wiener, source:Coppersmith, source:Coron, source:Kharin, source:Matveev_2019, source:Matveev_2018}.

\onlyinsubfile{
    \subfile{_10_bibliography}
}

\end{document}
