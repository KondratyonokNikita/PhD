\documentclass[_00_dissertation.tex]{subfiles}
\begin{document}

\onlyinsubfile{
    \renewcommand{\contentsname}{ОГЛАВЛЕНИЕ}
    \setcounter{tocdepth}{3}
    \tableofcontents
}

\begin{center}
    \chapter*{ВВЕДЕНИЕ}
    \addcontentsline{toc}{chapter}{ВВЕДЕНИЕ}
\end{center}

Простейшим алгоритмом проверки числа $n$ на простоту является перебор чисел от $2$ до $\sqrt{n}$ и проверка на делимость.
Вплоть до XVII века это был единственный тест на простоту для чисел общего вида.
В XVII веке Ферма сформулировал необходимое условие простоты числа $n$: если $n$ простое, то для любого $a$, не делящегося на $n$, выполняется сравнение $a^{n-1} \equiv 1{\pmod{n}}$.
Это утверждение было названо малой теоремой Ферма и доказано Эйлером в более сильной форме.
На основе этого условия был сформулирован тест Ферма проверки чисел на простоту.

Поскольку малая теорема Ферма не является критерием, то встает вопрос о нахождении составных чисел, удовлетворяющих ей.
В 1899 году А.Корсельт доказал критерий согласно которому число будет составным и удовлетворять условию малой теоремы Ферма.
В 1910 году Кармайкл сформулировал очень похожий критерий и привел пример такого числа.

Начиная со второй половины XX века начала активно развиваться информатика и криптография.
Это послужило одной из причин увеличения активности работы в области алгоритмической теории чисел.
В частности, был получен ряд существенных результатов связанных со свойствами целых простых чисел и способами проверки их на простоту.

Г. Миллером в 1976 году был предложен детерминированный полиномиальный тест на простоту \cite{source:Miller}.
Тест основывается на том, что нечётное составное число $n$ либо является степенью некоторого простого числа, либо существует простое число, лежащее в интервале от $2$ до некоторой функции $f(n)$, зависящей от $n$, не являющееся свидетелем простоты данного числа по Миллеру.
В качестве функции $f(n)$ исходя из теоремы Анкени берется $70 \ln^2 n$.
Отметим, что этот тест опирается на работу Анкени о нахождении наименьшего квадратичного невычета, в которой предполагается, что выполнена обобщенная гипотеза Римана.
В дальнейшем различные математики развивали идеи Миллера.
Например, в 1990 году в работе Баха \cite{source:Bach} был доказан аналог теоремы Анкени для случая кольца целых алгебраических чисел в предположении расширенной гипотезы Римана.
Используя это обобщение, константа в функции $f(n)$ теста Миллера была уменьшена до $2$.

В 1980 году в работе Рабина \cite{source:Rabin} результат Миллера был улучшен и получен алгоритм Миллера-Рабина.
Этот алгоритм является безусловным (не опирается на расширенную гипотезу Римана), но является вероятностным.
В работе Рабина доказано, что для составного числа $n$ существует не более $\varphi(n)/4$ свидетелей простоты.
Идея теста заключается в том, чтобы проверять для случайно выбранных чисел $a < n$, являются ли они свидетелями простоты числа $n$.
Если найдётся свидетель того, что число составное, то число действительно является составным.
Если было проверено $k$ чисел, и все они оказались свидетелями простоты, то число считается простым.
Для такого теста вероятность принять составное число за простое будет меньше $(1/4)^{k}$.

Тест Соловея-Штрассена использует похожую идею на малую теорему Ферма, но несколько модифицирует ее, используя символ Якоби.
Этот тест был предложен в 1977 году в работе Соловея и Штрассена \cite{source:Solovay}.
Тест основан на том факте, что для составного числа $n$ количество целых чисел $a < n$, взаимнопростых с $n$, удовлетворяющих сравнению $a^{{(n-1)/2}}\equiv \left({\frac{a}{n}}\right){\pmod{n}}$, не превосходит $\frac{n}{2}$.
Если случайное число $a$ не удовлетворяет этому сравнению, то число $n$ является составным.
В противном случае $a$ называют свидетелем простоты числа $n$.
Этот алгоритм редко применяют на практике так как тест Миллера-Рабира имеет немного больший уровень достоверности.

Так же задачей разработки теста на простоту занимались такие математики как Фибоначчи, Катальди, Мерсенн, Люка и Лемер, Эйлер, Лежандр, Гаусс, Адлеман.
В их работах было получено много различных тестов на простоту.

В 2002 году в работе Агравала, Каяла и Саксены \cite{source:AKS} было конструктивно доказано, что задача проверки на простоту является полиномиально разрешимой.
Вычислительная сложность алгоритма оценивается как $\mathcal{O}(\log ^{21/2}n)$.
В предположении верности гипотезы Артина, время выполнения улучшается до $\mathcal{O}(\log ^{6}n)$.
В предположении верности гипотезы Софи Жермен время выполнения составляет $\mathcal{O}(\log ^{6}n)$.
Данный алгоритм очень важен с теоретической точки зрения, но на практике не используется, так как имеет большую вычислительную сложность, что было описано в работе Бараша \cite{source:Barash}.

Задача проверки на простоту несколько усложняется при переходе от целых чисел в более общим алгебраическим структурам.
Например, при изучении простых элементов кольца целых алгебраических элементов квадратичного числового поля, следует разделять рациональные целые и остальные элементы кольца.
В 2016 году в работе Васьковского, Кондратёнка и Прохорова [\ref{source:JNT_2016}] были доказаны обобщения критериев простоты Миллера, Эйлера и других на случай кольца целых квадратичного числового поля.
Так же был построен аналог алгоритма Миллера-Рабина для этого случая.
В 2020 году результаты были улучшены в работе Прохорова \cite{source:Prochorov}.
Был доказан аналог критерия Миллера-Рабина для идеалов колец целых алгебраических чисел конечных расширений $\mathbb{Q}$.

Дальнейшее развитие тестирования на простоту приводит к изучению идеалов колец.
Идею идеалов предложил Куммер для замены недостающих множителей в кольцах, где не выполняется основная теорема арифметики \cite{source:Stillwell}.
В 1876 году Дедекинд в книге ''Лекции по теории чисел''\cite{source:Dedekind} сформулировал определение идеала для числовых колец.
В дальнейшем определение было расширено на кольца многочленом и другие коммутативные кольца в работах Гильберта и Нётер.
В работе Гекке \cite{source:Gekke} приводятся необходимые и достаточные условия простоты идеалов.

С задачей проверки на простоту тесно связана задача факторизации.
Тривиальный алгоритм факторизации состоит в переборе делителей числа.
Один из первых нетривиальных алгоритмов факторизации был предложен Ферма в 1643 году в его переписке с другими математиками.
Он предложил представлять числа в виде разницы квадратов $n = x^2 - y^2$, а далее, вычисляя $\gcd(n,x-y)$, находить делитель $n$.
Данный способ позволяет находить два мало различающихся по величине делителя числа быстрее, чем простой перебор делителей \cite{source:Yaschenko}.
Лежандр обнаружил, что достаточно найти числа $x, y$, для которых выполняется сравнение $x^{2}\equiv y^{2}\pmod{n}$.
В дальнейшем над задачей работали такие математики как Эйлер, Гаусс и другие.
Среди экспоненциальных алгоритмов факторизации имеются $\rho$-метод Полларда, алгоритм Ленстры, алгоритм Полларда-Штрассена, $(p-1)$-метод Полларда и другие.

Основным недостатком этих алгоритмов является их вычислительная сложность.
В 1988 году английский математик Поллард описал новый для того времени метод факторизации целых чисел специальной формы $2^n \pm C$ \cite{source:Pollard}.
А. Ленстра, Х. Ленстра, Марк Манассе и Поллард показали, что на числах специального вида алгоритм работает быстрее, чем все остальные известные методы факторизации.
Алгоритм был существенно улучшен в работе Бухлера, Х. Ленстры и Померанса ''Факторизация целых чисел с помощью решета числового поля'' \cite{source:Buhler} описали метод решета числового поля в применении к числам общего вида.
Минусом предложенного улучшения было то, что алгоритм содержал шаг, предполагающий вычисления с использованием чрезвычайно больших чисел.
Кувейгнес в своей работе \cite{source:Couveignes} описал способ обойти это.

Теорема Дедекинда связывает разложение идеала числового кольца с разложением многочлена над конечным полем.
Используя эту теорему, можно показать, что задача факторизации в произвольном числовом кольце полиномиально сводится к задаче факторизации целых чисел.
Однако при рассмотрении колец, которые не являются числовыми, аналогичного результата неизвестно.
В 2019 году в работе Даки-Менса \cite{source:Darkey-Mensah} был получен алгоритм факторизации идеалов в координатных кольцах.
Однако задача нахождения общего алгоритма факторизации в дедекиндовых кольцах остается открытой.

Важной частью многих теоретико-числовых алгоритмов являются алгоритмы вычисления наибольшего общего делителя.
В кольце целых чисел таким алгоритмом является алгоритм Евклида.
В общем случае под алгоритмом Евклида можно понимать любую конечную цепочку делений для элементов $a, b$ из некоторого кольца $K$.
Цепочкой делений в кольце $K$ для элементов $a, b$ называется последовательность элементов $q_i, r_i \in K$, для которых выполнены равенства $r_i=r_{i-2}-q_ir_{i-1}, i=1,2,\ldots,k$.
При этом $r_{-1} = a$, $r_0 = b$ и $r_k=0$.
Если эта цепочка конечна, то из не можно найти наибольший общий делитель элементов $a$ и $b$.
Если не существует конечной цепочки делений, то говорят, что она имеет бесконечную длину.
При изучении алгоритма Евклида естественным является вопрос нахождения алгоритма построения цепочки делений с минимальной длиной, а так же нахождения оценок на длину цепочки.

Согласно работе Баха \cite{source:Bach_Algorithmic_number_theory}, Вален и Кронекер доказали, что в кольце целых чисел алгоритм Евклида с выбором минимального по модулю остатка требует не больше делений, чем любой другой алгоритм Евклида, который на каждом шагу выбирает между остатками $(a\pmod{b})$ or $((a\pmod{b})-b)$.

Алгоритм Евклида, в котором на каждом шаге выбирается минимальный по модулю остаток называется жадным алгоритмов Евклида.

В 1977 году в работе Лазара \cite{source:Lazard} результат Кронекера и Валена был усилен.
Было доказано, что в кольце целых чисел жадный алгоритм Евклида приводит к цепочке делений с минимальной длиной.
Зачастую этот результат и называют теоремой Кронекера-Валена.
Так же в работе Лазара был доказан аналог теоремы Кронекера-Валена в кольце многочленов над полем.

В 1985-1986 годах в работах Калтофена и Роллетчека \cite{source:Kaltofen, source:Rolletschek_1986} был доказан аналог теоремы Кронекера-Валена в специальном классе мнимых квадратичных колец.
А в 1990 году в работе Роллетчека \cite{source:Rolletschek_1990} было доказано, что теорема Кронекера-Валена верна в $\mathbb{Z}[\sqrt{d}]$, $d < 0$ тогда и только тогда, когда $d\neq-11c^{2}$, $c\in\mathbb{N}$.
В описанных работах доказательство сильно опиралось на структуру элементов рассматриваемых колец и их геометрическое представление.

В случае $\mathbb{Z}[\sqrt{d}]$, $d > 0$ ситуация другая.
В 1975 году в работе Кука и Вайнбергера \cite{source:Cooke} доказано, что в кольцах с бесконечной группой единиц длину кратчайшей цепочки делений можно ограничить сверху константой.
Доказательство, изложенное в этой работе не является конструктивным и привести алгоритм построения такого короткого алгоритма Евклида не представляется возможным.
Примерами таких колец могут служить действительные квадратичные норменно-евклидовы кольца.

В 1844 году в работе Ламе было доказано, что количество делений, которое нужно выполнить для нахождения наибольшего общего делителя двух целых чисел, не превосходит количества цифр в меньшем числа, умноженного на $5$.
Таким образом было показано, что длина кратчайшей цепочки делений в кольце целых чисел растет логарифмически относительно самих чисел.

Развитием этих результатов послужила работа Васьковского и Кондратёнка [\ref{source:JSC_2016}], в которой был предложен новый подход к доказательству теоремы Кронекера-Валена, основанный на свойствах цепочек делений.
Используя полученные результаты был предложен метод автоматического доказательства выполнимости теоремы Кронекера-Валена в факториальном кольце.
Наряду с этим был получен результаты аналогичные теореме Ламе для специального класса колец с единственной факторизацией.

В 2021 году в работе Васьковского и Кондратёнка [\ref{source:JSC_2021}] вопрос выполнимости теоремы Кронекера-Валена был рассмотрен в действительных квадратичных норменно-евклидовых кольцах.
Опираясь на результаты Серри и Лезовски \cite{source:Cerri, source:Lezowski} было показано, что для каждого из этих колец существует пара элементов, для которой жадный алгоритм Евклида не является оптимальным.

Представленные выше цепочки делений сильно связаны с цепными дробями \cite{source:Vinogradov}.
Одно из основных назначений цепных дробей состоит в том, что они позволяют находить хорошие приближения вещественных чисел в виде рациональных дробей.
Цепные дроби широко используются в теории чисел и вычислительной математике, а их обобщения оказались чрезвычайно полезны в математическом анализе и других разделах математики.
Например, в работах Беняш-Кривца, Платонова \cite{source:Benyash-Krivets_1, source:Benyash-Krivets_2} исследуется свойства непрерывных дробей в гиперэллиптических и функциональных полях.

Исследование этих вопросов в дедекиндовых кольцах представляют большой интерес не только с теоретической точки зрения, но и с практической.
Например, алгоритмы проверки на простоту используются при генерации ключей ряда криптосистем (например RSA, Рабина).
Особенно актуально это становится в связи с развитием квантовых вычислений.
Например, уже существуют квантовые алгоритмы, позволяющие факторизовать числа за полиномиальное время.

Одной из криптосистем активно использующих простые числа и основанных на сложности задачи факторизации является RSA-криптосистема.
Впервые она была предложена Ривестом, Шамиром и Адлеманом в 1977 году в журнале ''Communications of the ACM'' \cite{source:Rivest}.
Для этой криптосистемы доказано существование вероятностного алгоритма факторизации модуля, если известна секретная экспонента.
Слабым местом криптосистемы является выбор секретных параметров криптосистемы.
Например, при выборе близких простых чисел для генерации модуля, этот модуль будет легко факторизовать методом Ферма.
Дальнейшие исследования в этой области сводились к тому, чтобы найти ограничения на параметры криптосистемы для исключения возможности легкого взлома.
Например, в 1990 году в работе Винера \cite{source:Wiener} было доказано, что маленькие секретные экспоненты не являются безопасными.
В 1997 году в работе Копперсмиса \cite{source:Coppersmith} был разработан метод поиска нулей с маленькой нормой у многочлена над конечным полем.
Используя этот метод было показано, что использование шаблонов в сообщениях может быть небезопасно при использовании RSA-криптосистемы.
В 2007 году в работе Корона и Мэй \cite{source:Coron} был улучшен результат, изложенный в первой версии криптосистемы об эквивалентности нахождения секретного ключа и факторизации модуля.
В этой работе было доказано, что существует детерминированный алгоритм факторизации модуля RSA-криптосистемы, если известна секретная экспонента.
Этот и многие другие результаты, связанные с криптосистемами подробно описаны в работах Харина, Агиевича, Васильева и Матвеева \cite{source:Kharin, source:Matveev_2019, source:Matveev_2018}.

Одним из вариантов улучшения является построение аналога RSA-крипто\-системы в более общих кольцах.
В 2005 году в работах Ли и Эль-Кассара \cite{source:Li, source:El_Kassar} были предложены аналоги RSA-криптосистемы в кольце многочленов над конечным полем.
В работах \cite{source:Elkamchouchi, source:Koval, source:El_Kassar} изучается аналог RSA-криптосистемы в кольце гауссовых чисел.
В 2016 году в работе Васьковского, Кондратёнка и Прохорова [\ref{source:JNT_2016}] был предложен аналог RSA-криптосистемы в квадратичных кольцах.
Для предложенной криптосистемы были доказаны аналоги теоремы Винера о малой секретной экспоненте и теоремы об эквивалентности взлома криптосистемы и факторизации ее модуля.
Так же исследованы способы взлома криптосистемы методом повторного шифрования.

Исходя из изложенной выше эквивалентности задачи факторизации в числовых кольцах и задачи факторизации в целых числах, в работе [\ref{source:BSU_Journal_2020}] был сделан вывод, что исследование RSA-криптосистемы в числовых кольцах е принесет существенного улучшения.

В работе Петуховой и Тронина \cite{source:Petukhova} предложено обобщение RSA-крипто\-системы на случай дедекиндовых колец.
В качестве элементов криптосистемы предлагается брать не элементы кольца, а его идеалы.
Однако исследований свойств предложенной криптосистемы не приводится.
В 2020 году в работе Кондратёнка [\ref{source:BSU_Journal_2020}] был исследован, предложенный ранее аналог RSA-крипто\-системы в дедекиндовых кольцах.

\onlyinsubfile{
    \subfile{_10_bibliography}
    \subfile{_11_pub}
}

\end{document}
