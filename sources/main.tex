\documentclass{article}
\usepackage[russian]{babel}
\usepackage[utf8]{inputenc}
\usepackage[top=2cm,bottom=2cm,left=2cm,right=2cm]{geometry}

\title{Publications}
\author{NIKITA KONDRATYONOK}
\date{October 2021}

\begin{document}

\section{Статьи в журналах}

\begin{enumerate}
    \item Vaskouski M. The Kronecker-Vahlen theorem fails in real quadratic norm-Euclidean fields / M. Vaskouski, N. Kondratyonok, // Journal of Symbolic Computation. --- 2021. ---V.~104, --- P.~134-141.

    \item Vaskouski M. Shortest division chains in unique factorization domains / M. Vaskouski, N. Kondratyonok // Journal of Symbolic Computation. --- 2016. --- V.~77. --- P.~175-188.

    \item Vaskouski M. Primes in quadratic unique factorization domains / M. Vaskouski, N. Kondratyonok, N. Prochorov // Journal of Number Theory. --- 2016. --- V.~168. --- P.~101-116.
    
    \item Кондратёнок Н.В. Анализ RSA-криптосистемы в абстрактных числовых кольцах. / Н.В. Кондратёнок // Журнал Белорусского государственного университета. Математика. Информатика. --- 2020. --- \textnumero~1. --- С.~13-21.

    \item Васьковский М.М. Конечные обобщенные цепные дроби в евклидовых кольцах / М.М. Васьковский, Н.В. Кондратёнок // Вестник БГУ Серия 1 ''Физика, Математика, Информатика''. --- 2013. --- \textnumero~3. --- С.~117-123.

    \item Васьковский М.М. Аналог теста Соловея-Штрассена в квадратичных евклидовых кольцах / М.М. Васьковский, Н.В. Кондратёнок, Н.П. Прохоров // Доклады Национальной Академии Наук Беларуси. --- 2017. --- Т.~61. --- \textnumero~5. --- С.~28-32.

    \item Vaskouski M. Analogue of the RSA-cryprosystem in quadratic unique factorization domains / M. Vaskouski, N. Kondratyonok // Reports of the National Academy of Sciences of Belarus. --- 2015. --- V.~59, \textnumero~5. --- P.~18-23.
\end{enumerate}

\section{Статьи в сборниках}

\begin{enumerate}
    \item Кондратёнок Н.В. Критерии простоты в квадратичных кольцах с единственной факторизацией / Н.В. Кондратёнок, Н.П. Прохоров // Сборник статей лауреатов и авторов научных работ, получивших первую категорию XXIV Республиканского конкурса научных работ студентов. --- 2017. --- С.~19-20.
    
    \item Кондратёнок Н.В. Аналог теоремы Кронекера-Валена и полиномиальные алгоритмы тестирования на простоту в числовых полях / Н.В. Кондратёнок, Н.П. Прохоров // Сборник статей лауреатов и авторов научных работ, получивших первую категорию XXV Республиканского конкурса научных работ студентов. --- 2018. --- С.~25.
\end{enumerate}

\section{Тезисы}

\begin{enumerate}
    \item Васьковский М.М. Построение и анализ RSA-криптосистемы в числовых полях / М.М. Васьковский, Н.В. Кондратёнок // ''Всероссийская конференция ''Алгебра и теория алгоритмов'', посвященная 100-летию факультета математики и компьютерных наук Ивановского государственного университета'' --- г. Иваново, 21-24 марта 2018 г.

    \item Васьковский М.М. Тест Соловея-Штрассена в квадратичных евклидовых кольцах / М.М. Васьковский, Н.В. Кондратёнок, Н.П. Прохоров // Материалы международной научной конференции ''XII Белорусская математическая конференция'' --- Сентябрь 5-10, 2016. --- Часть~5. --- С.~15-16
    
    \item Кондратёнок Н.В. Свойства RSA-криптосистемы в абстрактных числовых кольцах / Н.В. Кондратёнок // Материалы международной научной конференции ''XIII Белорусская математическая конференция'' --- Ноябрь 22-25, 2021. --- Часть~2. --- С.~63-64


    \item Кондратёнок Н.В. Цепные дроби в евклидовых кольцах / Н.В. Кондратёнок, М.М. Васьковский // Материалы XVI Республиканской научной конференции студентов и аспирантов --- Март 25-27, 2013. --- Часть~1. --- С.~63-64.
\end{enumerate}

% \section{Список научных конференций}

% \begin{enumerate}
%     \item Intel ISEF
    
%     \item Всероссийская конференция ''Алгебра и теория алгоритмов'', посвященная 100-летию факультета математики и компьютерных наук Ивановского государственного университета
    
%     \item XVI Республиканская научная конференция студентов и аспирантов
    
%     \item Конференция БГУ 2018
    
%     \item Конференция БГУ 2019
    
%     \item XII Белорусская математическая конференция
% \end{enumerate}

\end{document}
