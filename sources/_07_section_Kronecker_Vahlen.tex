\documentclass[_00_dissertation.tex]{subfiles}
\begin{document}

\onlyinsubfile{
    \renewcommand{\contentsname}{ОГЛАВЛЕНИЕ}
    \setcounter{tocdepth}{3}
    \tableofcontents
}

\newpage
\begin{center}
    \refstepcounter{section}
    \section*{ГЛАВА \arabic{section}.\\ ТЕОРЕМА КРОНЕКЕРА-ВАЛЕНА В ФАКТОРИАЛЬНЫХ КОЛЬЦАХ}\label{ch:Kronecker-Vahlen theorem}
    \addcontentsline{toc}{chapter}{ГЛАВА \arabic{section}. ТЕОРЕМА КРОНЕКЕРА-ВАЛЕНА В ФАКТОРИАЛЬНЫХ КОЛЬЦАХ}
\end{center}

\subsection{Предварительные сведения}

\begin{definition}
    Пусть $R$ факториальное кольцо.
    Функцию $\elementnorm{\cdot}: R \to \mathbb{N} \cup \{0, -\infty\}$ будем называть нормой в $R$, если
    \begin{itemize}
        \item $\elementnorm{x} = -\infty$ тогда и только тогда, когда $x = 0$;

        \item $\elementnorm{xy} \ge \elementnorm{x}$;

        \item для $x, y \in \zeroless{R}$ равенство $\elementnorm{xy} = \elementnorm{x}$ выполнено тогда и только тогда, когда $y \in \invertible{K}$.
    \end{itemize}
\end{definition}

\begin{remark}
    Для любого факториального кольца $R$ существует норма.
    Рассмотрим разложение элемента $x$ на простые множители $x = \varepsilon p_1^{\alpha_1} \dots p_k^{\alpha_k}$, где $\varepsilon \in \invertible{R}$, $p_1, \dots, p_k$ -- простые элементы $R$.
    Тогда функция
    \begin{equation*}
        \elementnorm{x} = \left\{\begin{split}
            \sum_{i=1}^k \alpha_{i}, & \textrm{ если } x \neq 0\\
            -\infty, & \textrm{ если } x = 0
        \end{split}\right.
    \end{equation*}
    является нормой в $R$.
\end{remark}

Далее в этой главе будем считать, что факориальное кольцо $R$ задано вместе с нормой $\elementnorm{\cdot}$.

\begin{definition}
    Пусть $R$ факториальное кольцо и $F$ его поле частных.
    Функию $\fr{\cdot}: F \to F$ будем называть дробной частью в $F$, если
    \begin{itemize}
        \item $\fr{\alpha + q} = \fr{\alpha}$ для любых $\alpha \in F$, $q \in R$;

        \item если $m \in R$, $n \in \zeroless{R}$ и $(m, n) = 1$, то $\fr{m/n} = r/n$, где $r \in R$, $(m-r)/n \in R$ и $\elementnorm{r} = \min \{\elementnorm{s} | s \in R, (m-s)/n \in R\}$.
    \end{itemize}
    Функцию $\int{\cdot}: F \to R$ будем называть целой частью, если
    \begin{equation*}
        \int{\alpha} = \alpha - \fr{\alpha}.
    \end{equation*}
\end{definition}

\begin{remark}
    Для любого факториального кольца $R$ существует дробная и целая часть.
    Рассмотрим случайный элемент $X \in F/R$.
    Этот элемент можно представить в виде $X = \{m/n + t | t \in R\}$, где $m \in R$, $n \in \zeroless{R}$ $(m, n) = 1$.
    Существует элемент $t_0 \in R$, минимизирующий норму $\elementnorm{m + n t_0}$.
    Тогда для любого элемента $x \in X$ положим $\fr{x} = m/n + t_0$.
    Несложно заметить, что эта функция является дробной частью.
    Целая часть определяется из дробной и равна $\int{x} = x - \fr{x}$.
\end{remark}

Далее в этой главе будем считать, что факориальное кольцо $R$ задано вместе дробной частью $\fr{\cdot}$ и целой частью $\int{\cdot}$.

\begin{definition}
    Пусть $a$ и $b$ ненулевые элементы факториального кольца $R$.
    Для любых $k \in \mathbb{N}$ и $q_1, \dots, q_k \in R$ обозначим
    \begin{equation*}
        \mathcal{D}_{a, b}(q_1, \dots, q_k) = (r_{-1}, r_0, \dots, r_{k-1}, r_k) \in R^{k+2},
    \end{equation*}
    где $r_{-1} = a$, $r_0 = b$, $r_i = r_{i-2} - q_i r_{i-1}$, для $i = 1, \dots, k$.
    Выражение $\mathcal{D}_{a, b}(q_1, \dots, q_k)$ будем называть цепочкой делений для $a, b \in R$.
    Через $\mathcal{E}_{a, b}$ обозначим множество всех цепочек делений для $a, b \in R$, которые заканчиваются на $0$:
    \begin{equation*}
        \mathcal{E}_{a, b} = \left\{
        \mathcal{D}_{a, b}(q_1, \dots, q_k) = (r_{-1}, \dots, r_k) \big| r_1, \dots, r_{k-1} \in \zeroless{R}, r_k = 0
        \right\}.
    \end{equation*}
\end{definition}

\begin{definition}
    Пусть $a$ и $b$ ненулевые элементы факториального кольца $R$.
    Цепочкой делений с выбором минимального по норме остатка для $a, b \in R$ будем называть такую цепочку делений, что $q_i = \textrm{int}(r_{i-2}/r_{i-1})$ для любого $i = 1, \dots, k$.

    Если цепочка делений с выбором минимального по норме остатка существует, то обозначим через $\mathcal{L}_{a, b}$ ее длину.
    Если ее не существует, то обозначим $\mathcal{L}_{a, b} = \infty$.
\end{definition}

\begin{definition}
    Обозначим через $\mathpzc{l}_{a, b}$ длину кратчайшей цепочки делений для $a, b \in \zeroless{R}$.
    \begin{equation*}
        \mathpzc{l}_{a, b} = \left\{\begin{split}
            \min_{\mathcal{D}_{a, b}(q_1, \dots, q_k) \in \mathcal{E}_{a, b}} k, & \textrm{ если } \mathcal{E}_{a, b} \neq \emptyset\\
            \infty, & \textrm{ если } \mathcal{E}_{a, b} = \emptyset
        \end{split}\right..
    \end{equation*}
\end{definition}

\begin{definition}
    Через $l_n(R)$ обозначим максимальную длину цепочки делений с выбором минимального по норме остатка для $a, b \in \zeroless{R}$ с ограниченной нормой.
    \begin{equation*}
        l_n(R) = \max \left\{
            \mathcal{L}_{a, b} \big| a, b \in \zeroless{R}, \elementnorm{a} \le \elementnorm{b} \le n
        \right\}.
    \end{equation*}
\end{definition}

\begin{remark}
    Теорему Кронекера-Валена в кольце целых чисел можно сформулировать в терминах этой главы следующим образом.
    Пусть $R = \mathbb{Z}$, $\elementnorm{x} = |x|$, $\fr{\alpha} = \alpha - [\alpha + 1/2]$.
    Тогда цепочка делений с выбором минимального по норме остатка является кратчайшей, т.е. $\mathcal{L}_{a, b} = \mathpzc{l}_{a, b}$ для всех $a, b \in \zeroless{R}$.

    Далее в этой главе определяются достаточные условия на факториальное кольцо $R$ с заданной нормой и дробной частью, при которых теорема Кронекера-Валена будет выполняться в этом кольце.
\end{remark}

\subsection{Теорема Кронекера-Валена в специальном классе факториальных колец}

\begin{definition}
    Обозначим через $F_1$ множество всех несократимых дробей $F$
    \begin{equation*}
        F_1 = \{
            \alpha \in F \big| \alpha = \fr{\alpha}
        \}.
    \end{equation*}
    Определим функцию $\omega: F_1 \to F_1$ следующим образом
    \begin{equation*}
        \omega(\alpha) = \left\{\begin{split}
            \fr{\alpha^{-1}}, \textrm{ если } \alpha \neq 0\\
            0, \textrm{ если } \alpha = 0
        \end{split}\right.
    \end{equation*}
\end{definition}

\begin{definition}
    Тройку $(x_0, \alpha, n) \in \zeroless{R} \times \zeroless{F_1} \times \mathbb{N}$ будем называть регулярной, если существуют
    \begin{itemize}
        \item $p, l \in \mathbb{N}$, $p \le n$ и $l \le p+1$,

        \item $\varepsilon_i \in \invertible{R}$, $b_i, c_i \in R$ для $i = 1, \dots, l-1$,

        \item $\varepsilon \in \{0, 1\}$,
    \end{itemize}
    для которых выполнены следующие условия
    \begin{itemize}
        \item $\beta_1 = \omega^{(p)}\left(\fr{(\alpha - x_0)^{-1}}\right)$;

        \item $\beta_{i+1} = (\varepsilon_i \beta_i + c_i)^{-1} + b_i$, $i = 1, \dots, l-1$;

        \item $\beta_{l} = \alpha^{(-1)^{\varepsilon}}$.
    \end{itemize}
\end{definition}

\begin{definition}
    Через $\mathcal{T}$ обозначим множество всех таких факориальных колец $R$, для которых существует $D_K \in \mathbb{N}$, что выполнено
    \begin{itemize}
        \item для всех $x_0 \in \zeroless{R}$, $\alpha \in \zeroless{F_1}$ тройка $(x_0, \alpha, D_K - 1)$ регулярная;

        \item если $D_K \ge 3$, то для любого $k \in [3, D_K] \cap \mathbb{N}$ и любых $x_0 \in \zeroless{R}$, $\alpha \in \zeroless{F_1}$ из равенства $\omega^{(k-2)}(\fr{(\alpha - x_0)^{-1}}) = 0$ следует, что тройка $(x_0, \alpha, k-2)$ регулярная.
    \end{itemize}
\end{definition}

\begin{definition}
    Обозначим
    \begin{equation*}
        [x_1: x_2: \dots: x_k] = x_{1} + \cfrac{1}{
            x_{2} + \cfrac{1}{
                x_{3} + \cfrac{1}{
                    \ddots + \cfrac{1}{
                        x_{k}
                    }
                }
            }
        }.
    \end{equation*}
    Будем говорить, что имеет место $(\alpha, k)$-разрешимость, если разрешимо уравнение
    \begin{equation*}
        \frac{a}{b} = [x_1: x_2: \dots: x_k].
    \end{equation*}
\end{definition}

\begin{lemma}\label{lemma:omega_and_euclidean_algorithm}
    Пусть кольцо $R \in \mathcal{T}$.
    Для любых $\alpha \in F_1$ и $k \in \mathbb{N}$ имеет место $(\alpha, k)$-разрешимость тогда и только тогда, когда $\omega^{(k-1)}(\alpha) = 0$.
\end{lemma}
\begin{proof}
    Пусть $\alpha = m/n$, где $(m, n) = 1$.
    Если $\alpha = 0$ или $k = 1$, то утверждение леммы очевидно.

    Пусть $k = 2$.
    Заметим, что имеет место $(\alpha, 2)$-разрешимость тогда и только тогда, когда $m \equiv \varepsilon \pmod{n]}$ для некоторого $\varepsilon \in \invertible{R}$.
    Пусть $q \in R$ и $m = qn + \varepsilon$, тогда $m/n = \fr{m/n} = \fr{\varepsilon/n}$.
    Если $n \in \invertible{R}$, то $\fr{\varepsilon/n} = 0$ и утверждение леммы очевидно.
    Если $n \not\in \invertible{R}$, то $\fr{\varepsilon/n} = \varepsilon/n$ и $\omega(m/n) = \fr{n/\varepsilon} = 0$.

    Пусть $k \ge 3$.
    Докажем лемму индукцией по $k$.
    Заметим, что имеет место $(\alpha, k)$-разрешимость тогда и только тогда, когда существует $z \in R$, что имеет место $((\alpha - z)^{-1}, k-1)$-разрешимость.
    Таким образом надо показать, что для $\omega^{(k-1)}(\alpha) = 0$ тогда и только тогда, когда существует $z \in R$, что $\omega^{(k-2)}(\fr{(\alpha - z)^{-1}}) = 0$.
    Если $\omega^{(k-1)}(\alpha) = 0$, то по определению $\omega^{(k-2)}(\fr{\alpha^{-1}}) = 0$.
    Если $\omega^{(k-2)}(\fr{(\alpha - z)^{-1}}) = 0$ и $z = 0$, то по определению $\omega^{(k-1)}(\alpha) = 0$.

    Предположим, что существует $z \neq 0$, что $\omega^{(k-2)}(\fr{(\alpha - z)^{-1}}) = 0$.
    Если $D_K \ge 3$ и $k \le D_K$, то обозначим $X = k-2$, иначе $X = D_K - 1$.
    Заметим, что $X \le k-2$.
    Из определения класса $\mathcal{T}$ следует, что тройка $(z, \alpha, X)$ регулярная.
    Следовательно, сущесвуют $p, l \in \mathbb{N}$, $p \le X$, $l \le p+1$, $\varepsilon \in \{0, 1\}$, $\varepsilon_i \in \invertible{R}$, $b_i, c_i \in R$, $i = 1, \dots, l-1$, что
    \begin{equation*}
        \begin{split}
            \beta_1 = \omega^{(p)}\left(\fr{(\alpha - z)^{-1}}\right)\\
            \beta_{i+1} = (\varepsilon_i \beta_i + c_i)^{-1} + b_i, i = 1, \dots, l-1\\
            \beta_{l} = \alpha^{(-1)^{\varepsilon}}
        \end{split}.
    \end{equation*}

    Заметим, что $0 \le k-p-2 < k$, так как $p \le X \le k-2$.
    Из предположения индукции и равенства $\omega^{(k-p-2)}(\beta_1) = 0$ следует, что имеет место $(\beta_1, k-p-1)$-разрешимость.
    Из определения $\beta_i$ следует, что для любых $i = 1, \dots, l-1$ и $j \in \mathbb{N}$ из того, что имеет место $(\beta_i, j)$-разрешимость следует, что имеет место $(\beta_{i+1}, j+1)$-разрешимость.
    Следовательно, имеет место $(\alpha^{(-1)^{\varepsilon}}, k-p+l-2)$-разрешимость.
    Так как $k - p + l - 2 \le k - 1$, то по предположению индукции $\omega^{(k-p+l-3)}(\fr{\alpha^{(-1)^{\varepsilon}}}) = 0$.
    Следовательно, $\omega^{(k-2)}(\fr{\alpha^{(-1)^{\varepsilon}}}) = 0$.
    Тогда из определения $\omega$ получаем $\omega^{(k-1)}(\alpha) = 0$.
    Лемма доказана.
\end{proof}

\begin{lemma}\label{lemma:euclidean_algorithm_and_minima}
    Пусть кольцо $R$ факториальное.
    Тогда для любых двух элементов $a, b \in \zeroless{R}$ выполнено равенство
    \begin{equation*}
        \mathcal{L}_{a, b} = \min\{
            k \in \mathbb{N} \big| \omega^{(k-1)}(a\/b) = 0
        \},
    \end{equation*}
    где $\min \emptyset = \infty$.
\end{lemma}
\begin{proof}
    Рассмотрим произвольные $a, b \in \zeroless{R}$.
    Предположим, что $\mathcal{L}_{a, b} = k < \infty$ и $\mathcal{D}_{a, b}(q_1, \dots, q_k) = (r_{-1}, r_0, \dots, r_k)$ это цепочка делений с выбором минимального по норме остатка.

    Из определения цепочки делений с выбором минимального по норме остатка следует, что
    \begin{equation*}
        \frac{r_{i-2}}{r_{i-1}} = q_i + \frac{r_i}{r_{i-1}}
    \end{equation*}
    для $i = 1, \dots, k-1$ и
    \begin{equation*}
        \frac{r_{k-2}}{r_{k-1}} = q_k.
    \end{equation*}
    Заметим, что
    \begin{equation*}
        \omega\left(
            \fr{\frac{r_{i-2}}{r_{i-1}}}
        \right) = \fr{\frac{r_{i-1}}{r_i}}.
    \end{equation*}
    Следовательно, получаем
    \begin{equation*}
        \omega^{(k-1)}\left(
            \fr{\frac{a}{b}}
        \right) = \fr{\frac{r_{k-2}}{r_{k-1}}} = 0.
    \end{equation*}
    Так как
    \begin{equation*}
        \omega^{(j-1)}\left(
            \fr{\frac{a}{b}}
        \right) = \fr{\frac{r_{j-2}}{r_{j-1}}} = 0,
    \end{equation*}
    для $j = 1, \dots, k-1$, то $\mathcal{L}_{a, b} = k = \min\{k \in \mathbb{N} \big| \omega^{(k-1)}(a\/b) = 0\}$.

    Предположим, что $\mathcal{L}_{a, b} = \infty$, но условия леммы не выполняются.
    Выберем минимальное такое $k \in \mathbb{N}$, что $\omega^{(k-1)}\left(\fr{\frac{a}{b}}\right) = 0$.

    Обозначим $r_{-1} = a$, $r_0 = b$.
    Тогда имеем $\omega^{(k-1)}\left(\fr{\frac{r_{-1}}{r_0}}\right) = 0$.

    Пусть $\omega^{(k-i)}\left(\fr{\frac{r_{i-2}}{r_{i-1}}}\right) = 0$.
    Пусть $\int{\frac{r_{i-2}}{r_{i-1}}} = q_i$, $\fr{\frac{r_{i-2}}{r_{i-1}}} = \frac{r_i}{r_{i-1}}$.
    Если $r_i = 0$, то $\mathcal{L}_{a, b} = i$, и получаем противоречие.
    Следовательно, имеем $r_i \neq 0$ и по определению $\omega$ получаем равенствво $\omega^{(k-i-1)}\left(\fr{\frac{r_{i-1}}{r_i}}\right) = 0$.

    Используя это рассуждение $k-1$ раз, получаем, что
    \begin{equation*}
        \omega^{(0)}\left(
            \fr{\frac{r_{k-2}}{r_{k-1}}}
        \right) = 0.
    \end{equation*}
    Однако имеем
    \begin{equation*}
        0 = \omega^{(0)}\left(
            \fr{\frac{r_{k-2}}{r_{k-1}}}
        \right) = \fr{\frac{r_{k-2}}{r_{k-1}}} = \frac{r_{k}{r_{k-1}}}.
    \end{equation*}
    Следовательно, получаем $r_k = 0$.
    Из построения следует, что построили цепочку делений с выбором минимального по норме остатка, что противоречит $\mathcal{L}_{a, b} = \infty$.
    Лемма доказана.
\end{proof}

\begin{theorem}
    Пусть кольцо $R \in \mathcal{T}$.
    Тогда цепочка делений с выбором минимального по норме остатка является кратчайшей, т.е. $\mathcal{L}_{a, b} = \mathpzc{l}_{a, b}$ для всех $a, b \in \zeroless{R}$.
\end{theorem}
\begin{proof}
    Пусть $\mathpzc{l}_{a, b} = \infty$.
    Тогда $\mathcal{L}_{a, b} = \infty$, так как иначе цепочка делений с выбором минимального по норме остатка имела бы длину меньше, чем кратчайшая.

    Пусть $\mathpzc{l}_{a, b} = k < \infty$ и $\mathcal{D}_{a, b}(q_1, \dots, q_k) = (r_{-1}, r_0, \dots, r_k)$ кратчайшая цепочка делений.
    Тогда, по лемме~\ref{lemma:omega_and_euclidean_algorithm} получаем, что $\omega^{(k-1)}\left(\frac{a}{b}\right) = 0$.
    Тогда, по лемме~\ref{lemma:euclidean_algorithm_and_minima} получаем, что $\mathcal{L}_{a, b} \le k$.
    Следовательно, получаем $\mathcal{L}_{a, b} = \mathpzc{l}_{a, b}$.
    Теорема доказана.
\end{proof}

\subsection{Теорема Ламе в факориальных кольцах}

Ранее было показано, что при определенных условиях на кольцо цепочка делений с выбором минимального по норме остатка является кратчайшей.
Однако существуют кольца, для которых эти условия не выполняются и в которых цепочка делений с выбором минимального по норме остатка не является кратчайшей.
Например, в работе~\cite{source:Rolletschek_1990} было показано, что теорема Кронекера-Валена не выполняется в кольце $\mathbb{Z}[\sqrt{-11}]$.
В работе ~\cite{source:Cooke} было показано, что для колец целых алгебраических чисел с бесконечной группой единиц длина кратчайшей цепочки делений с выбором минимальноо по норме остатка ограничена константой.
Следовательно, важным вопросом является исследование ассимптотического поведения длины кратчайшей цепочки делений.

В работе~\cite{source:Selfridge} доказана следующее полезное утверждение.
\begin{proposition}
    Пусть $d \neq 1$ -- целое число свободное от квадратов.
    Обозначим
    \begin{equation*}
        F(d) = \left\{\begin{split}
            \left(
                \left[0, \frac{1}{2}\right] \times \left[0, \frac{1}{2}\right]
            \right) \cap \left(
                \mathbb{Q} \times \mathbb{Q}
            \right), \textrm{ если } d \not\equiv 1 \pmod 4\\
            \left(
                \left[0, \frac{1}{2}\right] \times \left[0, \frac{1}{4}\right]
            \right) \cap \left(
                \mathbb{Q} \times \mathbb{Q}
            \right), \textrm{ если } d \equiv 1 \pmod 4\\
        \end{split}\right.,
    \end{equation*}
    а так же для $r > 0$ и $\lambda = \lambda_1 + \lambda_2 \sqrt{d} \in \mathbb{Z}[\sqrt{d}]$
    \begin{equation*}
        U(\lambda, r) = \left\{
            q_1 + q_2 \sqrt{d} \in Q[\sqrt{d}] \big| |(q_1 - \lambda_1)^2 - d(q_2 - \lambda_2)^2| < r
        \right\}.
    \end{equation*}
    Тогда кольцо $\mathbb{Z}[\sqrt{d}]$ является евклидовым относительно нормы числового поля $\mathbb{Q}[\sqrt{d}]$ тогда и только тогда, когда существует $\lambda \in \mathbb{Z}[\sqrt{d}]$, что $F(d) \subseteq U(\lambda, 1)$.
\end{proposition}

\begin{definition}
    Обозначим $\Lambda_K = \sup_{m/n \in F_1} |m/n|$, где $|m/n| = \frac{\elementnorm{m}}{\elementnorm{n}}$ для $m/n \in \zeroless{F_1}$, $(m, n) = 1$ и $|0| = 0$.
\end{definition}

\begin{theorem}
    Если $R$ -- евклидово кольцо относительно нормы $\elementnorm{\cdot}$, то $\Lambda_R \in [0, 1]$.

    Если $R$ -- факториальное кольцо и $\Lambda_R \in [0, 1)$, то $R$ -- евклидово относительно нормы $\elementnorm{\cdot}$ и $l_n(R) \le [\log_{\Lambda_R^{-1}} n] + 2$ для всех $n \in \mathbb{N}$, где $\log_{\infty} n = 0$.
\end{theorem}
\begin{proof}

\end{proof}

\begin{theorem}
    Пусть $d \neq 1$ целое число свободное от квадратов.
    Если кольцо $\mathbb{Z}[\sqrt{d}]$ евклидово, то $l_n(\mathbb{Z}[\sqrt{d}]) = O(\log n)$.
\end{theorem}
\begin{proof}

\end{proof}

\onlyinsubfile{
    \subfile{_10_bibliography}
    \subfile{_11_pub}
}

\end{document}
