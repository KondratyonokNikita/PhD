\documentclass[_00_dissertation.tex]{subfiles}
\begin{document}

\onlyinsubfile{
    \renewcommand{\contentsname}{ОГЛАВЛЕНИЕ}
    \setcounter{tocdepth}{3}
    \tableofcontents
}

\newpage
\begin{center}
    \refstepcounter{section}
    \section*{ГЛАВА \arabic{section}.\\ ТЕОРЕМА КРОНЕКЕРА-ВАЛЕНА В ФАКТОРИАЛЬНЫХ КОЛЬЦАХ}\label{ch:Kronecker-Vahlen theorem}
    \addcontentsline{toc}{chapter}{ГЛАВА \arabic{section}. ТЕОРЕМА КРОНЕКЕРА-ВАЛЕНА В ФАКТОРИАЛЬНЫХ КОЛЬЦАХ}
\end{center}

\begin{definition}
    Пусть $R$ факториальное кольцо.
    Функцию $\elementnorm{\dot}: R \to \mathbb{N} \cap \{0, -\infty\}$ будем называть нормой в $R$, если
    \begin{itemize}
        \item $\elementnorm{x} = -\infty$ тогда и только тогда, когда $x = 0$

        \item $\elementnorm{xy} \ge \elementnorm{x}$

        \item для $x, y \in \zeroless{R}$ равенство $\elementnorm{xy} = \elementnorm{x}$ выполнено тогда и только тогда, когда $y \in \invertible{K}$.
    \end{itemize}
\end{definition}

\begin{remark}
    Для любого факториального кольца $R$ существует норма.
    Рассмотрим разложение элемента $x$ на простые множители $x = \varepsilon p_1^{\alpha_1} \dots p_k^{\alpha_k}$, где $\varepsilon \in \invertible{R}$, $p_1, \dots, p_k$ -- простые элементы $R$.
    Тогда функция
    \begin{equation*}
        \elementnorm{x} = \left\{\begin{split}
            \sum_{i=1}^k \alpha_{i}, & \textrm{ если } x \neq 0\\
            -\infty, & \textrm{ если } x = 0
        \end{split}\right.
    \end{equation*}
    является нормой в $R$.
\end{remark}

\begin{definition}

\end{definition}

\begin{definition}
    Пусть $a$ и $b$ ненулевые элементы факториального кольца $R$.
    Для любых $k \in \mathbb{N}$ и $q_1, \dots, q_k \in R$ обозначим
    \begin{equation*}
        \mathcal{D}_{a, b}(q_1, \dots, q_k) = (r_{-1}, r_0, \dots, r_{k-1}, r_k) \in R^{k+2},
    \end{equation*}
    где $r_{-1} = a$, $r_0 = b$, $r_i = r_{i-2} - q_i r_{i-1}$, для $i = 1, \dots, k$.
    Выражение $\mathcal{D}_{a, b}(q_1, \dots, q_k)$ будем называть цепочкой делений для $a, b \in R$.
    Через $\mathcal{E}_{a, b}$ обозначим множество всех цепочек делений для $a, b \in R$, которые заканчиваются на $0$:
    \begin{equation*}
        \mathcal{E}_{a, b} = \left\{
        \mathcal{D}_{a, b}(q_1, \dots, q_k) = (r_{-1}, \dots, r_k) \big| r_1, \dots, r_{k-1} \in \zeroless{R}, r_k = 0
        \right\}.
    \end{equation*}
\end{definition}

\begin{definition}
    Пусть $a$ и $b$ ненулевые элементы факториального кольца $R$.
    Цепочкой делений с выбором минимального по норме остатка для $a, b \in R$ будем называть такую цепочку делений, что $q_i = \textrm{int}(r_{i-2}/r_{i-1})$ для любого $i = 1, \dots, k$.

    Если цепочка делений с выбором минимального по норме остатка существует, то обозначим через $\mathcal{L}_{a, b}$ ее длину.
    Если ее не существует, то обозначим $\mathcal{L}_{a, b} = \infty$.
\end{definition}

\begin{definition}
    Обозначим через $\mathcal{l}_{a, b}$ длину кратчайшей цепочки делений для $a, b \in \zeroless{R}$.
    \begin{equation*}
        \mathcal{l}_{a, b} = \left\{\begin{split}
            \min_{\mathcal{D}_{a, b}(q_1, \dots, q_k) \in \mathcal{E}_{a, b}} k, & \textrm{ если } \mathcal{E}_{a, b} \neq \emptyset\\
            \infty, & \textrm{ если } \mathcal{E}_{a, b} = \emptyset
        \end{split}\right..
    \end{equation*}
\end{definition}

\begin{definition}
    Через $l_n(R)$ обозначим максимальную длину цепочки делений с выбором минимального по норме остатка для $a, b \in \zeroless{R}$ с ограниченной нормой.
    \begin{equation*}
        l_n(R) = \max \left\{
            \mathcal{L}_{a, b} \big| a, b \in \zeroless{R}, \elementnorm{a} \le \elementnorm{b} \le n
        \right\}.
    \end{equation*}
\end{definition}

\onlyinsubfile{
    \subfile{_10_bibliography}
    \subfile{_11_pub}
}

\end{document}
