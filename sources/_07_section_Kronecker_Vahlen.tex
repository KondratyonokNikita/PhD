\documentclass[_00_dissertation.tex]{subfiles}
\begin{document}

\onlyinsubfile{
    \renewcommand{\contentsname}{ОГЛАВЛЕНИЕ}
    \setcounter{tocdepth}{3}
    \tableofcontents
}

\refstepcounter{chapter}
\chapter*{\MakeUppercase{Глава \arabic{chapter}\\ Теорема Кронекера-Валена в факториальных кольцах}}\label{chapter:Kronecker-Vahlen theorem}
\addcontentsline{toc}{chapter}{Глава \arabic{chapter} Теорема Кронекера-Валена в факториальных кольцах}

\begin{definition}
    Пусть $a$ и $b$ ненулевые элементы факториального кольца $R$.
    Для любых $k \in \mathbb{N}$ и $q_1, \dots, q_k \in R$ обозначим
    \begin{equation*}
        \mathcal{D}_{a, b}(q_1, \dots, q_k) = (r_{-1}, r_0, \dots, r_{k-1}, r_k) \in R^{k+2},
    \end{equation*}
    где $r_{-1} = a$, $r_0 = b$, $r_i = r_{i-2} - q_i r_{i-1}$, для $i = 1, \dots, k$.
    Выражение $\mathcal{D}_{a, b}(q_1, \dots, q_k)$ будем называть цепочкой делений для $a, b \in R$.
    Обозначим через $\mathcal{E}_{a, b}$ следующее множество:
    \begin{equation*}
        \mathcal{E}_{a, b} = \left\{
        \mathcal{D}_{a, b}(q_1, \dots, q_k) = (r_{-1}, \dots, r_k) \big| r_1, \dots, r_{k-1} \in \zeroless{R}, r_k = 0
        \right\}.
    \end{equation*}
\end{definition}

\begin{definition}
    Пусть $a$ и $b$ ненулевые элементы факториального кольца $R$.
    Цепочкой делений с выбором минимального по норме остатка для $a, b \in R$ будем называть такую цепочку делений, что $q_i = \textrm{int}(r_{i-2}/r_{i-1})$ для любого $i = 1, \dots, k$.

    Если цепочка делений с выбором минимального по норме остатка существует, то обозначим через $\mathcal{L}_{a, b}$ ее длину.
    Если ее не существует, то обозначим $\mathcal{L}_{a, b} = \infty$.
\end{definition}

\begin{definition}
    Обозначим через $\mathpzc{l}_{a, b}$ длину кратчайшей цепочки делений для $a, b \in \zeroless{R}$.
    \begin{equation*}
        \mathpzc{l}_{a, b} = \begin{cases}
            \min_{\mathcal{D}_{a, b}(q_1, \dots, q_k) \in \mathcal{E}_{a, b}} k, & \textrm{ если } \mathcal{E}_{a, b} \neq \emptyset\\
            \infty, & \textrm{ если } \mathcal{E}_{a, b} = \emptyset
        \end{cases}.
    \end{equation*}
\end{definition}

\begin{definition}
    Через $l_n(R)$ обозначим максимальную длину цепочки делений с выбором минимального по норме остатка для $a, b \in \zeroless{R}$ с ограниченной нормой.
    \begin{equation*}
        l_n(R) = \max \left\{
            \mathcal{L}_{a, b} \big| a, b \in \zeroless{R}, \elementnorm{a} \le \elementnorm{b} \le n
        \right\}.
    \end{equation*}
\end{definition}

\begin{remark}
    Теорему Кронекера-Валена в кольце целых чисел можно сформулировать в терминах этой главы следующим образом.
    Пусть $R = \mathbb{Z}$, $\elementnorm{x} = |x|$, $\fr{\alpha} = \alpha - [\alpha + 1/2]$.
    Тогда $\mathcal{L}_{a, b} = \mathpzc{l}_{a, b}$ для всех $a, b \in \zeroless{R}$, или другими словами, цепочка делений с выбором минимального по норме остатка является кратчайшей.

    Далее в этой главе определяются достаточные условия на факториальное кольцо $R$ с заданной нормой и дробной частью, при которых теорема Кронекера-Валена будет выполняться в этом кольце.
\end{remark}

\section{Теорема Кронекера-Валена в специальном классе факториальных колец}

\begin{definition}
    Определим функцию $\omega: F_1 \to F_1$ следующим образом
    \begin{equation*}
        \omega(\alpha) = \begin{cases}
            \fr{\alpha^{-1}}, \textrm{ если } \alpha \neq 0\\
            0, \textrm{ если } \alpha = 0
        \end{cases}
    \end{equation*}
\end{definition}

\begin{definition}
    Будем говорить, что $(x_0, \alpha, n) \in \zeroless{R} \times \zeroless{F_1} \times \mathbb{N}$ \emph{регулярная тройка}, если существуют
    \begin{itemize}
        \item $p, l \in \mathbb{N}$, $p \le n$ и $l \le p+1$,

        \item $\varepsilon_i \in \invertible{R}$, $b_i, c_i \in R$ для $i = 1, \dots, l-1$,

        \item $\varepsilon \in \{0, 1\}$,
    \end{itemize}
    для которых выполнены следующие условия
    \begin{itemize}
        \item $\beta_1 = \omega^{(p)}\left(\fr{(\alpha - x_0)^{-1}}\right)$;

        \item $\beta_{i+1} = (\varepsilon_i \beta_i + c_i)^{-1} + b_i$, $i = 1, \dots, l-1$;

        \item $\beta_{l} = \alpha^{(-1)^{\varepsilon}}$.
    \end{itemize}
\end{definition}

\begin{definition}
    Через $\mathcal{T}$ обозначим множество всех таких факториальных колец $R$, для которых существует $D_R \in \mathbb{N}$, что выполнено
    \begin{itemize}
        \item для всех $x_0 \in \zeroless{R}$, $\alpha \in \zeroless{F_1}$ тройка $(x_0, \alpha, D_R - 1)$ регулярная;

        \item если $D_R \ge 3$, то для любого $k \in [3, D_R] \cap \mathbb{N}$ и любых $x_0 \in \zeroless{R}$, $\alpha \in \zeroless{F_1}$ из равенства $\omega^{(k-2)}(\fr{(\alpha - x_0)^{-1}}) = 0$ следует, что тройка $(x_0, \alpha, k-2)$ регулярная.
    \end{itemize}
\end{definition}

\begin{definition}
    Обозначим
    \begin{equation*}
        [x_1: x_2: \dots: x_k] = x_{1} + \cfrac{1}{
            x_{2} + \cfrac{1}{
                x_{3} + \cfrac{1}{
                    \ddots + \cfrac{1}{
                        x_{k}
                    }
                }
            }
        }.
    \end{equation*}
    Будем говорить, что для $\alpha \in F$ и $k \in \mathbb{N}$ имеет место $(\alpha, k)$-разрешимость, если разрешимо уравнение
    \begin{equation*}
        \alpha = [x_1: x_2: \dots: x_k].
    \end{equation*}
\end{definition}

\begin{lemma}\label{lemma:omega_and_euclidean_algorithm}
    Пусть кольцо $R \in \mathcal{T}$.
    Для любых $\alpha \in F_1$ и $k \in \mathbb{N}$ имеет место $(\alpha, k)$-разрешимость тогда и только тогда, когда $\omega^{(k-1)}(\alpha) = 0$.
\end{lemma}
\begin{proof}
    Пусть $\alpha = m/n$, где $(m, n) = 1$.
    Если $\alpha = 0$ или $k = 1$, то утверждение леммы очевидно.

    Пусть $k = 2$.
    Заметим, что $(\alpha, 2)$-разрешимость имеет место тогда и только тогда, когда $m \equiv \varepsilon \pmod{n}$ для некоторого $\varepsilon \in \invertible{R}$.
    Пусть $q \in R$ и $m = q n + \varepsilon$, тогда $m/n = \fr{m/n} = \fr{\varepsilon/n}$.
    Если $n \in \invertible{R}$, то $\fr{\varepsilon/n} = 0$ и утверждение леммы очевидно.
    Если $n \not\in \invertible{R}$, то $\fr{\varepsilon/n} = \varepsilon/n$ и $\omega(m/n) = \fr{n/\varepsilon} = 0$.

    Пусть $k \ge 3$.
    Докажем лемму индукцией по $k$.
    Заметим, что $(\alpha, k)$-разрешимость имеет место тогда и только тогда, когда существует $z \in R$, что имеет место $((\alpha - z)^{-1}, k-1)$-разрешимость.
    Таким образом надо показать, что $\omega^{(k-1)}(\alpha) = 0$ тогда и только тогда, когда существует $z \in R$, что $\omega^{(k-2)}(\fr{(\alpha - z)^{-1}}) = 0$.
    Если $\omega^{(k-1)}(\alpha) = 0$, то по определению $\omega^{(k-2)}(\fr{\alpha^{-1}}) = 0$.
    Если $\omega^{(k-2)}(\fr{(\alpha - z)^{-1}}) = 0$ и $z = 0$, то по определению $\omega^{(k-1)}(\alpha) = 0$.

    Предположим, что существует $z \neq 0$, что $\omega^{(k-2)}(\fr{(\alpha - z)^{-1}}) = 0$.
    Если $D_R \ge 3$ и $k \le D_R$, то обозначим $X = k-2$, иначе $X = D_R - 1$.
    Заметим, что $X \le k-2$.
    Из определения класса $\mathcal{T}$ следует, что тройка $(z, \alpha, X)$ регулярная.
    Следовательно, существуют $p, l \in \mathbb{N}$, $p \le X$, $l \le p+1$, $\varepsilon \in \{0, 1\}$, $\varepsilon_i \in \invertible{R}$, $b_i, c_i \in R$, $i = 1, \dots, l-1$, что
    \begin{equation*}
        \begin{split}
            \beta_1 = \omega^{(p)}\left(\fr{(\alpha - z)^{-1}}\right),\\
            \beta_{i+1} = (\varepsilon_i \beta_i + c_i)^{-1} + b_i, i = 1, \dots, l-1,\\
            \beta_{l} = \alpha^{(-1)^{\varepsilon}}.
        \end{split}
    \end{equation*}

    Заметим, что $0 \le k-p-2 < k$, так как $p \le X \le k-2$.
    Из предположения индукции и равенства $\omega^{(k-p-2)}(\beta_1) = 0$ следует, что имеет место $(\beta_1, k-p-1)$-разрешимость.
    Из определения $\beta_i$ следует, что для любых $i = 1, \dots, l-1$ и $j \in \mathbb{N}$ из того, что имеет место $(\beta_i, j)$-разрешимость следует, что имеет место $(\beta_{i+1}, j+1)$-разрешимость.
    Следовательно, имеет место $(\alpha^{(-1)^{\varepsilon}}, k-p+l-2)$-разрешимость.
    Так как $k - p + l - 2 \le k - 1$, то по предположению индукции $\omega^{(k-p+l-3)}(\fr{\alpha^{(-1)^{\varepsilon}}}) = 0$.
    Следовательно, $\omega^{(k-2)}(\fr{\alpha^{(-1)^{\varepsilon}}}) = 0$.
    Тогда из определения $\omega$ получаем $\omega^{(k-1)}(\alpha) = 0$.
\end{proof}

\begin{lemma}\label{lemma:euclidean_algorithm_and_minima}
    Пусть кольцо $R$ факториальное.
    Тогда для любых двух элементов $a, b \in \zeroless{R}$ выполнено равенство
    \begin{equation*}
        \mathcal{L}_{a, b} = \min\left\{
            k \in \mathbb{N} \big| \omega^{(k-1)}\left(\frac{a}{b}\right) = 0
        \right\},
    \end{equation*}
    где $\min \emptyset = \infty$.
\end{lemma}
\begin{proof}
    Рассмотрим произвольные $a, b \in \zeroless{R}$.
    Предположим, что $\mathcal{L}_{a, b} = k < \infty$ и $\mathcal{D}_{a, b}(q_1, \dots, q_k) = (r_{-1}, r_0, \dots, r_k)$ это цепочка делений с выбором минимального по норме остатка.

    Из определения цепочки делений с выбором минимального по норме остатка следует, что
    \begin{equation*}
        \frac{r_{i-2}}{r_{i-1}} = q_i + \frac{r_i}{r_{i-1}}
    \end{equation*}
    для $i = 1, \dots, k-1$ и
    \begin{equation*}
        \frac{r_{k-2}}{r_{k-1}} = q_k.
    \end{equation*}
    Заметим, что
    \begin{equation*}
        \omega\left(
            \fr{\frac{r_{i-2}}{r_{i-1}}}
        \right) = \fr{\frac{r_{i-1}}{r_i}}.
    \end{equation*}
    Следовательно, получаем
    \begin{equation*}
        \omega^{(k-1)}\left(
            \fr{\frac{a}{b}}
        \right) = \fr{\frac{r_{k-2}}{r_{k-1}}} = 0.
    \end{equation*}
    Так как
    \begin{equation*}
        \omega^{(j-1)}\left(
            \fr{\frac{a}{b}}
        \right) = \fr{\frac{r_{j-2}}{r_{j-1}}} = 0,
    \end{equation*}
    для $j = 1, \dots, k-1$, то $\mathcal{L}_{a, b} = k = \min\{k \in \mathbb{N} \big| \omega^{(k-1)}(a\/b) = 0\}$.

    Предположим, что $\mathcal{L}_{a, b} = \infty$, но условия леммы не выполняются.
    Выберем минимальное такое $k \in \mathbb{N}$, что $\omega^{(k-1)}\left(\fr{\frac{a}{b}}\right) = 0$.

    Обозначим $r_{-1} = a$, $r_0 = b$.
    Тогда имеем $\omega^{(k-1)}\left(\fr{\frac{r_{-1}}{r_0}}\right) = 0$.

    Пусть $\omega^{(k-i)}\left(\fr{\frac{r_{i-2}}{r_{i-1}}}\right) = 0$.
    Пусть $\int{\frac{r_{i-2}}{r_{i-1}}} = q_i$, $\fr{\frac{r_{i-2}}{r_{i-1}}} = \frac{r_i}{r_{i-1}}$.
    Если $r_i = 0$, то $\mathcal{L}_{a, b} = i$, и получаем противоречие.
    Следовательно, имеем $r_i \neq 0$ и по определению $\omega$ получаем равенствво $\omega^{(k-i-1)}\left(\fr{\frac{r_{i-1}}{r_i}}\right) = 0$.

    Используя это рассуждение $k-1$ раз, получаем, что
    \begin{equation*}
        \omega^{(0)}\left(
            \fr{\frac{r_{k-2}}{r_{k-1}}}
        \right) = 0.
    \end{equation*}
    Однако имеем
    \begin{equation*}
        0 = \omega^{(0)}\left(
            \fr{\frac{r_{k-2}}{r_{k-1}}}
        \right) = \fr{\frac{r_{k-2}}{r_{k-1}}} = \frac{r_{k}}{r_{k-1}}.
    \end{equation*}
    Следовательно, получаем $r_k = 0$.
    Из построения следует, что построили цепочку делений с выбором минимального по норме остатка, что противоречит $\mathcal{L}_{a, b} = \infty$.
\end{proof}

\begin{theorem}\label{theorem:Kroneker_Vahlen_theorem_in_UFD}
    Пусть $R \in \mathcal{T}$.
    Тогда $\mathcal{L}_{a, b} = \mathpzc{l}_{a, b}$ для всех $a, b \in \zeroless{R}$, или другими словами, цепочка делений с выбором минимального по норме остатка является кратчайшей.
\end{theorem}
\begin{proof}
    Пусть $\mathpzc{l}_{a, b} = \infty$.
    Тогда $\mathcal{L}_{a, b} = \infty$, так как иначе цепочка делений с выбором минимального по норме остатка имела бы длину меньше, чем кратчайшая.

    Пусть $\mathpzc{l}_{a, b} = k < \infty$ и $\mathcal{D}_{a, b}(q_1, \dots, q_k) = (r_{-1}, r_0, \dots, r_k)$ кратчайшая цепочка делений.
    Из леммы~\ref{lemma:omega_and_euclidean_algorithm} следует, что $\omega^{(k-1)}\left(\frac{a}{b}\right) = 0$.
    Тогда, по лемме~\ref{lemma:euclidean_algorithm_and_minima} получаем, что $\mathcal{L}_{a, b} \le k$.
    Следовательно, получаем $\mathcal{L}_{a, b} = \mathpzc{l}_{a, b}$.
\end{proof}

% \subsection{Упрощенный метод проверки $R \in \mathcal{T}$}

Пусть дано некоторое факториальное кольцо $R$.
Рассмотрим алгоритм проверки принадлежности этого кольца классу $T$.

\begin{definition}
    Через $\mathcal{S}$ обозначим множество всех таких факториальных колец $R$, что для всех $x \in \zeroless{R}$ и $\alpha \in \zeroless{F_1}$ выполнено одно из условий
    \begin{itemize}
        \item $\int{(\alpha - x)^{-1}} \in \invertible{R} \cup \{0\}$;
        
        \item $x \int{(\alpha - x)^{-1}} + 1 \in \invertible{R}$.
    \end{itemize}
\end{definition}

\begin{lemma}
    Множество $\mathcal{S}$ содержится в $\mathcal{T}$.
\end{lemma}
\begin{proof}
    Предположим, что факториальное кольцо $R \in \mathcal{S}$.
    Возьмем $D_R = 2$.
    Необходимо доказать, что для любых $x \in \zeroless{R}$ и $\alpha \in \zeroless{F_1}$ тройка $(x, \alpha, D_R - 1)$ является регулярной.
    
    Возьмем $p = 1$, $l = 2$ и обозначим $b = \int{(\alpha - x)^{-1}}$.
    Тогда имеем
    \begin{equation*}
        \beta_1 = \omega\left(
            \fr{(\alpha - x)^{-1}}
        \right) = \omega\left(
            (\alpha - x)^{-1} - b
        \right) = \fr{\frac{\alpha - x}{1 - b\alpha + bx}}.
    \end{equation*}

    Обозначим $c = \int{\beta_1}$, тогда
    \begin{equation*}
        \beta_1 = \frac{\alpha - x}{1 - b\alpha + bx} - c.
    \end{equation*}

    Предположим, что $b \in \invertible{R} \cup \{0\}$.
    Если $b = 0$, то
    \begin{equation*}
        \beta_2 = \left(
            \beta_1 + x + c
        \right)^{-1} =
        \left(
            \frac{\alpha - x}{1} - c + x + c
        \right)^{-1} = \alpha^{-1}.
    \end{equation*}

    Если $b \in \invertible{R}$, то
    \begin{equation*}
        \begin{split}
            \beta_2 = \left(
                -b^2 \beta_1 - b - b^2 c
            \right)^{-1} + b^{-1} + x = \\
            = \left(
                -b^2 \left(
                    \frac{\alpha - x}{1 - b\alpha + bx} - c
                \right) - b - b^2 c
            \right)^{-1} + b^{-1} + x = \\
            = \left(
                \frac{-b^2\alpha + b^2x}{1 - b\alpha + bx} - b
            \right)^{-1} + b^{-1} + x = \\
            = \left(
                \frac{-b^2\alpha + b^2x - b + b^2\alpha - b^2x}{1 - b\alpha + bx}
            \right)^{-1} + b^{-1} + x = \\
            = \left(
                \frac{- b}{1 - b\alpha + bx}
            \right)^{-1} + b^{-1} + x = \frac{-1 + b\alpha - bx}{b} + b^{-1} + x = \alpha.
        \end{split}
    \end{equation*}
    
    Предположим, что $xb + 1 \in \invertible{R}$.
    Обозначим $xb + 1 = \varepsilon$.
    Тогда
    \begin{equation*}
        \begin{split}
            \beta_2 = \left(
                \beta_1 \varepsilon^2 + c\varepsilon^2 + x\varepsilon
            \right)^{-1} + b\varepsilon^{-1} = \frac{
                \varepsilon^{-1}
            }{
                \beta_1 \varepsilon + c\varepsilon + x
            } + b\varepsilon^{-1} = \\
            = \frac{
                1 - bx\varepsilon^{-1}
            }{
                \beta_1 \varepsilon + c\varepsilon + x
            } + b\varepsilon^{-1} = \frac{
                1 - bx\varepsilon^{-1} + b\varepsilon^{-1}\beta_1 \varepsilon + b\varepsilon^{-1}c\varepsilon + b\varepsilon^{-1}x
            }{
                \beta_1 \varepsilon + c\varepsilon + x
            } = \\
            = \frac{
                1 + b\beta_1 + bc
            }{
                \beta_1 \varepsilon + c\varepsilon + x
            } = \frac{
                1 + b\beta_1 + bc
            }{
                \beta_1(xb + 1) + c(xb + 1) + x
            } = \\
            = \frac{
                1 + b\left(
                    \frac{\alpha - x}{1 - b\alpha + bx} - c
                \right) + bc
            }{
                \left(
                    \frac{\alpha - x}{1 - b\alpha + bx} - c
                \right)(xb + 1) + c(xb + 1) + x
            } = \frac{
                1 + \frac{b\alpha - bx}{1 - b\alpha + bx}
            }{
                \frac{xb\alpha + \alpha - x^2b - x}{1 - b\alpha + bx} + x
            } = \\
            = \frac{
                \frac{1}{1 - b\alpha + bx}
            }{
                \frac{\alpha}{1 - b\alpha + bx}
            } = \alpha^{-1},
        \end{split}
    \end{equation*}
    
    Следовательно, тройка $(x, \alpha, D_R - 1)$ регулярная.
\end{proof}

Используя доказанную выше лемму, сформулируем метод проверки включения $R \in \mathcal{S}$.

\begin{algorithm}\label{algorithm:R_in_S}
    На вход подается факториальное кольцо $R$.
    
    \begin{enumerate}
        \item Построить множество
        \begin{equation*}
            J = \left\{
                x \in \zeroless{R} \big| \forall \alpha \in \zeroless{F_1}, \int{(\alpha - x)^{-1}} \in \invertible{R} \cup \{0\}
            \right\}
        \end{equation*}
        
        \item Для каждого $x_0 \in \zeroless{R} \setminus J$ построить множество
        \begin{equation*}
            Y(x_0) = \left\{
                f_{x_0}(\alpha) = \int{(\alpha - x_0)^{-1}} | \alpha \in \zeroless{F_1}
            \right\}
        \end{equation*}
        
        \item Для каждого $x_0 \in \zeroless{R} \setminus J$ построить множество
        \begin{equation*}
            U(x_0) = \left(
                \left\{
                    \frac{\varepsilon - 1}{x_0} | \varepsilon \in \invertible{R}
                \right\} \cap R
            \right) \cup \invertible{R}
        \end{equation*}
        
        \item Если $Y(x_0) \subseteq U(x_0)$ для всех $x_0 \in \zeroless{R} \setminus J$, то ответ ''$R \in \mathcal{S}$'', иначе ответ ''неизвестно''
    \end{enumerate}
\end{algorithm}

Докажем корректность этого алгоритма.

\begin{proposition}
    Если алгоритм~$\ref{algorithm:R_in_S}$ вернул ответ ''$R \in \mathcal{S}$'', то $R \in \mathcal{S}$.
\end{proposition}
\begin{proof}
    Предположим, что ответ ''$R \in \mathcal{S}$''.
    Рассмотрим произвольный $x_0 \in \zeroless{R}$.
    Если $x_0 \in J$, то
    \begin{equation*}
        \int{(\alpha - x)^{-1}} \in \invertible{R} \cup \{0\},
    \end{equation*}
    и выполнено первое условие определения множества $\mathcal{S}$.
    
    Предположим, что $x_0 \not\in J$.
    Рассмотрим $y = \int{(\alpha - x_0)^{-1}} \in Y(x_0)$.
    С другой стороны, $y \in U(x_0)$, следовательно, $y = \frac{\varepsilon - 1}{x_0}$.
    Тогда получаем, что
    \begin{equation*}
        \frac{\varepsilon - 1}{x_0} = \int{(\alpha - x_0)^{-1}}.
    \end{equation*}
    Следовательно, получаем
    \begin{equation*}
        \varepsilon = x_0 \int{(\alpha - x_0)^{-1}} + 1.
    \end{equation*}
    Значит выполнено второе условие определения множества $\mathcal{S}$.
\end{proof}

% \subsection{Примеры проверки $R \in \mathcal{T}$}

\begin{example}\label{example:Z}
    Пусть $R = \mathbb{Z}$.
    Норма задана функцией $\elementnorm{x} = |x|$.
    Дробная часть $\fr{\alpha} = \alpha - \left[\alpha + \frac{1}{2}\right]$.
    Тогда
    \begin{itemize}
        \item $J = \left\{x \in \mathbb{Z} \big| |x| > 1\right\}$
        
        \item $Y(1) = \{-2, -1\}$, $Y(-1) = \{1, 2\}$
        
        \item $U(1) = \{-2, -1, 0, 1\}$, $U(-1) = \{-1, 0, 1, 2\}$
        
        \item $Y(1) \subseteq U(1)$, $Y(-1) \subseteq U(-1)$
    \end{itemize}
    
    Следовательно, $\mathbb{Z} \in \mathcal{S}$.
\end{example}

\begin{example}\label{example:P[t]}
    Пусть $P$ поле.
    Положим $R = \mathbb{P}[t]$.
    Норма задана функцией $\upsilon(f)=\deg f$, $f \in \mathbb{P}[t]$.
    Дробная часть $\textrm{fr}(m(t)/n(t))=r(t)/n(t)$, $m(t)\equiv r(t)(\textrm{mod}\ n(t))$, $\deg r < \deg n$, $m(t)/n(t) \in \mathbb{P}(t).$

    \begin{itemize}
        \item $\mathbb{J}=\mathbb{K}_{*}$;

        \item $\mathbb{K}_{*}\backslash\mathbb{J} = \emptyset$;
    \end{itemize}

    Следовательно, $\mathbb{P}[t] \in \mathcal{S}$.
\end{example}

\begin{example}\label{example:coordinate_ring_of_circle}
    Пусть $R$ координатное кольцо $\mathbb{Q}(i)[x, y]/(x^2 + y^2 + 1)$.
    Известно, что $R$ изоморфно $\mathbb{Q}(i)[t,  t^{-1}]$, где отображение задано следующим образом
    \begin{equation*}
        \begin{split}
            X \to \cos \theta \to \frac{e^{ix} + e^{-ix}}{2} \to \frac{t + t^{-1}}{2},\\
            Y \to \sin \theta \to \frac{-ie^{ix} + ie^{-ix}}{2} \to \frac{-it + it^{-1}}{2}
        \end{split}
    \end{equation*}
    Так же известно, что $Q[i][t, t^{-1}]$ является факториальным кольцом.
    
    Обратимыми элементами в этом кольце  являются одночлены вида $\alpha t^k$, где $\alpha\in\mathbb{Q}(i)$, $k\in\mathbb{Z}$.
    Введем норму в этом кольце следующим образом
    \begin{equation*}
        \upsilon(x) = \upsilon\left(
            \sum_{i=k_1}^{k_2} \alpha_i t^i
        \right) = k_2 - k_1
    \end{equation*}
    где $\alpha_i\in\mathbb{Q}(i)$, если $x \neq 0$ и $\upsilon(0) = -\infty$.
    Проверим критерии.
    Рассмотрим два элемента кольца
    \begin{equation*}
        \begin{split}
            x = \sum_{i=k_1}^{k_2} \alpha_i t^i,\\
            y = \sum_{i=l_1}^{l_2} \beta_i t^i
        \end{split}
    \end{equation*}
    При  умножении максимальная степень будет равна $k_2 + l_2$, а минимальная $k_1 + l_1$.
    Тогда $\upsilon(xy) = k_2 + l_2 - k_1 - l_1 \ge k_2 - k_1$.
    Это равенство будет выполняться только при $l_2 - l_1 = 0$.
    А это будет означать, что $y\in\mathbb{I}$.
    
    Покажем, что $\textrm{fr}(\alpha) = \varepsilon^{-1}\textrm{fr}(\varepsilon\alpha)$ для $\varepsilon\in\mathbb{I}$, $\alpha\in\mathbb{K}$.
    Предположим, что $\textrm{fr}\left(\frac{m}{n}\right) = \frac{r}{n}$.
    Это означает, что $n | (m-r)$ и $r$ имеет минимальную норму из возможных.
    Так как норма не меняется при умножении на обратимый элемент, то $\varepsilon r$ будет иметь минимальную норму, а так же $n | (\varepsilon m - \varepsilon r)$.
    Следовательно, имеем $\textrm{fr}(\alpha) = \varepsilon^{-1}\textrm{fr}(\varepsilon\alpha)$ для $\varepsilon\in\mathbb{I}$, $\alpha\in\mathbb{K}$.
    Тогда $\textrm{int}(\alpha) = \varepsilon^{-1}\textrm{int}(\varepsilon\alpha)$ для $\varepsilon\in\mathbb{I}$, $\alpha\in\mathbb{K}$.
    
    Заметим, что умножив $\textrm{int}((\alpha-x)^{-1})$ на некоторый обратимый элемент можно сделать так, что $\alpha$ принадлежит полю частных кольца многочленов, а $x$ это многочлен.
    Тогда этот пример будет аналогичным примеру с многочленами.
    Следовательно, для данного координатного кольца выполнен аналог теоремы Кронекера-Валена.
\end{example}

\begin{example}\label{example:Z[t]}
    Пусть $R = \mathbb{Z}[t]$, $\upsilon(f)=\deg f$, $f \in \mathbb{Z}[t]$.
    Определим дробную часть в $\mathbb{F}=\mathbb{Z}(t)$ следующим образом.
    Рассмотрим отображение $\mathcal{A}:\mathbb{F}/\mathbb{K}\to\mathbb{F}$:
    \begin{equation*}
        \mathcal{A}(\mathbb{A})=m(t)/n(t),
    \end{equation*}
    где $\mathbb{A}=\{m(t)/n(t)+q(t)|q(t)\in\mathbb{Z}[t]\}.$
    Для произвольных $\mathbb{A}\in\mathbb{F}/\mathbb{K},$ $\alpha\in\mathbb{A}$ положим
    \begin{equation*}
        \begin{array}{c}
            \textrm{int}(\alpha)=r(t),\\
            \textrm{fr}(\alpha)=\mathcal{A}(\mathbb{A})-r(t),
        \end{array}
    \end{equation*}
    где $r(t)\in\mathbb{Z}[t]$ и для любого $p(t)\in\mathbb{Z}[t]$
    \begin{equation*}
        \lim_{t\to+\infty}\left|\frac{\mathcal{A}(\mathbb{A})-r(t)}{\mathcal{A}(\mathbb{A})-p(t)}\right|\le 1.
    \end{equation*}

    Докажем, что
    \begin{equation*}
        \mathbb{J}\supseteq\{f\in\mathbb{Z}[t]|\deg f>0\ \textrm{или}\ |f(t)|\equiv|x_{0}|>2\}.
    \end{equation*}
    Рассмотрим произвольные $\mathbb{A}\in\mathbb{F}/\mathbb{K}$ и $\alpha=m(t)/n(t)\in\mathbb{A}$, $\alpha=\textrm{fr}(\alpha)$.

    Имеем два случая.
    Предположим, что $\deg m > \deg n$.
    Из того, что $\alpha=\textrm{fr}(\alpha)$ и определения дробной части следует, что
    \begin{equation*}
        1 \ge \lim_{t\to+\infty}\left|\frac{\frac{m(t)}{n(t)}}{\frac{m(t)}{n(t)} - p(t)}\right| = \lim_{t\to+\infty}\left|\frac{m(t)}{m(t) - n(t)p(t)}\right|.
    \end{equation*}
    Следовательно, получаем, что для любых $p(t)\in \mathbb{Z}[t]$ выполнено $\deg m(t) \le \deg(m(t) - n(t)p(t))$.
    Предположим, что $\textrm{int}((\alpha-x_{0})^{-1})=r(t)\not\equiv0$ для некоторого $x_{0}(t)\in\mathbb{Z}[t]$.
    Тогда
    \begin{equation*}
        \begin{split}
            \textrm{fr}((\alpha - x_{0})^{-1}) = (\alpha - x_0)^{-1} - r =\\
            = \left(\frac{m}{n} - x_0\right)^{-1} - r = \frac{n}{m - x_0 n} - r =\\
            = \frac{n - r(m - nx_{0})}{m - nx_{0}}.
        \end{split}
    \end{equation*}
    Из того, что $\deg n < \deg m\le \deg(m-nx_{0})$ следует, что $\deg(n-r(m-nx_{0})) > \deg n$.
    Однако выше показано, что $\deg(n-r(m-nx_{0})) < \deg n$, исходя из определения дробной части.
    Получаем противоречие.
    Следовательно, если $\deg m > \deg n$, то для любого $x_0 \in \mathbb{Z}[t]$ выполнено $\textrm{int}((\alpha-x_{0})^{-1}) = 0$.

    Теперь предположим, что $\deg m \le \deg n$.
    Предположим, что $\textrm{int}((\alpha-x_{0})^{-1})=r(t)\not\equiv0$ для некоторого $x_{0}\in\mathbb{Z}[t]$, $\deg x_{0} > 0$.
    Тогда $\deg(m-nx_{0}) > \deg n$.
    Следовательно, имеем $\deg(n-r(m-nx_{0})) > \deg n$.
    А это противоречит определению дробной части.
    Следовательно, если $\deg m \le \deg n$, то для любого $x_{0}\in\mathbb{Z}[t]$, $\deg x_{0} > 0$ выполнено $\textrm{int}((\alpha-x_{0})^{-1}) = 0$.
    
    Предположим, что $\textrm{int}((\alpha-x_{0})^{-1})=r(t)\not\equiv0$ для некоторого $x_{0} = c \in \mathbb{Z}\setminus\{0, \pm 1, \pm 2\}$.
    Если $\deg m < \deg n$, то $\deg n < \deg(n-r(m-nc))$ или $r=const$.
    Первый вариант противоречит определению дробной части.
    Если $r=const$, то $\lim_{t\to+\infty}\left|\frac{n(t)-r(m(t)-n(t)c)}{n(t)}\right|=|1+rc|\ge2$, но это противоречит определению дробной части.
    Таким образом имеем $\deg m=\deg n$.
    Учитывая то, что $\alpha=\textrm{fr}(\alpha)$ и $\deg m=\deg n$ получаем $\lim_{t\to+\infty}\left|\frac{m(t)}{n(t)}\right|\le 0.5$.
    Если $\deg r>0,$ то $ \deg n<\deg(n-r(m-nc)),$ но это противоречие.
    Таким образом $r=const$ и
    \begin{align*}
        \lim_{t\to+\infty}\left|\frac{n(t)-r(m(t)-n(t)c)}{n(t)}\right| \ge |1+rc|-|r|/2 \ge\\
        \ge |r|(|c|-1/2)-1 \ge \frac{3}{2},
    \end{align*}
    но это противоречит определению дробной части.

    Таким образом $\mathbb{J}\supseteq\{f\in\mathbb{Z}[t]|\deg f>0\ \textrm{или}\ |f(t)|\equiv|x_{0}|>2\}$.
    Если $x_{0}\equiv\pm2,$ то $\textrm{int}((\alpha-x_{0})^{-1})=r(t)\not\equiv0$ тогда и только тогда, когда $\deg m=\deg n$ и $r(t)\equiv\pm1$, откуда следует, что $\pm2\in\mathbb{J}$.
    Несложно показать, что $0,\pm1\notin\mathbb{J}$.
    
    Тогда $\mathbb{K}_{*}\backslash\mathbb{J}\subseteq\{\pm1\}$.
    Вычисляем $\mathbb{Y}(1)=\{-2,-1\}$, $\mathbb{Y}(-1)=\{1,2\}$ и $\mathbb{U}(1)=\{-2,-1,0,1\}$, $\mathbb{U}(-1)=\{-1,0,1,2\}$.

    Следовательно, $\mathbb{Z}[t] \in \mathcal{S}.$
\end{example}

\begin{example}\label{example:a|b or b|a}
    Пусть $R$ такое евклидово кольцо, что для любых $a,b\in\mathbb{K}$ или $a|b$, или $b|a$.
    В этом случае $\mathbb{F}=\mathbb{K}\cup\{1/a|a\in\mathbb{K}_{*}\backslash\mathbb{I}\}$.

    Докажем, что $\mathbb{J}=\mathbb{K}_{*}$.
    Рассмотрим произвольный $\alpha\in\mathbb{F}_{1}^{*}$, тогда существует $b\in\mathbb{K}_{*}\backslash\mathbb{I}$ такой, что $\alpha=1/b$.
    Пусть $x\in\mathbb{K}_{*}$.
    Покажем, что $\textrm{int}((\alpha-x)^{-1}) = (\alpha-x)^{-1}$.
    Имеем $(\alpha-x)^{-1}=\frac{b}{1-bx}$.
    Предположим, что $\textrm{fr}(\frac{b}{1-bx}) \neq 0$, что эквивалентно утверждению, что $\frac{b}{1 - b x} = \frac{1}{c}$ для некоторого $c\in\mathbb{K}_{*}\backslash\mathbb{I}$.
    Так как $b$ и $1 - b x$ взаимнопростые, то $b\in\mathbb{I}$.
    Следовательно, имеем $\alpha-x=b^{-1}-x\in\mathbb{K}$.
    Это означает, что $\alpha\in\mathbb{K}$, но это противоречит условию $\alpha \in \mathbb{F}_{1}^{*}$.
    Таким образом, $x\ \textrm{int}((\alpha-x)^{-1})+1=\frac{1}{1-bx}$.
    Предположим, что $1 - b x \in \mathbb{K}_{*}\backslash\mathbb{I}$.
    Так как $b$ и $1-bx$ взаимнопростые, то $b|(1-bx)$.
    Следовательно, $b\in\mathbb{I}$.
    Из последнего следует, что $\alpha-x=b^{-1}-x\in\mathbb{K}$, но это противоречит условию $\alpha\in\mathbb{F}_{1}^{*}$.
    Итого получаем $1-bx\in\mathbb{I}$.
    Следовательно, $x\ \textrm{int}((\alpha-x)^{-1}) + 1 \in \mathbb{I}$.
    
    Тогда $\mathbb{K}_{*}\backslash\mathbb{J} = \emptyset$.
    Следовательно, $\mathbb{K}\in \mathcal{S}$.
\end{example}

Рассмотрим пример факториального кольца, не лежащего в $\mathcal{S}$, но лежащего в $\mathcal{T}$.

\begin{example}\label{example:Z[i]}
    Пусть $R = \mathbb{Z}[i]$.
    Покажем, что кольцо $\mathbb{Z}[i]$ не принадлежит $\mathcal{S}.$
    Выберем $\alpha=\frac{9-4i}{20}$, $x=1$.
    Тогда $\textrm{int}((\alpha-x)^{-1})=-2+i \notin \mathbb{I} \cup \{0\}$ и $x \ \textrm{int}((\alpha-x)^{-1})+1=-1+i \notin \mathbb{I}.$

    Покажем, что $\mathbb{K}\in\mathcal{T}$.
    Заметим, что
    \begin{equation*}
        \mathbb{F}_1 = \left\{
            z \in \mathbb{C} | Re(z), Im(z) \in \mathbb{Q} \cap [-1/2,1/2[
        \right\}.
    \end{equation*}
    Положим $D_{\mathbb{K}}=3$ в определении множества $\mathcal{T}$.

    Проверим первое условие из определения $\mathcal{T}$.
    Рассмотрим произвольные $x_0\in\mathbb{Z}[i]\setminus\{0\}$ и $\alpha\in\mathbb{F}^*_1$.
    Обозначим $b=\textrm{int}((\alpha-x_0)^{-1})$.
    Пусть $p$~-- число из первого условия определения $\mathcal{T}$.
    Если $\upsilon(x_0)>5$, то при $p=1$ имеем $\beta_1=\alpha$.
    Если $b\in\mathbb{I}\cup\{0\}$, то для $p=1$ имеем $\beta_1=\textrm{fr}((\alpha-x_0)/(bx_0+1-\alpha))$, тогда $\beta_2=\alpha$.
    Далее предполагаем, что $b\not\in\mathbb{I}\cup\{0\}$ и $\upsilon(x_0)\le 5$.
    Заметим, что $\upsilon(x_0)\in\{1,2\}$.
    Достаточно рассмотреть только случаи $x_0=1$ и $x_0=1+i$ (так как для любого такого $x_0$, что $\upsilon(x_0)\in\{1,2\}$, существует элемент такой $\varepsilon\in\mathbb{I}$, что $x_0=\varepsilon$ или $x_0=(1+i)\varepsilon$).

    Пусть $x_0=1+i$, тогда $b=-1+i$.
    Положим $p=1$, и получим
    \begin{equation*}
        \beta_1=\textrm{fr}((\alpha-(1+i))/(\alpha(1-i)-1)),\ \beta_2=\alpha^{-1}.
    \end{equation*}

    Пусть $x_0=1$, тогда $b\in\{-2,-1\pm i,-2\pm i\}$.

    Если $b=-2$, то для $p=1$ имеем
    \begin{equation*}
        \beta_1=\textrm{fr}((\alpha-1)/(2\alpha-1)),\ \beta_2=\alpha^{-1}.
    \end{equation*}

    Если $b=-1\pm i$, то для $p=1$ имеем
    \begin{equation*}
        \beta_1=\textrm{fr}((\alpha-1)/(\alpha(1\mp i)\pm i)),\ \beta_2=\alpha^{-1}.
    \end{equation*}

    Пусть $b=-2+i$. Рассмотрим элемент
    \begin{equation*}
        \beta=\omega(\textrm{fr}((\alpha-1)^{-1}))=\textrm{fr}((\alpha-1)/(\alpha(2-i)-(1-i))).
    \end{equation*}

    Заметим, что
    \begin{equation*}
        \gamma=\textrm{int}((\alpha-1)/(\alpha(2-1)-(1-i)))\in\{1+i,1+2i,2+i,2+2i\}.
    \end{equation*}

    Если $\gamma=1+i$, то
    \begin{equation*}
        \beta=(1-\alpha(2+i))/(\alpha(2-i)-(1-i)).
    \end{equation*}

    Положим $p=2$, тогда имеем
    \begin{equation*}
        \begin{array}{c}
            \beta_1=\omega^{(2)}(\textrm{fr}((\alpha-1)^{-1}))=\textrm{fr}((\alpha(2-i)-(1-i))/(1-\alpha(2+i))),\\
            \beta_2=\alpha/(1-\alpha(2+i)),\\
            \beta_3=\alpha^{-1}.
        \end{array}
    \end{equation*}

    Если $\gamma=1+2i$, то
    \begin{equation*}
        \beta=((2+i)-\alpha(3+3i))/(\alpha(2-i)-(1-i)).
    \end{equation*}

    Положим $p=2$, тогда имеем
    \begin{equation*}
        \begin{array}{c}
            \beta_1=\omega^{(2)}(\textrm{fr}((\alpha-1)^{-1}))=\textrm{fr}((\alpha(2-i)-(1-i))/((2+i)-\alpha(3+3i))),\\
            \beta_2=\alpha/(1-\alpha(2+i)),\\
            \beta_3=\alpha^{-1}.
        \end{array}
    \end{equation*}

    Если $\gamma=2+i$, то
    \begin{equation*}
        \beta=((2-i)-4\alpha)/(\alpha(2-i)-(1-i)).
    \end{equation*}

    Положим $p=2$, тогда имеем
    \begin{equation*}
        \begin{array}{c}
            \beta_1=\omega^{(2)}(\textrm{fr}((\alpha-1)^{-1}))=\textrm{fr}((\alpha(2-i)-(1-i))/((2-i)-4\alpha)),\\
            \beta_2=\alpha/(1-\alpha(2+i)),\\
            \beta_3=\alpha^{-1}.
        \end{array}
    \end{equation*}

    Если $\gamma=2+2i$, то
    \begin{equation*}
        \beta=(3-\alpha(5+2i))/(\alpha(2-i)-(1-i)).
    \end{equation*}

    Положим $p=3$, тогда имеем
    \begin{equation*}
        \begin{array}{c}
            \beta_1=\omega^{(3)}(\textrm{fr}((\alpha-1)^{-1}))=\textrm{fr}((3-\alpha(5+2i))/(2-2i-\alpha(5-2i))),\\
            \beta_2=\alpha/(1-\alpha(2_i),\\
            \beta_3=\alpha^{-1}.
        \end{array}
    \end{equation*}

    Случай $b=-2-i$ аналогичен случаю $b=-2+i$.

    Проверим второе условие определения множества $\mathcal{T}$.
    Рассмотрим произвольные $x_0\in\mathbb{Z}[i]\setminus\{0\}$ и $\alpha\in\mathbb{F}^*_1$, для которых выполнено равенство $\omega(\textrm{fr}((\alpha-x_0)^{-1}))=0$.
    Так как во всех случаях, кроме $x_0=1$, $b=-2\pm i$, можно взять $D_{\mathbb{K}}=2$ вместо $D_{\mathbb{K}}=3$, то необходимо рассмотреть только случай $x_0=1$, $b=-2\pm i$.
    Пусть $b=-2+i$, тогда из условия $\omega(\textrm{fr}((\alpha-x_0)^{-1}))=0$ следует, что $\alpha\in\{1/(2+i),(2+i)/(3+3i),(2-i)/4,3/(5+2i)\}$.
    Для первого, второго и третьего элемента выполнено $\omega^{(2)}(\alpha)=0$.
    Для четвертого элемента условие $\alpha=\textrm{fr}(\alpha)$ не выполняется.
    Случай $b=-2-i$ аналогичен случаю $b=-2+i$.
    Следовательно, $\mathbb{K}=\mathbb{Z}[i]$ не принадлежит классу $\mathcal{S}$, но принадлежит классу $\mathcal{T}$.
\end{example}

В работе Роллетчека~\cite[с.~341]{source:Rolletschek_1990} приведен пример кольца, для которого цепочка делений с выбором минимального по норме остатка не является кратчайшей.
Следовательно, это кольцо не может принадлежать классу $\mathcal{T}$.

\begin{example}\label{example:Z[sqrt{-11}]} \cite[с.~341]{source:Rolletschek_1990}
    Пусть $R = \mathcal{O}_{\mathbb{Q}(\sqrt{-11})}$.
    Заметим, что множество
    \begin{equation*}
        \{6,-2i\sqrt{11},6,-3+i\sqrt{11},-1-i\sqrt{11},2,0\}
    \end{equation*}
    образует цепочку делений с выбором минимального по норме остатка для пары $(a,b)=(6,-2i\sqrt{11})$.
    Тогда $\mathcal{L}_{a,b}=5$.
    С другой стороны существует цепочка делений
    \begin{equation*}
        \{6,-2i\sqrt{11},-5+i\sqrt{11},3+i\sqrt{11},2,0\}.
    \end{equation*}
    
    Из этого следует, что $\mathpzc{l}_{a,b}\le 4\le \mathcal{L}_{a,b}$.
    Следовательно, теорема~\ref{theorem:Kroneker_Vahlen_theorem_in_UFD} не выполняется.
    Предположим, что $\mathcal{O}_{\mathbb{Q}(\sqrt{-11})} \in \mathcal{T}$, но тогда по теореме~\ref{theorem:Kroneker_Vahlen_theorem_in_UFD} получаем, что для любой пары $(c,d)\in\mathbb{K}_* \times\mathbb{K}_*$ выполнено $\mathpzc{l}_{c,d}=\mathcal{L}_{c,d}$.
    Получаем противоречие с приведенным примером.
    Аналогично можно показать, что лемма~\ref{lemma:omega_and_euclidean_algorithm} не выполняется для $\mathcal{O}_{\mathbb{Q}(\sqrt{-11})}$.
\end{example}

% \subsection{Общий метод проверки $R \in \mathcal{T}$}

Приведем общий метод проверки включения $R \in \mathcal{T}$.

\begin{algorithm}\label{algorithm:R_in_T}
    На вход подается факториальное кольцо $R$.
    
    \begin{enumerate}
        \item Выбрать $D_R, M \in \mathbb{N}$
        
        \item Построить множество
        \begin{equation*}
            \mathbb{J} = \left\{
                x_0 \in R \Bigg| \int{\frac{1}{\alpha-x_0}} \in \invertible{R} \cup \{0\} \forall \alpha \in \zeroless{F}
            \right\}
        \end{equation*}
        
        \item Создать список $\mathbb{L}$, в котором будут храниться элементы из $R^i$, где $i \in \{1, \ldots, D_R - 1\}$

        \item Вычислить $\mathbb{L}=\mathbb{K}^*\setminus\mathbb{J}$. Мы будем хранить элементы из $\mathbb{L}$, в которых более двух компонент в множестве $\mathbb{L}_M$

        \item\label{step:every_element_in_L} Выбрать элемент $(x_0, \ldots, x_l) \in \mathbb{L}$ и удалить его из $\mathbb{L}$

        \item Вычислить
        \begin{equation*}
            \delta = \left(
                \left(
                    \ldots\left(
                        \left(
                            \alpha - x_0
                        \right)^{-1} - x_1
                    \right)^{-1} - \ldots
                \right)^{-1} - x_l
            \right)^{-1}
        \end{equation*}

        \item Построить множество
        \begin{equation*}
            \mathbb{A} = \left\{
                b \in R \big| b = \int{\delta}
            \right\}
        \end{equation*}

        \item\label{step:every_element_in_A} Для каждого элемента $x_{l+1} \in \mathbb{A}$ выполнить
        \begin{enumerate}
            \item Вычислить
            \begin{equation*}
                \beta_1 = \omega^{(l+1)} \left(
                    \int{(\alpha-x_0)^{-1}}
                \right) = \fr{(\delta-x_{l+1})^{-1}}
            \end{equation*}

            \item Попробовать найти такие $(\varepsilon_i) \in \invertible{R}$ и $(a_i), (b_i) \in R$, что $\elementnorm{a_i},\elementnorm{b_i} \le M$ и
            \begin{equation*}
                \begin{split}
                    \beta_{i+1}=\frac{1}{\varepsilon_i \beta_i + a_i} + b_i,\\
                    \beta_{l+2}=\alpha \textrm{ или } \beta_{l+2}=\alpha^{-1}
                \end{split}
            \end{equation*}

            \item Если такие элементы не нашлись и $l+1 \ge D_R - 1$, то вернуть ''выберите большие $D_R$ и $M$'' и завершить алгоритм
            
            \item Если такие элементы не нашлись и $l+1 < D_R - 1$, то добавить в множество $\mathbb{L}$ элемент $(x_0,\ldots,x_l,x_{l+1})$
            
            \item Если такие элементы нашлись, то перейти к следующему элементу в шаге~\ref{step:every_element_in_A}
        \end{enumerate}

        \item Если множество $\mathbb{L}$ не пустое, то выбрать другой элемент на шаге~\ref{step:every_element_in_L}
        
        \item Если множество $\mathbb{L}$ пустое и $D_R < 3$, то вернуть ''$R \in \mathcal{T}$'' и завершить алгоритм
        
        \item Для всех $k \in [3, D_R]$
        \begin{enumerate}
            \item Построить множество
            \begin{equation*}
                \mathbb{B} = \left\{
                    \alpha \in \zeroless{F} \big| \omega^{(k-2)}\left(
                        \fr{(\alpha-x_0)^{-1}}
                    \right) = 0
                \right\}
            \end{equation*}

            \item Для всех $\alpha_0 \in \mathbb{B}$ попробовать найти такие $(\varepsilon_i) \in \invertible{R}$ и $(a_i), (b_i) \in R$, что $\elementnorm{a_i}, \elementnorm{b_i} \le M$ и
            \begin{equation*}
                \begin{split}
                    \beta_{i+1}=\frac{1}{\varepsilon_i \beta_i + a_i} + b_i,\\
                    \beta_{k-1}=\alpha \textrm{ или } \beta_{k-1}=\alpha^{-1},
                \end{split}
            \end{equation*}
            где $\beta_1 = 0$

            \item Если такие элементы не были найдены, то вернуть ''выберите большие $D_R$ и $M$'' и завершить алгоритм
            
            \item Если такие элементы были найдены и проверены все $x_0 \in \zeroless{\mathbb{L}}$, то вернуть ''$R \in \mathcal{T}$'' и завершить алгоритм
        \end{enumerate}
    \end{enumerate}
\end{algorithm}

\begin{proposition}
    Пусть алгоритм~$\ref{algorithm:R_in_T}$ вернул ''$R \in \mathcal{T}$''.
    Тогда $R \in \mathcal{T}$.
\end{proposition}
\begin{proof}
    Рассмотрим произвольный $x_0 \in R$.
    Если $x_0 \in \mathbb{J}$, то тройка $(x_0, \alpha, D_R - 1)$ правильная для всех $\alpha \in \zeroless{F}$ и $D_R \in \mathbb{N}$.

    Рассмотрим множество $\mathbb{L}_1$ всех элементов, что добавлялись в $\mathbb{L}$.
    Если $(x_0, \ldots, x_k) \in \mathbb{L}_1$, $(y_0, \ldots, y_l) \in \mathbb{L}_1$ и $x_i = y_i$ для всех $i = \overline{1,k}$, $k \le l$, то удалим $(x_0, \ldots, x_k)$ из $\mathbb{L}_1$.

    Зафиксируем $x_0$ и покажем, что тройка $(x_0, \alpha, D_R - 1)$ регулярная.
    Заметим, что, если тройка $(x_0, \alpha, k)$ регулярная, то тройка $(x_0, \alpha, k + i)$ регулярная для всех $i \in \mathbb{N}$.

    Рассмотрим элемент $(x_0, \ldots, x_l) \in \mathbb{L}_1$.
    Этот элемент генерирует множество таких элементов $\alpha$, что
    \begin{equation*}
        x_l = \left(
            \left(
                \ldots \left(
                    \left(
                        \alpha - x_0
                    \right)^{-1} - x_1
                \right)^{-1} - \ldots
            \right)^{-1} - x_{l-1}
        \right)^{-1}.
    \end{equation*}
    Очевидно, что все такие элементы $\mathbb{L}_1$ с фиксированным $x_0$ генерируют $\zeroless{F}$.
    
    Так как ответ алгоритма ''$R \in \mathcal{T}$'', то тройка $(x_0, \alpha, D_R - 1)$ регулярная для всех $x_0, \alpha$.
    Следовательно, выполняется первая часть определения множества $\mathcal{T}$.
    
    Если $D_R \ge 3$, тот доказательство проводится аналогично.
\end{proof}

\section{Теорема Кронекера-Валена в кольце алгебраических целых элементов числового поля}

% \subsection{Метод деления с выбором минимального по норме остатка}

\begin{lemma}
    Пусть $\xi \in K$ и $k > 0$.
    Обозначим $x = \Phi(\xi)$.
    Выберем такие элементы $z' \in \textrm{Orb}(x)$ и $Z'\in\mathcal{I}_{z', k}$, что $\mathcal{M}_k = |\mathcal{N}(z'-Z')|$.
    Если $\mathcal{M}_k \le k$, то $\int{\xi}$ можно вычислить за $O(1)$ арифметических операций в $K$.
\end{lemma}
\begin{proof}
    Из утверждения~\ref{proposition:division_with_least_norm_remainder} следует, что
    \begin{equation*}
        m_{\overline{K}}(x) = \mathcal{M}_k.
    \end{equation*}
    
    Используя определение $m_{\overline{K}}(x)$ и условие этого утверждения, получаем, что
    \begin{equation*}
        \inf\limits_{z \in \mathcal{O}_K} |\mathcal{N}(x - \Phi(z))| = |\mathcal{N}(z' - Z')|.
    \end{equation*}

    Из того, что $z' \in \textrm{Orb}(x)$ следует, что существует $\varepsilon \in \invertible{\mathcal{O}_K}$ такое, что $z' = \Phi(\varepsilon)x$.
    Так как $\mathcal{N}(x) = \mathcal{N}(\Phi(\varepsilon)x)$ для всех $x \in \mathbb{R}^n$ и $\varepsilon \in \invertible{\mathcal{O}_K}$ выполнено следующее равенство
    \begin{equation*}
        \inf\limits_{z \in \mathcal{O}_K} |\mathcal(x - \Phi(z))| = |\mathcal{N}(x - Z'\Phi(\varepsilon^{-1}))|.
    \end{equation*}

    Таким образом, получаем, что $\int{\xi} = Z'\Phi(\varepsilon^{-1})$, где $\varepsilon$ задана соотношением $z' = \Phi(\varepsilon)x$.
    Следовательно, надо для $\xi \in R$ уметь вычислять $\textrm{Orb}(\Phi(\xi))$.
    Определим следующую операцию
    \begin{equation*}
        x = \sum\limits_{i=1}^n q_i e_i \longmapsto \overline{x} = \sum\limits_{i=1}^n (q_i - \lfloor q_i \rfloor)e_i.
    \end{equation*}
    
    Обозначим
    \begin{equation*}
        \mathcal{O}(\xi) = \left\{
            \overline{\varepsilon\xi} \Big| \varepsilon \in \invertible{\mathcal{O}_K}
    	\right\}.
    \end{equation*}
    
    Для каждого $1 \le i \le r$ существует такое положительное число $m$, что $\overline{\varepsilon_i}^m\overline{\xi} = \overline{\xi}$.
    Обозначим $l_i$ наименьшее такое число.
    Для $\nu$ наименьшее такое число обозначим $l$.
    Далее будем полагать, что $l$ и $l_i$ известны для $K$.
    
    С этими обозначениями, из работы \cite[с.~1413]{source:Lezowski} следует, что
    \begin{equation*}
        \mathcal{O}(\xi) = \left\{
            \overline{\nu}^m \prod\limits_{i=1}^r \overline{\varepsilon_i}^{m_i} \overline{\xi}:
                0 \le m < l, 0 \le m_i < l_i, 1 \le i \le r
        \right\}
    \end{equation*}
    и $\textrm{Orb}(\Phi(\xi)) = \{\Phi(\zeta):\zeta \in \mathcal{O}(\xi)\}$ для любого $\xi\in K$.

    Предположим, что мы знаем такие $m$ и $m_i$ что $\varepsilon = \nu^m\prod_{i=1}^r \varepsilon_i^{m_i}$.
    Так как $m$ и $m_i$ ограничены $l$ и $l_i$, то можно вычислить $\Phi(\varepsilon^{-1})$ за $O(1)$ арифметических операций.
\end{proof}

\begin{remark}
    Для произвольного $x \in \Phi(K)$ и $z \in \textrm{Orb}(x)$ будем предполагать, что известно такое $\varepsilon_z \in \invertible{\mathcal{O}_K}$, что $z = \Phi(\overline{\varepsilon_z \Phi^{-1}(x)})$, и более того можно вычислить $\varepsilon_z^{-1}$ за $O(1)$ арифметических операций.
\end{remark}

Используя доказанные выше утверждения, получаем следующий алгоритм.

\begin{algorithm}\label{algorithm:least_norm_remainder}
    Дано числовое поле $K$ и два элемента $a, b \in \mathcal{O}_K$.
    Необходимо вычислить наименьший остаток $r$ при делении $a$ на $b$.

    \begin{enumerate}
        \item Вычислить $x = \Phi(a/b) \in \Phi(K)$;
        
        \item Вычислить $\textrm{Orb}(x)$;

        \item Выбрать произвольное действительное $k > 0$;

        \item Вычислить $\Gamma(k)$ \label{loop:1};

        \item Объявить переменные $z'$ и $Z'$, которые будут инициализированы позже;

        \item Для всех $z \in \textrm{Orb}(x)$
        \begin{enumerate}
            \item Вычислить $\mathcal{I}_{z, k}$;

            \item Для всех $Z \in \mathcal{I}_{z, k}$, если $z'$ и $Z'$ не инициализированы или $\mathcal{N}(z' - Z') > \mathcal{N}(z - Z)$ положить $z' = z$ и $Z' = Z$;
        \end{enumerate}

        \item Вычислить $\mathcal{M}_k = \mathcal{N}(z' - Z')$
        
        \item Если $\mathcal{M}_k > k$, то положить $k = \mathcal{M}_k$ и перейти к шагу~\ref{loop:1}

        \item Вычислить $\int{\frac{a}{b}} = Z'\Phi((\varepsilon_z')^{-1})$

        \item Вернуть $r = a - b \int{\frac{a}{b}}$
    \end{enumerate}
\end{algorithm}

Оценим вычислительную сложность этого алгоритма.

\begin{proposition}
    Для любых $a, b \in \zeroless{\mathcal{O}_K}$ наименьший по норме остаток $r$ при делении $a$ на $b$ можно найти, используя алгоритм~$\ref{algorithm:least_norm_remainder}$, за $O(1)$ арифметических операций в $K$.
\end{proposition}
\begin{proof}
    Из того, что
    \begin{equation*}
        |\textrm{Orb}(x)| \le l\prod\limits_{i=1}^r l_i
    \end{equation*}
    следует, что множество $\textrm{Orb}(x)$ можно найти за $O(1)$ арифметических операций в $K$.
    Так же мощность множества $\mathcal{I}_{z, k}$ ограничена некоторой константой, зависящей только от $\Gamma(k)$.
    Заметим, что $\Gamma(k)$ на зависит от $\frac{a}{b}$.
    Так как функция $k \to \mathcal{M}_k$ неубывающая, из предложения~\ref{proposition:division_with_least_norm_remainder} следует, что цикл необходимо будет повторить не более двух раз.
    Таким образом, можно заключить, что наименьший по норме остаток при делении $a$ на $b$ можно найти за $O(1)$ арифметических операций в $K$.
\end{proof}

% \subsection{Метод доказательства невыполнимости теоремы Кронекера-Валена}

В этой части работы будет представлен метод автоматического доказательства невыполнимости теоремы Кронекера-Валена в кольце целых алгебраических элементов числового поля $K$.
Метод доказательства приведен в алгоритме~\ref{algorithm:kronecker_vahlen_common}.

\begin{algorithm}\label{algorithm:kronecker_vahlen_common}
    Дано числовое поле $K$.
    Требуется доказать, что теорема Кронекера-Валена не выполняется в $\mathcal{O}_K$.
    
    \begin{enumerate}
        \item Взять произвольные $a, b \in \mathcal{O}_K$;

        \item Используя алгоритм~\ref{algorithm:least_norm_remainder} вычислить цепочку делений с выбором минимального по норме остатка $\mathcal{D}_{a, b}$;

        \item Найти такое $c \in \mathcal{O}_K$, что $a = bx + c$ для некоторого $x \in \mathcal{O}_K$;

        \item Используя алгоритм~\ref{algorithm:least_norm_remainder} вычислить цепочку делений с выбором минимального по норме остатка $\mathcal{D}'_{b,c}$;

        \item Если $\textrm{len}(\mathcal{D}_{a, b}) > \textrm{len}(\mathcal{D}'_{b, c}) + 1$, то теорема Кронекера-Валена не выполняется в $K$.
    \end{enumerate}
\end{algorithm}

% \subsection{Теорема Кронекера-Валена в действительных квадратичных норменно-евклидовых кольцах}

Применим описанный выше метод в частном случае.
Предположим, что поле $K$ такое, что $\mathcal{O}_K$ действительное квадратичное норменно-евклидово кольцо.
В методе будем рассматривать $a, b \in \mathbb{Z}$.
А так же будем искать $c$ используя остаток при делении $a$ на $b$ на $\mathbb{Z}$.
Модифицированный метод приведен в алгоритме~\ref{algorithm:kronecker_vahlen_special}.

\begin{algorithm}\label{algorithm:kronecker_vahlen_special}
    Дано такое числовое поле $K$, что $\mathcal{O}_K$ действительное квадратичное норменно-евклидово кольцо
    Требуется доказать, что теорема Кронекера-Валена не выполняется в $\mathcal{O}_K$.

    \begin{enumerate}
        \item Взять произвольные $a, b \in \mathbb{Z}$;

        \item Используя алгоритм~\ref{algorithm:least_norm_remainder} вычислить цепочку делений с выбором минимального по норме остатка $\mathcal{D}_{a,b}$;

        \item Вычислить остаток $c$ при делении $a$ на $b$;

        \item Используя алгоритм~\ref{algorithm:least_norm_remainder} вычислить цепочку делений с выбором минимального по норме остатка $\mathcal{D}'_{b, c}$;

        \item Если $\textrm{len}(\mathcal{D}_{a, b}) > \textrm{len}(\mathcal{D}'_{b, c}) + 1$, то теорема Кронекера-Валена не выполняется в $K$;

        \item Вычислить $c = a \% b - b$;

        \item Используя алгоритм~\ref{algorithm:least_norm_remainder} вычислить цепочку делений с выбором минимального по норме остатка $\mathcal{D}''_{b,c}$;

        \item Если $\textrm{len}(\mathcal{D}_{a, b}) > \textrm{len}(\mathcal{D}''_{b, c}) + 1$, то теорема Кронекера-Валена не выполняется в $K$;
    \end{enumerate}
\end{algorithm}

Реализуем этот алгоритм на R и применим его для всех действительных квадратичных норменно-евклидовых колец.

\begin{theorem}\label{theorem:kronecker}
    Пусть поле $K$ такое, что $\mathcal{O}_K$ действительное квадратичное норменно-евклидово кольцо.
    Тогда теорема Кронекера-Валена не выполняется в $\mathcal{O}_K$.
\end{theorem}
\begin{proof}
    Известно, что существует ровно $16$ полей $K = \mathbb{Q}(\sqrt{d})$, удовлетворяющих условиям теоремы, со значениями $d$ из множества $\{2, 3, 5, 6, 7, 11, 13, 17, 19, 21, 29, 33, 37, 41, 57, 73\}$.
    Используя код, приведенный в приложении Б, получаем контрпримеры для всех $\mathbb{Q}(\sqrt{d})$, которые опубликованы в работе~[\ref{source:JSC_2021}]:
    \begin{itemize}
        \item $\mathcal{O}_{\mathbb{Q}(\sqrt{2})}$.
        Цепочка делений с выбором минимального по норме остатка:
        \begin{equation*}
            \{50, 29, -8, -11-8\sqrt{2}, -7-5\sqrt{2}, 0\}.
        \end{equation*}
        Более короткая цепочка делений:
        \begin{equation*}
            \{50, 29, 21, 19601-13860\sqrt{2}, 0\}.
        \end{equation*}
    
        \item $\mathcal{O}_{\mathbb{Q}(\sqrt{3})}$.
        Цепочка делений с выбором минимального по норме остатка:
        \begin{equation*}
            \{52, 38, 14, 24+14\sqrt{3}, 14+8\sqrt{3}, 0\}.
        \end{equation*}
        Более короткая цепочка делений:
        \begin{equation*}
            \{52, 38, -24, 2702-1560\sqrt{3}, 0\}.
        \end{equation*}
    
        \item $\mathcal{O}_{\mathbb{Q}(\sqrt{5})}$.
        Цепочка делений с выбором минимального по норме остатка:
        \begin{multline*}
            \{58, 39, 17101-10569\frac{1+\sqrt{5}}{2},\\
            -1974+1220\frac{1+\sqrt{5}}{2}, -377+233\frac{1+\sqrt{5}}{2}, 0\}.
        \end{multline*}
        Более короткая цепочка делений:
        \begin{equation*}
            \{58, 39, 19, 1, 0\}.
        \end{equation*}
    
        \item $\mathcal{O}_{\mathbb{Q}(\sqrt{6})}$.
        Цепочка делений с выбором минимального по норме остатка:
        \begin{equation*}
            \{50, 33, -2425+990\sqrt{6}, -9602+3920\sqrt{6}, -485+198\sqrt{6}, 0\}.
        \end{equation*}
        Более короткая цепочка делений:
        \begin{equation*}
            \{50, 33, 17, -1, 0\}.
        \end{equation*}
    
        \item $\mathcal{O}_{\mathbb{Q}(\sqrt{7})}$.
        Цепочка делений с выбором минимального по норме остатка:
        \begin{equation*}
            \{50, 23, -5355+2024\sqrt{7}, -20830+7873\sqrt{7}, 2024-765\sqrt{7}, 0\}.
        \end{equation*}
        Более короткая цепочка делений:
        \begin{equation*}
            \{50, 23, 4, -1, 0\}.
        \end{equation*}
    
        \item $\mathcal{O}_{\mathbb{Q}(\sqrt{11})}$.
        Цепочка делений с выбором минимального по норме остатка:
        \begin{equation*}
            \{50, 34, -902+272\sqrt{11}, -252+76\sqrt{11}, -398+120\sqrt{11}, 0\}.
        \end{equation*}
        Более короткая цепочка делений:
        \begin{equation*}
            \{50, 34, 16, 2, 0\}.
        \end{equation*}
    
        \item $\mathcal{O}_{\mathbb{Q}(\sqrt{13})}$.
        Цепочка делений с выбором минимального по норме остатка:
        \begin{multline*}
            \{52, 19, -355666+154451\frac{1+\sqrt{13}}{2},\\
            -128511+55807\frac{1+\sqrt{13}}{2}, -98644+42837\frac{1+\sqrt{13}}{2}, 0\}.
        \end{multline*}
        Более короткая цепочка делений:
        \begin{equation*}
            \{52, 19, -5, -1, 0\}.
        \end{equation*}
    
        \item $\mathcal{O}_{\mathbb{Q}(\sqrt{17})}$.
        Цепочка делений с выбором минимального по норме остатка:
        \begin{multline*}
            \{51, 14, 11153-4354\frac{1+\sqrt{17}}{2},\\
            3038-1186\frac{1+\sqrt{17}}{2}, 333-130\frac{1+\sqrt{17}}{2}, 0\}.
        \end{multline*}
        Более короткая цепочка делений:
        \begin{equation*}
            \{51, 14, -5, -1, 0\}.
        \end{equation*}
    
        \item $\mathcal{O}_{\mathbb{Q}(\sqrt{19})}$.
        Цепочка делений с выбором минимального по норме остатка:
        \begin{equation*}
            \{70, 29, -6194+1421\sqrt{19}, 3012-691\sqrt{19}, -170+39\sqrt{19}, 0\}.
        \end{equation*}
        Более короткая цепочка делений:
        \begin{equation*}
            \{70, 29, 12, -57799+13260\sqrt{19}, 0\}.
        \end{equation*}
    
        \item $\mathcal{O}_{\mathbb{Q}(\sqrt{21})}$.
        Цепочка делений с выбором минимального по норме остатка:
        \begin{multline*}
          \{50, 33, 36845-13200\frac{1+\sqrt{21}}{2},\\
          14738-5280\frac{1+\sqrt{21}}{2}, 7369-2640\frac{1+\sqrt{21}}{2}, 0\}.
        \end{multline*}
        Более короткая цепочка делений:
        \begin{equation*}
            \{50, 33, 17, -1, 0\}.
        \end{equation*}
    
        \item $\mathcal{O}_{\mathbb{Q}(\sqrt{29})}$.
        Цепочка делений с выбором минимального по норме остатка:
        \begin{equation*}
            \{54, 21, 201-63\frac{1+\sqrt{29}}{2}, 96-30\frac{1+\sqrt{29}}{2}, -9+3\frac{1+\sqrt{29}}{2}, 0\}.
        \end{equation*}
        Более короткая цепочка делений:
        \begin{equation*}
            \{54, 21, 12, -3, 0\}.
        \end{equation*}
    
        \item $\mathcal{O}_{\mathbb{Q}(\sqrt{33})}$.
        Цепочка делений с выбором минимального по норме остатка:
        \begin{multline*}
            \{50, 27, 58999613-17495460\frac{1+\sqrt{33}}{2},\\
            -30811543+9136706\frac{1+\sqrt{33}}{2}, -2623473+777952\frac{1+\sqrt{33}}{2}, 0\}.
        \end{multline*}
        Более короткая цепочка делений:
        \begin{equation*}
            \{50, 27, -4, -1, 0\}.
        \end{equation*}
    
        \item $\mathcal{O}_{\mathbb{Q}(\sqrt{37})}$.
        Цепочка делений с выбором минимального по норме остатка:
        \begin{multline*}
            \{51, 23, -2525+713\frac{1+\sqrt{37}}{2},\\
            471-133\frac{1+\sqrt{37}}{2}, 1027-290\frac{1+\sqrt{37}}{2}, 0\}.
        \end{multline*}
        Более короткая цепочка делений:
        \begin{equation*}
            \{51, 23, 5, -1027+290\frac{1+\sqrt{37}}{2}, 0\}.
        \end{equation*}
    
        \item $\mathcal{O}_{\mathbb{Q}(\sqrt{41})}$.
        Цепочка делений с выбором минимального по норме остатка:
        \begin{multline*}
            \{51, 33, -12093+3267\frac{1+\sqrt{41}}{2},\\
            -47628+12867\frac{1+\sqrt{41}}{2}, 454959-122910\frac{1+\sqrt{41}}{2}, 0\}.
        \end{multline*}
        Более короткая цепочка делений:
        \begin{equation*}
            \{51, 33, 18, -3, 0\}.
        \end{equation*}
    
        \item $\mathcal{O}_{\mathbb{Q}(\sqrt{57})}$.
        Цепочка делений с выбором минимального по норме остатка:
        \begin{equation*}
            \{51, 31, -11, -2, -1, 0\}.
        \end{equation*}
        Более короткая цепочка делений:
        \begin{equation*}
            \{51, 31, 20, 131+40\frac{1+\sqrt{57}}{2}, 0\}.
        \end{equation*}
    
        \item $\mathcal{O}_{\mathbb{Q}(\sqrt{73})}$.
        Цепочка делений с выбором минимального по норме остатка:
        \begin{multline*}
            \{57, 32, -7, -580996+121751\frac{1+\sqrt{73}}{2},\\
            2548249-534000\frac{1+\sqrt{73}}{2}, 0\}.
        \end{multline*}
        Более короткая цепочка делений:
        \begin{equation*}
            \{57, 32, 25, -943-250\frac{1+\sqrt{73}}{2}, 0\}.
        \end{equation*}
    \end{itemize}
    
\end{proof}

\begin{corollary}
    Пусть $R = \mathcal{O}_{\mathbb{Q}(\sqrt{d})}$~-- квадратичное норменно-евклидово кольцо.
    Теорема Кронекера-Валена выполняется в $R$ тогда и только тогда, когда $d=-1, -2, -3, -7$.
\end{corollary}
\begin{proof}
    Если $d \ge 0$, то утверждение следует из теоремы~\ref{theorem:kronecker}.
    Если $d < 0$, то утверждение следует из работы~\cite[с.~322]{source:Rolletschek_1990}.
\end{proof}

\section{Теорема Ламе в факториальных кольцах}

Ранее было показано, что при определенных условиях на кольцо цепочка делений с выбором минимального по норме остатка является кратчайшей.
Однако существуют кольца, для которых эти условия не выполняются.
В таких кольцах цепочка делений с выбором минимального по норме остатка не является кратчайшей.
Например, в работе~\cite[с.~322]{source:Rolletschek_1990} было показано, что теорема Кронекера-Валена не выполняется в кольце $\mathcal{O}_{\mathbb{Q}(\sqrt{-11})}$.
В работе~\cite[с.~483]{source:Cooke} было показано, что для колец целых алгебраических чисел с бесконечной группой единиц длина кратчайшей цепочки делений с выбором минимального по норме остатка ограничена константой.
Следовательно, важным вопросом является исследование асимптотического поведения длины кратчайшей цепочки делений.

\begin{definition}
    Пусть $d \neq 1$~-- целое число свободное от квадратов.
    Под \emph{фундаментальной областью поля} $\mathbb{Q}(\sqrt{d})$ понимают множество, заданное следующим образом
    \begin{equation*}
        F(d) = \begin{cases}
            \left(
                \left[0, \frac{1}{2}\right] \times \left[0, \frac{1}{2}\right]
            \right) \cap \left(
                \mathbb{Q} \times \mathbb{Q}
            \right), \textrm{ если } d \not\equiv 1 \pmod 4\\
            \left(
                \left[0, \frac{1}{2}\right] \times \left[0, \frac{1}{4}\right]
            \right) \cap \left(
                \mathbb{Q} \times \mathbb{Q}
            \right), \textrm{ если } d \equiv 1 \pmod 4
        \end{cases}
    \end{equation*}
\end{definition}

\begin{definition}
    Пусть $d \neq 1$~-- целое число свободное от квадратов.
    Под окрестностью точки $\lambda = \lambda_1 + \lambda_2 \sqrt{d} \in \mathcal{O}_{\mathbb{Q}(\sqrt{d})}$ радиуса $r > 0$ понимают множество заданное следующим образом
    \begin{equation*}
        U(\lambda, r) = \left\{
            q_1 + q_2 \sqrt{d} \in Q[\sqrt{d}] \big| |(q_1 - \lambda_1)^2 - d(q_2 - \lambda_2)^2| < r
        \right\}.
    \end{equation*}

    Под окрестностью множества точек $\Lambda \subseteq \mathcal{O}_{\mathbb{Q}(\sqrt{d})}$ радиуса $r > 0$ понимают множество заданное следующим образом
    \begin{equation*}
        U(\Lambda, r) = \bigcup_{\lambda \in \Lambda} U(\lambda, r).
    \end{equation*}
\end{definition}

В работе~\cite[с.~832]{source:Selfridge} доказано следующее утверждение.

\begin{statement}\label{proposition:fundamental_in_circle} \cite[с.~832]{source:Selfridge}
    Пусть $d \neq 1$~-- целое число свободное от квадратов.
    Тогда кольцо $\mathcal{O}_{\mathbb{Q}(\sqrt{d})}$ является евклидовым относительно нормы числового поля $\mathbb{Q}(\sqrt{d})$ тогда и только тогда, когда существует множество $\Lambda \subseteq \mathcal{O}_{\mathbb{Q}(\sqrt{d})}$, что
    \begin{equation*}
        F(d) \subseteq U(\Lambda, 1).
    \end{equation*}
\end{statement}

\begin{definition}\label{definition:euclidean_lambda}
    Для $m/n \in F_1$ рассмотрим функцию
    \begin{equation*}
        |m/n| = \begin{cases}
            \frac{\elementnorm{m}}{\elementnorm{n}}, & m \neq 0, (m, n) = 1\\
            0, & m = 0
        \end{cases}.
    \end{equation*}
    Обозначим
    \begin{equation*}
        \Lambda_K = \sup_{m/n \in F_1} |m/n|.
    \end{equation*}
\end{definition}

\begin{theorem}\label{theorem:euclidean_and_lambda}
    Если $R$~-- евклидово кольцо относительно нормы $\elementnorm{\cdot}$, то $\Lambda_R \in [0, 1]$.

    Если $R$~-- факториальное кольцо с мультипликативной нормой $\elementnorm{\cdot}$ и $\Lambda_R \in [0, 1)$, то $R$~-- евклидово относительно нормы $\elementnorm{\cdot}$ и $l_n(R) \le [\log_{\Lambda_R^{-1}} n] + 2$ для всех $n \in \mathbb{N}$, где $\log_{\infty} n = 0$.
\end{theorem}
\begin{proof}
    Пусть $R$~-- евклидово кольцо относительно нормы $\elementnorm{\cdot}$.
    Рассмотрим произвольный элемент $\frac{a}{b} \in \zeroless{F_1}$, где $a, b \in \zeroless{R}$, $(a, b) = 1$.
    Так как кольцо евклидово, то существуют такие $q, r \in R$, что $a = bq + r$ и $\elementnorm{r} < \elementnorm{b}$.
    Так как $\fr{\frac{a}{b}} = \frac{a}{b}$, то $\elementnorm{a} \le \elementnorm{r}$.
    Следовательно, $\left|\frac{a}{b}\right| = \frac{\elementnorm{a}}{\elementnorm{b}} \le \frac{\elementnorm{v}}{\elementnorm{b}} < 1$.
    Из этого следует, что $\Lambda_R \in [0, 1]$.

    Пусть $R$~-- факториальное кольцо и $\Lambda_R \in [0, 1)$.
    Рассмотрим произвольные $a, b \in \zeroless{K}$.
    Положим $q = \int{\frac{a}{b}}$, $r = b\fr{\frac{a}{b}}$.
    Тогда $a = bq + r$ и $\elementnorm{r} \le \Lambda_R \elementnorm{b}$ из определения $\Lambda_R$.
    Если $\elementnorm{b} > 0$, то $\elementnorm{r} \le \Lambda_R \elementnorm{b} < \elementnorm{b}$.
    Если $\elementnorm{b} = 0$, то $b$ обратимый элемент.
    Тогда получаем $q = 0$, $r = 0$ и $a = 0$.
    Следовательно, $R$~-- евклидово.
\end{proof}

\begin{theorem}
    Пусть $d \neq 1$ целое число свободное от квадратов.
    Если кольцо $\mathcal{O}_{\mathbb{Q}(\sqrt{d})}$ евклидово относительно нормы числового поля $\elementnorm{\cdot}$, то $l_n(\mathcal{O}_{\mathbb{Q}(\sqrt{d})}) = O(\log n)$.
\end{theorem}
\begin{proof}
    Обозначим
    \begin{equation*}
        E(d) = \begin{cases}
            \left[0, \frac{1}{2}\right] \times \left[0, \frac{1}{2}\right], \textrm{ если } d \not\equiv 1 \pmod 4\\
            \left[0, \frac{1}{2}\right] \times \left[0, \frac{1}{4}\right], \textrm{ если } d \equiv 1 \pmod 4
        \end{cases}
    \end{equation*}

    Для $\lambda = \lambda_1 + \lambda_2 \sqrt{d} \in \mathcal{O}_{\mathbb{Q}(\sqrt{d})}$ и $r > 0$ обозначим
    \begin{equation*}
        V(\lambda, r) = \left\{
            (x, y) \in R \times R \Big| \left|(x - \lambda_1)^2 - d(y - \lambda_2)^2\right| < r
        \right\}.
    \end{equation*}

    Для множества точек $\Lambda \subseteq \mathcal{O}_{\mathbb{Q}(\sqrt{d})}$ и $r > 0$ обозначим
    \begin{equation*}
        V(\Lambda, r) = \bigcup_{\lambda \in \Lambda} V(\lambda, r).
    \end{equation*}

    Из предложения~\ref{proposition:fundamental_in_circle} следует, что существует множество $\Lambda \subseteq \mathcal{O}_{\mathbb{Q}(\sqrt{d})}$, что $F(d) \subseteq U(\Lambda, 1)$.
    Докажем, что $E(d) \subseteq V(\Lambda, 1)$.

    Предположим, что $(x_0, y_0) \in E(d) \setminus V(\Lambda, 1)$.
    Так как $F(d) \setminus U(\Lambda, 1) = \emptyset$, то или $x_0 \not\in \mathbb{Q}$, или $y_0 \not\in \mathbb{Q}$.

    Предположим, что $x_0 \not\in \mathbb{Q}$.
    Тогда существует $\varepsilon_x > 0$ и $x \in [x_0 - \varepsilon_x, x_0 + \varepsilon_x] \cap \mathbb{Q}$, что $(x, y_0) \in E(d) \setminus V(\Lambda, 1)$.
    Следовательно, можно считать, что $x_0 \in \mathbb{Q}$ и $y_0 \not\in \mathbb{Q}$.
    Тогда существует $\varepsilon_y > 0$ и $y \in [y_0 - \varepsilon_y, y_0 + \varepsilon_y] \cap \mathbb{Q}$, что $(x_0, y) \in E(d) \setminus V(\Lambda, 1)$.
    Получаем противоречие с условием $F(d) \setminus U(\Lambda, 1) = \emptyset$.
    Следовательно, выполнено $E(d) \subseteq V(\Lambda, 1)$.

    Докажем, что существует $r_0 < 1$, что $F(d) \subseteq U(\Lambda, r_0)$.
    Используя рассуждения выше, заметим, что для этого достаточно показать, что $E(d) \subseteq V(\Lambda, r_0)$ для некоторого $r_0 < 1$.
    
    Множество $E(d)$ замкнутое.
    Следовательно, существует такая точка $(x_0, y_0) \in E(d)$, что
    \begin{equation*}
        r_{x_0} = \min_{\substack{
            \lambda \in V(\Lambda, 1),\\
            \lambda = \lambda_1 + \lambda_2 \sqrt{d}
        }} \left|
            (x_0 - \lambda_1)^2 - d(y_0 - \lambda_2)^2
        \right| = \min_{\substack{
            x \in E(d),\\
            \lambda \in V(\Lambda, 1),\\
            \lambda = \lambda_1 + \lambda_2 \sqrt{d}
        }} \left|
            (x - \lambda_1)^2 - d(y - \lambda_2)^2
        \right|.
    \end{equation*}
    Заметим, что $r_{x_0} < 1$, так как $E(d) \subseteq V(\Lambda, 1)$.

    Следовательно, существует $r_0 < 1$, что $F(d) \subseteq U(\Lambda, r_0)$.
    Рассмотрим произвольные $\gamma, \delta \in \mathcal{O}_{\mathbb{Q}(\sqrt{d})}$, $\delta \neq 0$.
    Из доказанного выше следует, что существует $q \in \mathcal{O}_{\mathbb{Q}(\sqrt{d})}$, что $\elementnorm{\frac{\gamma}{\delta} - q} \le r_0$.
    Так как $\elementnorm{\frac{\gamma}{\delta} - q} = \frac{\elementnorm{\gamma - q\delta}}{\elementnorm{\delta}}$, то получаем, что существуют $q, r \in \mathcal{O}_{\mathbb{Q}(\sqrt{d})}$, что $\gamma = q\delta + r$ и $\elementnorm{r} \le r_0 \elementnorm{\delta}$.
    Следовательно, $\Lambda_{\mathcal{O}_{\mathbb{Q}(\sqrt{d})}} \le r_0$.

    Тогда, из теоремы~\ref{theorem:euclidean_and_lambda} следует, что $l_n(\mathcal{O}_{\mathbb{Q}(\sqrt{d})}) \le [\log_{r_0^{-1}} n] + 2$.
\end{proof}

\section*{Выводы по главе~\ref{chapter:Kronecker-Vahlen theorem}}
\addcontentsline{toc}{section}{Выводы по главе~\ref{chapter:Kronecker-Vahlen theorem}}

В данной главе диссертации получены достаточные условия для выполнения аналога теоремы Кронекера-Валена в факториальном кольце.
Разработан метод проверки этого достаточного условия.
Разработан метод автоматического доказательства невыполнимости теоремы Кронекера-Валена в факториальном кольце.
Используя этот метод доказано, что теорема Кронекера-Валена не выполняется во всех действительных квадратичных норменно-евклидовых кольцах.
Получены оценки длины цепочки делений с выбором минимального по норме остатка в евклидовых кольцах.

Результаты, полученные в данной главе опубликованы в работах [\ref{source:Vestnik_BSU_2013}; \ref{source:JSC_2016}; \ref{source:JSC_2021}; \ref{source:Republican_Scientific_Conference_of_Students_and_Postgraduates_2013}].

\onlyinsubfile{
    \subfile{_10_bibliography}
}

\end{document}
