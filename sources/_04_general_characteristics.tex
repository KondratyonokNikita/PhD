\documentclass[_00_dissertation.tex]{subfiles}
\begin{document}

\onlyinsubfile{
    \renewcommand{\contentsname}{ОГЛАВЛЕНИЕ}
    \setcounter{tocdepth}{3}
    \tableofcontents
}

% =========================
%      РУССКИЙ
% =========================
\addcontentsline{toc}{chapter}{ОБЩАЯ ХАРАКТЕРИСТИКА РАБОТЫ}

\begin{center}
    \chapter*{ОБЩАЯ ХАРАКТЕРИСТИКА РАБОТЫ}
\end{center}

\section*{Связь работы с научными программами (проектами), темами}

Исследования проводились в рамках следующих научных проектов:
\begin{itemize}
    \item 
\end{itemize}

\section*{Цель и задачи исследования}

Целью диссертации является получение приводящих к эффективным алгоритмам критериев простоты идеалов в дедекиндовых кольцах, исследовании свойств алгоритма Евклида в неевклидовых кольцах и доказательстве накладывающих необходимые условия на параметры RSA-криптосистемы для обеспечения ее криптографической устойчивости.

\section*{Объект и предмет исследования}

Объектом исследование являются идеалы дедекиндовых колец, алгоритм Евклида и RSA-криптосистема.
Предметом исследования являются алгебраическая теория чисел.

\section*{Научная новизна}

Полученные в диссертации результаты являются обобщениями известных результатов.
Доказаны аналоги критериев простоты Эйлера и Миллера, показано, что на основе этих аналогов строится эффективный алгоритм проверки на простоту.
Доказаны аналоги теоремы Кронекера-Валена и Ламе об экстремальных свойствах алгоритма Евклида в факториальных кольцах.
Получены необходимые условия для параметров аналога RSA-криптосистемы в дедекиндовых кольцах.

\section*{Положения, выносимые на защиту}

Критерии простоты идеалов в дедекиндовых кольцах, аналогичные критериям Эйлера и Миллера в кольце целых чисел.

Теоремы об экстремальных свойствах алгоритма Евклида в факториальных кольцах, аналогичные теореме Кронекера-Валена и Ламе в кольца целых чисел.

Теоремы накладывающие необходимые условия на параметры аналога RSA-криптосистемы в дедекиндовых кольцах для обеспечения устойчивости к взлому.

\section*{Личный вклад соискателя ученой степени}

Работы [\ref{source:JNT_2016}, \ref{source:Collection_of_articles_by_laureates_2018}, \ref{source:Collection_of_articles_by_laureates_2017}, \ref{source:NANB_2017}, \ref{source:XII_Belarussian_math_conference_2016}], написанные в соавторстве с Н.П.~Прохоровым и научным руководителем М.М.~Васьковским, посвящены исследованию простых элементов и идеалов в кольцах.
Автору диссертации принадлежит разработка подхода, позволяющего переносить доказательства критериев простоты на различные классы колец.
Н.П.~Прохорову принадлежит применение этих подходов для конкретных классов колец, указанных в работе.
Научному руководителю М.М.~Васьковскому принадлежит постановка задачи, выбор методов исследования и обсуждение полученных результатов.

Работы [\ref{source:JSC_2016}, \ref{source:JSC_2021}, \ref{source:Vestnik_BSU_2013}, \ref{source:Republican_Scientific_Conference_of_Students_and_Postgraduates_2013}], написанные в соавторстве с научным руководителем М.М.~Васьковским, посвящены исследованию экстремальных свойств алгоритма Евклида в различных  классах колец.
Результаты, полученные  в этих работах получены  автором самостоятельно.
Научному руководителю М.М.~Васьковскому принадлежит постановка задачи, выбор методов исследования и обсуждение полученных результатов.

Работы [\ref{source:XIII_Belarussian_math_conference_2021}, \ref{source:BSU_Journal_2020}, \ref{source:Algebra_and_theory_of_algorithms}, \ref{source:NANB_2015}], написанные в соавторстве с научным руководителем М.М.~Васьковским, посвящены исследованию свойств аналогов RSA-криптосистемы в различных кольцах.
Результаты, полученные  в этих работах получены  автором самостоятельно.
Научному руководителю М.М.~Васьковскому принадлежит постановка задачи, выбор методов исследования и обсуждение полученных результатов.

\section*{Апробация диссертации и информация об использовании ее результатов}

Результаты работы докладывались и обсуждались на международной научной конференции ''XIII Белорусская математическая конференция'' (Минск, 2021), на всероссийской конференции ''Алгебра и теория алгоритмов'', посвященной 100-летию факультета математики и компьютерных наук Ивановского государственного университета (Иваново, 2018), на международной научной конференции ''XII Белорусская математическая конференция'' (Минск, 2016), на XVI Республиканской научной конференции студентов и аспирантов (Гомель, 2013).

Результаты, включенные в диссертацию, отмечены дипломом лауреата XXVI республиканского конкурса научных работ студентов (2020), стипендией Президенра Республики Беларусь (2020), дипломом лауреата конкурса факультета прикладной математики и информатики на лучшую студенческую научную работу за 2018 год, дипломом 1-й категории XXV республиканского конкурса научных работ студентов (2019), второй премией Специального фонда Президента Республики Беларусь по социальной поддержк одаренных учащихся и студентов (2019), дипломом 1-й категории XXIV республиканского конкурса научных работ студентов (2018), второй премией Специального фонда Президента Республики Беларусь по социальной поддержк одаренных учащихся и студентов (2018), дипломом 1-й премией конкурса факультета прикладной математики и информатики на лучшую студенческую научную работу за 2017 год, грамотой за конкурс лучших научных работ студентов БГУ за 2017 год, дипломом лауреата конкурса факультета прикладной математики и информатики на лучшую студенческую научную работу за 2016 год.

Результаты диссертации внедрены в учебный процесс БГУ, что подтверждается $1$ актом о внедрении.

\section*{Публикации результатов диссертации}

Основные результаты диссертационного исследования опубликованы в $14$ научных работах, в том числе: $4$ статьи в научных журналах в соответствии с Положением ВАК о присуждении ученых степеней и присвоении ученых званий в Республике Беларусь и $3$ статьи в международных журналах (общим объемом $4$ авторских листа), $7$ статей в сборниках трудов конференций.
% добавил уже работу с csist 2022

\section*{Структура и объем диссертации}

Диссертация состоит из перечня условных обозначений, введения, общей характеристики работы, основной части, включающей $4$ главы, заключения, библиографического списка, $2$ приложения, содержащие необходимые подробности неоторых результатов и приложения, содержащего документы о практическом применении результатов диссертации.
Полный объем диссертации -- $\pageref{LastPage}$ страниц, библиографический список содержит $64$ наименования, включая собственные публикации автора, на $5$ страницах, приложения занимают $15$ страниц.

\onlyinsubfile{
    \subfile{_10_bibliography}
}

\end{document}
