\documentclass[_00_dissertation.tex]{subfiles}
\begin{document}

\onlyinsubfile{
    \renewcommand{\contentsname}{ОГЛАВЛЕНИЕ}
    \setcounter{tocdepth}{3}
    \tableofcontents
}

\chapter*{ОБЩАЯ ХАРАКТЕРИСТИКА РАБОТЫ}
\addcontentsline{toc}{chapter}{ОБЩАЯ ХАРАКТЕРИСТИКА РАБОТЫ}

\begin{center}
\textbf{Связь работы с научными программами (проектами), темами}
\end{center}

Исследования проводились в рамках следующих научных проектов:
\begin{itemize}
    \item Анализ асимптотических свойств решений дифференциальных и алгебраических систем (2016--2020 гг., номер госрегистрации 20162496);

    \item Анализ общих и асимптотических свойств решений стохастических дифференциальных уравнений с приложениями в криптографии и теории кредитных рисков (2021--2025 гг., номер госрегистрации 20213106)
\end{itemize}

\begin{center}
\textbf{Цель и задачи исследования}
\end{center}

Целью диссертации является доказательство новых критериев простоты идеалов в дедекиндовых кольцах и исследование свойств теоретико-числовых и криптографических алгоритмов в дедекиндовых кольцах.

\begin{center}
\textbf{Объект и предмет исследования}
\end{center}

Объектом исследования являются простые идеалы в дедекиндовых кольцах.
Предметом исследования являются арифметические свойства идеалов в дедекиндовых кольцах

\begin{center}
\textbf{Научная новизна}
\end{center}

В диссертации доказаны новые критерии простоты идеалов в дедекиндовых кольцах, обобщающие известные критерии Миллера и Эйлера, которые являются основой для построения эффективных алгоритмов тестирования простоты; разработаны новые методы исследования экстремальных свойств алгоритма Евклида в факториальных кольцах, на основе которых доказаны аналоги теорем Кронекера-Валена и Ламе; впервые исследованы свойства аналога криптосистемы RSA в дедекиндовых кольцах и найдены необходимые условия криптостойкости аналога криптосистемы RSA в дедекиндовых кольцах.

\pagebreak
\begin{center}
\textbf{Положения, выносимые на защиту}
\end{center}

\begin{enumerate}
    \item Аналоги критериев Миллера и Эйлера простоты идеалов в дедекиндовых кольцах с конечной нормой.
    
    \item Аналоги теорем Кронекера-Валена и Ламе в факториальных кольцах.
    
    \item Необходимые условия криптографической стойкости аналога криптосистемы RSA в дедекиндовых кольцах.
\end{enumerate}

\begin{center}
\textbf{Личный вклад соискателя ученой степени}
\end{center}

Работы [\ref{source:JNT_2016}, \ref{source:CSIST_2022}, \ref{source:Collection_of_articles_by_laureates_2018}, \ref{source:Collection_of_articles_by_laureates_2017}, \ref{source:NANB_2017}, \ref{source:XII_Belarussian_math_conference_2016}], написанные в соавторстве с Н.П.~Прохоровым и научным руководителем М.М.~Васьковским, посвящены исследованию простых элементов и идеалов в кольцах.
Автору диссертации принадлежит разработка подхода, позволяющего переносить доказательства критериев простоты на различные классы колец.
Н.П.~Прохорову принадлежит применение этих подходов для конкретных классов колец, указанных в работе.
Научному руководителю М.М.~Васьковскому принадлежит постановка задачи, выбор методов исследования и обсуждение полученных результатов.

Работы [\ref{source:JSC_2016}, \ref{source:JSC_2021}, \ref{source:Vestnik_BSU_2013}, \ref{source:Republican_Scientific_Conference_of_Students_and_Postgraduates_2013}], написанные в соавторстве с научным руководителем М.М.~Васьковским, посвящены исследованию экстремальных свойств алгоритма Евклида в различных  классах колец.
Результаты, полученные  в этих работах получены  автором самостоятельно.
Научному руководителю М.М.~Васьковскому принадлежит постановка задачи, выбор методов исследования и обсуждение полученных результатов.

Работы [\ref{source:XIII_Belarussian_math_conference_2021}, \ref{source:BSU_Journal_2020}, \ref{source:Algebra_and_theory_of_algorithms}, \ref{source:NANB_2015}], написанные в соавторстве с научным руководителем М.М.~Васьковским, посвящены исследованию свойств аналогов криптосистемы RSA в различных кольцах.
Результаты, полученные  в этих работах получены  автором самостоятельно.
Научному руководителю М.М.~Васьковскому принадлежит постановка задачи, выбор методов исследования и обсуждение полученных результатов.

\begin{center}
\textbf{Апробация диссертации и информация об использовании ее результатов}
\end{center}

Результаты работы докладывались и обсуждались на международном конгрессе по информатике CSIST'2022 (Минск, 2022), международной научной конференции ''XIII Белорусская математическая конференция'' (Минск, 2021), на всероссийской конференции ''Алгебра и теория алгоритмов'', посвященной 100-летию факультета математики и компьютерных наук Ивановского государственного университета (Иваново, 2018), на международной научной конференции ''XII Белорусская математическая конференция'' (Минск, 2016), на международной конференции ICYS (Измир, 2015), на международном конкурсе Intel ISEF (Лос-Анджелес, 2014), на международной конференции EUCYS 2014 (Варшава, 2014), на XVI Республиканской научной конференции студентов и аспирантов (Гомель, 2013), на балтийском научно-инженерном конкурсе (2015, 2014, 2013), на республиканском конкурсе работ исследовательского характера (2015, 2014, 2013).

Результаты, включенные в диссертацию, отмечены дипломом лауреата XXVI республиканского конкурса научных работ студентов (2020), стипендией Президента Республики Беларусь (2020), дипломом лауреата конкурса факультета прикладной математики и информатики на лучшую студенческую научную работу за 2018 год, дипломом 1-й категории XXV республиканского конкурса научных работ студентов (2019), второй премией Специального фонда Президента Республики Беларусь по социальной поддержке одаренных учащихся и студентов (2019), дипломом 1-й категории XXIV республиканского конкурса научных работ студентов (2018), второй премией Специального фонда Президента Республики Беларусь по социальной поддержке одаренных учащихся и студентов (2018), дипломом 1-й премией конкурса факультета прикладной математики и информатики на лучшую студенческую научную работу за 2017 год, грамотой за конкурс лучших научных работ студентов БГУ за 2017 год, дипломом лауреата конкурса факультета прикладной математики и информатики на лучшую студенческую научную работу за 2016 год, Mu Alpha Theta Award за работу на BelSEF-2014 (2014), дипломами 1 степени за успешное выступление на республиканском конкурсе работ исследовательского характера (2015, 2014, 2013), дипломами 1 и 3 степени на научно-инженерном конкурсе учащихся BelSEF (2015, 2014, 2013), дипломами 1 степени и дипломами лауреата на балтийском научно-инженерном конкурсе (2015, 2014, 2013).

Результаты диссертации внедрены в учебный процесс БГУ, что подтверждается одним актом о практическом использовании результатов исследования в образовательном процессе \textnumero 24/14 от 26.01.2021 г., который приведен в приложении А.

\pagebreak
\begin{center}
\textbf{Опубликованность результатов диссертации}
\end{center}

Основные научные результаты диссертационного исследования опубликованы в $14$ научных работах, в том числе: $7$ статьях в научных журналах в соответствии с Положением ВАК о присуждении ученых степеней и присвоении ученых званий в Республике Беларусь и зарубежных научных журналах (общим объемом $4$ авторских листа); $7$ статьях в сборниках трудов конференций.

\begin{center}
\textbf{Структура и объем диссертации}
\end{center}\mbox{}\\

Диссертация состоит из перечня условных обозначений, введения, общей характеристики работы, основной части, включающей $4$ главы, заключения, библиографического списка, приложений, включающих код программы, использующийся в работе, и документы о практическом применении результатов диссертации.
Полный объем диссертации~--- $102$ страницы, библиографический список содержит $67$ наименований, включая собственные публикации автора, на $6$ страницах, приложения занимают $14$ страниц.

\onlyinsubfile{
    \subfile{_10_bibliography}
}

\end{document}
