\documentclass[_00_dissertation.tex]{subfiles}
\begin{document}

\onlyinsubfile{
    \renewcommand{\contentsname}{ОГЛАВЛЕНИЕ}
    \setcounter{tocdepth}{3}
    \tableofcontents
}

\newpage
\begin{center}
    \refstepcounter{section}
    \section*{ГЛАВА \arabic{section}.\\ ТЕСТИРОВАНИЕ ИДЕАЛОВ НА ПРОСТОТУ В ДЕДЕКИНДОВЫХ КОЛЬЦАХ}\label{ch:Primality}
    \addcontentsline{toc}{chapter}{ГЛАВА \arabic{section}. ТЕСТИРОВАНИЕ ИДЕАЛОВ НА ПРОСТОТУ В ДЕДЕКИНДОВЫХ КОЛЬЦАХ}
\end{center}

\subsection{Аналог критерия Эйлера}

\begin{theorem}\label{theorem:euler_criteria}
    Пусть $\ideal{n}$ -- нетривиальный идеал нечетной нормы дедекиндового кольца $R$.
    Тогда $\ideal{n}$ -- простой идеал тогда и только тогда, когда для любого идеала $\ideal{a} \in \multiplicative{(R/\ideal{n})}$ выполнено
    \begin{equation*}
        a^{\frac{\Nm{\ideal{n}} - 1}{2}} \equiv \left[\frac{\ideal{a}}{\ideal{n}}\right] \pmod{\ideal{n}}.
    \end{equation*}
\end{theorem}
\begin{proof}
    Предположим, что $\ideal{n}$ -- простой идеал.
    Рассмотрим произвольный идеал $\ideal{a} \in \multiplicative{(R/\ideal{n})}$.
    Пусть $g$ -- первообразный корень $\multiplicative{(R/\ideal{n})}$.

    Так как $\Nm{\ideal{n}}$ нечетный, то $\ideal{a}$ является квадратичным вычетом тогда и только тогда, когда существует такое $t' = 2t \in \{0, 2, \dots, \Nm{\ideal{n}} - 1\}$, что $a \equiv g^{t'} \pmod{\ideal{n}}$.
    Так как порядок $g$ равен $\Nm{\ideal{n}} - 1$, то последнее сравнение выполняется тогда и только тогда, когда $a^{\frac{\Nm{\ideal{n}} - 1}{2}} \equiv 1 \pmod{\ideal{n}}$.
    Это завершает доказательство необходимости.

    Предположим, что $\ideal{n}$ -- не простой идеал.
    Пусть $\ideal{n}$ раскладывается в произведение простых идеалов следующим образом $\ideal{n} = \prod_{i=1}^{r} \ideal{p}_i^{\alpha_i}$.
    Так как норма простого идеала примарная, то обозначим $\Nm{\ideal{p}_i} = q_i^{f_i}$, где $q_i$ -- простой в $\mathbb{Z}$.
    Пусть для любого $\ideal{a} \in \multiplicative{(R/\ideal{n})}$ выполнено $a^{\frac{\Nm{\ideal{n}} - 1}{2}} \equiv \left[\frac{\ideal{a}}{\ideal{n}}\right] \pmod{\ideal{n}}$.

    Пусть существует такой $j \in \{1, \dots, r\}$, что $\alpha_j > 1$ в разложении $\ideal{n}$ на множители.
    Из теоремы Коши для групп \ref{statement:cauchy} и свойств функции Эйлера \ref{statement:euler_function} следует, что существует $\ideal{a} \in \multiplicative{(R/\ideal{n})}$ порядка $q_j$.
    Тогда $q_j | \Nm{\ideal{n}} - 1$, что невозможно.

    Следовательно, $\alpha_j = 1$ для любого $j \in \{1, \ldots, r\}$.
    Так как $\ideal{n}$ -- составное, то $r \ge 2$.
    Рассмотрим произвольный квадратичный невычет $\ideal{b} \in \multiplicative{(R/\ideal{p}_1)}$.
    Согласно аналогу Китайской теоремы об остатках \ref{statement:chinese_remainder_theorem} существует такой $\ideal{a} \in \multiplicative{(R/\ideal{n})}$, что выполнено $a \equiv b \pmod{\ideal{p}_1}$ и $a \equiv 1 \pmod{\ideal{p}_2\dots\ideal{p}_r}$.
    Но в этом случае $\left[\frac{\ideal{a}}{\ideal{n}}\right] = -1$.
    Из условия теоремы следует, что $a^{\frac{\Nm{\ideal{n}} - 1}{2}} \equiv -1 \pmod{\ideal{n}}$, что противоречит условию $a \equiv 1 \pmod{\ideal{p}_2}$.
    Это завершает доказательство достаточности.

    Теорема доказана.
\end{proof}

\begin{algorithm}\label{algorithm:solovay_strassen}
    Дан нетривиальный идеал $\ideal{n} \subset R$.
    Необходимо определить является ли он простым.

    \begin{enumerate}
        \item Вычислить $\Nm{\ideal{n}}$;
        
        \item Выбрать случайное $\ideal{a} \subset \multiplicative{(R/\ideal{n})}$;

        \item Вычислить $r_0 = \ideal{a}^{\frac{\Nm{\ideal{n}} -- 1}{2}} \pmod{\ideal{n}}$;

        \item Вычислить $r_1 = \left[\frac{\ideal{a}}{\ideal{n}}\right]$;

        \item Если $r_0 \equiv r_1 \pmod{\ideal{n}}$, то вернуть ''неизвестно'' и завершить алгоритм;

        \item Вернуть ''$\ideal{n}$ не простой'' и завершить алгоритм.
    \end{enumerate}
\end{algorithm}

\begin{remark}
    Алгоритм \ref{algorithm:solovay_strassen} является вероятностным.
    Если был получен ответ "неизвестно", то можно выполнить алгоритм еще раз.
\end{remark}

\begin{proposition}
    Пусть $\ideal{n}$ -- не простой идеал.
    Тогда вероятность ответа "$\ideal{n}$ не простой" у алгоритма \ref{algorithm:solovay_strassen} не менее $1/2$.
\end{proposition}
\begin{proof}
    Рассмотрим множество
    \begin{equation*}
        G = \left\{
            \ideal{a} \in \multiplicative{(R/\ideal{n})} \big| \ideal{a}^{\frac{\Nm{\ideal{n}} - 1}{2}} \equiv \left[\frac{\ideal{a}}{\ideal{n}}\right] \pmod{\ideal{n}}
        \right\}.
    \end{equation*}
    Алгоритм~\ref{algorithm:solovay_strassen} возвращает ответ ''неизвестно'' только для элементов из множества $G$.

    Заметим, что если алгоритм~\ref{algorithm:solovay_strassen} возвращает ответ ''неизвестно'' для $\ideal{a}$ и $\ideal{b}$, то он вернет ответ ''неизвестно'' и для $\ideal{a}\ideal{b}$.
    Следовательно, $G$ образует подгруппу группы $\multiplicative{(R/\ideal{n})}$.

    Исходя из критерия Эйлера эта подгруппа собственная.
    Из теоремы Лагранжа~\ref{statement:lagrange} выполнено $|G|/|\multiplicative{(R/\ideal{n})}| \le \frac{1}{2}$.
\end{proof}

\begin{remark}
    Если $\ideal{n}$ -- составной, то при выполнении алгоритма \ref{algorithm:miller_rabin} $k$ раз вероятность получить ответ ''$\ideal{n}$ не простой'' не меньше $1 - \frac{1}{2^k}$.
\end{remark}

\subsection{Аналог критерия Миллера}

\begin{theorem}\label{theorem:miller_criteria}
    Пусть $\ideal{n}$ -- нетривиальный идеал нечетной нормы дедекиндового кольца $R$.
    Пусть $\Nm{\ideal{n}} - 1 = 2^t u$, $(u, 2) = 1$.
    Тогда $\ideal{n}$ -- простой идеал тогда и только тогда, когда для любого идеала $\ideal{a} \in \multiplicative{(R/\ideal{n})}$, $(\ideal{a}, \ideal{n}) = 1$, $\ideal{a}^u \not\equiv 1 \pmod{\ideal{n}}$ существует $k\in \{0, \dots, t-1\}$, такое что $\ideal{a}^{2^{k}u} \equiv -1 \pmod{\ideal{n}}$.
\end{theorem}
\begin{proof}
    Предположим, что $\ideal{n}$ -- простой идеал.
    Рассмотрим произвольный идеал $\ideal{a} \in \multiplicative{(R/\ideal{n})}$, $(\ideal{a}, \ideal{n}) = 1$, $\ideal{a}^u \not\equiv 1 \pmod{\ideal{n}}$.
    Из теоремы Эйлера \ref{statement:euler_function} следует, что:
    \begin{equation*}
        \ideal{a}^{2^{t} u} = \ideal{a}^{\varphi(\ideal{n})} \equiv 1 \pmod{\ideal{n}}
    \end{equation*}

    Раскладываем на множители и получаем, что выполнено
    \begin{equation*}
        (\ideal{a}^{u} - 1)(\ideal{a}^{u} + 1)(\ideal{a}^{2u} + 1)\dots(\ideal{a}^{2^{t-1}u} + 1) \equiv 0 \pmod{\ideal{n}}
    \end{equation*}

    Из того, что $\ideal{a}^{u} \not\equiv 1 \pmod{\ideal{n}}$ следует, что $\ideal{a}^{2^{k}u} + 1 \equiv 0 \pmod{\ideal{n}}$ для некоторого $k\in \{0, \dots, t-1\}$.
    Это завершает доказательство необходимости.

    Предположим, что $\ideal{n}$ -- не простой идеал.
    Пусть $\ideal{n}$ раскладывается в произведение простых идеалов следующим образом $\ideal{n} = \prod_{i=1}^{r} \ideal{p}_i^{\alpha_i}$.
    Так как норма простого идеала примарная, то обозначим $\Nm{\ideal{p}_i} = q_i^{f_i}$, где $q_i$ -- простой в $\mathbb{Z}$.

    Пусть существует такой $j \in \{1, \dots, r\}$, что $\alpha_j > 1$ в разложении $\ideal{n}$ на множители.
    Из теоремы Коши для групп \ref{statement:cauchy} и свойств функции Эйлера~\ref{statement:euler_function} следует, что существует $\ideal{a} \in \multiplicative{(R/\ideal{n})}$ порядка $q_j$.
    Так как $u \not\equiv 0 \pmod{q_j}$, то $\ideal{a}^u \not\equiv 1 \pmod{\ideal{n}}$.
    Следовательно, существует число $k \in \{1, \dots, t-1\}$, такое что выполнено сравнение $\ideal{a}^{2^{k}u} \equiv -1 \pmod{\ideal{n}}$.
    Тогда $\ideal{a}^{2^{k+1}u} \equiv 1 \pmod{\ideal{n}}$.
    Значит выполнено $2^{k+1}u \equiv 0 \pmod{q_j}$.
    Из последнего сравнения следует, что $\Nm{\ideal{n}} - 1 \equiv 0 \pmod{q_j}$, что невозможно.
    
    Следовательно, $\alpha_j = 1$ для любого $j \in \{1, \ldots, r\}$.
    Так как $\ideal{n}$ -- составное, то $r \ge 2$.
    Из аналога Китайской теоремы об остатках~\ref{statement:chinese_remainder_theorem} и того, что элемент $-1$ имеет порядок $2$ в каждой группе $\multiplicative{(R/\ideal{p}_j)}$ следует, что существует по крайней мере $2^r-1 \ge 3$ элемента $\multiplicative{(R/\ideal{n})}$ порядка $2$.
    Пусть $\ideal{a} \not\equiv \pm 1 \pmod{\ideal{n}}$ является произвольным элементом порядка $2$ в группе $\multiplicative{(R/\ideal{n})}$.
    Из того, что $(u, 2) = 1$ следует, что $\ideal{a}^u \equiv \ideal{a} \not\equiv \pm 1 \pmod{\ideal{n}}$.
    Таким образом, существует $k \in \{0,\ldots, t-1\}$, такое что верно $\ideal{a}^{2^{k}u} \equiv -1 \pmod{\ideal{n}}$.
    Это противоречит тому, что порядок $\ideal{a}$ равен $2$.
    Это завершает доказательство достаточности.

    Теорема доказана.
\end{proof}

\begin{algorithm}\label{algorithm:miller_rabin}
    Дан идеал $\ideal{n} \subset R$.
    Необходимо определить является ли он простым.

    \begin{enumerate}
        \item Найти $u, t \in \mathbb{N}$, что $\Nm{\ideal{n}} - 1 = 2^t u$ и $(2, u) = 1$;
        
        \item Выбрать случайное $\ideal{a} \subset \multiplicative{(R/\ideal{n})}\setminus\{0\}$;

        \item Вычислить $r_0 = \ideal{a}^u \pmod{\ideal{n}}$;

        \item Если $r_0 = 1$, то вернуть ''неизвестно'' и завершить алгоритм;

        \item Положить $k = 0$;

        \item Пока $k < t$ выполнять:
        \begin{enumerate}
            \item Если $r_k = -1$, то вернуть ''неизвестно'' и завершить алгоритм;

            \item Увеличить $k$ на $1$;

            \item Вычислить $r_{k+1} \equiv r_k^2 \pmod{\ideal{n}}$;
        \end{enumerate}

        \item Вернуть ''$\ideal{n}$ не простой'' и завершить алгоритм.
    \end{enumerate}
\end{algorithm}

\begin{remark}
    Алгоритм \ref{algorithm:miller_rabin} является вероятностным.
    Если был получен ответ "неизвестно", то можно выполнить алгоритм еще раз.
\end{remark}

\begin{proposition}
    Пусть $\ideal{n}$ -- не простой идеал.
    Тогда вероятность ответа "$\ideal{n}$ не простой" у алгоритма \ref{algorithm:miller_rabin} не менее $1/2$.
\end{proposition}
\begin{proof}
    Рассмотрим множество всех $\ideal{a}$, для которых алгоритм дает ответ "неизвестно".
    Это в точности множество таких $\ideal{a}$, что $\ideal{a}^u \equiv 1 \pmod{\ideal{n}}$ или для которых существует $j \in \{0, \dots, t-1\}$, что $\ideal{a}^{2^{j}u} \equiv -1 \pmod{\ideal{n}}$.
    Из этого следует, что $\ideal{a}^{\Nm{\ideal{n}} - 1} \equiv 1 \pmod{\ideal{n}}$ для всех таких $\ideal{a}$.

    Рассмотрим множество
    \begin{equation*}
        G = \left\{
            \ideal{a} \in \multiplicative{(R/\ideal{n})} \big| \ideal{a}^{\Nm{\ideal{n}} - 1} \equiv 1 \pmod{\ideal{n}}
        \right\}.
    \end{equation*}
    Предположим, что $G$ -- нетривиальная подгруппа.
    Тогда из теоремы Лагранжа следует, что $|G| / |\multiplicative{(R/\ideal{n})}| \le 1/2$.
    Из этого следует верность теоремы для этого случая.

    Предположим, что $\ideal{n}$ такой, что $G = \multiplicative{(R/\ideal{n})}$.
    Пусть $\ideal{n}$ раскладывается в произведение простых идеалов следующим образом $\ideal{n} = \prod_{i=1}^r \ideal{p}_i^{\alpha_i}$.
    Так как норма простого идеала примарная, то обозначим $\Nm{p_i} = q_i^{f_i}$, где $q_i$ -- простой в $\mathbb{Z}$.

    Предположим, что существует такой $j \in \{1, \dots, r\}$, что $\alpha_i > 1$ в разложении $\ideal{n}$ на множители.
    Тогда $\Nm{\ideal{p}_j} | \varphi(\ideal{n})$, следовательно $q_j | \varphi(\ideal{n})$.
    Из теоремы Коши для групп~\ref{statement:cauchy} следует, что в группе $\multiplicative{(R/\ideal{n})}$ существует элемент $a$ порядка $q_j$.
    Из теоремы Эйлера~\ref{statement:euler_function} следует, что $a^{\Nm{\ideal{n}} - 1} \equiv 1 \pmod{\ideal{n}}$, следовательно, $a^{\Nm{\ideal{n}} - 1} \equiv 1 \pmod{\ideal{p}}$.
    Из утверждений выше получаем, что $q_j | \Nm{\ideal{n}} - 1$.
    Это противоречит тому, что $q_j | \Nm{\ideal{n}}$.

    Следовательно, $\alpha_j = 1$ для любого $j \in \{1, \dots, r\}$.
    Так как $\ideal{n}$ -- составное, то $r \ge 2$.
    Обозначим $\Nm{\ideal{p}_i} - 1 = 2^{t_i} u_i$, где $(u_i, 2) = 1$.
    Так же обозначим $s = \min_{i=\overline{1, r}} t_i$, $P = \prod_{i=1}^r (u_i, u)$.

    Из аналога Китайской теоремы об остатках~\ref{statement:chinese_remainder_theorem} следует, что
    \begin{equation*}
        a^u \equiv 1 \pmod{\ideal{n}}
        \Leftrightarrow
        a^u \equiv 1 \pmod{\ideal{p}_i}, i=\overline{1, r}.
    \end{equation*}

    Из теоремы Эйлера~\ref{statement:euler_function} следует, что
    \begin{equation*}
        a^u \equiv 1 \pmod{\ideal{p}_i}
        \Leftrightarrow
        \lambda u \equiv 0 \pmod{\Nm{\ideal{p}_i} - 1},
    \end{equation*}
    где $a \equiv g^\lambda \pmod{\ideal{p}_i}$ и $g$ -- первообразный корень $\multiplicative{(R/\ideal{p}_i)}$.
    Так как последнее сравнение имеет $(u, \Nm{\ideal{p}_i} - 1)$ решений, то количество решений сравнения $a^u \equiv 1 \pmod{\ideal{n}}$ равно
    \begin{equation*}
        \prod_{i=1}^r (u, \Nm{\ideal{p}_i} - 1) = P.
    \end{equation*}

    Аналогично получаем, что
    \begin{equation*}
        a^{2^j u} \equiv -1 \pmod{\ideal{n}}
        \Leftrightarrow
        \lambda 2^j u \equiv \frac{\Nm{\ideal{p}_i} - 1}{2} \pmod{\Nm{\ideal{p}_i} - 1}, i=\overline{1, r}.
    \end{equation*}
    Заметим, что сравнение $\lambda 2^j u \equiv \frac{\Nm{\ideal{p}_i} - 1}{2} \pmod{\Nm{\ideal{p}_i} - 1}$ не имеет решений при $j \ge t_i$ и имеет $(2^j u, \Nm{\ideal{p}_i} - 1)$ решений при $j < t_i$.
    Тогда количество решений $a^{2^j u} \equiv -1 \pmod{\ideal{n}}$ равно
    \begin{equation*}
        \prod_{i=1}^r (2^j u, \Nm{\ideal{p}_i} - 1) = 2^{jr} \prod_{i=1}^r (u, \Nm{\ideal{p}_i} - 1) = 2^{jr} P.
    \end{equation*}

    Следовательно, количество идеалов $a \in \multiplicative{(R/\ideal{n})}$, на которых алгоритм дает ответ ''неизвестно'' равно
    \begin{equation*}
        P + \sum_{j=1}^{s-1} 2^{jr} P = P\left(1 + \frac{2^{rs} - 1}{2^r - 1}\right) = P\frac{2^{rs} + 2^r - 2}{2^r - 1}
    \end{equation*}

    Исходя из определения, получаем
    \begin{equation*}
        |\multiplicative{(R/\ideal{n})}| = \varphi(\ideal{n}) = \prod_{i=1}^r \varphi(\ideal{p}_i) = \prod_{i=1}^r (\Nm{\ideal{p}_i} - 1) = \prod_{i=1}^r 2^{t_i} u_i \ge 2^{rs} P.
    \end{equation*}

    Таким образом
    \begin{equation*}
        |G|/|\multiplicative{(R/\ideal{n})}| \le \frac{2^{rs} + 2^r - 2}{2^{rs}(2^r - 1)} \le \frac{1}{2}.
    \end{equation*}
\end{proof}

\begin{remark}
    Если $\ideal{n}$ -- составной, то при выполнении алгоритма \ref{algorithm:miller_rabin} $k$ раз вероятность получить ответ ''$\ideal{n}$ не простой'' не меньше $1 - \frac{1}{2^k}$.
\end{remark}

\subsection{Детерминированное тестирование на простоту}

\begin{definition}
    Характером группы $G$ называется гомоморфизм $\chi: G \to \mathbb{C}^*$, где $\mathbb{C}^*$ -- мультипликаттивная группа поля $\mathbb{C}$.
    Характер называется тривиальным, если его образ является тривиальной группой, т.е. состоит из $1$ элемента.
\end{definition}

\begin{definition}
    Характером Дирихле по модулю $\ideal{n}$ называется гомоморфизм $\chi: \multiplicative{(R/\ideal{n})} \setminus \{0\} \to \mathbb{C}$, для которой выполнено:
    \begin{itemize}
        \item если $(\ideal{a}, \ideal{n}) > 1$, то $\chi(\ideal{a}) = 0$;

        \item если $(\ideal{a}, \ideal{n}) = 1$, то $\chi(\ideal{a}) \neq 0$;

        \item если $\ideal{a} \equiv \ideal{b} \pmod{\ideal{n}}$, то $\chi(\ideal{a}) = \chi(\ideal{b})$.
    \end{itemize}
\end{definition}

\begin{definition}
    Главным характером Дирихле называется
    \begin{equation*}
        \chi_0(\ideal{a}) = \left\{\begin{split}
            0 & \textrm{ если }\; (\ideal{a}, \ideal{n}) > 1\\
            1 & \textrm{ если }\; (\ideal{a}, \ideal{n}) = 1.
        \end{split}\right.
    \end{equation*}
\end{definition}

\begin{remark}
    Пусть $\rho$ -- характер группы $\multiplicative{(R/\ideal{n})}\setminus\{0\}$.
    Его можно доопределить до характера Дирихле по модулю $\ideal{n}$ следующим образом
    \begin{equation*}
        \chi(\ideal{a}) = \left\{\begin{split}
            0 & \textrm{ если }\; \ideal{a} \not\in \multiplicative{(R/\ideal{n})}\setminus\{0\}\\
            \rho(\ideal{a}) & \textrm{ если }\; \ideal{a} \in \multiplicative{(R/\ideal{n})}\setminus\{0\}
        \end{split}\right.
    \end{equation*}
\end{remark}

\begin{definition}
    Кольцо $R$ удовлетворяет условию A, если существует многочлен $f_A$, что для любого нетривиального характера Дирихле $\chi$ существует простой идеал $\ideal{p}$ взаимнопростой с $\ideal{n}$ и такой, что $\chi(\ideal{p}) \neq 1$ и
    \begin{equation*}
        \Nm{\ideal{p}} \le f(\log{\Nm{\ideal{n}}}).
    \end{equation*}
\end{definition}

\begin{remark}
    Обобщенная гипотеза Римана гласит, что для любого характера Дирихле $\chi$ и любого $s \in \mathbb{C}$ такого, что $L(\chi, s) = 0$ выполняется: если $s \not\in \mathbb{R}_{-}$, то $s = \frac{1}{2}$.

    Из работы Анкени~\cite{Ankeny} следует, что, если обобщенная гипотеза Римана выполнена, то условие A выполнено для кольца целых чисел и $f_{\mathbb{Z}}(x) = 2x^2$.
\end{remark}

\begin{remark}
    Пусть $K$ числовое поле с кольцом целых алгебраических чисел $\mathcal{O}_K$.
    Расширенная гипотеза Римана гласит, что для любого $s \in \mathbb{C}$ такого, что $\zeta_K(s) = 0$ выполняется: если $\Re{s} \in [0, 1]$, то $\Re{s} = \frac{1}{2}$.

    Из работы Баха~\cite{Bach} следует, что, если расширенная гипотеза Римана выполнена, то условие A выполнено для всех колец целых алгебраических чисел и $f_{\mathcal{O}_K}(x) = 12x^2 + 12\log^2 \Delta$.
\end{remark}

\begin{proposition}
    Пусть дедекиндово кольцо $R$ удовлетворяет условию A.
    Пусть $G$ является конечной
\end{proposition}

\begin{theorem}\label{theorem:euler_criteria_deterministic}
    Пусть $\ideal{n}$ -- нетривиальный идеал нечетной нормы дедекиндового кольца $R$, удовлетворяющего условию A.
    Тогда $\ideal{n}$ -- простой идеал тогда и только тогда, когда для любого идеала $\ideal{a} \in \multiplicative{(R/\ideal{n})}$, $\Nm{\ideal{a}} \le f_R(\Nm{\ideal{n}})$ выполнено
    \begin{equation*}
        a^{\frac{\Nm{\ideal{n}} - 1}{2}} \equiv \left[\frac{\ideal{a}}{\ideal{n}}\right] \pmod{\ideal{n}}.
    \end{equation*}
\end{theorem}

\begin{theorem}\label{theorem:miller_criteria_deterministic}
    Пусть $\ideal{n}$ -- нетривиальный идеал нечетной нормы дедекиндового кольца $R$, удовлетворяющего условию A.
    Пусть $\Nm{\ideal{n}} - 1 = 2^t u$, $(u, 2) = 1$.
    Тогда $\ideal{n}$ -- простой идеал тогда и только тогда, когда для любого идеала $\ideal{a} \in \multiplicative{(R/\ideal{n})}$, $\Nm{\ideal{a}} \le f_R(\Nm{\ideal{n}})$, $(\ideal{a}, \ideal{n}) = 1$, $\ideal{a}^u \not\equiv 1 \pmod{\ideal{n}}$ существует $k\in \{0, \dots, t-1\}$, такое что $\ideal{a}^{2^{k}u} \equiv -1 \pmod{\ideal{n}}$.
\end{theorem}
%\begin{proof}
%    Необходимость следует из теоремы~\ref{theorem:euler_criteria}.
%    Докажем достаточность.
%
%    Пусть $\ideal{n}$ -- составной идеал кольца $R$.
%    Предположим, что существует такой $\ideal{p}$, что $\ideal{p}^2 | \ideal{N}$.
%    Рассмотрим отображение $\chi(\ideal{a}) = \ideal{a}^{\Nm{\ideal{p}} - 1}$.
%    Заметим, что это отображение является гомоморфизмом
%\end{proof}

\onlyinsubfile{
    \subfile{_10_bibliography}
    \subfile{_11_pub}
}

\end{document}
