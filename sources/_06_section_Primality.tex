\documentclass[_dissertation.tex]{subfiles}
\begin{document}

\onlyinsubfile{
    \renewcommand{\contentsname}{ОГЛАВЛЕНИЕ}
    \setcounter{tocdepth}{3}
    \tableofcontents
}

\newpage
\begin{center}
    \refstepcounter{section}
    \section*{ГЛАВА \arabic{section}.\\ ТЕСТИРОВАНИЕ ИДЕАЛОВ НА ПРОСТОТУ В ДЕДЕКИНДОВЫХ КОЛЬЦАХ}\label{ch:Primality}
    \addcontentsline{toc}{chapter}{ГЛАВА \arabic{section}. ТЕСТИРОВАНИЕ ИДЕАЛОВ НА ПРОСТОТУ В ДЕДЕКИНДОВЫХ КОЛЬЦАХ}
\end{center}

\subsection{Аналог критерия Эйлера}

\begin{theorem}\label{theorem:euler_criteria}
    Пусть $\ideal{n}$ -- нетривиальный идеал нечетной нормы дедекиндового кольца $R$.
    Тогда $\ideal{n}$ -- простой идеал тогда и только тогда, когда для любого идеала $\ideal{a} \in (R/\ideal{n})^{\times}$ выполнено
    \begin{equation*}
        a^{\frac{\Nm{\ideal{n}} - 1}{2}} \equiv \left[\frac{\ideal{a}}{\ideal{n}}\right](\modul \ideal{n}).
    \end{equation*}
\end{theorem}
\begin{proof}
    Предположим, что $\ideal{n}$ -- простой идеал.
    Рассмотрим произвольный идеал $\ideal{a} \in (R/\ideal{n})^{\times}$.
    Пусть $g$ -- первообразный корень $(R/\ideal{n})^{\times}$.

    Так как $\Nm{\ideal{n}}$ нечетный, то $\ideal{a}$ является квадратичным вычетом тогда и только тогда, когда существует такое $t' = 2t \in \{0, 2, \dots, \Nm{\ideal{n}} - 1\}$, что $a \equiv g^{t'} (\modul \ideal{n})$.
    Так как порядок $g$ равен $\Nm{\ideal{n}} - 1$, то последнее сравнение выполняется тогда и только тогда, когда $a^{\frac{\Nm{\ideal{n}} - 1}{2}} \equiv 1(\modul \ideal{n})$.
    Это завершает доказательство необходимости.

    Предположим, что $\ideal{n}$ -- не простой идеал.
    Пусть $\ideal{n}$ раскладывается в произведение простых идеалов следующим образом $\ideal{n} = \prod_{i=1}^{r} \ideal{p}_i^{\alpha_i}$.
    Так как норма простого идеала примарная, то обозначим $\Nm{\ideal{p}_i} = q_i^{f_i}$, где $q_i$ -- простой в $\mathbb{Z}$.
    Пусть для любого $\ideal{a} \in (R/\ideal{n})^{\times}$ выполнено $a^{\frac{\Nm{\ideal{n}} - 1}{2}} \equiv \left[\frac{\ideal{a}}{\ideal{n}}\right](\modul \ideal{n})$.

    Пусть существует такой $j \in \{1, \dots, r\}$, что $\alpha_j > 1$ в разложении $\ideal{n}$ на множители.
    Из теоремы Коши для групп \ref{statement:cauchy} и свойств функции Эйлера \ref{statement:euler_function} следует, что существует $\ideal{a} \in (R/\ideal{n})^{\times}$ порядка $q_j l$.
    Тогда $q_j | \Nm{\ideal{n}} - 1$, что невозможно.

    Следовательно, $\alpha_j = 1$ для любого $j \in \{1, \ldots, r\}$.
    Так как $\ideal{n}$ -- составное, то $r \ge 2$.
    Рассмотрим произвольный квадратичный невычет $\ideal{b} \in (R/\ideal{p}_1)^{\times}$.
    Согласно аналогу Китайской теоремы об остатках \ref{statement:chinese_remainder_theorem} существует такой $\ideal{a} \in (R/\ideal{n})^{\times}$, что выполнено $a \equiv b(\modul \ideal{p}_1)$ и $a \equiv 1(\modul \ideal{p}_2\dots\ideal{p}_r)$.
    Но в этом случае $\left[\frac{\ideal{a}}{\ideal{n}}\right] = -1$.
    Из условия теоремы следует, что $a^{\frac{\Nm{\ideal{n}} - 1}{2}} \equiv -1(\modul \ideal{n})$, что противоречит условию $a \equiv 1(\modul \ideal{p}_2)$.
    Это завершает доказательство достаточности.

    Теорема доказана.
\end{proof}

\begin{algorithm}\label{algorithm:solovay_strassen}
    Дан идеал $\ideal{n} \subset R$.
    Необходимо определить является ли он простым.

    \begin{enumerate}
        \item Вычислить $\Nm{\ideal{n}}$;
        
        \item Выбрать случайное $\ideal{a} \subset (R/\ideal{n})^{\times}$;

        \item Вычислить $r_0 = \ideal{a}^{\frac{\Nm{\ideal{n}} -- 1}{2}} (\modul \ideal{n})$;

        \item Вычислить $r_1 = \left[\frac{\ideal{a}}{\ideal{n}}\right]$;

        \item Если $r_0 \equiv r_1 (\modul \ideal{n})$, то вернуть ''неизвестно'' и завершить алгоритм;

        \item Вернуть ''$\ideal{n}$ не простой'' и завершить алгоритм.
    \end{enumerate}
\end{algorithm}

\begin{remark}
    Алгоритм \ref{algorithm:solovay_strassen} является вероятностным.
    Если был получен ответ "неизвестно", то можно выполнить алгоритм еще раз.
\end{remark}

\begin{proposition}
    Пусть $\ideal{n}$ -- не простой идеал.
    Тогда вероятность ответа "$\ideal{n}$ не простой" у алгоритма \ref{algorithm:solovay_strassen} не менее $1/2$.
\end{proposition}
\begin{proof}
    Рассмотрим множество
    \begin{equation*}
        G = \left\{
            a \in (R/\ideal{n})^{\times} | \ideal{a}^{\frac{\Nm{\ideal{n}} - 1}{2}} \equiv \left[\frac{\ideal{a}}{\ideal{n}}\right](\modul \ideal{n})
        \right\}.
    \end{equation*}
    Алгоритм \ref{algorithm:solovay_strassen} возвращает ответ ''неизвестно'' только для элементов из множества $G$.

    Заметим, что если алгоритм \ref{algorithm:solovay_strassen} возвращает ответ ''неизвестно'' для $\ideal{a}$ и $\ideal{b}$, то он вернет ответ ''неизвестно'' и для $\ideal{a}\ideal{b}$.
    Следовательно, $G$ образует подгруппу группы $(R/\ideal{n})^{\times}$.

    Исходя из критерия Эйлера эта подгруппа собственная.
    Из теоремы Лагранжа \ref{statement:lagrange} выполнено $|G|/|(R/\ideal{n})^{\times}| \le \frac{1}{2}$.
\end{proof}

\begin{remark}
    Если $\ideal{n}$ -- составной, то при выполнении алгоритма \ref{algorithm:miller_rabin} $k$ раз вероятность получить ответ ''$\ideal{n}$ не простой'' не меньше $1 - \frac{1}{2^k}$.
\end{remark}

\subsection{Аналог критерия Миллера}

\begin{theorem}\label{theorem:miller_criteria}
    Пусть $\ideal{n}$ -- нетривиальный идеал нечетной нормы дедекиндового кольца $R$.
    Пусть $\Nm{\ideal{n}} - 1 = 2^t u$, $(u, 2) = 1$.
    Тогда $\ideal{n}$ -- простой идеал тогда и только тогда, когда для любого идеала $\ideal{a} \in (R/\ideal{n})^{\times}$, $(\ideal{a}, \ideal{n}) = 1$, $\ideal{a}^u \not\equiv 1(\modul \ideal{n})$ существует $k\in \{0, \dots, t-1\}$, такое что $\ideal{a}^{2^{k}u} \equiv -1 (\modul \ideal{n})$.
\end{theorem}
\begin{proof}
    Предположим, что $\ideal{n}$ -- простой идеал.
    Рассмотрим произвольный идеал $\ideal{a} \in (R/\ideal{n})^{\times}$, $(\ideal{a}, \ideal{n}) = 1$, $\ideal{a}^u \not\equiv 1(\modul \ideal{n})$.
    Из теоремы Эйлера \ref{statement:euler_function} следует, что:
    \begin{equation*}
        \ideal{a}^{2^{t} u} = \ideal{a}^{\varphi(\ideal{n})} \equiv 1 (\modul \ideal{n})
    \end{equation*}

    Раскладываем на множители и получаем, что выполнено
    \begin{equation*}
        (\ideal{a}^{u} - 1)(\ideal{a}^{u} + 1)(\ideal{a}^{2u} + 1)\dots(\ideal{a}^{2^{t-1}u} + 1) \equiv 0 (\modul \ideal{n})
    \end{equation*}

    Из того, что $\ideal{a}^{u} \not\equiv 1 (\modul \ideal{n})$ следует, что $\ideal{a}^{2^{k}u} + 1 \equiv 0 (\modul \ideal{n})$ для некоторого $k\in \{0, \dots, t-1\}$.
    Это завершает доказательство необходимости.

    Предположим, что $\ideal{n}$ -- не простой идеал.
    Пусть $\ideal{n}$ раскладывается в произведение простых идеалов следующим образом $\ideal{n} = \prod_{i=1}^{r} \ideal{p}_i^{\alpha_i}$.
    Так как норма простого идеала примарная, то обозначим $\Nm{\ideal{p}_i} = q_i^{f_i}$, где $q_i$ -- простой в $\mathbb{Z}$.

    Пусть существует такой $j \in \{1, \dots, r\}$, что $\alpha_j > 1$ в разложении $\ideal{n}$ на множители.
    Из теоремы Коши для групп \ref{statement:cauchy} и свойств функции Эйлера \ref{statement:euler_function} следует, что существует $\ideal{a} \in (R/\ideal{n})^{\times}$ порядка $q_j l$.
    Так как $u \not\equiv 0 (\modul q_j)$, то $\ideal{a}^u \not\equiv 1 (\modul \ideal{n})$.
    Следовательно, существует число $k \in \{1, \dots, t-1\}$, такое что выполнено сравнение $\ideal{a}^{2^{k}u} \equiv -1(\modul \ideal{n})$.
    Тогда $\ideal{a}^{2^{k+1}u} \equiv 1(\modul \ideal{n})$.
    Значит выполнено $2^{k+1}u \equiv 0 (\modul q_j)$.
    Из последнего сравнения следует, что $\Nm{\ideal{n}} - 1 \equiv 0(\modul q_j)$, что невозможно.
    
    Следовательно, $\alpha_j = 1$ для любого $j \in \{1, \ldots, r\}$.
    Так как $\ideal{n}$ -- составное, то $r \ge 2$.
    Из аналога Китайской теоремы об остатках \ref{statement:chinese_remainder_theorem} и того, что элемент $-1$ имеет порядок $2$ в каждой группе $(R/\ideal{p}_j)^{\times}$ следует, что существует по крайней мере $2^r-1 \ge 3$ элемента $(R/\ideal{n})^{\times}$ порядка $2$.
    Пусть $\ideal{a} \not\equiv \pm 1(\modul \ideal{n})$ является произвольным элементом порядка $2$ в группе $(R/\ideal{n})^{\times}$.
    Из того, что $(u, 2) = 1$ следует, что $\ideal{a}^u \equiv \ideal{a} \not\equiv \pm 1(\modul \ideal{n})$.
    Таким образом, существует $k \in \{0,\ldots, t-1\}$, такое что верно $\ideal{a}^{2^{k}u} \equiv -1(\modul \ideal{n})$.
    Это противоречит тому, что порядок $\ideal{a}$ равен $2$.
    Это завершает доказательство достаточности.

    Теорема доказана.
\end{proof}

\begin{algorithm}\label{algorithm:miller_rabin}
    Дан идеал $\ideal{n} \subset R$.
    Необходимо определить является ли он простым.

    \begin{enumerate}
        \item Найти $u, t \in \mathbb{N}$, что $\Nm{\ideal{n}} - 1 = 2^t u$ и $(2, u) = 1$;
        
        \item Выбрать случайное $\ideal{a} \subset (R/\ideal{n})^{\times}\setminus\{0\}$;

        \item Вычислить $r_0 = \ideal{a}^u (\modul \ideal{n})$;

        \item Если $r_0 = 1$, то вернуть ''неизвестно'' и завершить алгоритм;

        \item Положить $k = 0$;

        \item Пока $k < t$ выполнять:
        \begin{enumerate}
            \item Если $r_k = -1$, то вернуть ''неизвестно'' и завершить алгоритм;

            \item Увеличить $k$ на $1$;

            \item Вычислить $r_{k+1} \equiv r_k^2 (\modul \ideal{n})$;
        \end{enumerate}

        \item Вернуть ''$\ideal{n}$ не простой'' и завершить алгоритм.
    \end{enumerate}
\end{algorithm}

\begin{remark}
    Алгоритм \ref{algorithm:miller_rabin} является вероятностным.
    Если был получен ответ "неизвестно", то можно выполнить алгоритм еще раз.
\end{remark}

\begin{proposition}
    Пусть $\ideal{n}$ -- не простой идеал.
    Тогда вероятность ответа "$\ideal{n}$ не простой" у алгоритма \ref{algorithm:miller_rabin} не менее $1/2$.
\end{proposition}
\begin{proof}
    Рассмотрим множество всех $\ideal{a}$, для которых алгоритм дает ответ "неизвестно".
    Это в точности множество таких $\ideal{a}$, что $\ideal{a}^u \equiv 1(\modul \ideal{n})$ или для которых существует $j \in \{0, \dots, t-1\}$, что $\ideal{a}^{2^{j}u} \equiv -1(\modul \ideal{n})$.
    Из этого следует, что $\ideal{a}^{\Nm{\ideal{n}} - 1} \equiv 1(\modul \ideal{n})$ для всех таких $\ideal{a}$.

    Рассмотрим множество
    \begin{equation*}
        G = \{\ideal{a} \in (R/\ideal{n})^{\times} | \ideal{a}^{\Nm{\ideal{n}} - 1} \equiv 1(\modul \ideal{n})\}.
    \end{equation*}
    Предположим, что $G$ -- нетривиальная подгруппа.
    Тогда из теоремы Лагранжа следует, что $|G| / |(R/\ideal{n})^{\times}| \le 1/2$.
    Из этого следует верность теоремы для этого случая.

    Предположим, что $\ideal{n}$ такой, что $G = (R/\ideal{n})^{\times}$.

    Предположим, что существует простой идеал $\ideal{p}$, что $\ideal{p}^2 | \ideal{n}$.
    Тогда $\varphi(\ideal{p}^2) | \varphi(\ideal{n})$, следовательно, $\Nm{\ideal{p}} | \varphi(\ideal{n})$.
    Из теоремы Коши для групп \ref{statement:cauchy} следует, что в группе $(R/\ideal{n})^{\times}$ существует элемент $a$ порядка $\Nm{\ideal{p}}$.
    Из теоремы Эйлера \ref{statement:euler_function} следует, что $a^{\Nm{\ideal{n}} - 1} \equiv 1(\modul \ideal{n})$, следовательно, $a^{\Nm{\ideal{n}} - 1} \equiv 1(\modul \ideal{p})$.
    Из утверждений выше получаем, что $\Nm{\ideal{p}} | \Nm{\ideal{n}} - 1$.
    Следовательно, не существует простого идеала $\ideal{p}$, что $\ideal{p}^2 | \ideal{n}$.

    Пусть $\ideal{n} = \prod_{i=1}^r \ideal{p}_i$, где $\ideal{p}_i$ попарно различные простые идеалы и $r \ge 2$.
    Обозначим $\Nm{\ideal{p}_i} = 2^{t_i} u_i$, $(u_i, 2) = 1$, $s = \min_{i=\overline{1, r}} t_i$, $P = \prod_{i=1}^r (u_i, u)$.

    Из аналога Китайской теоремы об остатках \ref{statement:chinese_remainder_theorem} следует, что
    \begin{equation*}
        a^u \equiv 1(\modul \ideal{n})
        \Leftrightarrow
        a^u \equiv 1(\modul \ideal{p}_i), i=\overline{1, r}.
    \end{equation*}

    Из теоремы Эйлера \ref{statement:euler_function} следует, что
    \begin{equation*}
        a^u \equiv 1(\modul \ideal{p}_i)
        \Leftrightarrow
        \lambda u \equiv 0(\modul \Nm{\ideal{p}_i} - 1),
    \end{equation*}
    где $a \equiv g^\lambda(\modul \ideal{p}_i)$ и $g$ -- первообразный корень $(R/\ideal{p}_i)^{\times}$.
    Так как последнее сравнение имеет $(u, \Nm{\ideal{p}_i} - 1)$ решений, то количество решений сравнения $a^u \equiv 1(\modul \ideal{n})$ равно
    \begin{equation*}
        \prod_{i=1}^r (u, \Nm{\ideal{p}_i} - 1) = P.
    \end{equation*}

    Аналогично получаем, что
    \begin{equation*}
        a^{2^j u} \equiv -1(\modul \ideal{n})
        \Leftrightarrow
        \lambda 2^j u \equiv \frac{\Nm{\ideal{p}_i} - 1}{2}(\modul \Nm{\ideal{p}_i} - 1), i=\overline{1, r}.
    \end{equation*}
    Заметим, что сравнение $\lambda 2^j u \equiv \frac{\Nm{\ideal{p}_i} - 1}{2}(\modul \Nm{\ideal{p}_i} - 1)$ не имеет решений при $j \ge t_i$ и имеет $(2^j u, \Nm{\ideal{p}_i} - 1)$ решений при $j < t_i$.
    Тогда количество решений $a^{2^j u} \equiv -1(\modul \ideal{n})$ равно
    \begin{equation*}
        \prod_{i=1}^r (2^j u, \Nm{\ideal{p}_i} - 1) = 2^{jr} \prod_{i=1}^r (u, \Nm{\ideal{p}_i} - 1) = 2^{jr} P.
    \end{equation*}

    Следовательно, количество идеалов $a \in(R/\ideal{n})^{\times}$, на которых алгоритм дает ответ ''неизвестно'' равно
    \begin{equation*}
        P + \sum_{j=1}^{s-1} 2^{jr} P = P\left(1 + \frac{2^{rs} - 1}{2^r - 1}\right) = P\frac{2^{rs} + 2^r - 2}{2^r - 1}
    \end{equation*}

    Исходя из определения, получаем
    \begin{equation*}
        |(R/\ideal{n})^{\times}| = \varphi(\ideal{n}) = \prod_{i=1}^r \varphi(\ideal{p}_i) = \prod_{i=1}^r (\Nm{\ideal{p}_i} - 1) = \prod_{i=1}^r 2^{t_i} u_i \ge 2^{rs} P.
    \end{equation*}

    Таким образом
    \begin{equation*}
        |G|/|(R/\ideal{n})^{\times}| \le \frac{2^{rs} + 2^r - 2}{2^{rs}(2^r - 1)} \le \frac{1}{2}.
    \end{equation*}
\end{proof}

\begin{remark}
    Если $\ideal{n}$ -- составной, то при выполнении алгоритма \ref{algorithm:miller_rabin} $k$ раз вероятность получить ответ ''$\ideal{n}$ не простой'' не меньше $1 - \frac{1}{2^k}$.
\end{remark}

\subsection{Детерминированное тестирование на простоту}

\begin{definition}
    Характером группы $G$ называется гомоморфизм $\chi: G \to \mathbb{C}^*$.
\end{definition}

\begin{definition}
    Характером Дирихле по модулю $\ideal{n}$ называется функция $\chi$ из множества идеалов $R$ в $\mathbb{C}$, для которой выполнено:
    \begin{itemize}
        \item $\chi(\ideal{a}\ideal{b}) = \chi(\ideal{a})\chi(\ideal{b})$;

        \item \begin{equation*}
            \chi(\ideal{a})\left\{\begin{split}
                = 0 & \text{если }\; (\ideal{a}, \ideal{n}) > 1\\
                \neq 0 & \text{если }\; (\ideal{a}, \ideal{n}) = 1.
            \end{split}\right.
        \end{equation*}
        
        \item если $\ideal{a} \equiv \ideal{b} (\modul \ideal{n})$, то $\chi(\ideal{a}) = \chi(\ideal{b})$.
    \end{itemize}
\end{definition}

\begin{definition}
    Главным характером Дирихле называется
    \begin{equation*}
        \chi_0(\ideal{a}) = \left\{\begin{split}
            0 & \text{если }\; (\ideal{a}, \ideal{n}) > 1\\
            1 & \text{если }\; (\ideal{a}, \ideal{n}) = 1.
        \end{split}\right.
    \end{equation*}
\end{definition}

Пусть $\rho$ -- характер группы $(R/\ideal{n})^{\times}\setminus\{0\}$.
Доопределим его до характера Дирихле по модулю $\ideal{n}$ следующим образом
\begin{equation*}
    \chi(\ideal{a}) = \left\{\begin{split}
        0 & \text{если }\; \ideal{a} \not\in (R/\ideal{n})^{\times}\setminus\{0\}
        \rho(\ideal{a}) & \text{если }\; \ideal{a} \in (R/\ideal{n})^{\times}\setminus\{0\}
    \end{split}\right.
\end{equation*}

Характер называется нетривиальны

\onlyinsubfile{
    \subfile{_10_bibliography}
    \subfile{_11_pub}
}

\end{document}
