\documentclass[_00_dissertation.tex]{subfiles}
\begin{document}

\onlyinsubfile{
    \renewcommand{\contentsname}{ОГЛАВЛЕНИЕ}
    \setcounter{tocdepth}{3}
    \tableofcontents
}

\newpage
\begin{center}
    \refstepcounter{section}
    \section*{ГЛАВА \arabic{section}.\\ ПРЕДВАРИТЕЛЬНЫЕ СВЕДЕНИЯ}\label{ch:Prelimiaries}
    \addcontentsline{toc}{chapter}{ГЛАВА \arabic{section}. ПРЕДВАРИТЕЛЬНЫЕ СВЕДЕНИЯ}
\end{center}

\subsection{Дедекиндовы кольца}

Пусть $R$ дедекиндово кольцо.
Идеалом кольца $R$ называется его подкольцо $\ideal{n}$, замкнутое относительно умножения на элементы $R$.
А именно для любого $a \in R$ выполнено включение $a\ideal{n} \subseteq \ideal{n}$.
Идеал $\ideal{n}$ называется тривиальным, если он совпадает с $R$ или нулевым идеалом $0$.
Идеал $\ideal{n}$ называется собственным, если он не совпадает с $R$.

Говорят, что дедекиндово кольцо имеет конечные поля частных (finite quotient field), если для любого простого идеала $\ideal{p} \subseteq R$ факторкольцо $R/\ideal{p}$ конечно.
Далее в работе будем рассматривать только дедекиндовы кольца с конечными полями частных.
Нормой $\Nm{\ideal{n}}$ идеала $\ideal{n} \subset R$ называется мощность факторкольца $R/\ideal{n}$.

\begin{example}
    Рассмотрим примеры дедекиндовых колец с конечными полями частных.
    \begin{itemize}
        \item Пусть $K$ -- числовое поле.
        Кольцо $\mathbb{Z}_K$, образованное алгебраическими целыми элементами этого поля является дедекиндовым с конечным полем остатков.
        Частными случаями этого примера являются кольцо целых чисел и гауссовых чисел.
        
        \item Пусть $f(x, y) = y - mx - b$ -- прямая.
        Тогда $K[x, y]/(f(x, y)) \cong K[x]$.
        Следовательно, это координатное кольцо является факториальным.
        
        \item Пусть $f(x, y) = y - x^2$ -- парабола.
        Тогда $K[x, y]/(f(x, y)) \cong K[x]$.
        Следовательно, это координатное кольцо является факториальным.
    
        \item Пусть $f(x, y) = x^2 + y^2 - 1$.
        Если $K = \mathbb{Q}$, то координатное кольцо $\mathbb{Q}[x, y]/(f)$ не изоморфно $K[x]$, так как первое не является факториальным кольцом.
        Это можно показать, рассмотрев элементы $y^2 = yy$ и $1-x^2 = (1-x)(1+x)$.
        Однако, если $K = \mathbb{C}$, то координатное кольцо $\mathbb{C}[x, y]/(f) \cong \mathbb{C}[x, x^{-1}]$ уже будет факториальным, так как это локализация факториального кольца.
    \end{itemize}
\end{example}

Собственный идеал $\ideal{n} \subset R$ называется простым, если факторкольцо $R/\ideal{n}$ является областью целостности.
Собственный идеал $\ideal{n} \subset R$ называется максимальным, если он не содержится ни в каком другом собственном идеале.
Любой максимальный идеал является простым.

\begin{definition}\cite{source:Petukhova}
    Функцией Эйлера идеала $\ideal{n} \subset R$ называется функция
    \begin{equation*}
        \varphi(\ideal{n}) = \left|
            \invertible{(R/\ideal{n})}
        \right|.
    \end{equation*}
\end{definition}

\begin{statement}[Обобщенная теорема Эйлера]\cite{source:Petukhova}
    Пусть $m \in R$ и $\ideal{n} \subset R$ --- идеал.
    Если $Rm + \ideal{n} = R$, то
    \begin{equation*}
        m^{\varphi(\ideal{n})}\equiv 1 \pmod{\ideal{n}}.
    \end{equation*}
\end{statement}

Первообразным корнем по модулю идеала $\ideal{n}$ будем называть такой элемент $g \in R$, что $g^{\varphi (\ideal{n})}\equiv 1 \pmod{\ideal{n}}$ и $g^{l} \not\equiv 1 \pmod{\ideal{n}}$ при $1 \leq l < \varphi(\ideal{n})$.

\begin{statement}\label{statement:chinese_remainder_theorem}
    Пусть $\ideal{n}_1, \ideal{n}_2, \dots, \ideal{n}_k$ -- попарно взаимнопростые идеалы кольца $R$.
    Тогда
    \begin{equation*}
        \begin{split}
            R/(\ideal{n}_1\ideal{n}_2\dots\ideal{n}_k) \cong & (R/\ideal{n}_1) \times (R/\ideal{n}_2) \times \dots \times (R/\ideal{n}_k)\\
            \invertible{(R/(\ideal{n}_1\ideal{n}_2\dots\ideal{n}_k))} \cong & \invertible{(R/\ideal{n}_1)} \times \invertible{(R/\ideal{n}_2)} \times \dots \times \invertible{(R/\ideal{n}_k)}
        \end{split}
    \end{equation*}
\end{statement}

Пусть $a, b \in R$ элементы дедекиндового кольца, а $\ideal{n} \subseteq R$ идеал дедекиндового кольца.
Будем говорить, что $a$ сравнимо с $b$ по модулю $\ideal{n}$ и писать $a \equiv b \pmod{\ideal{n}}$, если $a - b \in \ideal{n}$.

Элемент $a \in R$ будем называть квадратичным вычетом по модулю идеала $\ideal{n}$, если существует $b \in R$, что $b^2 \equiv a \pmod{\ideal{n}}$.

Для простого идеала $\ideal{p}$ и $a \in \invertible{R/\ideal{p}}$ определим символ Лежандра следующим образом
\begin{equation*}
    \jacobi{a}{\ideal{p}} = \begin{cases}
        1, \textrm{ если } a \textrm{ квадратичный вычет по модулю } \ideal{p}\\
        -1, \textrm{ иначе}.
    \end{cases}
\end{equation*}

Для идеала $\ideal{n} = \ideal{p}_1  \dots \ideal{p}_k$ и $a \in \invertible{R/\ideal{n}}$ определим символ Якоби следующим образом
\begin{equation*}
    \jacobi{a}{\ideal{n}} = \left(\frac{a}{\ideal{p}_1}\right) \dots \left(\frac{a}{\ideal{p}_k}\right)
\end{equation*}

Пусть $R$ дедекиндово кольцо с полем частных $K$.
Пусть расширение $L/K$ конечное, сепарабельное и нормальное и группа Галуа этого расширения абелева.
Положим $S$ алгебраическое замыкание $R$ в $L$.

\begin{definition}
    Пусть $\ideal{p}$ простой идеал кольца $R$.
    Рассмотрим идеал $\ideal{p}S$, который он генерирует в кольце $S$.
    Этот идеал не обязан быть простым, но существует единственная факторизация его на простые идеалы
    \begin{equation*}
        \ideal{p}S = \prod_{\ideal{q}} \ideal{q}^{e_{\ideal{q}}},
    \end{equation*}
    где произведение берется по всем простым идеалам кольца $S$ и $e_{\ideal{q}} > 0$ только для конечного количества простых $\ideal{q}$.
    Если $e_{\ideal{q}} > 0$ для некоторого $\ideal{q}$, то говорят, что $\ideal{q}$ лежит над (lie over, lie above) простым идеалом $\ideal{p}$.

    Число $e_{\ideal{q}}$ из разложения $\ideal{p}S = \prod_{\ideal{q}} \ideal{q}^{e_{\ideal{q}}}$ называется индексом разветвления (ramification index) $\ideal{q}$.
    Число $f_{\ideal{q}} = [S/\ideal{q} : R/\ideal{p}]$ называется степенью инертности $\ideal{q}$.

    Если для простого идеала $\ideal{p} \subseteq R$ выполнено $e_{\ideal{q}} > 1$ для некоторого $\ideal{q} \subseteq S$, то говорят, что идеал $\ideal{p}$ ветвится в $L$.
    Если идеал $\ideal{p}S$ простой в $S$, то говорят, что $\ideal{p}$ инертный (inert).
    Если $e_{\ideal{q}} = 0$ или $e_{\ideal{q}} = 1$ для всех $\ideal{q}$, то говорят, что $\ideal{p}$ разлагается полностью (splits completely) в $L$.
\end{definition}

\begin{definition}
    Пусть $\ideal{p}$ простой идеал кольца $R$, не ветвящийся в $L$ и пусть $\ideal{P} = \ideal{p}S$ соответствующий идеал в $S$.
    Тогда существует единственный такой элемент $\sigma \in \Gal{L/K}$, что для любого $\alpha \in L$
    \begin{equation*}
        \sigma(\alpha) \equiv \alpha^{\Nm{\ideal{p}}} \pmod{\ideal{P}}.
    \end{equation*}
    Этот элемент обозначают $((L/K), \ideal{p})$ и называют символом Артина.
\end{definition}

\begin{definition}
    Пусть $\phi: \Gal{L/K} \to \invertible{\mathbb{C}}$ гомоморфизм.
    Рассмотрим функцию
    \begin{equation*}
        \chi(\ideal{p}) = \begin{cases}
            \phi(((L/K), \ideal{p})), & \textrm{если } \ideal{p} \textrm{ не ветвится}\\
            0, & \textrm{иначе}
        \end{cases}
    \end{equation*}
    где $\ideal{p}$ простой.
    Используя мультипликативность, эту функцию можно определить для всех идеалов $R$.
    Полученную функцию $\chi$ будем называть характером.
    Характер принимающий только значения $0$ и $1$ называется главным.
\end{definition}

\begin{definition}
    Будем говорить, что характер $\chi$ задан по модулю идеала $\ideal{f} \subset R$
    Если для характера $\chi$ существует такой идеал $\ideal{f} \subset R$, что для всех идеалов $\ideal{x} \subseteq R$, если $\ideal{x} \equiv 1 \pmod{\ideal{f}}$, то $\chi(\ideal{x}) = 1$.
\end{definition}

\begin{definition}
    Для произвольного характера $\chi: R \to \invertible{\mathbb{C}}$, не являющегося главным и определенного по модулю идеала $\ideal{f} \subset R$, рассмотрим идеал $\ideal{p}_{\chi}$ минимальной нормы, для которого $\chi(\ideal{p}) \neq 0, 1$.

    Будем говорить, что кольцо $R$ удовлетворяет условию A, если существует многочлен $f_R$, что для любого характера $\chi$, не являющегося главным, выполнено
    \begin{equation*}
        \Nm{\ideal{p}} \le f_R(\log{\Nm{\ideal{n}}}).
    \end{equation*}
\end{definition}

\begin{remark}
    Пусть $K$ числовое поле с кольцом целых алгебраических чисел $\mathcal{O}_K$.
    Расширенная гипотеза Римана гласит, что для любого $s \in \mathbb{C}$ такого, что $\zeta_K(s) = 0$ выполняется: если $\Re{s} \in [0, 1]$, то $\Re{s} = \frac{1}{2}$.

    Из работы Баха~\cite{source:Bach} следует, что, если расширенная гипотеза Римана выполнена, то условие A выполнено для всех колец целых алгебраических чисел и $f_{\mathcal{O}_K}(x) = 12x^2 + 12\log^2 \Delta$.
\end{remark}

\begin{remark}
    Обобщенная гипотеза Римана гласит, что для любого характера Дирихле $\chi$ и любого $s \in \mathbb{C}$ такого, что $L(\chi, s) = 0$ выполняется: если $s \not\in \mathbb{R}_{-}$, то $s = \frac{1}{2}$.

    Из работы Анкени~\cite{source:Ankeny} следует, что, если обобщенная гипотеза Римана выполнена, то условие A выполнено для кольца целых чисел и $f_{\mathbb{Z}}(x) = 2x^2$.
\end{remark}

\subsection{Факториальные кольца}

Идеал $\ideal{n}$ дедекиндового кольца $R$ называется главным, если его можно представить в виде $nR$ для некоторого $n \in R$.
Дедекиндово кольцо, у которого все идеалы являются главными называется кольцом главных идеалов или факториальным кольцом.
В кольце главных идеалов можно построить биективное отображение между элементами кольца и идеалами.
Далее в работе при рассмотрении факториальных колец будем писать вместо идеалов элементы кольца.

\begin{definition}
    Пусть $R$ факториальное кольцо.
    Функцию $\elementnorm{\cdot}: R \to \mathbb{N} \cup \{0, -\infty\}$ будем называть нормой в $R$, если
    \begin{itemize}
        \item $\elementnorm{x} = -\infty$ тогда и только тогда, когда $x = 0$;

        \item $\elementnorm{xy} \ge \elementnorm{x}$;

        \item для $x, y \in \zeroless{R}$ равенство $\elementnorm{xy} = \elementnorm{x}$ выполнено тогда и только тогда, когда $y \in \invertible{K}$.
    \end{itemize}
\end{definition}

\begin{remark}
    Для любого факториального кольца $R$ можно определить норму.
    Рассмотрим разложение элемента $x$ на простые множители $x = \varepsilon p_1^{\alpha_1} \dots p_k^{\alpha_k}$, где $\varepsilon \in \invertible{R}$, $p_1, \dots, p_k$ -- простые элементы $R$.
    Тогда функция
    \begin{equation*}
        \elementnorm{x} = \left\{\begin{split}
            \sum_{i=1}^k \alpha_{i}, & \textrm{ если } x \neq 0\\
            -\infty, & \textrm{ если } x = 0
        \end{split}\right.
    \end{equation*}
    является нормой в $R$.
\end{remark}

Далее будем считать, что факториальное кольцо $R$ задано вместе с нормой $\elementnorm{\cdot}$.

\begin{definition}
    Пусть $R$ факториальное кольцо и $F$ его поле частных.
    Функцию $\fr{\cdot}: F \to F$ будем называть дробной частью в $F$, если
    \begin{itemize}
        \item $\fr{\alpha + q} = \fr{\alpha}$ для любых $\alpha \in F$, $q \in R$;

        \item если $m \in R$, $n \in \zeroless{R}$ и $(m, n) = 1$, то $\fr{m/n} = r/n$, где $r \in R$, $(m-r)/n \in R$ и $\elementnorm{r} = \min \{\elementnorm{s} | s \in R, (m-s)/n \in R\}$.
    \end{itemize}
    Функцию $\int{\cdot}: F \to R$ будем называть целой частью, если
    \begin{equation*}
        \int{\alpha} = \alpha - \fr{\alpha}.
    \end{equation*}
\end{definition}

\begin{remark}\label{remark:easy_fr}
    Для любого факториального кольца $R$ существует дробная и целая часть.
    Рассмотрим произвольный элемент $X \in F/R$.
    Этот элемент можно представить в виде $X = \{m/n + t | t \in R\}$, где $m \in R$, $n \in \zeroless{R}$ $(m, n) = 1$.
    Существует элемент $t_0 \in R$, минимизирующий норму $\elementnorm{m + n t_0}$.
    Тогда для любого элемента $x \in X$ положим
    \begin{equation*}
        \fr{x} = m/n + t_0.
    \end{equation*}

    Несложно заметить, что эта функция является дробной частью.
    Целую часть определим следующим образом
    \begin{equation*}
        \int{x} = x - \fr{x}.
    \end{equation*}
\end{remark}

Далее будем считать, что факториальное кольцо $R$ задано вместе дробной частью $\fr{\cdot}$ и целой частью $\int{\cdot}$.

\begin{statement}[Теорема Копперсмита]\label{statement:coppersmith}
  Пусть
  \begin{equation*}
      f(x, y) = \sum\limits_{i, j} p_{i, j} x^i y^j
  \end{equation*}
  неприводимый многочлен от двух переменных над $\mathbb{Z}$ со степенью $\delta$ по каждой переменной отдельно.
  Пусть $X$, $Y$ границы предполагаемого решения $(x_0, y_0)$.
  Обозначим
  \begin{equation*}
      W = \max_{i, j} |p_{i, j}| X^i Y^j.
  \end{equation*}
  Пусть $XY < W^{\frac{3}{2\delta}}$, то существует полиномиальный относительно $\log W$ и $2^\delta$ алгоритм, который позволяет найти такую пару $(x_0, y_0)$, что $f(x_0, y_0) = 0$, $|x_0| \le X$ и $|y_0| \le Y$.
\end{statement}

\begin{statement}\label{statement:cauchy}(Теорема Коши)
    Если порядок конечной группы $G$ делится на простое число $p$, то $G$ содержит элементы порядка $p$.
\end{statement}

\begin{statement}\label{statement:lagrange}(Теорема Лагранжа)
    Пусть группа $G$ конечна, и $H$ -- её подгруппа.
    Тогда порядок $G$ равен порядку $H$, умноженному на индекс подгруппы.
\end{statement}

\onlyinsubfile{
    \subfile{_10_bibliography}
    \subfile{_11_pub}
}

\end{document}
