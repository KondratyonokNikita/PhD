\documentclass[_00_dissertation.tex]{subfiles}
\begin{document}

\onlyinsubfile{
    \renewcommand{\contentsname}{ОГЛАВЛЕНИЕ}
    \setcounter{tocdepth}{3}
    \tableofcontents
}

\newpage
\begin{center}
    \refstepcounter{section}
    \section*{ГЛАВА \arabic{section}.\\ ПРЕДВАРИТЕЛЬНЫЕ СВЕДЕНИЯ}\label{ch:Prelimiaries}
    \addcontentsline{toc}{chapter}{ГЛАВА \arabic{section}. ПРЕДВАРИТЕЛЬНЫЕ СВЕДЕНИЯ}
\end{center}

Пусть $R$ дедекиндово кольцо.
Идеалом кольца $R$ называется его подкольцо $\ideal{n}$, замкнутое относительно умножения на элементы $R$.
А именно для любого $a \in R$ выполнено $a\ideal{n} \subseteq \ideal{n}$.
Идеал $\ideal{n}$ называется  тривиальным, если он совпадает с $R$ или нулевым идеалом $0$.
Идеал $\ideal{n}$ называется собственным, если он не совпадает с $R$.

Во всей работе будем предполагать, что для любого максимального идеала $\ideal{n} \subseteq R$ факторкольцо $R/\ideal{n}$ конечно.
Нормой $\Nm{\ideal{n}}$ идеала $\ideal{n} \subset R$ называется мощность факторкольца $R/\ideal{n}$.

Собственный идеал $\ideal{n} \subset R$ называется простым, если факторкольцо $R/\ideal{n}$ является областью целостности.
Собственный идеал $\ideal{n} \subset R$ называется максимальным, если он не содержится ни в каком другом собственном идеале.
Любой максимальный идеал является простым.

\begin{definition}\cite{source:Petukhova}
    Функцией Эйлера идеала $\ideal{n} \subset R$ называется функция
    \begin{equation*}
        \varphi(\ideal{n}) = \left|
            \invertible{(R/\ideal{n})}
        \right|.
    \end{equation*}
\end{definition}

\begin{statement}[Обобщенная теорема Эйлера]\cite{source:Petukhova}
    Пусть $m \in R$ и $\ideal{n} \subset R$ --- идеал.
    Если $Rm + \ideal{n} = R$, то
    \begin{equation*}
        m^{\varphi(\ideal{n})}\equiv 1 \pmod{\ideal{n}}.
    \end{equation*}
\end{statement}

Первообразным корнем по модулю идеала $\ideal{n}$ будем называть такой элемент $g \in R$, что $g^{\varphi (\ideal{n})}\equiv 1 \pmod{\ideal{n}}$ и $g^{l} \not\equiv 1 \pmod{\ideal{n}}$ при $1 \leq l < \varphi(\ideal{n})$.

\begin{statement}\label{statement:cauchy}(Теорема Коши)
    Если порядок конечной группы $G$ делится на простое число $p$, то $G$ содержит элементы порядка $p$.
\end{statement}

\begin{statement}\label{statement:lagrange}(Теорема Лагранжа)
    Пусть группа $G$ конечна, и $H$ -- её подгруппа.
    Тогда порядок $G$ равен порядку $H$, умноженному на индекс подгруппы.
\end{statement}

\begin{statement}\label{statement:chinese_remainder_theorem}
    Пусть $\ideal{n}_1, \ideal{n}_2, \dots, \ideal{n}_k$ -- попарно взаимнопростые идеалы кольца $R$.
    Тогда
    \begin{equation*}
        \begin{split}
            R/(\ideal{n}_1\ideal{n}_2\dots\ideal{n}_k) \cong & (R/\ideal{n}_1) \times (R/\ideal{n}_2) \times \dots \times (R/\ideal{n}_k)\\
            \invertible{(R/(\ideal{n}_1\ideal{n}_2\dots\ideal{n}_k))} \cong & \invertible{(R/\ideal{n}_1)} \times \invertible{(R/\ideal{n}_2)} \times \dots \times \invertible{(R/\ideal{n}_k)}
        \end{split}
    \end{equation*}
\end{statement}

Пусть $a, b \in R$ элементы дедекиндового кольца, а $\ideal{n} \subseteq R$ идеал дедекиндового кольца.
Будем говорить, что $a$ сравнимо с $b$ по модулю $\ideal{n}$ и писать $a \equiv b \pmod{\ideal{n}}$, если $a - b \in \ideal{n}$.

Элемент $a \in R$ будем называть квадратичным вычетом по модулю идеала $\ideal{n}$, если существует $b \in R$, что $b^2 \equiv a \pmod{\ideal{n}}$.

Для простого идеала $\ideal{p}$ и $a \in \invertible{R/\ideal{p}}$ определим символ Лежандра следующим образом
\begin{equation*}
    \jacobi{a}{\ideal{p}} = \begin{cases}
        1, \textrm{ если } a \textrm{ квадратичный вычет по модулю } \ideal{p}\\
        -1, \textrm{ иначе}.
    \end{cases}
\end{equation*}

Для идеала $\ideal{n} = \ideal{p}_1  \dots \ideal{p}_k$ и $a \in \invertible{R/\ideal{n}}$ определим символ Якоби следующим образом
\begin{equation*}
    \jacobi{a}{\ideal{n}} = \left(\frac{a}{\ideal{p}_1}\right) \dots \left(\frac{a}{\ideal{p}_k}\right)
\end{equation*}

\onlyinsubfile{
    \subfile{_10_bibliography}
    \subfile{_11_pub}
}

\end{document}
