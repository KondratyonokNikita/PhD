\documentclass[_00_dissertation.tex]{subfiles}
\begin{document}

\onlyinsubfile{
    \renewcommand{\contentsname}{ОГЛАВЛЕНИЕ}
    \setcounter{tocdepth}{3}
    \tableofcontents
}

\newpage
\begin{center}
    \refstepcounter{section}
    \section*{ГЛАВА \arabic{section}.\\ ПРЕДВАРИТЕЛЬНЫЕ СВЕДЕНИЯ}\label{ch:Prelimiaries}
    \addcontentsline{toc}{chapter}{ГЛАВА \arabic{section}. ПРЕДВАРИТЕЛЬНЫЕ СВЕДЕНИЯ}
\end{center}

\subsection{Дедекиндовы кольца}

\begin{definition}
    \emph{Идеалом} кольца $R$ называется его подкольцо $\ideal{n}$, замкнутое относительно умножения на элементы $R$.
    А именно для любого $a \in R$ выполнено включение $a\ideal{n} \subseteq \ideal{n}$.
    Идеал $\ideal{n}$ называется \emph{тривиальным}, если он совпадает с $R$ или нулевым идеалом $0$.
    Идеал $\ideal{n}$ называется \emph{собственным}, если он не совпадает с $R$.
\end{definition}

\begin{definition}
    Собственный идеал $\ideal{n} \subset R$ называется \emph{простым}, если факторкольцо $R/\ideal{n}$ является областью целостности.
\end{definition}

\begin{definition}
    Собственный идеал $\ideal{n} \subset R$ называется \emph{максимальным}, если он не содержится ни в каком другом собственном идеале.
\end{definition}

\begin{remark}
    Любой максимальный идеал является простым.
\end{remark}

\begin{definition}
    Кольцо $R$ называется \emph{дедекиндовым}, если любой нетривиальный идеал раскладывается в произведение простых идеалов иденственным способом с точностью до порядка множителей.
\end{definition}

\begin{definition}
    \emph{Нормой} $\Nm{\ideal{n}}$ идеала $\ideal{n} \subset R$ называется мощность факторкольца $R/\ideal{n}$.

    Говорят, что дедекиндово кольцо $R$ является \emph{дедекиндовым кольцом с конечной нормой} (finite norm property), если для любого собственного идеала $\ideal{n} \subseteq R$ факторкольцо $R/\ideal{n}$ конечно.
\end{definition}

Далее в работе будем рассматривать только дедекиндовы кольца с конечной нормой.

\begin{example}
    Рассмотрим примеры дедекиндовых колец с конечной нормой.
    \begin{itemize}
        \item Пусть $K$ -- числовое поле.
        Кольцо $\mathbb{Z}_K$, образованное алгебраическими целыми элементами этого поля является дедекиндовым с конечным полем остатков.
        Частными случаями этого примера являются кольцо целых чисел и гауссовых чисел.
        
        \item Пусть $f(x, y) = y - mx - b$ -- прямая.
        Тогда $K[x, y]/(f(x, y)) \cong K[x]$.
        Следовательно, это координатное кольцо является факториальным.
        
        \item Пусть $f(x, y) = y - x^2$ -- парабола.
        Тогда $K[x, y]/(f(x, y)) \cong K[x]$.
        Следовательно, это координатное кольцо является факториальным.
    
        \item Пусть $f(x, y) = x^2 + y^2 - 1$.
        Если $K = \mathbb{Q}$, то координатное кольцо $\mathbb{Q}[x, y]/(f)$ не изоморфно $K[x]$, так как первое не является факториальным кольцом.
        Это можно показать, рассмотрев элементы $y^2 = yy$ и $1-x^2 = (1-x)(1+x)$.
        Однако, если $K = \mathbb{C}$, то координатное кольцо $\mathbb{C}[x, y]/(f) \cong \mathbb{C}[x, x^{-1}]$ уже будет факториальным, так как это локализация факториального кольца.
    \end{itemize}
\end{example}

\begin{definition}
    Идеал $\ideal{n}$ называется \emph{главным}, если $\ideal{n} = aR$, где $a \in R$.
    Если все идеалы кольца $R$ являются главными, то говорят, что $R$ \emph{кольцо главных идеалов}.
\end{definition}

\subsection{Операции над идеалами}

\begin{definition}
    \emph{Произведением} двух идеалов $\ideal{a}$ и $\ideal{b}$ называется идеал, порожденный всеми произведениями $ab$, где $a \in \ideal{a}$, $b \in \ideal{b}$.
\end{definition}

\begin{definition}
    Пусть $a, b \in R$ и $\ideal{n} \subseteq R$.
    Будем говорить, что $a$ \emph{сравнимо} с $b$ по модулю $\ideal{n}$ и писать $a \equiv b \pmod{\ideal{n}}$, если $a - b \in \ideal{n}$.
\end{definition}

\begin{definition}\cite{source:Petukhova}
    Функцией Эйлера нетривиального идеала $\ideal{n} \subset R$ называется функция
    \begin{equation*}
        \varphi(\ideal{n}) = \left|
            \invertible{(R/\ideal{n})}
        \right|.
    \end{equation*}
\end{definition}

\begin{statement}[Обобщенная теорема Эйлера]\cite{source:Petukhova}
    Пусть $m \in R$ и $\ideal{n} \subset R$ --- идеал.
    Если $Rm + \ideal{n} = R$, то
    \begin{equation*}
        m^{\varphi(\ideal{n})}\equiv 1 \pmod{\ideal{n}}.
    \end{equation*}
\end{statement}

\begin{definition}
    \emph{Первообразным корнем по модулю идеала} $\ideal{n}$ будем называть такой элемент $g \in R$, что $g^{\varphi (\ideal{n})} \equiv 1 \pmod{\ideal{n}}$ и $g^{l} \not\equiv 1 \pmod{\ideal{n}}$ при $1 \leq l < \varphi(\ideal{n})$.
\end{definition}

\begin{statement}[Обобщенная китайская теорема об остатках]\label{statement:chinese_remainder_theorem}
    Пусть $\ideal{n}_1, \ideal{n}_2, \dots, \ideal{n}_k$ -- попарно взаимнопростые идеалы кольца $R$.
    Тогда
    \begin{equation*}
        \begin{split}
            R/(\ideal{n}_1\ideal{n}_2\dots\ideal{n}_k) \cong & (R/\ideal{n}_1) \times (R/\ideal{n}_2) \times \dots \times (R/\ideal{n}_k)\\
            \invertible{R/(\ideal{n}_1\ideal{n}_2\dots\ideal{n}_k)} \cong & \invertible{R/\ideal{n}_1} \times \invertible{R/\ideal{n}_2} \times \dots \times \invertible{R/\ideal{n}_k}
        \end{split}
    \end{equation*}
\end{statement}

\begin{definition}
    Элемент $a \in R$ будем называть \emph{квадратичным вычетом по модулю идеала} $\ideal{n}$, если существует $b \in R$, что $b^2 \equiv a \pmod{\ideal{n}}$.

    Для простого идеала $\ideal{p}$ и $a \in \invertible{R/\ideal{p}}$ определим \emph{символ Лежандра} следующим образом
    \begin{equation*}
        \jacobi{a}{\ideal{p}} = \begin{cases}
            1, \textrm{ если } a \textrm{ квадратичный вычет по модулю } \ideal{p}\\
            -1, \textrm{ иначе}.
        \end{cases}
    \end{equation*}

    Для нетривиального идеала $\ideal{n} = \ideal{p}_1  \dots \ideal{p}_k$ и $a \in \invertible{R/\ideal{n}}$ определим \emph{символ Якоби} следующим образом
    \begin{equation*}
        \jacobi{a}{\ideal{n}} = \left(\frac{a}{\ideal{p}_1}\right) \dots \left(\frac{a}{\ideal{p}_k}\right)
    \end{equation*}
\end{definition}

\subsection{Характер дедекиндового кольца}

\begin{definition}
    \emph{Расширение поля} $K$ это такое поле $E$, которое содержит поле $K$ в качестве подполя.
    Для любого расширения $E \supset K$ поле $E$ является векторным пространством над полем $K$.
    Размерность этого векторного пространства называется \emph{степенью расширения} и обозначается $[E:K]$.
    Говорят, что $E \supset K$ \emph{конечное расширение}, если его степень конечна.
\end{definition}

\begin{definition}
    Пусть $E$ расширение поля $K$.
    Элемент $E$ называется \emph{алгебраическим} над $K$, если он является корнем ненулевого многочлена с коэффициентами в $K$.
    Элементы, не являющиеся алгебраическими, называются \emph{трансцендентными}.
    
    Если каждый элемент расширения $E \supset K$ является алгебраическим над $K$, то $E \supset K$ называется \emph{алгебраическим расширением}. 
\end{definition}

\begin{definition}
    Алгебраическое расширение $E \supset K$ называется \emph{нормальным}, если каждый неприводимый многочлен $f(x)$ над $K$, имеющий хотя бы один корень в $E$, разлагается в $E$ на линейные множители.

    Алгебраическое расширение $E \supset K$ называется \emph{сепарабельным}, если каждый элемент $E$ является сепарабельным, то есть его минимальный многочлен не имеет кратных корней.
    В частности, теорема о примитивном элементе утверждает, что любое конечное сепарабельное расширение имеет примитивный элемент.
    
    Расширение $E \supset K$ называется \emph{расширением Галуа}, если оно одновременно сепарабельное и нормальное.

    Для любого расширения $E \supset K$ можно рассмотреть группу автоморфизмов поля $E$, действующих тождественно на поле $K$.
    Когда расширение является расширением Галуа, эта группа называется \emph{группой Галуа} данного расширения и обозначается $\Gal{L/K}$.
\end{definition}

Пусть $R$ дедекиндово кольцо с полем частных $K$.
Пусть расширение $L \supset K$ конечное, сепарабельное и нормальное расширение, а $\Gal{L/K}$ является абелевой.
Положим $S$ алгебраическое замыкание $R$ в $L$.

\begin{definition}
    Пусть $\ideal{p}$ простой идеал кольца $R$.
    Рассмотрим идеал $\ideal{p}S$, который он генерирует в кольце $S$.
    Этот идеал не обязан быть простым, но существует единственная факторизация его на простые идеалы
    \begin{equation*}
        \ideal{p}S = \prod_{\ideal{q}} \ideal{q}^{e_{\ideal{q}}},
    \end{equation*}
    где произведение берется по всем простым идеалам кольца $S$ и $e_{\ideal{q}} > 0$ только для конечного количества простых $\ideal{q}$.
    Если $e_{\ideal{q}} > 0$ для некоторого $\ideal{q}$, то говорят, что $\ideal{q}$ лежит над (lie over, lie above) простым идеалом $\ideal{p}$.

    Число $e_{\ideal{q}}$ из разложения $\ideal{p}S = \prod_{\ideal{q}} \ideal{q}^{e_{\ideal{q}}}$ называется \emph{индексом разветвления} (ramification index) $\ideal{q}$.

    Если для простого идеала $\ideal{p} \subseteq R$ выполнено $e_{\ideal{q}} > 1$ для некоторого $\ideal{q} \subseteq S$, то говорят, что идеал $\ideal{p}$ \emph{ветвится} в $L$.
    Если идеал $\ideal{p}S$ простой в $S$, то говорят, что $\ideal{p}$ \emph{инертный} (inert).
    Если для всех $\ideal{q}$ выполняется или $e_{\ideal{q}} = 0$, или $e_{\ideal{q}} = 1$, то говорят, что $\ideal{p}$ \emph{полностью разлагается} (splits completely) в $L$.
\end{definition}

\begin{definition}
    Пусть $\ideal{p}$ простой идеал кольца $R$, не ветвящийся в $L$ и пусть $\ideal{P} = \ideal{p}S$ соответствующий идеал в $S$.
    Тогда существует единственный такой элемент $\sigma \in \Gal{L/K}$, что для любого $\alpha \in L$
    \begin{equation*}
        \sigma(\alpha) \equiv \alpha^{\Nm{\ideal{p}}} \pmod{\ideal{P}}.
    \end{equation*}
    Этот элемент называют \emph{символом Артина} идеала $\ideal{p}$.
\end{definition}

\begin{definition}
    Пусть $\phi: \Gal{L/K} \to \invertible{\mathbb{C}}$ гомоморфизм.
    Рассмотрим функцию
    \begin{equation*}
        \chi(\ideal{p}) = \begin{cases}
            \phi(\sigma_{\ideal{p}}), & \textrm{если } \ideal{p} \textrm{ не ветвится}\\
            0, & \textrm{иначе}
        \end{cases}
    \end{equation*}
    где $\ideal{p}$ простой и $\sigma_{\ideal{p}}$ символ Артина идеала $\ideal{p}$.
    Используя мультипликативность, эту функцию можно определить для всех идеалов $R$.
    Полученную функцию $\chi$ будем называть \emph{характером}.
    Характер принимающий только значения $0$ и $1$ называется \emph{главным}.
\end{definition}

\begin{definition}
    Будем говорить, что характер $\chi$ \emph{задан по модулю идеала} $\ideal{f} \subset R$, если для всех идеалов $\ideal{n} \subseteq R$, из сравнения $\ideal{n} \equiv 1 \pmod{\ideal{f}}$ следует равенство $\chi(\ideal{n}) = 1$.
\end{definition}

\subsection{Факториальные кольца}

\begin{definition}
    Кольцо $R$ называется \emph{факториальным}, если каждый его ненулевой элемент $x \in R$ либо обратим, либо однозначно представляется в виде произведения неприводимых элементов с точностью до перестановки множителей.

    Факториальное кольцо является кольцом главных идеалов.
    Далее рассматривая идеалы факториальных колец будем писать элементы, порождающие эти идеалы.
\end{definition}

\begin{definition}
    Пусть $R$ факториальное кольцо.
    Функцию $\elementnorm{\cdot}: R \to \mathbb{N} \cup \{0, -\infty\}$ будем называть нормой в $R$, если
    \begin{itemize}
        \item $\elementnorm{x} = -\infty$ тогда и только тогда, когда $x = 0$;

        \item $\elementnorm{xy} \ge \elementnorm{x}$;

        \item для $x, y \in \zeroless{R}$ равенство $\elementnorm{xy} = \elementnorm{x}$ выполнено тогда и только тогда, когда $y \in \invertible{K}$.
    \end{itemize}
\end{definition}

\begin{remark}
    Для любого факториального кольца $R$ можно определить норму.
    Рассмотрим разложение элемента $x$ на простые множители $x = \varepsilon p_1^{\alpha_1} \dots p_k^{\alpha_k}$, где $\varepsilon \in \invertible{R}$, $p_1, \dots, p_k$ -- простые элементы $R$.
    Тогда функция
    \begin{equation*}
        \elementnorm{x} = \left\{\begin{split}
            \sum_{i=1}^k \alpha_{i}, & \textrm{ если } x \neq 0\\
            -\infty, & \textrm{ если } x = 0
        \end{split}\right.
    \end{equation*}
    является нормой в $R$.
\end{remark}

Далее будем считать, что факториальное кольцо $R$ задано вместе с нормой $\elementnorm{\cdot}$.

\begin{definition}
    Пусть $R$ факториальное кольцо и $F$ его поле частных.
    Функцию $\fr{\cdot}: F \to F$ будем называть дробной частью в $F$, если
    \begin{itemize}
        \item $\fr{\alpha + q} = \fr{\alpha}$ для любых $\alpha \in F$, $q \in R$;

        \item если $m \in R$, $n \in \zeroless{R}$ и $(m, n) = 1$, то $\fr{m/n} = r/n$, где $r \in R$, $(m-r)/n \in R$ и $\elementnorm{r} = \min \{\elementnorm{s} | s \in R, (m-s)/n \in R\}$.
    \end{itemize}
    Функцию $\int{\cdot}: F \to R$ будем называть целой частью, если
    \begin{equation*}
        \int{\alpha} = \alpha - \fr{\alpha}.
    \end{equation*}
\end{definition}

\begin{definition}
    Обозначим через $F_1$ множество всех несократимых дробей $F$
    \begin{equation*}
        F_1 = \{
            \alpha \in F \big| \alpha = \fr{\alpha}
        \}.
    \end{equation*}
\end{definition}

\begin{remark}\label{remark:easy_fr}
    Для любого факториального кольца $R$ можно определить дробную и целую часть.
    Рассмотрим произвольный элемент $X \in F/R$.
    Этот элемент можно представить в виде $X = \{m/n + t | t \in R\}$, где $m \in R$, $n \in \zeroless{R}$ $(m, n) = 1$.
    Существует элемент $t_0 \in R$, минимизирующий норму $\elementnorm{m + n t_0}$.
    Тогда для любого элемента $x \in X$ положим
    \begin{equation*}
        \fr{x} = m/n + t_0.
    \end{equation*}

    Несложно заметить, что эта функция является дробной частью.
    Целую часть определим следующим образом
    \begin{equation*}
        \int{x} = x - \fr{x}.
    \end{equation*}
\end{remark}

Далее будем считать, что факториальное кольцо $R$ задано вместе дробной частью $\fr{\cdot}$ и целой частью $\int{\cdot}$.

\subsection{Способы представления идеалов}

% \subsubsection{Способы представления идеалов}

% Рассмотрим способы представления идеалов.

% \begin{definition}
%     Пусть $\ideal{n} \subseteq R$ произвольный идеал дедекиндового кольца $R$.
%     Тогда существуют элементы $\alpha_1, \dots, \alpha_r \in R$ для $r \leq n$, что
%     \begin{equation*}
%         \ideal{n} = \{\xi_1 \alpha_1 + \dots + \xi_r \alpha_r | \xi_i \in R\}.
%     \end{equation*}
    
%     Такое представление будем обозначать $\ideal{n} = (\alpha_1, \dots, \alpha_r)$ и называть базисным представлением идеала.
% \end{definition}

% \begin{definition}
%     Пусть $\ideal{n} \subseteq R$ произвольный идеал дедекиндового кольца $R$.
%     Тогда существуют элементы $e_1, \dots, e_r \in R$ для $r \leq n$, что
%     \begin{equation}
%         \ideal{n} = \{x_1 e_1 + \dots + x_m e_m | x_i \in \mathbb{Z}\}.
%     \end{equation}

%     Такое представление будем обозначать $\ideal{n} = (e_1, \dots, e_n)_{\mathbb{Z}}$ и называть $\mathbb{Z}$-представлением идеала.
% \end{definition}

% Далее под $\mathbb{Z}$-представлением будем понимать матрицу $A \in \mathbb{\mathbb{Z}}^{n \times n}$, такую что её столбец под номером $i$ - это коэффициенты разложения $e_i$ в фиксированный целый базис $\mathcal{O}_K$.

% В источниках \cite{source:Cohen}, \cite{source:Post} может быть найдено следующее утверждение.

% \begin{statement}
%     Любой идеал имеет $\mathbb{Z}$-представление.
% \end{statement}

% $\mathbb{Z}$-представление требует не меньше памяти для хранения по сравнению с базисным представлением.

% \begin{definition}
%     Представление 
%     \begin{equation}
%         \ideal{a} = (a, \alpha)_2 = \{a\xi_1 + \alpha \xi_2 | \xi_1, \xi_2 \in \mathcal{O}_K\},
%     \end{equation}
%     где $a \in \mathbb{N}_0, \alpha \in \mathcal{O}_K$, будет называть 2-представление идеала $\ideal{a}$.
% \end{definition}

% В источниках \cite{Cohen}, \cite{Post} может быть найдено следующее утверждение.

% \begin{statement}
%     Любой идеал имеет 2-представление.
% \end{statement}

% Далее под 2-представлением будем понимать вектор $\mathbb{Z}^n$ -- коэффициенты разложения $\alpha$ в целый базис и целое неотрицательное число $a$.

% 2-представление является частным случаем базисного представления и любое 2-представление задаёт идеал. Отметим, что оно требует меньше памяти для хранения по сравнению с $\mathbb{Z}$-представлением и базисным представлением.   

% В \cite{Post} приведены соответствующие алгоритмы.

% \begin{statement}
%     Существует полиномиальный алгоритм перехода от 2 - представления к $\mathbb{Z}$-представлению и обратно.
% \end{statement}

% К сожалению, не смотря на полиномиальность алгоритма, он может оказаться достаточно трудоёмким.

% Рассмотрим один важный частный случай $\mathbb{Z}$-представления.

% \begin{definition}
%     Будем говорить, что матрица $A \in \mathbb{Z}^{n \times n}$ записана в нормальной эрмитовой форме, если выполнены следующие условия:
	
%     \begin{enumerate}
%         \item $m_{i,j} = 0$, если $i > j$.
        
%         \item $m_{i,i} > 0$ для любого $i$.
        
%         \item Для любого $i > j$ выполнено $0\leq m_{i,j}\leq m_{i,i}$.
%     \end{enumerate}
% \end{definition}

% \begin{definition}
%     Представление идеала $\ideal{a}$ в нормальной эрмитовой форме будем называть такое его $\mathbb{Z}$-представление
%     \begin{equation}
%         \ideal{a} = (e_1,\dots,e_n)_{\mathbb{Z}},
%     \end{equation}
%     что соответствующая матрица является матрицей в эрмитовой нормальной форме.
% \end{definition}

% В источниках \cite{Cohen}, \cite{Post} может быть найдено следующее утверждение.

% \begin{statement}
%     Любой идеал может быть записан в нормальной эрмитовой форме, причём такое представление единственно.
% \end{statement}

% В статье \cite{PolynomialHermitForm} был построен необходимый алгоритм:

% \begin{statement}
%     Существует полиномиальный алгоритм получения представления идеала в нормальной эрмитовой форме из его $\mathbb{Z}$-представления.
% \end{statement}

% \begin{corollary}
%     Таким образом, за полиномиальное время можно переходить от $\mathbb{Z}$-представления, 2-представления или представления в нормальной эрмитовой форме к любому из них.
% \end{corollary}

% \begin{definition}
%     Пусть дан идеал $\ideal{a}$, зафиксирован целый базис $E$ кольца $\mathcal{O}_K$ и $\mathbb{Z}$-представление идеала $\ideal{a} = (e_1,\dots,e_n)_{\mathbb{Z}}$. Тогда введём абсолютное значение идеала $\ideal{a}$ как
%     \begin{equation}
%         l(\ideal{a}) = \max\limits_{i = \overline{1,n}, j = \overline{1,n}} |a_{ij}|, 
%     \end{equation}
%     где $A = (a_{ij}) \in \mathbb{Z}^{n \times n}$ -- матрица соответствующая указанному $\mathbb{Z}$-представлению.
% \end{definition}

% Нетрудно видеть, что логарифм абсолютного значения идеала характеризует количество памяти необходимое для того, чтобы закодировать его $\mathbb{Z}$-представление.

% В случае, когда кольцо $\mathcal{O}_K$ факториально, любой идеал является главным, а значит любой идеал может быть задан с помощью порождающего его элемента. Поэтому в таких кольцах идеал, как и любой элемент, будет кодироваться в виде вектора $\mathbb{Z}^n$ коэффициентов разложения в целый базис кольца $\mathcal{O}_K$.

% \subsubsection{Операции над элементами}

% В данном параграфе исследуем некоторые арифметические и модулярные операции над элементами колец целых алгебраических элементов и сложности их выполнения.

% Пусть $f(L),$ $g(L)$ две различные функции натурального аргумента $L.$ Мы будем писать $f(L)=\tilde O(g(L))$, если существует положительная функция $h(L)$, такая что $f(L)\le h(L)g(L)$ для любых $L\in \mathbb{N},$ и $h(L)=O(\log g(L)\log \log g(L))$. Данное обозначение вводится в связи с известной оценкой сложности перемножения двух натуральных чисел по алгоритму Шaнхаге-Штрассена. Любое положительная действительное число $C$ будет называться эффективно вычислимой константной (или просто константой), если оно зависит только от инвариантов поля $K$ (например, степени, дискриминанта, интегрального базиса, системы фундаментальных единиц) и существует алгоритм нахождения данного числа.

% \begin{definition}
%     Для любого $a \in \mathcal{O}_K^*$ обозначим через $\overline{a}\in \mathcal{O}_K^*$ сопряжённый элемент определяемый как $\overline{a}=\Nm(a)/a.$
% \end{definition}

% Далее предполагаем, что элементы заданы с помощью коэффициентов своего разложения в целый базис $\mathcal{O}_K$.

% \begin{statement}\label{statement:operations}
%     Пусть $a,b \in \mathcal{O}_K^*$ и $l(a)\leq L,\,l(b)\leq L$, тогда $a+b$, $ab,$ $b/a$ (включая проверку условия $a|b$), $\Nm(a),$ $\overline{a}$ могут быть  вычислены за $\tilde{O}(\log L)$ бинарных операций.
% \end{statement}
% \begin{proof}
%     Рассмотрим произвольные элементы $a = \sum_{i = 1}^n \alpha_ie_i$, $b = \sum_{i = 1}^n \beta_i e_i \in \mathcal{O}_K$, такие что $l(a)\leq L,\,l(b)\leq L$.
%     Утверждение для суммы $a+b$ очевидно.
%     Используя алгоритм Шанхаге-Штрассена быстрого перемножения чисел, нетрудно получить необходимое утверждение для произведения $ab$.
%     Известно, что $\Nm(a) = |\mathrm{det}A|$, где $A = (a_{ij}) \in \mathbb{Z}^{n \times n}$ матрица, такая что $a e_i = \sum_{j = 1}^n a_{ij}e_j$ ($i=1,\ldots,n$).
%     Определитель $\mathrm{det} A$ может быть найден с помощью операций сложения и умножения за $\tilde{O}(\log L)$ бинарных операций.
%     Пусть $\overline{a}=\sum_{i = 1}^n x_ie_i,$ где $x_i$ неизвестные целые коэффициенты.
%     Пусть
%     \begin{equation}
%         \Nm(a)=\sum_{i,j=1}^n  \alpha_i x_j e_i e_j=\sum_{k=1}^n \biggl( \sum_{i = 1}^n\sum_{j = 1}^n \alpha_i x_j \alpha_k^{i,j}\biggr)e_k,
%     \end{equation}
%     где $e_ie_j=\sum_{k=1}^n \alpha_k^{i, j} e_k,$ тогда выполнено соотношение
%     \begin{equation}
%         H (x_1,x_2,\ldots, x_n)^T= (\Nm(a),0,\ldots,0)^T,
%     \end{equation}
%     где $H$ матрица элементов $h_{ij} \in \mathbb{Z}$ ($i,j = 1,\ldots,n$), такая что $h_{ij} = O(L)$ ($i,j = 1,\ldots,n$).
%     Тогда существует константа $D$, такая что $\Nm(a)\le D l(a)^n.$
%     Следовательно решение $(x_1,x_2,\ldots, x_n)^T$ может быть найдено за $\tilde{O}(\log L)$ бинарных операций.
%     Пусть $b\overline{a}=\sum_{i=1}^n y_i e_i,$ $y_i \in \mathbb{Z}.$ Тогда $b/a=\frac{b\overline{a}}{\Nm(a)},$ условие $a|b$ эквивалентно условию $\Nm(a)|y_i$ для любых $i=1,\ldots,n.$ Элемент $b/a$ может быть определён за $\tilde{O}(\log L)$ используя произведение в $\mathcal{O}_K$ и деление рациональных чисел. 
% \end{proof}

% \begin{remark}
%     В доказательстве утверждения было показано, что существует константа $D$, такая что $\Nm(a) \leq Dl(a)^n$ для любого $a \in \mathcal{O}_K$. Из предыдущего утверждения и правила Крамера следует, что существуют константы $R$ и $q$, такие что $l(\overline{a}) \leq RL^q$ для любого $a \in \mathcal{O}_K$.
% \end{remark}

% \begin{statement}\label{statement:mod}
%     Существует константа $M$, такая что для любых $a, m \in \mathcal{O}_K^*$ может быть найдено $z \in \mathcal{O}_K$ удовлетворяющее условию $a\equiv z(\modul\,m)$ and $l(z)\leq Ml(m)$. Если $l(a)\leq L, l(m)\leq L$, тогда такой элемент $z$ может быть определён за $\tilde{O}(\log L)$ бинарных операций.
% \end{statement}

% \begin{proof}
%     Пусть $a = \sum_{i=1}^n a_i e_i$, $m = \sum_{i=1}^n m_i e_i \in \mathcal{O}_K^*$. Тогда мы получаем
%     \begin{equation}
%         \frac{a}{m} = \frac{a\overline{m}}{\Nm(m)} = \frac{1}{\Nm(m)}\sum\limits_{i=1}^n b_i e_i = \sum\limits_{i=1}^n \left\lfloor\frac{b_i}{\Nm(m)}\right\rfloor e_i + \sum\limits_{i=1}^n \frac{b_i'}{\Nm(m)} e_i,
%     \end{equation}
%     где $b_i' \in \mathbb{Z},$ $|b_i'| < \Nm(m)$, $i=1,\ldots,n$.
%     Так как $a\overline{m} \equiv \sum_{i=1}^n b_i' e_i(\modul\,\Nm(m))$, мы получаем $\overline{m}\big|\sum_{i=1}^n b_i' e_i$.
%     Тогда $a\equiv z(\modul\,m)$, где $z = \frac{1}{\overline{m}}\sum_{i=1}^n b_i' e_i$.
%     Так как
%     \begin{equation} \label{eq_z}
%         z= \frac{1}{\Nm(m)}\sum\limits_{k=1}^n e_k\biggl(\sum\limits_{i, j = 1}^n b_i' m_j \alpha_k^{i, j}\biggr),
%     \end{equation}
%     получаем
%     \begin{equation}
%         l(z) = \max_{k = \overline{1, n}} \biggl|\frac{1}{\Nm(m)}\sum\limits_{i, j = 1}^n b_i' m_j \alpha_k^{i, j}\biggr| < \max_{k = \overline{1, n}} \sum_{i, j = 1}^n |m_j \alpha_k^{i, j}| \le \nonumber
%     \end{equation}
%     \begin{equation}
%         \le l(m) \max_{k = \overline{1, n}} \sum_{i, j = 1}^n |\alpha_k^{i, j}|= Ml(m).
%     \end{equation}
%     Предположим, что $l(a)\leq L, l(m)\leq L$. Тогда $l(\overline{m}) \le RL^q,$ где $q$ и $R$ эффективно вычислимые константы.
%     Так как существует константа $D$, такая что $\Nm(m)\le D l(m)^n$, числа $b_i,$ $b_i'$ могут быть найдены за $\tilde{O}(\log L)$ бинарных операций.
%     Поэтому элемент $z$ может быть определён в $K$ по формуле (\ref{eq_z}) используя не более $\tilde{O}(\log L)$ бинарных операций.
% \end{proof}

% \begin{corollary}\label{corollary_mod}
%     Пусть $k \in \mathbb{N},$ и для $a,$ $b,$ $m \in \mathcal{O}_K^*$ выполнено $l(a)\leq L,$ $l(b)\leq L$, $l(m)\leq L$.
%     Элементы $z_1,$ $z_2\in \mathcal{O}_K$ такие что $a+b\equiv z_1(\modul\,m),$ $a^k\equiv z_2(\modul\,m),$ $l(z_1)\leq Ml(m)$, $l(z_2)\leq Ml(m)$, могут быть определены за $\tilde{O}(\log L)$, $\tilde{O}(\log k \log L)$  бинарных операций соответственно.
% \end{corollary}

% \subsubsection{Операции над идеалами}

% Далее будут доказаны утверждения, связанные с вычислительной сложностью различных операций над идеалами.

% \begin{statement}\label{equality}
%     Пусть идеалы $\ideal{a}$ и $\ideal{b}$ заданы в виде нормальной эрмитовой формы и $l(\ideal{a}), l(\ideal{b}) \leq L$, то проверка указанных идеалов на равенство может быть выполнена за $O(\log L)$ бинарных операций.
% \end{statement}
% \begin{proof}
%     Как было указано ранее, любой идеал однозначно задаётся своей нормально эрмитовой формой.
%     Таким образом, достаточно проверить на поэлементнffое равенство две целочисленные матрицы $n \times n$ с коэффициентами размера $O(L)$.
% \end{proof}

% \begin{statement}\label{particular_equality}
%     Пусть $\ideal{p}$ -- простой идеал, заданный в виде 2-представления, а $\ideal{n}$ произвольный идеал заданный в виде $\mathbb{Z}$-представления, причём $l(\ideal{p}), l(\ideal{n}) \leq L$.
%     Тогда проверка равенства $\ideal{p}$ и $\ideal{n}$ может быть выполнена за $\tilde{O}(\log L)$ операций.
% \end{statement}
% \begin{proof}
%     Пусть $\ideal{p} = (\alpha, a)_2$ -- 2-представление простого идеала. Нетрудно видеть, что проверка равенства $\ideal{p} = \ideal{n}$ равносильна проверке того, что $\ideal{p}$ делится на $\ideal{n}$, что по Утверждению \ref{inclusion} равносильно включению $\ideal{p}$ в $\ideal{n}$. Последнее выполнено тогда и только тогда, когда $\alpha, a \in \ideal{n}$ и проверка сводится к проверке разрешимости систем линейных уравнений $n \times n$ с коэффициентами размера $O(L)$, что может быть выполнено за $\tilde{O}(\log L)$ операций
% \end{proof}

% \begin{statement}\label{norm}
%     Пусть идеал $\ideal{a}$ задан в виде $\mathbb{Z}$-представления и $l(\ideal{a}) \leq L$, то $\Nm(\ideal{a})$ может быть вычислено за $\tilde{O}(\log L)$ бинарных операций.
% \end{statement}
% \begin{proof}
%     Исходя из утверждения описанного в \cite{Post} выполнено равенство $\Nm{\ideal{a}} = |\mathrm{det}(A)|$, где $A$ - матрица соответствующая $\mathbb{Z}$-представлению.
%     Нетрудно видеть, что определитель целочисленной матрицы $n \times n$ с коэффициентами размера $O(L)$  может быть вычислен за указанное число операций.
% \end{proof}

% \begin{statement}\label{congruence}
%     Пусть идеал $\ideal{a}$ задан в виде $\mathbb{Z}$-представления и $l(\ideal{a}), l(a), l(b) \leq L$, то проверка сравнения $a \equiv b(\modul \ideal{a})$ может быть выполнена за $\tilde{O}(\log L)$ бинарных операций.
% \end{statement}
% \begin{proof}
%     Пусть изначально $\ideal{a}$ задан в виде $\mathbb{Z}$-представления и $l(\ideal{a}), l(a), l(b) \leq L$.
%     Требуется проверить делимость главного идеала $(a - b)$ на идеал $\ideal{a}$. Исходя из  Утверждения \ref{inclusion} это эквивалентно проверке включения главного идеала $(a - b)$ в идеал $\ideal{a}$. А это, в свою очередь, равносильно тому, что $a - b \in \ideal{a}$. То есть проверка сравнения сводится к проверке разложимости $a - b$ по базису идеала $\ideal{a}$, то есть проверке разрешимости системы линейных уравнений $n \times n$ с коэффициентами размера $O(L)$. Нетрудно видеть, что это может быть сделано за $\tilde{O}(\log L)$ бинарных операций.
% \end{proof}

% \begin{statement}\label{residue_modulo_ideal}
%     Пусть $\ideal{n}$ -- нетривиальный идеал отличный и $a \in \mathcal{O}_K$. Тогда существует $z \in \mathcal{O}_K$, такое что $z \equiv a (\modul \ideal{n})$ и $l(z) \leq N l(\ideal{n})^n$. 
    
%     Если $\ideal{n}$ задан с помощью $\mathbb{Z}$-представления, причём $l(\ideal{n}), l(a) \leq L$, то такой элемент $z$ может быть вычислен за $\tilde{O}(\log L)$ бинарных операций.
% \end{statement}
% \begin{proof}
%     Пусть $E = (e_1,\dots,e_n)$ -- целый базис в $\mathcal{O}_K$, $\ideal{n} = (\omega_1,\dots,\omega_n)_{\mathbb{Z}}$ -- $\mathbb{Z}$-представление идеала $\ideal{n}$.
    
%     Пусть далее $a = \sum\limits_{i = 1}^n \alpha_i e_i$ и $\theta = \mathrm{\text{НОК}}(\Nm(\omega_1),\dots,\Nm(\omega_n))$.
    
%     Нетрудно видеть, что $\theta \in \ideal{n}$, в следствии чего $\theta\equiv 0(\modul \ideal{n})$.
%     Отсюда следует, что $a \equiv a - \beta\theta(\modul \ideal{n})$ для любого $\beta \in \mathcal{O}_K$.
    
%     Положим $\beta = \sum\limits_{i = 1}^n \beta_ie_i,$ где $\alpha_i = \theta \beta_i + r_i$, $r_i < \theta,\,i = \overline{1,n}$, а также $z = a - \beta\theta$.
    
%     Тогда
%     \begin{equation}
%         l(z)  = \max\limits_{i = \overline{1,n}} |r_i| \leq |\theta| \leq \prod\limits_{i = 1}^n \Nm(\omega_i) \leq D^n \prod\limits_{i = 1}^n l(\omega_i) \leq D^n l(\ideal{n})^n = N l(\ideal{n})^n.
%     \end{equation}
    
%     Нетрудно видеть, что рассмотренные операции могут быть выполнены за $\tilde{O}(\log L)$ бинарных операций.
% \end{proof}

% \begin{corollary}
%     Пусть $k \in \mathbb{N}$ и $a,$ $b,$ $ \in \mathcal{O}_K^*$, $\ideal{n}$ -- нетривиальный идеал заданный с помощью $\mathbb{Z}$-представления. Пусть выполнено $l(a)\leq L,$ $l(b)\leq L$, $l(\ideal{n})\leq L.$ Элементы $z_1,$ $z_2\in \mathcal{O}_K$ такие что $a+b\equiv z_1(\modul\,\ideal{n}),$ $a^k\equiv z_2(\modul\,\ideal{n}),$ $l(z_1)\leq Nl(\ideal{n})^n$, $l(z_2)\leq Nl(\ideal{n})^n$, могут быть определены за $\tilde{O}(\log L)$, $\tilde{O}(\log k \log L)$  бинарных операций соответственно.
% \end{corollary}

% \begin{remark}
%     Отметим, что все указанные операции могут быть выполнены за полиномиальное время в случае, когда идеалы заданы с помощью одного из представлений: $\mathbb{Z}$-представление, 2-представление, нормальная эрмитовая форма; в силу того, что из одного представления может быть получено другое за полиномиальное время.
% \end{remark}

\subsection{Вспомогательные утверждения}

\begin{statement}[Теорема Копперсмита]\label{statement:coppersmith}
  Пусть
  \begin{equation*}
      f(x, y) = \sum\limits_{i, j = 0}^{\delta} p_{i, j} x^i y^j
  \end{equation*}
  неприводимый многочлен от двух переменных над $\mathbb{Z}$.
  Пусть $X, Y \ge 0$ такие, что $|x_0| \le X$ и $|y_0| \le Y$, где $(x_0, y_0)$ решение уравнения $f(x, y) = 0$.
  Обозначим
  \begin{equation*}
      W = \max_{i, j} |p_{i, j}| X^i Y^j.
  \end{equation*}
  Пусть $XY < W^{\frac{3}{2\delta}}$, то существует полиномиальный относительно $\log W$ и $2^\delta$ алгоритм, который позволяет найти такую пару $(x_0, y_0)$, что $f(x_0, y_0) = 0$, $|x_0| \le X$ и $|y_0| \le Y$.
\end{statement}

\begin{statement}\label{statement:cauchy}(Теорема Коши)
    Если порядок конечной группы $G$ делится на простое число $p$, то $G$ содержит элементы порядка $p$.
\end{statement}

\begin{statement}\label{statement:lagrange}(Теорема Лагранжа)
    Пусть группа $G$ конечна, и $H$ -- её подгруппа.
    Тогда порядок $G$ равен порядку $H$, умноженному на индекс подгруппы.
\end{statement}

\onlyinsubfile{
    \subfile{_10_bibliography}
    \subfile{_11_pub}
}

\end{document}
