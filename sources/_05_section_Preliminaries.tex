\documentclass[_00_dissertation.tex]{subfiles}
\begin{document}

\onlyinsubfile{
    \renewcommand{\contentsname}{ОГЛАВЛЕНИЕ}
    \setcounter{tocdepth}{3}
    \tableofcontents
}

\newpage
\begin{center}
    \refstepcounter{section}
    \section*{ГЛАВА \arabic{section}.\\ ПРЕДВАРИТЕЛЬНЫЕ СВЕДЕНИЯ}\label{ch:Prelimiaries}
    \addcontentsline{toc}{chapter}{ГЛАВА \arabic{section}. ПРЕДВАРИТЕЛЬНЫЕ СВЕДЕНИЯ}
\end{center}

Пусть $R$ -- дедекиндово кольцо.
Нормой $\Nm{\ideal{n}}$ идеала $\ideal{n} \subset R$ называется мощность факторкольца $R/\ideal{n}$.

% \begin{definition}
%     Идеал $\mathfrak{p}$ называется простым, если выполняются следующие два условия

%     \begin{enumerate}
%         \item Для любых двух $a, b\in R$, если $ab\in \mathfrak{p}$, то $a\in\mathfrak{p}$ или $b\in\mathfrak{p}$.

%         \item Идеал $\mathfrak{p}$ не совпадает с $R$.
%     \end{enumerate}
% \end{definition}

% \begin{definition}
%     Идеал $\mathfrak{m}$ называется максимальным, если выполняются следующие условия

%     \begin{enumerate}
%         \item Это собственный идеал, т.е. $\mathfrak{m}$ собственное подмножество $R$.

%         \item Не существует собственных идеалов, для которых $\mathfrak{m}$ является собственным подмножеством.
%     \end{enumerate}
% \end{definition}

Идеал $\ideal{n} \subset R$ называется простым, если факторкольцо $R/\ideal{n}$ является областью целостности.
Собственный идеал $\ideal{n} \subset R$ называется максимальным, если он не содержится ни в каком другом собственном идеале.
Любой максимальный идеал является простым.

Далее предполагаем, что для любого  максимального идеала $\ideal{n} \subset R$ факторкольцо $R/\ideal{n}$ конечно.

Для любого кольца $K$ через $\multiplicative{K}$ обозначим множество обратимых элементов кольца $K$ с нулем.

\begin{definition}
    Функцией Эйлера идеала $\ideal{n} \subset R$ называется функция
    \begin{equation*}
        \varphi(\ideal{n}) = |\multiplicative{(R/\ideal{n})}|.
    \end{equation*}
\end{definition}

% \begin{statement}\cite{source:Petukhova}
%     Для функции Эйлера в абстрактном числовом кольце выполнено свойство мультипликативности:
%     \begin{eqnarray}
%         \varphi\left(
%             \prod\limits_{i=1}^k \mathfrak{m}^{\alpha_i}
%         \right) = \prod\limits_{i=1}^k \varphi(\mathfrak{m}^{\alpha_i})\nonumber
%     \end{eqnarray}
    
%     Для максимального идеала $\mathfrak{m}$ значение функции Эйлера можно вычислить по формуле:
%     \begin{eqnarray}
%         \varphi(\mathfrak{m}^{\alpha}) = \mathcal{N}(\mathfrak{m})^{\alpha} - \mathcal{N}(\mathfrak{m})^{\alpha-1}.\nonumber
%     \end{eqnarray}
% \end{statement}

% \begin{statement}[Обобщенная теорема Эйлера]\cite{Petukhova}
%   Пусть $R$ --- дедекиндово кольцо, обладающее свойствами, описанными выше.
%   Пусть $m\in R$, $\mathfrak{U}\subset R$ --- идеал.
%   Если $Rm+\mathfrak{U} = R$, то $m^{\varphi(\mathfrak{U})}\equiv 1(\modul \mathfrak{U})$.
% \end{statement}

\begin{statement}\cite{Narkiewicz}\label{statement:euler_function}
    Пусть $\ideal{n}$ -- идеал кольца $R$.
    Тогда
    \begin{equation*}
        \varphi(\ideal{n}) = \Nm{\ideal{n}} \prod_{\ideal{p} | \ideal{n}} \left(
            1 - \frac{1}{\Nm{\ideal{p}}}
        \right),
    \end{equation*}
    где произведение берется по всем простым идеалам $\ideal{p}$, делящим $\ideal{n}$.
    Если $x \in R$ и $(xR, \ideal{n}) = 1$, то
    \begin{equation*}
        x^{\varphi(\ideal{n})} \equiv 1 (\modul \ideal{n})
    \end{equation*}
\end{statement}

\begin{statement}\label{statement:cauchy}(Теорема Коши)
    Если порядок конечной группы $G$ делится на простое число $p$, то $G$ содержит элементы порядка $p$.
\end{statement}

\begin{statement}\label{statement:lagrange}(Теорема Лагранжа)
    Пусть группа $G$ конечна, и $H$ -- её подгруппа.
    Тогда порядок $G$ равен порядку $H$, умноженному на индекс подгруппы.
\end{statement}

\begin{statement}\label{statement:chinese_remainder_theorem}
    Пусть $\ideal{n}_1, \ideal{n}_2, \dots, \ideal{n}_k$ -- попарно взаимнопростые идеалы кольца $R$.
    Тогда
    \begin{equation*}
        \begin{split}
            R/(\ideal{n}_1\ideal{n}_2\dots\ideal{n}_k) \cong & (R/\ideal{n}_1) \times (R/\ideal{n}_2) \times \dots \times (R/\ideal{n}_k)\\
            \multiplicative{(R/(\ideal{n}_1\ideal{n}_2\dots\ideal{n}_k))} \cong & \multiplicative{(R/\ideal{n}_1)} \times \multiplicative{(R/\ideal{n}_2)} \times \dots \times \multiplicative{(R/\ideal{n}_k)}
        \end{split}
    \end{equation*}
\end{statement}

\onlyinsubfile{
    \subfile{_10_bibliography}
    \subfile{_11_pub}
}

\end{document}
