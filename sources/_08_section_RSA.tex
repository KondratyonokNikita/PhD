\documentclass[_00_dissertation.tex]{subfiles}
\begin{document}

\onlyinsubfile{
    \renewcommand{\contentsname}{ОГЛАВЛЕНИЕ}
    \setcounter{tocdepth}{3}
    \tableofcontents
}

\newpage
\begin{center}
    \refstepcounter{section}
    \section*{ГЛАВА \arabic{section}.\\ АНАЛОГ RSA-КРИПТОСИСТЕМЫ В ДЕДЕКИНДОВЫХ КОЛЬЦАХ}\label{ch:RSA-cryptosystem}
    \addcontentsline{toc}{chapter}{ГЛАВА \arabic{section}. АНАЛОГ RSA-КРИПТОСИСТЕМЫ В ДЕДЕКИНДОВЫХ КОЛЬЦАХ}
\end{center}

\subsection{Предварительные сведения}

\begin{definition}
    Говорят, что дедекиндово кольцо $R$ имеет конечное поле остатков, если $|R/\ideal{m}| < \infty$ для любого максимального ненулевого идеала $\ideal{m} \subseteq R$.
\end{definition}

Далее в этой главе полагаем, что дедекиндово кольцо $R$ имеет конечное поле остатков.

\begin{example}
    Рассмотрим примеры дедекиндовых колец, удовлетворяющих этим условиям.
    \begin{itemize}
        \item Пусть $K$ -- числовое поле.
        Кольцо $\mathbb{Z}_K$, образованное алгебраическими целыми элементами этого поля является дедекиндовым с конечным полем остатков.
        Частными случаями этого примера являются кольцо целых чисел и гауссовых чисел.
        
        \item Пусть $f(x, y) = y - mx - b$ -- прямая.
        Тогда $K[x, y]/(f(x, y)) \cong K[x]$.
        Следовательно, это координатное кольцо является факториальным.
        
        \item Пусть $f(x, y) = y - x^2$ -- парабола.
        Тогда $K[x, y]/(f(x, y)) \cong K[x]$.
        Следовательно, это координатное кольцо является факториальным.
    
        \item Пусть $f(x, y) = x^2 + y^2 - 1$.
        Если $K = \mathbb{Q}$, то координатное кольцо $\mathbb{Q}[x, y]/(f)$ не изоморфно $K[x]$, так как первое не является факториальным кольцом.
        Это можно показать, рассмотрев элементы $y^2 = yy$ и $1-x^2 = (1-x)(1+x)$.
        Однако, если $K = \mathbb{C}$, то координатное кольцо $\mathbb{C}[x, y]/(f) \cong \mathbb{C}[x, x^{-1}]$ уже будет факториальным, так как это локализация факториального кольца.
    \end{itemize}
\end{example}

% Пусть $\mathcal{O}_K$ -- кольцо алгебраических целых чисел числового поля $K$.
% Известно, что $\mathcal{O}_K$ является абстрактным числовым кольцом.
% Обозначим через $\mathcal{O}_K^\times$ группу единиц кольца $\mathcal{O}_K$, через $\mathcal{O}_{K, \mathfrak{m}} = \mathcal{O}_K/\mathfrak{m}$ аддитивную группу вычетов по модулю $\mathfrak{m}$ и через $\mathcal{O}_{K, \mathfrak{m}}^\times$ мультипликативную группу вычетов по модулю $\mathfrak{m}$.
% Пусть задана некоторая норма $\textrm{Nm}(m)$, где $m\in\mathcal{O}_K$.

% \begin{statement}
%     Пусть $m\in\mathcal{O}_K$.
%     Тогда $\mathcal{N}((m)) = |\Nm{m}|$, а так же $(m)|(\Nm{m})$.
% \end{statement}

\begin{statement}[Теорема Копперсмита]\label{statement:coppersmith}
  Пусть
  \begin{equation*}
      f(x, y) = \sum\limits_{i, j} p_{i, j} x^i y^j
  \end{equation*}
  неприводимый многочлен от двух переменных над $\mathbb{Z}$ со степенью $\delta$ по каждой переменной отдельно.
  Пусть $X$, $Y$ границы предполагаемого решения $(x_0, y_0)$.
  Обозначим
  \begin{equation*}
      W = \max_{i, j} |p_{i, j}| X^i Y^j.
  \end{equation*}
  Пусть $XY < W^{\frac{3}{2\delta}}$, то существует полиномиальный относительно $\log W$ и $2^\delta$ алгоритм, который позволяет найти такую пару $(x_0, y_0)$, что $f(x_0, y_0) = 0$, $|x_0| \le X$ и $|y_0| \le Y$.
\end{statement}

\subsection{Анализ аналога RSA-криптосистемы}

Изложенный далее алгоритм аналога RSA-криптосистемы был предложен в работе Петуховой и Тронина~\cite{source:Petukhova}.
Была показана корректность полученной криптосистемы и представлены ограничения  на кольцо для ее эффектитвного применения.
В этой части исследуется RSA-криптосистема в дедекиндовых кольцах с конечным полем остатков.
Целью является получение доказательств теорем, связанных с ее криптостойкостью.
Например теоремы Винера, теоремы об эквивалентности факторизации и взлома криптосистемы, а так же изучение методов взлома криптосистемы.

\begin{algorithm}\label{algorithm:RSA_in_dedekind}
    Аналог RSA-криптосистемы в дедекиндовых кольцах.

    \begin{enumerate}
        \item Выбираются максимальные идеалы $\ideal{p}$, $\ideal{q}\in R$

        \item Вычисляется $\varphi(\ideal{N}),$ где $\ideal{N} = \ideal{p} \ideal{q}$

        \item Выбирается случайное целое $e \in [1, \varphi(\ideal{N})],$ $(e, \varphi(\ideal{N}))=1$

        \item Вычисляется целое положительное $d$ такое, что $ed \equiv 1 \pmod{\varphi(\ideal{N})}$
    \end{enumerate}

    Пара $(\ideal{N}, e)$ это публичный ключ $A$, пара $(\ideal{N}, d)$ секретный ключ $A$.
    Функцией шифрования называется

    \begin{equation*}
        \begin{array}{c}
            f: R/\ideal{N} \to R/\ideal{N},\\
            f(x) \equiv x^{e} \pmod{\varphi(\ideal{N})}.
        \end{array}
    \end{equation*}

    Функцией дешифровки называется

    \begin{equation*}
        \begin{array}{c}
            f^{-1}: R/\ideal{N} \to R/\ideal{N},\\
            f^{-1}(x) \equiv x^{d}\pmod{\varphi(\ideal{N})}.
        \end{array}
    \end{equation*}
\end{algorithm}

\begin{remark}
    Корректность приведенной криптосистемы гарантируется аналогом теоремы Эйлера для абстрактных числовых колец.
\end{remark}

\begin{remark}
    В работе \cite{Petukhova} описаны условия, при которых применение алгоритма шифрования в абстрактном числовом кольце будет возможно на компьютере.
    Необходимо, чтобы для некоторых максимальных идеалов $\ideal{M}_1, \ideal{M}_2 \in R$ и $\ideal{U} = \ideal{M}_1\ideal{M}_2$ было известно множество $W \subset R$ такое, что:
    \begin{enumerate}
        \item Любое подмножество из $R/\ideal{U}$ пересекается с $W$ в ровно одном элементе.
        
        \item Есть вычислимая биекция между $W$ и $\{1, \dots, T\}$ для некоторого $T$.
    \end{enumerate}
\end{remark}

Нетрудно заметить, что зная разложение на множители $\ideal{N}=\ideal{p}\ideal{q}$ для RSA-модуля можно эффективно найти секретный ключ.
В некоторых случаях можно доказать обратное утверждение.

\begin{theorem}\label{theorem:factor}\ref{source:BSU_Journal_2020}
    Пусть $K$ -- числовое поле и $\mathbb{Z}_K$ его кольцо целых алгебраических чисел.
    Пусть $\mathbb{Z}_K$ -- кольцо с единственной факторизацией, $((N), e, d)$ параметры RSA-криптосистемы в $\mathbb{Z}_K$.
    Если $d$ известно, то $N$ можно эффективно разложить на множители с вероятностью не менее $\frac{1}{2}$ за полиномиальное относительно $\log \Nm{(N)}$ количество арифметических операций в $\mathbb{Z}_K$.
\end{theorem}
\begin{proof}
    Пусть $s = ed - 1 = 2^t u$, где $t, u \in \mathbb{N}$ и $u$ нечетное.
    Так как $\varphi((N)) | s$, то $x^s \equiv 1 \pmod{(N)}$ для всех $x \in \mathbb{Z}_K$.
    
    Обозначим $\mathcal{S}_N$ множество таких элементов $x \in \invertible{\mathbb{Z}_N / (N)}$, что $x^u \equiv 1 \pmod{(N)}$ или существует $j \in \{0, \dots, t-1\}$, для которого $a^{2^j u} \equiv -1 \pmod{(N)}$.
    Пусть $A = \invertible{\mathbb{Z}_K} \setminus \mathcal{S}_N$.
    
    Рассмотрим произвольный элемент $a \in A$ и выберем наименьшее $j \in \mathbb{N}$, что $a^{2^j u} \equiv 1 \pmod{(N)}$.
    Пусть $b = a^{2^{j-1} u} \pmod{(N)}$.
    Из того, что $b^2 \equiv 1 \pmod{(N)}$ и $b \not\equiv \pm 1 \pmod{(N)}$ следует, что $(b - 1, N)$ -- собственный делитель $N$.
    
    В работе Викстрома~\cite{source:Wikstrom} показано, что наибольший общий делитель $(b - 1, N)$ можно вычислить за полиномиальное относительно $\log \Nm{(N)}$ число арифметических операций в $\mathbb{Z}_K$.
    
    Пусть $(N) = (p)(q)$, где $p, q$ -- простые элементы в $\mathbb{Z}_K$.
    Положим $\varphi((p)) = 2^{v_1} u-1$, $\varphi((q)) = 2^{v_2} u_2$, где $v_1, v_2, u_1, u_2 \in \mathbb{N}$, $u_1, u_2$ -- нечетные.
    Обозначим $v = \min\{v_1, v_2\}$ и $K = (u, u_1)(u, u_2)$.
    
    Из свойств сравнений следует, что
    \begin{equation*}
        x^u \equiv 1 \pmod{(N)}
        \Leftrightarrow
        \begin{cases}
            u \log_{\alpha} x \equiv 0 \pmod{\varphi((p))}\\
            u \log_{\beta} x \equiv 0 \pmod{\varphi((q))}
        \end{cases},
    \end{equation*}
    где $\alpha$ и $\beta$ примитивные элементы в $\invertible{\mathbb{Z}_K / (p)}$ и $\invertible{\mathbb{Z}_K / (q)}$ соответственно.
    Следовательно, это сравнение имеет $K$ решений.
    
    Рассмотрим сравнение $x^{2^j u} \equiv -1 \pmod{(N)}$, где $j \in \{0, \dots, t - 1\}$.
    Если $j \ge v$, то решений нет.
    Если $j < v$, то количество решений равно $4^j K$.
    
    Следовательно, получаем, что
    \begin{equation*}
        |\mathcal{S}_N| = \left(
            1 + 1 + 4 + \dots + 4^{v-1}
        \right)K = \frac{4^v + 2}{3}K.
    \end{equation*}
    Тогда
    \begin{equation*}
        \frac{|\mathcal{S}_N|}{\varphi((N))} \le \frac{\frac{4^v + 2}{3}K}{4^v K} \le \frac{1}{2}.
    \end{equation*}
\end{proof}

Следующая теорема является аналогом теоремы Винера о малой секретной экспоненте \cite{source:Wiener}.

\begin{theorem}\label{theorem:Wiener}\ref{source:BSU_Journal_2020}
    Пусть $(\ideal{N},e,d)$, $\ideal{N}=\ideal{p} \ideal{q}$ -- параметры RSA-криптосистемы в дедекиндовом кольце $R$.
    Пусть $\Nm{\ideal{q}} < \Nm{\ideal{p}} < \alpha^2 \Nm{\ideal{q}},$ где $\alpha > 1.$
    Если $d<\frac{1}{\sqrt{2\alpha+2}}(\Nm{\ideal{N}})^{1/4},$ то $d$ можно эффективно вычислить за полиномиальное относительно $\log \Nm{\ideal{N}}$ число бинарных операций.
\end{theorem}
\begin{proof}
    Пусть $ed - 1 = k \varphi(\ideal{N})$.
    Так как $\Nm{\ideal{p}} + \Nm{\ideal{q}} < (\alpha + 1)\sqrt{\Nm{\ideal{N}}}$, то
    \begin{equation*}
        \Nm{\ideal{N}} - \varphi(\ideal{N}) = \Nm{\ideal{p}} + \Nm{\ideal{q}} - 1 < (\alpha + 1)\sqrt{\Nm{\ideal{N}}}.
    \end{equation*}
    
    Так как $k \varphi(\ideal{N}) < ed$ и $e < \varphi(\ideal{N})$, то $k < d$ и
    \begin{equation*}
        \frac{(\alpha + 1)k}{d\sqrt{\Nm{\ideal{N}}}} < \frac{\alpha + 1}{\sqrt{\Nm{\ideal{N}}}} < \frac{1}{2 d^2}.
    \end{equation*}
    
    Тогда получаем
    \begin{equation*}
        \begin{split}
            \left|
                \frac{e}{\Nm{\ideal{N}}} - \frac{k}{d}
            \right| = \left|
                \frac{1 - k(\Nm{\ideal{N}} - \varphi(\ideal{N}))}{\Nm{\ideal{N}}d}
            \right| \le \left|
                \frac{k(\Nm{\ideal{N}} - \varphi(\ideal{N}))}{\Nm{\ideal{N}}d}
            \right| \le \\
            \le \left|
                \frac{k(\alpha + 1)\sqrt{\Nm{\ideal{N}}}}{\Nm{\ideal{N}}d}
            \right| < \frac{1}{2 d^2}.
        \end{split}
    \end{equation*}
    Следовательно, $\frac{k}{d}$ -- подходящая цепная дробь для дроби $\frac{e}{\Nm{\ideal{N}}}$, которая не является секретной.
    Тогда $\frac{k}{d}$ можно вычислить, используя алгоритм Евклида в $\mathbb{Z}$ за полиномиальное относительно $\log \Nm{\ideal{N}}$ число бинарных операций.
\end{proof}

\begin{remark}
    Доказанная выше теорема является основой для атаки Винера на RSA-криптосистему.
    При соблюдении определенных условий на параметры криптосистемы, можно сделать использование этой атаки невозможным.
    Однако существуют атаки, от которых невозможно полностью защититься.
    
    Метод повторного шифрования является примером такой атаки.
    Предположим, что было перехвачено некоторое зашифрованное сообщение $y = x^e \pmod{\ideal{N}}$, где $x \in \mathbb{Z}_{K} / \ideal{N}$ -- некоторое сообщение.
    Построим последовательность $y_i = y^{e^i} \pmod{\ideal{N}}$, где $i \in \{1, 2, \ldots\}$.
    Используя свойства возведения в степень и то, что $\mathbb{Z}_{K} / \ideal{N}$ конечно, получаем, что существует таое $m \in \mathbb{N}$, что $y_m = y$.
    Тогда $y_{m-1} = x$.
    
    Единственный способ защиты от этого метода взлома состоит в том, чтобы сделать $m$ достаточно большим.
\end{remark}

Обозначим через $R_{\mathfrak{m}}$ и $R_{\mathfrak{m}}^{\times}$ аддитивную и мультипликативную группы вычетов по модулю $\mathfrak{m}$.
Заметим, что если $\mathfrak{m}=\mathfrak{m}_1 \mathfrak{m}_2$, то $R_{\mathfrak{m}}^{\times} \cong R_{\mathfrak{m}_1}^{\times} \times R_{\mathfrak{m}_2}^{\times}$.

\begin{theorem}\label{theorem:iterated}
    Пусть $\ideal{N} = \ideal{p} \ideal{q}$ -- модуль RSA-криптосистемы в дедекиндовом кольце $R$.
    Предположим, что существуют различные простые числа $r$, $s$ и положительные целые числа $k$, $l$ такие, что $\varphi(\ideal{p}) = rk$, $\varphi(\ideal{q}) = sl$ и числа $r - 1$, $s - 1$ имеют различные простые делители $r_1$, $s_1$ соответственно.

    Пусть $y$ и $e$ -- независимые равномерно распределенные случайные величины со значениями в $R / \ideal{N}$ и $\invertible{\mathbb{Z}_{\varphi(\ideal{N})}}$ соответственно.
    Обозначим
    \begin{equation*}
        m_{e,y} = \min \{m \in \mathbb{N} | y_m = y\}.
    \end{equation*}
    Тогда выполняется неравенство
    \begin{equation*}
        P(m_{e,y} \ge r_1s_1)\ge(1-r^{-1})(1-s^{-1})(1-r_1^{-1})(1-s_1^{-1}).
    \end{equation*}
\end{theorem}
\begin{proof}
    Оценим вероятность $P\{rs | \textrm{ord}_{\invertible{R / \ideal{N}}}(y)\}$.

    Так как $\invertible{R / \ideal{N}} \cong \invertible{R / \ideal{p}} \times \invertible{R / \ideal{q}}$ и группы $\invertible{R / \ideal{p}}$, $\invertible{R / \ideal{q}}$ циклические, то можно записать $y = (\alpha^i, \beta^j),$ где $\alpha$ и $\beta$ примитивные элементы $\invertible{R / \ideal{p}}$ и $\invertible{R / \ideal{q}}$ соответственно, $i$ и $j$ случайные величины со значениями в $\{1, \ldots, rk\}$ и $\{1, \ldots, sl\}$ соответственно.
    
    Следовательно, получаем, что $\textrm{ord}_{\invertible{R / \ideal{N}}}(y) = \textrm{lcm}\left(\frac{rk}{(rk,i)}, \frac{sl}{(sl,j)}\right)$.
    Если $r \nmid i$ и $s \nmid j$ то $\textrm{ord}_{\invertible{R / \ideal{N}}}(y) \vdots rs$.
    Таким образом получаем
    \begin{equation*}
        P\left\{
            rs \big| \textrm{ord}_{\invertible{R / \ideal{N}}}(y)
        \right\} \ge P\left\{
            r\nmid i, s \nmid j
        \right\} = \frac{\varphi(r)k\varphi(s)l}{rksl} = \left(
            1-\frac{1}{r}
        \right)\left(
            1-\frac{1}{s}
        \right).
    \end{equation*}

    Так как $e \in \invertible{\mathbb{Z}_{rs}}$, то аналогично можно получить неравенство
    \begin{equation*}
        P\left\{
            r_1s_1|\textrm{ord}_{\mathbb{Z}^*_{rs}}(e)
        \right\} \ge \left(
            1-\frac{1}{r_1}
        \right)\left(
            1-\frac{1}{s_1}
        \right).
    \end{equation*}

    Легко заметить, что $\textrm{ord}_{\invertible{R / \ideal{N}}}(y) | (e^{m_{e,y}}-1),$ следовательно, имеем
    \begin{equation*}
        \left\{
            rs | \textrm{ord}_{\invertible{R / \ideal{N}}}(y)
        \right\} \subseteq \left\{
            \textrm{ord}_{\mathbb{Z}^*_{rs}}(e) | m_{e,y}
        \right\}.
    \end{equation*}

    Следовательно, получаем
    \begin{equation*}
        \begin{split}
            P\left\{
                m_{e,y} \ge r_1s_1
            \right\} \ge P\left\{
                r_1s_1 | m_{e,y}
            \right\} \ge P\left\{
                r_1s_1|\textrm{ord}_{\mathbb{Z}^*_{rs}}(e), rs|\textrm{ord}_{\invertible{R / \ideal{N}}}(y)
            \right\} \ge \\
            \ge (1-r^{-1})(1-s^{-1})(1-r_1^{-1})(1-s_1^{-1}).
        \end{split}
    \end{equation*}
\end{proof}

\begin{theorem}\label{theorem:d_is_known_2}
    Пусть $(\ideal{N}, e, d)$ параметры RSA-криптосистемы в дедекиндовом кольце $R$, где $\Nm{\ideal{p}}$ и $\Nm{\ideal{q}}$ имеют одинаковую битовую длину.
    Пусть $ed \le (\Nm{\ideal{N}})^2$, $\Nm{\ideal{N}} \ge 3$.
    Если $d$ известно, то существует эффективный алгоритм, который позволяет найти $\Nm{\ideal{p}}$ и $\Nm{\ideal{q}}$.
\end{theorem}
\begin{proof}
    Пусть $ed - 1 = k\varphi(\ideal{N})$.

    Не нарушая общности, предположим, что $\Nm{\ideal{p}} \le \Nm{\ideal{q}}$.
    Тогда $\Nm{\ideal{p}} \le (\Nm{\ideal{N}})^{1/2} \le \Nm{\ideal{q}} < 2\Nm{\ideal{p}} \le 2(\Nm{\ideal{N}})^{1/2}$.
    Следовательно,
    \begin{equation*}
        \Nm{\ideal{N}} - \varphi(\ideal{N}) = \Nm{\ideal{p}} + \Nm{\ideal{q}} - 1 < 3(\Nm{\ideal{N}})^{1/2} < 3(\Nm{\ideal{N}})^{1/2} \le \frac{\Nm{\ideal{N}}}{2}.
    \end{equation*}

    Тогда $\varphi(\ideal{N}) \in [\Nm{\ideal{N}}-3(\Nm{\ideal{N}})^{1/2}, \Nm{\ideal{N}}]$.
    Разделим этот интервал на $6$ интервалов длины $\frac{1}{2}(\Nm{\ideal{N}})^{1/2}$ с центрами в точках $\Nm{\ideal{N}} - \frac{2i-1}{4}(\Nm{\ideal{N}})^{1/2}$, где $i \in \{1, \ldots, 6\}$.
    Рассмотрим $i \in \{1, \ldots, 6\}$ такое, что
    \begin{equation*}
        \left|
            \Nm{\ideal{N}}-\frac{2i-1}{4}(\Nm{\ideal{N}})^{1/2}-\varphi(\ideal{N})
        \right| \le \frac{1}{4}(\Nm{\ideal{N}})^{1/2}.
    \end{equation*}
  
    Обозначим
    \begin{equation*}
        g = \left\lceil
            \frac{2i-1}{4}\Nm{\ideal{N}}
        \right\rceil
    \end{equation*}

    Тогда
    \begin{equation*}
        \left|
            \Nm{\ideal{N}} - g - \varphi(\ideal{N})
        \right| < \frac{1}{4}(\Nm{\ideal{N}})^{1/2} + 1.
    \end{equation*}

    Обозначим $\overline{k} = \frac{ed-1}{\Nm{\ideal{N}}}$.
    Тогда

    \begin{equation*}
        \begin{split}
            k - \overline{k} =
            \frac{(\Nm{\ideal{N}}-\varphi(\ideal{N}))(ed-1)}{\Nm{\ideal{N}}\varphi(\ideal{N})} =\\
            = \frac{(\Nm{\ideal{p}}+\Nm{\ideal{q}}-1)(ed-1)}{\Nm{\ideal{N}}\varphi(\ideal{N})} < \frac{3(\Nm{\ideal{N}})^{1/2}(ed-1)}{\Nm{\ideal{N}}\frac{\Nm{\ideal{N}}}{2}} =\\
            = 6(\Nm{\ideal{N}})^{-3/2}(ed-1)
        \end{split}
    \end{equation*}
  
    Рассмотрим многочлен
    \begin{equation*}
        \begin{split}
            f(x,y) = ed - 1 - (\lceil\overline{k}\rceil + x)(\Nm{\ideal{N}} - g - y) =\\
            = xy - (\Nm{\ideal{N}} - g)x + \lceil\overline{k}\rceil y - \lceil\overline{k}\rceil(\Nm{\ideal{N}} - g) + ed - 1.
        \end{split}
    \end{equation*}
    
    Заметим, что пара $(x_0, y_0) = (k - \lceil\overline{k}\rceil, \Nm{\ideal{N}} - g - \varphi(\ideal{N}))$ является его корнем.
    
    Воспользуемся теоремой Копперсмита.
    Положим $\delta = 1$.
    Заметим, что
    \begin{equation*}
        \begin{split}
            |x_0| = \left|
                k - \lceil\overline{k}\rceil
            \right| \le \left|
                k - \overline{k}
            \right| < 6(\Nm{\ideal{N}})^{-3/2}(ed-1) < \\
            < 6(\Nm{\ideal{N}})^{-3/2}(\Nm{\ideal{N}})^2 = 6(\Nm{\ideal{N}})^{1/2}\\
            |y_0| = \left|
                \Nm{\ideal{N}} - g - \varphi(\ideal{N})
            \right| \le \frac{1}{4}(\Nm{\ideal{N}})^{1/2} + 1.
        \end{split}
    \end{equation*}
    
    Положим $X = 6(\Nm{\ideal{N}})^{1/2}$, $Y = \frac{1}{4}(\Nm{\ideal{N}})^{1/2} + 1$.
    Тогда $|x_0| < X$, $|y_0| < Y$.
    Из определения $W$ следует, что
    \begin{equation*}
        \begin{split}
            W \ge (\Nm{\ideal{N}} - g)X = 6(\Nm{\ideal{N}})^{1/2}\left(
                \Nm{\ideal{N}} - \left\lceil
                    \frac{2i-1}{4}\Nm{\ideal{N}}
                \right\rceil
            \right) >\\
            > \frac{3}{2}(\Nm{\ideal{N}})^{3/2}
        \end{split}
    \end{equation*}
    
    Для $\Nm{\ideal{N}} \ge 3$ выполнено
    \begin{equation*}
        \begin{split}
            (XY)^{2/3} = \left(
                6(\Nm{\ideal{N}})^{1/2} \left(
                    \frac{1}{4}(\Nm{\ideal{N}})^{1/2} + 1
                \right)
            \right)^{2/3} =\\
            = \left(
                \frac{3}{2}\Nm{\ideal{N}} + 6(\Nm{\ideal{N}})^{1/2}
            \right)^{2/3} < \frac{3}{2}(\Nm{\ideal{N}})^{3/2}.
        \end{split}
    \end{equation*}
    
    Следовательно, если $\Nm{\ideal{N}} \ge 3$, то $W^{\frac{3}{2}} > XY$.
    Тогда, используя теорему Копперсмита, мы можем найти корень $(x_0, y_0)$ за полиномиальное относительно $\log W$ время.
    Корень $(x_0, y_0)$ позволяет найти $\Nm{\ideal{p}}$ и $\Nm{\ideal{q}}$.
\end{proof}

% результат из Конкурс_студ_работ_по_теор_инф_и_диск_мат_им_Алана_Тьюринга
\begin{remark}
    Если в условии теоремы~\ref{theorem:d_is_known_2} заменить неравенство $ed \le (\Nm{\ideal{N}})^2$ на более строгое $ed \le (\Nm{\ideal{N}})^{3/2}$, то получим, что
    \begin{equation*}
        k - \overline{k} < 6(\Nm{\ideal{N}})^{-3/2}(ed-1) < 6.
    \end{equation*}
    
    Следовательно, вычислив $\overline{k} = \frac{ed-1}{\Nm{\ideal{N}}}$, можно перебрать все возможные $k$ и для каждого вычислить $\varphi(\ideal{N})$, $\Nm{\ideal{p}}$, $\Nm{\ideal{q}}$.
\end{remark}

\begin{theorem}
    Пусть дедекиндово кольцо $R$ является евклидовым относительно некоторой нормы $\upsilon(\cdot)$ и $\Lambda_{R} < 1$, где $\Lambda_{R}$ задано в определении~\ref{definition:euclidean_lambda}.
    Тогда это кольцо главных идеалов.
    Для простоты будем обозначать идеалы соответствующими элементами кольца.

    Пусть $(N, e_1, d_1)$ и $(N, e_2, d_2)$ параметры RSA-криптосистемы в $R$ и $(e_1, e_2) = 1$.
    Пусть перехвачены сообщения $c_1 \equiv m^{e_1} \pmod{N}$ и $c_2 \equiv m^{e_2} \pmod{N}$.
    Тогда сообщение $m$ можно вычислить за полиномиальное относительно $\log \upsilon(N)$ количество арифметических операций в $R$.
\end{theorem}
\begin{proof}
    Условие $(e_1, e_2) = 1$ означает, что существуют $s_1, s_2 \in \mathbb{Z}$ такие, что $e_1 s_1 + e_2 s_2 = 1$.
    Это означает, что можно вычислить
    \begin{equation*}
        c_1^{s_1}c_2^{s_2} = (m^{e_1})^{s_1} (m^{e_2})^{s_2} = m^{e_1 s_1 + e_2 s_2} = m
    \end{equation*}

    Так как $e_1, e_2 > 0$, то одно из чисел $s_1$, $s_2$ будет отрицательным.
    Не нарушая общности, предположим, что $s_2 < 0$.
    Тогда можно записать $c_2^{s_2} = (c_2^{-1})^{|s_2|}$.
    Надо вычислить $c_2^{-1} \pmod{N}$.
    Это эквивалентно нахождению решения уравнения $c_2 x + N y = (c_2, N)$.

    Если $(c_2, N) \neq 1$, то известно разложение $N$ на простые множители и сообщение $m$ можно вычислить стандартным способом.
    Предположим, что $(c_2, N) = 1$, тогда надо решить уравнение $c_2 x + N y = 1$.
    Решить это уравнение можно с помощью цепочек деления.
    Из условия этой теоремы и теоремы~\ref{theorem:euclidean_and_lambda} следует, что
    \begin{equation*}
        l_{n}(R) \le \left[
            \log_{\Lambda_{R}^{-1}} n
        \right] + 2.
    \end{equation*}

    Это означает, что длина цепочки с выбором минимального по норме остатка для $c_2$ и $N$ ограничена логарифмом от их модуля.
    Следовательно, ее можно вычислить за полиномиальное относительно $\log \upsilon(N)$ количество арифметических операций в $R$.
    Из построения RSA-криптосистемы следует, что $e_1 < \upsilon(N)$ и $e_2 < \upsilon(N)$, следовательно сообщение $m$ можно вычислить за полиномиальное относительно $\log \upsilon(N)$ количество арифметических операций в $R$.
\end{proof}

\begin{remark}
    В работе \cite{source:Vaskouski_CSIST} в доказательстве предложения 1 показано, что $\Lambda_{R} < 1$ во всех квадратичных норменно-евклидовых кольцах.
\end{remark}

Приведем примеры работы криптосистемы для некоторых координатных колец.

\begin{example}
	Рассмотрим кольцо многочленов от двух переменных над $\mathbb{Z}_2$.
	Рассмотрим координатное кольцо
	\begin{equation*}
		R = \frac{\mathbb{Z}_2(x, y)}{y-x} \cong \mathbb{Z}_2[x]
	\end{equation*}
	
	Это кольцо дедекиндово.
	Рассмотрим идеалы $\ideal{p} = (x^3 + x + 1)$ и $\ideal{q} = (x^3 + x^2 + 1)$.
	И $R/\ideal{p}$, и $R/\ideal{q}$ состоит из многочленов степени не более $2$.
	Следовательно, норма обоих идеалов равна $8$.
	
	Вычислим $\ideal{N} = \ideal{p}\ideal{q} = (x^6 + x^5 + x^4 + x^3 + x^2 + x + 1)$.
	Заметим, что $R/\ideal{N}$ состоит из многочленов степени не более $5$.
	Следовательно, $\Nm{\ideal{N}} = 64$.
	Вычислим $\varphi(\ideal{N}) = 7 * 7 = 49$.

	Заметим, что
	\begin{equation*}
		\begin{array}{l}
			x^{7i} = 1\\
			x^{7i+1} = x\\
			x^{7i+2} = x^2\\
			x^{7i+3} = x^3\\
			x^{7i+4} = x^4\\
			x^{7i+5} = x^5\\
			x^{7i+6} = x^5 + x^4 + x^3 + x^2 + x + 1
		\end{array}
	\end{equation*}

	Выберем случайное $e = 11$.
	Тогда $d = 9$.
	Возьмем элемент $x^4 + x^2 + 1$.
	Зашифруем его, а затем расшифруем
	\begin{equation*}
	    \begin{split}
    		(x^4 + x^2 + 1)^{11} = x^2 + x + 1\\
    		(x^2 + x + 1)^{9} = x^4 + x^2 + 1
	    \end{split}
	\end{equation*}
\end{example}

\begin{example}
	Рассмотрим кольцо многочленов от двух переменных над $\mathbb{Z}_2$. И рассмотрим координатное кольцо
	\begin{equation*}
		R = \frac{\mathbb{Z}_2(x, y)}{\langle xy - 1\rangle}.
	\end{equation*}
	
	Так как в этом координатном кольце $xy - 1 = 0$, то оно состоит из многочленов вида $xf(x) + yg(y) + c$.
	Из этого несложно заметить, что $R \cong \mathbb{Z}_2(x, x^{-1})$.
	
	Рассмотрим идеал $\ideal{p} = (x + 1)$.
	Его норма $2$, так как фактор кольцо $R/\ideal{p}$ состоит из элементов $0$ и $1$.
	
	Рассмотрим идеал $\ideal{q} = (x^2 + x + 1)$.
	Кольцо $R/\ideal{q}$ состоит из $x+1$, $x$, $1$, $0$.
	Следовательно, норма этого идеала $4$.

	Вычислим $\ideal{N} = \ideal{p}\ideal{q} = x^3 + 1$.
	Несложно заметить, что $R/\ideal{N}$ состоит из многочленов степени не более $2$.
	Значит норма $\Nm{\ideal{N}} = 8$.
	Вычислим $\varphi_K(\ideal{N}) = 3$

	Заметим, что
	\begin{equation*}
		\begin{array}{l}
			x^{3i} = 1\\
			x^{3i+1} = x\\
			x^{3i+2} = x^2
		\end{array}
	\end{equation*}

	Выберем случайное $e = 2$.
	Тогда $d = 2$.
	Возьмем элемент $x^2 + 1$.
	Зашифруем его, а затем расшифруем
	\begin{equation*}
	    \begin{split}
    		(x^2 + 1)^2 = x^4 + 1 = x + 1\\
	        (x + 1)^2 = x^2 + 1
	    \end{split}
	\end{equation*}
\end{example}

\subsection{Факторизация идеалов}

% Надо доделать еще

% Пусть $R$ кольцо целых алгебраических элементов числового поля $K = \mathbb{Q}(\theta)$.

% Рассмотрим представления идеалов.
% Существует два способа представить идеал

% \begin{enumerate}
%     \item Через целый базис.
%     \begin{equation}\label{eq:Z_basis}
%         \mathfrak{p} = \oplus_{i=1}^{n} \mathbb{Z}\alpha_i,
%     \end{equation}
%     где $\alpha_i\in\mathcal{O}_K$.

%     \item Через два элемента $\mathcal{O}_K$.
%     Для любого $\alpha\in\mathfrak{p}$ существует $\beta\in\mathfrak{p}$ такой, что
%     \begin{equation}\label{eq:2_element}
%         \mathfrak{p} = \alpha\mathcal{O}_K + \beta\mathcal{O}_K.
%     \end{equation}

%     Будем обозначать идеалы в таком представлении через $(\alpha, \beta)$.
%     Идеал, который генерируется одним элементом $\alpha\in\mathcal{O}_K$ будем обозначать $(\alpha) = \alpha\mathcal{O}_K$.
% \end{enumerate}

% Будем предполагать, что поле $K$ фиксировано и, следовательно, известен индекс $[\mathcal{O}_K:\mathbb{Z}[\theta]]$.
% А так же разложение на простые идеалы всех простых делителей индекса.

% \begin{statement}[Теорема Дедекинда]\label{thm:dedekind}
%     Пусть $f(T)$ минимальный многочлен алгебраического числа $\theta$ в $\mathbb{Z}[\theta]$.
%     Для простого рационального числа $p$, не делящего индекс $[\mathcal{O}_K:\mathbb{Z}[\theta]]$, запишем
%     \begin{equation}
%         f(T) \equiv \pi_1(T)^{e_1}\dots \pi_g(T)^{e_g} \modul p,\nonumber
%     \end{equation}
%     где $\pi_i(T)$ --- различные монические неприводимые многочлены в $\mathbb{F}_p[T]$.
%     Тогда
%     \begin{equation}
%         (p) = \mathfrak{p}_1^{e_1}\dots \mathfrak{p}_g^{e_g},\nonumber
%     \end{equation}
%     где $\mathfrak{p}_i = (p_i, T_i(\theta))$, $T_i(T) \equiv \pi_i(T)(\modul p)$.
% \end{statement}

% Используя теорему Дедекинда, можно показать, что задача факторизации в кольце целых чисел числового поля полиномиально эквивалентна задаче факторизации целых чисел.

% Сделать это можно с помощью следующего алгоритма.

% \begin{enumerate}
%     \item Пусть дан идеал $(N)$ в форме своего $2$-представления.

%     \item Считаем норму идеала, равную норме элемента $N$ и раскладываем норму на множители одним из известных алгоритмов для факторизации целых чисел.
%     Например методом решета числового поля или алгоритмом Шора.
%     Получаем разложение
%     $$
%         n = \Nm{N} = \prod_{i=1}^{k} p_i^{\alpha_i}.
%     $$
%     Таким образом, мы знаем, что
%     $$
%         (\Nm{N}) = \prod_{i=1}^{k} (p_i)^{\alpha_i}.
%     $$

%     \item Факторизуем идеал $(p_i)$ с помощью теоремы Дедекинда \ref{thm:dedekind} и получаем двухэлементные представления идеалов
%     $$
%         (p_i) = \prod_{j=1}^{l_i} (p_i, f_{i, j}(\theta))
%     $$

%     \item Преобразуем полученные простые идеалы в $\mathbb{Z}$-представление и объединяем равные.
%     Получаем представление
%     $$
%         (\Nm{N}) = \prod_{i=1}^{l} \mathfrak{p}_i^{\beta_i}
%     $$

%     \item Используем бинарный поиск для нахождения степеней, в которых $\mathfrak{p}_i$ входит в $(N)$.
% \end{enumerate}

% Покажем, что алгоритм полиномиальный

% \begin{statement}
%     Для вычисления нормы идеала $(N)$ необходимо $O(n^3 \log^2 |N|)$ бинарных операций, где $|N|$ обозначает максимальный по модулю элемент матричного представления $N$.
% \end{statement}
% \begin{proof}
%     Для вычисления нормы идеала $(N)$ необходимо найти определитель матрицы, которая получается при матричном представлении элемента $N$.
%     Для этого необходимо $O(n^3 \log^2 |N|)$ бинарных операций.
% \end{proof}

% \begin{statement}
%     Разложить идеал $(p)$, используя теорему Дедекинда, можно за $O((n\log n + \log p)n\log n\log\log n\log^2 p)$ бинарных операций.
% \end{statement}
% \begin{proof}
%     Оценим сложность разложения многочлена на множители в $\mathbb{F}_p$.
%     Сделать это можно с помощью вероятностной версии алгоритма Берлекэмпа за $O((n\log n + \log p)n\log n\log\log n\log^2 p)$ бинарных операций.
%     В результате получим не более $n$ многочленов.
%     Тогда вычислить значения многочленом в разложении можно за $O(n\log^2 p)$.

%     Итого получаем, что разложить идеал $(p)$, используя теорему Дедекинда, можно за $O((n\log n + \log p)n\log n\log\log n\log^2 p + n\log^2 p) = O((n\log n + \log p)n\log n\log\log n\log^2 p)$ бинарных операций.
% \end{proof}

% \begin{statement}
%     Преобразовать $2$-представление идеала $(p, \alpha)$ из теоремы Дедекинда в $\mathbb{Z}$-представление можно $O(P(n)Q(\log p))$ бинарных операций, где $P(T)$ и $Q(T)$ некоторые полиномы.
% \end{statement}
% \begin{proof}
%     В книге \cite{Pohst} описан алгоритм преобразования $2$-представления в $\mathbb{Z}$-представление.
%     Необходимо найти Эрмитову нормальную форму блочной матрицы
%     $$
%         \begin{pmatrix}
%             A\\
%             B
%         \end{pmatrix},
%     $$
%     где $A = diag(p, \dots, p)$, а $B$ является матричным представлением элемента $\alpha$.
%     В 1979 году было доказано, что эрмитову нормальную форму матрицу можно найти за строго полиномиальное время \cite{Kannan}.
%     Это означает, что алгоритму необходимо полиномиальное, относительно размеров матрицы, количество арифметических операций над числами не превосходящими полинома от бинарного представления элементов матрицы.
%     Таким образом эрмитову нормальную форму можно вычислить за $O(P(n)Q(\log p))$ бинарных операций, где $P(T)$ и $Q(T)$ некоторые полиномы.
% \end{proof}

% \begin{remark}
%     Таким образом, зная разложение $(p_i)$ на произведение простых идеалов, можно найти одинаковые идеалы и разложение $(\Nm{N})$ на произведение различных идеалов за $O(P(n)Q(\log |N|))$ бинарных операций, так как $k \le \log \Nm{N}$ и $l_i \le n$.
%     Таким образом, найти разложение идеала $(\Nm{N})$ на произведение различных простых идеалов можно за полиномиальное относительно $\log\Nm{N}$ количество бинарных операций, если разложение $\Nm{N}$ на множители известно.

%     Это показывает, что аналог RSA-криптосистемы в некотором смысле не дает никакого выигрыша при использовании в кольцах алгебраических целых чисел числовых полей.
% \end{remark}

% \begin{remark}
%     Задача факторизации в абстрактных числовых полях является более сложной, так как нет доказательства аналога теоремы Дедекинда в абстрактных числовых кольцах.
% \end{remark}

% В работе \cite{Darkey-Mensah} приводится алгоритм факторизации идеалов в дедекиндовых кольцах.
% Во время факторизации идеал проходит три алгоритма: разложение на радикалы, разложение на множители с одинаковыми степенями и разложение на множители с разными степенями.
% Рассмотрим эти алгоритмы подробнее.

% Пусть $R$ -- дедекиндово кольцо и $\mathfrak{a}$ -- идеал в $R$.
% Так как $R$ дедекиндово, то $\mathfrak{a}$ единственным образом раскладывается в произведение простых идеалов.
% Пусть это разложение имеет вид
% \begin{equation*}
%     \mathfrak{a} = \mathfrak{p}_1^{k_1} \dots \mathfrak{p}_s^{k_s},
% \end{equation*}
% где $p_1, \dots, p_s$ различные простые идеалы и $k_1, \dots, k_s > 0$.
% Соберем вместе простые множители с одинаковой степенью.
% Для $j \le m = \max\{k_1, \dots, k_s\}$ обозначим
% \begin{equation*}
%     \mathfrak{g}_j = \bigcap\limits_{k_i = j} \mathfrak{p}_i
% \end{equation*}

% Представление
% \begin{equation*}
%     \mathfrak{a} = \mathfrak{g}_1 \mathfrak{g}_1^2 \dots \mathfrak{g}_m^m
% \end{equation*}
% называется разложением на радикалы.

% После выполнения этого разложения задача факторизации сводится к задаче факторизации идеала вида $\mathfrak{a} = \mathfrak{p}_1\dots\mathfrak{p}_s$, где $\mathfrak{p}_1, \dots, \mathfrak{p}_s$ являются простыми идеалами.
% Обозначим
% \begin{equation*}
%     \mathfrak{h}_j = \prod\limits_{\mathfrak{p} | \mathfrak{a}, \textrm{deg} \mathfrak{p} = j} \mathfrak{p}
% \end{equation*}

% Представление
% \begin{equation*}
%     \mathfrak{a} = \mathfrak{h}_1 \dots \mathfrak{h}_m
% \end{equation*}
% называется разложением на множители с разными степенями.

% После двух шагов получается, что задача факторизации идеала сводится к задаче факторизации радикала, который является произведением различных простых идеалов с одинаковой степенью, которая нам неизвестна.
% Далее можно воспользоваться алгоритмом, похожим на алгоритм Кантотра-Зассенхауса.

% Несмотря на то, что алгоритм факторизации имеется, у него есть определенные ограничения.
% Они описаны в работе \cite{Darkey-Mensah} и состоят в том, что надо уметь вычислять радикал идеала, сумму идеалов и частное.

\onlyinsubfile{
    \subfile{_10_bibliography}
    \subfile{_11_pub}
}

\end{document}
