\documentclass[_00_dissertation.tex]{subfiles}
\begin{document}

\onlyinsubfile{
    \renewcommand{\contentsname}{ОГЛАВЛЕНИЕ}
    \setcounter{tocdepth}{3}
    \tableofcontents
}

\newpage
\begin{center}
    \refstepcounter{section}
    \section*{ГЛАВА \arabic{section}.\\ АНАЛОГ RSA-КРИПТОСИСТЕМЫ В ДЕДЕКИНДОВЫХ КОЛЬЦАХ}\label{ch:RSA-cryptosystem}
    \addcontentsline{toc}{chapter}{ГЛАВА \arabic{section}. АНАЛОГ RSA-КРИПТОСИСТЕМЫ В ДЕДЕКИНДОВЫХ КОЛЬЦАХ}
\end{center}

\subsection{Предварительные сведения}

\begin{definition}
    Говорят, что дедекиндово кольцо $R$ имеет конечное поле остатков, если $|R/\ideal{m}| < \infty$ для любого максимального ненулевого идеала $\ideal{m} \subseteq R$.
\end{definition}

Далее в этой главе полагаем, что дедекиндово кольцо $R$ имеет конечное поле остатков.

\begin{example}
    \begin{itemize}
        \item Пусть $K$ -- числовое поле.
        Кольцо $\mathbb{Z}_K$, образованное алгебраическими целыми элементами этого поля является дедекиндовым с конечным полем остатков.
        Частными случаями этого примера являются кольцо целых чисел и гауссовых чисел.
        
        \item Пусть $f(x, y) = y - mx - b$ -- прямая.
        Тогда $K[x, y]/(f(x, y)) \cong K[x]$.
        Следовательно, это координатное кольцо является факториальным.
        
        \item Пусть $f(x, y) = y - x^2$ -- парабола.
        Тогда $K[x, y]/(f(x, y)) \cong K[x]$.
        Следовательно, это координатное кольцо является факториальным.
    
        \item Пусть $f(x, y) = x^2 + y^2 - 1$.
        Если $K = \mathbb{Q}$, то координатное кольцо $\mathbb{Q}[x, y]/(f)$ не изоморфно $K[x]$, так как первое не является факториальным кольцом.
        Это можно показать, рассмотрев элементы $y^2 = yy$ и $1-x^2 = (1-x)(1+x)$.
        Однако, если $K = \mathbb{C}$, то координатное кольцо $\mathbb{C}[x, y]/(f) \cong \mathbb{C}[x, x^{-1}]$ уже будет факториальным, так как это локализация факториального кольца.
    \end{itemize}
\end{example}

% Пусть $\mathcal{O}_K$ -- кольцо алгебраических целых чисел числового поля $K$.
% Известно, что $\mathcal{O}_K$ является абстрактным числовым кольцом.
% Обозначим через $\mathcal{O}_K^\times$ группу единиц кольца $\mathcal{O}_K$, через $\mathcal{O}_{K, \mathfrak{m}} = \mathcal{O}_K/\mathfrak{m}$ аддитивную группу вычетов по модулю $\mathfrak{m}$ и через $\mathcal{O}_{K, \mathfrak{m}}^\times$ мультипликативную группу вычетов по модулю $\mathfrak{m}$.
% Пусть задана некоторая норма $\textrm{Nm}(m)$, где $m\in\mathcal{O}_K$.

% \begin{statement}
%     Пусть $m\in\mathcal{O}_K$.
%     Тогда $\mathcal{N}((m)) = |\Nm{m}|$, а так же $(m)|(\Nm{m})$.
% \end{statement}

\begin{statement}[Теорема Копперсмита]\label{statement:coppersmith}
  Пусть $f(x,y)$ неприводимый многочлен от двух переменных над $\mathbb{Z}$ со степенью $\delta$ по каждой переменной отдельно.
  Пусть $X$, $Y$ границы предполагаемого решения $(x_0, y_0)$.
  Пусть $W$ модуль максимального коэффициента $f(xX, yY)$.
  Если $XY\le W^{\frac{2}{3\delta}}$, то существует полиномиальный относительно $\log W$ и $2^\delta$ алгоритм, который позволяет найти все $(x_0,y_0)$ такие, что $f(x_0,y_0)=0$, $|x_0|\le X$ и $|y_0|\le Y$.
\end{statement}

\subsection{Формулировка и анализ аналога RSA-криптосистемы}

Изложенный далее алгоритм аналога RSA-криптосистемы был предложен в работе Петуховой и Тронина~\cite{source:Petukhova}.
Была показана корректность полученной криптосистемы и представлены ограничения  на кольцо для ее эффектитвного применения.
В этой части исследуется RSA-криптосистема в дедекиндовых кольцах с конечным полем остатков.
Целью является получение доказательств теорем, связанных с ее криптостойкостью.
Например теоремы Винера, теоремы об эквивалентности факторизации и взлома криптосистемы, а так же изучение методов взлома криптосистемы.

\begin{algorithm}\label{algorithm:RSA_in_dedekind}
    Аналог RSA-криптосистемы в дедекиндовых кольцах.

    \begin{enumerate}
        \item Выбираются максимальные идеалы $\ideal{p}$, $\ideal{q}\in R$

        \item Вычисляется $\varphi(\ideal{N}),$ где $\ideal{N} = \ideal{p} \ideal{q}$

        \item Выбирается случайное целое $e \in [1, \varphi(\ideal{N})],$ $(e, \varphi(\ideal{N}))=1$

        \item Вычисляется целое положительное $d$ такое, что $ed \equiv 1 \pmod{\varphi(\ideal{N})}$
    \end{enumerate}

    Пара $(\ideal{N}, e)$ это публичный ключ $A$, пара $(\ideal{N}, d)$ секретный ключ $A$.
    Функцией шифрования называется

    \begin{equation*}
        \begin{array}{c}
            f: R/\ideal{N} \to R/\ideal{N},\\
            f(x) \equiv x^{e} \pmod{\varphi(\ideal{N})}.
        \end{array}
    \end{equation*}

    Функцией дешифровки называется

    \begin{equation*}
        \begin{array}{c}
            f^{-1}: R/\ideal{N} \to R/\ideal{N},\\
            f^{-1}(x) \equiv x^{d}\pmod{\varphi(\ideal{N})}.
        \end{array}
    \end{equation*}
\end{algorithm}

\begin{remark}
    Корректность приведенной криптосистемы гарантируется аналогом теоремы Эйлера для абстрактных числовых колец.
\end{remark}

\begin{remark}
    В работе \cite{Petukhova} описаны условия, при которых применение алгоритма шифрования в абстрактном числовом кольце будет возможно на компьютере.
    Необходимо, чтобы для некоторых максимальных идеалов $\ideal{M}_1, \ideal{M}_2 \in R$ и $\ideal{U} = \ideal{M}_1\ideal{M}_2$ было известно множество $W \subset R$ такое, что:
    \begin{enumerate}
        \item Любое подмножество из $R/\ideal{U}$ пересекается с $W$ в ровно одном элементе.
        
        \item Есть вычислимая биекция между $W$ и $\{1, \dots, T\}$ для некоторого $T$.
    \end{enumerate}
\end{remark}

Нетрудно заметить, что зная разложение на множители $\ideal{N}=\ideal{p}\ideal{q}$ для RSA-модуля можно эффективно найти секретный ключ.
В некоторых случаях можно доказать обратное утверждение.

\begin{theorem}\label{theorem:factor}\ref{source:BSU_Journal_2020}
    Пусть $K$ -- числовое поле и $\mathbb{Z}_K$ его кольцо целых алгебраических чисел.
    Пусть $\mathbb{Z}_K$ -- кольцо с единственной факторизацией, $((N), e, d)$ параметры RSA-криптосистемы в $\mathbb{Z}_K$.
    Если $d$ известно, то $N$ можно эффективно разложить на множители с вероятностью не менее $\frac{1}{2}$ за полиномиальное относительно $\log \Nm{(N)}$ количество арифметических операций в $\mathbb{Z}_K$.
\end{theorem}
\begin{proof}
    Пусть $s = ed - 1 = 2^t u$, где $t, u \in \mathbb{N}$ и $u$ нечетное.
    Так как $\varphi((N)) | s$, то $x^s \equiv 1 \pmod{(N)}$ для всех $x \in \mathbb{Z}_K$.
    
    Обозначим $\mathcal{S}_N$ множество таких элементов $x \in \invertible{\mathbb{Z}_N / (N)}$, что $x^u \equiv 1 \pmod{(N)}$ или существует $j \in \{0, \dots, t-1\}$, для которого $a^{2^j u} \equiv -1 \pmod{(N)}$.
    Пусть $A = \invertible{\mathbb{Z}_K} \setminus \mathcal{S}_N$.
    
    Рассмотрим произвольный элемент $a \in A$ и выберем наименьшее $j \in \mathbb{N}$, что $a^{2^j u} \equiv 1 \pmod{(N)}$.
    Пусть $b = a^{2^{j-1} u} \pmod{(N)}$.
    Из того, что $b^2 \equiv 1 \pmod{(N)}$ и $b \not\equiv \pm 1 \pmod{(N)}$ следует, что $(b - 1, N)$ -- собственный делитель $N$.
    
    В работе Викстрома~\cite{source:Wikstrom} показано, что наибольший общий делитель $(b - 1, N)$ можно вычислить за полиномиальное относительно $\log \Nm{(N)}$ число арифметических операций в $\mathbb{Z}_K$.
    
    Пусть $(N) = (p)(q)$, где $p, q$ -- простые элементы в $\mathbb{Z}_K$.
    Положим $\varphi((p)) = 2^{v_1} u-1$, $\varphi((q)) = 2^{v_2} u_2$, где $v_1, v_2, u_1, u_2 \in \mathbb{N}$, $u_1, u_2$ -- нечетные.
    Обозначим $v = \min\{v_1, v_2\}$ и $K = (u, u_1)(u, u_2)$.
    
    Из свойств сравнений следует, что
    \begin{equation*}
        x^u \equiv 1 \pmod{(N)}
        \Leftrightarrow
        \begin{cases}
            u \log_{\alpha} x \equiv 0 \pmod{\varphi((p))}\\
            u \log_{\beta} x \equiv 0 \pmod{\varphi((q))}
        \end{cases},
    \end{equation*}
    где $\alpha$ и $\beta$ примитивные элементы в $\invertible{\mathbb{Z}_K / (p)}$ и $\invertible{\mathbb{Z}_K / (q)}$ соответственно.
    Следовательно, это сравнение имеет $K$ решений.
    
    Рассмотрим сравнение $x^{2^j u} \equiv -1 \pmod{(N)}$, где $j \in \{0, \dots, t - 1\}$.
    Если $j \ge v$, то решений нет.
    Если $j < v$, то количество решений равно $4^j K$.
    
    Следовательно, получаем, что
    \begin{equation*}
        |\mathcal{S}_N| = \left(
            1 + 1 + 4 + \dots + 4^{v-1}
        \right)K = \frac{4^v + 2}{3}K.
    \end{equation*}
    Тогда
    \begin{equation*}
        \frac{|\mathcal{S}_N|}{\varphi((N))} \le \frac{\frac{4^v + 2}{3}K}{4^v K} \le \frac{1}{2}.
    \end{equation*}
\end{proof}

Следующая теорема является аналогом теоремы Винера о малой секретной экспоненте \cite{source:Wiener}.

\begin{theorem}\label{theorem:Wiener}\ref{source:BSU_Journal_2020}
    Пусть $(\ideal{N},e,d)$, $\ideal{N}=\ideal{p} \ideal{q}$ -- параметры RSA-криптосистемы в дедекиндовом кольце $R$.
    Пусть $\Nm{\ideal{q}} < \Nm{\ideal{p}} < \alpha^2 \Nm{\ideal{q}},$ где $\alpha > 1.$
    Если $d<\frac{1}{\sqrt{2\alpha+2}}(\Nm{\ideal{N}})^{1/4},$ то $d$ можно эффективно вычислить за полиномиальное относительно $\log \Nm{\ideal{N}}$ число бинарных операций.
\end{theorem}
\begin{proof}
    Пусть $ed - 1 = k \varphi(\ideal{N})$.
    Так как $\Nm{\ideal{p}} + \Nm{\ideal{q}} < (\alpha + 1)\sqrt{\Nm{\ideal{N}}}$, то
    \begin{equation*}
        \Nm{\ideal{N}} - \varphi(\ideal{N}) = \Nm{\ideal{p}} + \Nm{\ideal{q}} - 1 < (\alpha + 1)\sqrt{\Nm{\ideal{N}}}.
    \end{equation*}
    
    Так как $k \varphi(\ideal{N}) < ed$ и $e < \varphi(\ideal{N})$, то $k < d$ и
    \begin{equation*}
        \frac{(\alpha + 1)k}{d\sqrt{\Nm{\ideal{N}}}} < \frac{\alpha + 1}{\sqrt{\Nm{\ideal{N}}}} < \frac{1}{2 d^2}.
    \end{equation*}
    
    Тогда получаем
    \begin{equation*}
        \begin{split}
            \left|
                \frac{e}{\Nm{\ideal{N}}} - \frac{k}{d}
            \right| = \left|
                \frac{1 - k(\Nm{\ideal{N}} - \varphi(\ideal{N}))}{\Nm{\ideal{N}}d}
            \right| \le \left|
                \frac{k(\Nm{\ideal{N}} - \varphi(\ideal{N}))}{\Nm{\ideal{N}}d}
            \right| \le \\
            \le \left|
                \frac{k(\alpha + 1)\sqrt{\Nm{\ideal{N}}}}{\Nm{\ideal{N}}d}
            \right| < \frac{1}{2 d^2}.
        \end{split}
    \end{equation*}
    Следовательно, $\frac{k}{d}$ -- подходящая цепная дробь для дроби $\frac{e}{\Nm{\ideal{N}}}$, которая не является секретной.
    Тогда $\frac{k}{d}$ можно вычислить, используя алгоритм Евклида в $\mathbb{Z}$ за полиномиальное относительно $\log \Nm{\ideal{N}}$ число бинарных операций.
\end{proof}

\begin{remark}
    Доказанная выше теорема является основой для атаки Винера на RSA-криптосистему.
    При соблюдении определенных условий на параметры криптосистемы, можно сделать использование этой атаки невозможным.
    Однако существуют атаки, от которых невозможно полностью защититься.
    
    Метод повторного шифрования является примером такой атаки.
    Предположим, что было перехвачено некоторое зашифрованное сообщение $y = x^e \pmod{\ideal{N}}$, где $x \in \mathbb{Z}_{K} / \ideal{N}$ -- некоторое сообщение.
    Построим последовательность $y_i = y^{e^i} \pmod{\ideal{N}}$, где $i \in \{1, 2, \ldots\}$.
    Используя свойства возведения в степень и то, что $\mathbb{Z}_{K} / \ideal{N}$ конечно, получаем, что существует таое $m \in \mathbb{N}$, что $y_m = y$.
    Тогда $y_{m-1} = x$.
    
    Единственный способ защиты от этого метода взлома состоит в том, чтобы сделать $m$ достаточно большим.
\end{remark}

Обозначим через $R_{\mathfrak{m}}$ и $R_{\mathfrak{m}}^{\times}$ аддитивную и мультипликативную группы вычетов по модулю $\mathfrak{m}$.
Заметим, что если $\mathfrak{m}=\mathfrak{m}_1 \mathfrak{m}_2$, то $R_{\mathfrak{m}}^{\times} \cong R_{\mathfrak{m}_1}^{\times} \times R_{\mathfrak{m}_2}^{\times}$.

\begin{theorem}\label{theorem:iterated}
    Пусть $\ideal{N} = \ideal{p} \ideal{q}$ -- модуль RSA-криптосистемы в дедекиндовом кольце $R$.
    Предположим, что существуют различные простые числа $r$, $s$ и положительные целые числа $k$, $l$ такие, что $\varphi(\ideal{p}) = rk$, $\varphi(\ideal{q}) = sl$ и числа $r - 1$, $s - 1$ имеют различные простые делители $r_1$, $s_1$ соответственно.

    Пусть $y$ и $e$ -- независимые равномерно распределенные случайные величины со значениями в $R / \ideal{N}$ и $\invertible{\mathbb{Z}_{\varphi(\ideal{N})}}$ соответственно.
    Обозначим
    \begin{equation*}
        m_{e,y} = \min \{m \in \mathbb{N} | y_m = y\}.
    \end{equation*}
    Тогда выполняется неравенство
    \begin{equation*}
        P(m_{e,y} \ge r_1s_1)\ge(1-r^{-1})(1-s^{-1})(1-r_1^{-1})(1-s_1^{-1}).
    \end{equation*}
\end{theorem}
\begin{proof}
    Оценим вероятность $P\{rs | \textrm{ord}_{\invertible{R / \ideal{N}}}(y)\}$.

    Так как $\invertible{R / \ideal{N}} \cong \invertible{R / \ideal{p}} \times \invertible{R / \ideal{q}}$ и группы $\invertible{R / \ideal{p}}$, $\invertible{R / \ideal{q}}$ циклические, то можно записать $y = (\alpha^i, \beta^j),$ где $\alpha$ и $\beta$ примитивные элементы $\invertible{R / \ideal{p}}$ и $\invertible{R / \ideal{q}}$ соответственно, $i$ и $j$ случайные величины со значениями в $\{1, \ldots, rk\}$ и $\{1, \ldots, sl\}$ соответственно.
    
    Следовательно, получаем, что $\textrm{ord}_{\invertible{R / \ideal{N}}}(y) = \textrm{lcm}\left(\frac{rk}{(rk,i)}, \frac{sl}{(sl,j)}\right)$.
    Если $r \nmid i$ и $s \nmid j$ то $\textrm{ord}_{\invertible{R / \ideal{N}}}(y) \vdots rs$.
    Таким образом получаем
    \begin{equation*}
        P\left\{
            rs \big| \textrm{ord}_{\invertible{R / \ideal{N}}}(y)
        \right\} \ge P\left\{
            r\nmid i, s \nmid j
        \right\} = \frac{\varphi(r)k\varphi(s)l}{rksl} = \left(
            1-\frac{1}{r}
        \right)\left(
            1-\frac{1}{s}
        \right).
    \end{equation*}

    Так как $e \in \invertible{\mathbb{Z}_{rs}}$, то аналогично можно получить неравенство
    \begin{equation*}
        P\left\{
            r_1s_1|\textrm{ord}_{\mathbb{Z}^*_{rs}}(e)
        \right\} \ge \left(
            1-\frac{1}{r_1}
        \right)\left(
            1-\frac{1}{s_1}
        \right).
    \end{equation*}

    Легко заметить, что $\textrm{ord}_{\invertible{R / \ideal{N}}}(y) | (e^{m_{e,y}}-1),$ следовательно, имеем
    \begin{equation*}
        \left\{
            rs | \textrm{ord}_{\invertible{R / \ideal{N}}}(y)
        \right\} \subseteq \left\{
            \textrm{ord}_{\mathbb{Z}^*_{rs}}(e) | m_{e,y}
        \right\}.
    \end{equation*}

    Следовательно, получаем
    \begin{equation*}
        \begin{split}
            P\left\{
                m_{e,y} \ge r_1s_1
            \right\} \ge P\left\{
                r_1s_1 | m_{e,y}
            \right\} \ge P\left\{
                r_1s_1|\textrm{ord}_{\mathbb{Z}^*_{rs}}(e), rs|\textrm{ord}_{\invertible{R / \ideal{N}}}(y)
            \right\} \ge \\
            \ge (1-r^{-1})(1-s^{-1})(1-r_1^{-1})(1-s_1^{-1}).
        \end{split}
    \end{equation*}
\end{proof}

\begin{theorem}\label{thm:d_is_known_2}
    Пусть $(\ideal{N}, e, d)$ параметры RSA-криптосистемы в абстрактном числовом кольце $R$, где $\mathcal{N}(\ideal{p})$ и $\mathcal{N}(\ideal{q})$ имеют одинаковую битовую длину.
    Пусть $ed \le (\mathcal{N}(\ideal{N}))^2,$ $\mathcal{N}(\ideal{N}) \ge 107$.
    Если $d$ известно, то существует эффективный алгоритм, который позволяет найти $\mathcal{N}(\ideal{p})$ и $\mathcal{N}(\ideal{q})$.
\end{theorem}
\begin{proof}
    Из того, что $ed\equiv 1(\modul\varphi(\ideal{N}))$ следует, что $ed=k\varphi(\ideal{N})+1$ для некоторого $k\in\mathbb{N}$.
    Предположим, что $\mathcal{N}(\ideal{p}) \le \mathcal{N}(\ideal{q})$.
    Тогда $\mathcal{N}(\ideal{P}) \le (\mathcal{N}(\ideal{N}))^{1/2} \le \mathcal{N}(\ideal{q}) < 2\mathcal{N}(\ideal{p}) \le 2(\mathcal{N}(\ideal{N}))^{1/2}$.
    Следовательно, имеем

    \begin{equation}\label{eq:d_is_known_3/2_1}
        \mathcal{N}(\ideal{p})+\mathcal{N}(\ideal{q}) < 3(\mathcal{N}(\ideal{N}))^{1/2} \le \frac{\mathcal{N}(\ideal{N})}{2}.
    \end{equation}

    Из (\ref{eq:d_is_known_3/2_1}) следует, что

    \begin{multline}\label{eq:d_is_known_3/2_2}
        \varphi(\ideal{N}) =
        (\mathcal{N}(\ideal{p})-1)(\mathcal{N}(\ideal{q})-1) =
        \mathcal{N}(\ideal{N}) + 1 - \mathcal{N}(\ideal{p}) - \mathcal{N}(\ideal{q}) >\\
        > \mathcal{N}(\ideal{N}) + 1 - \frac{\mathcal{N}(\ideal{N})}{2} >
        \frac{\mathcal{N}(\ideal{N})}{2}.
    \end{multline}

    Обозначим $\overline{k} = \frac{ed-1}{\mathcal{N}(\ideal{N})}$.
    Тогда

    \begin{equation}\label{eq:d_is_known_3/2_3}
        k-\overline{k} =
        \frac{(\mathcal{N}(\ideal{N})-\varphi(\ideal{N}))(ed-1)}{\mathcal{N}(\ideal{N})\varphi(\ideal{N})} =
        \frac{(\mathcal{N}(\ideal{p})+\mathcal{N}(\ideal{q})-1)(ed-1)}{\mathcal{N}(\ideal{N})\varphi(\ideal{N})}.
    \end{equation}
  
    Из (\ref{eq:d_is_known_3/2_1}), (\ref{eq:d_is_known_3/2_2}) и (\ref{eq:d_is_known_3/2_3}) следует, что

    \begin{equation}
        k-\overline{k} < 6(\mathcal{N}(\ideal{N}))^{-3/2}(ed-1).
    \end{equation}
  
    Пусть $x=k-\overline{k}$.
    Так же выполнено $\mathcal{N}(\ideal{N})-\varphi(\ideal{N}) = \mathcal{N}(\ideal{p})+\mathcal{N}(\ideal{q})-1 < 3(\mathcal{N}(\ideal{N}))^{1/2}$.
    Следовательно $\varphi(\ideal{N}) \in [\mathcal{N}(\ideal{N})-3(\mathcal{N}(\ideal{N}))^{1/2}, \mathcal{N}(\ideal{N})]$.
    Разделим этот интервал на $6$ интервалов длины $\frac{1}{2}(\mathcal{N}(\ideal{N}))^{1/2}$ с центрами в точках $\mathcal{N}(\ideal{N})-\frac{2i-1}{4}(\mathcal{N}(\ideal{N}))^{1/2}$, где $i=\overline{1,6}$.
    Рассмотрим $i\in \{1,\ldots,6\}$ такое, что

    \begin{equation}
        \left|
            \mathcal{N}(\ideal{N})-\frac{2i-1}{4}(\mathcal{N}(\ideal{N}))^{1/2}-\varphi(\ideal{N})
        \right| \le \frac{1}{4}(\mathcal{N}(\ideal{N}))^{1/2}.
    \end{equation}
  
    Пусть $g = \lceil\frac{2i-1}{4}\mathcal{N}(\ideal{N})\rceil$.
    Тогда $|\mathcal{N}(\ideal{N})-g-\varphi(\ideal{N})| < \frac{1}{4}(\mathcal{N}(\ideal{N}))^{1/2}+1$ и $\varphi(\ideal{N}) = \mathcal{N}(\ideal{N})-g-y$ для некоторого неизвестного $y$ такого, что $|y| \le \frac{1}{4}(\mathcal{N}(\ideal{N}))^{1/2}$.
    Далее имеем

    \begin{equation}
        ed-1 = k\varphi(\ideal{N}) = (\lceil\overline{k}\rceil+x)(\mathcal{N}(\ideal{N})-g-y).
    \end{equation}
  
    Рассмотрим многочлен

    \begin{equation}
        f(x,y) = xy-(\mathcal{N}(\ideal{N})-g)x+\lceil\overline{k}\rceil y-\lceil\overline{k}\rceil(\mathcal{N}(\ideal{N})-g)+ed-1.
    \end{equation}

    У него есть корень $(x_0,y_0) = (k-\lceil\overline{k}\rceil, \mathcal{N}(\ideal{p})+\mathcal{N}(\ideal{q})+1-g)$.
    Имеем $\delta = 1,$ где $\delta$ из теоремы Копперсмита.
    Определим $X = 6(\mathcal{N}(\ideal{N}))^{1/2}$ и $Y = \frac{1}{4}(\mathcal{N}(\ideal{N}))^{1/2}+1$.
    Тогда $|x_0| \le X$ и $|y_0| \le Y$.
    Пусть $W$ норма вектора коэффициентов многочлена $f(xX,yY)$.
    Тогда $W \ge (\mathcal{N}(\ideal{N})-g)X > 3(\mathcal{N}(\ideal{N}))^{3/2}$.
    Следовательно
    
    \begin{equation*}
        XY = \frac{3}{2}\mathcal{N}(\ideal{N}) + 6(\mathcal{N}(\ideal{N}))^{1/2} < W^{2/3} = W^{\frac{2}{3\delta}}
    \end{equation*}
    
    для $\mathcal{N}(\ideal{N}) \ge 107$.
  
    Тогда мы можем применить теорему Копперсмита.
    Следовательно, мы можем найти корень $(x_0,y_0)$ за полиномиальное относительно $\log W$ время.
    Корень $(x_0,y_0)$ позволяет найти $\mathcal{N}(\ideal{p})$ и $\mathcal{N}(\ideal{q})$.
\end{proof}

Приведем примеры работы криптосистемы для некоторых координатных колец.

\begin{example}
	Рассмотрим кольцо многочленов от двух переменных над $\mathbb{Z}_2$.
	Рассмотрим координатное кольцо
	
	\begin{equation*}
		R = \frac{\mathbb{Z}_2(x, y)}{y-x}
	\end{equation*}
	
	Оно изоморфно $\mathbb{Z}_2[x]$.
	Известно, что это кольцо дедекиндово.
	Рассмотрим идеалы $(x^3 + x + 1)$ и $(x^3 + x^2 + 1)$ и найдем их норму.
	И $R/(x^3 + x + 1)$, и $R/(x^3 + x^2 + 1)$ состоит из многочленов степени не более $2$.
	Следовательно, норма обоих идеалов равна $8$.
	
	Теперь найдем элементы $R/(x^3 + x + 1)(x^3 + x^2 + 1)$
	
	\begin{equation*}
		(x^3 + x + 1)(x^3 + x^2 + 1) = x^6 + x^5 + x^4 + x^3 + x^2 + x + 1
	\end{equation*}
	
	Следовательно, $x^6 = x^5 + x^4 + x^3 + x^2 + x + 1$.
	Следовательно, фактор-кольцо $R/(x^3 + x + 1)(x^3 + x^2 + 1)$ состоит из многочленов не более чем $5$ степени.
	Их $64$, следовательно, по норме все сходится.
	
	Заметим, что
	\begin{equation*}
		\begin{array}{l}
			x^6 = x^5 + x^4 + x^3 + x^2 + x + 1\\
			x^7 = 1\\
			x^8 = x\\
			x^9 = x^2\\
			x^{10} = x^3\\
			x^{11} = x^4\\
			x^{12} = x^5\\
			x^{13} = x^6 = x^5 + x^4 + x^3 + x^2 + x + 1\\
			x^{14} = x^7 = 1\\
			\ldots
		\end{array}
	\end{equation*}
	
	Следовательно, имеем
	\begin{equation*}
		\begin{array}{l}
			x^{7i} = 1\\
			x^{7i+1} = x\\
			x^{7i+2} = x^2\\
			x^{7i+3} = x^3\\
			x^{7i+4} = x^4\\
			x^{7i+5} = x^5\\
			x^{7i+6} = x^5 + x^4 + x^3 + x^2 + x + 1
		\end{array}
	\end{equation*}
	
	Вычисляем
	\begin{equation*}
		\varphi_K((x^3 + x + 1)(x^3 + x^2 + 1)) = 7 * 7 = 49
	\end{equation*}
	
	Выбираем случайное $e = 11$.
	Тогда $d=9$.
	
	Возьмем элемент $x^4 + x^2 + 1$.
	Зашифруем его
	\begin{equation*}
		(x^4 + x^2 + 1)^{11} = x^2 + x + 1
	\end{equation*}
	
	Теперь расшифруем
	\begin{equation*}
		(x^2 + x + 1)^{9} = x^4 + x^2 + 1
	\end{equation*}
\end{example}

\begin{example}
	Рассмотрим кольцо многочленов от двух переменных над $\mathbb{Z}_2$. И рассмотрим координатное кольцо
	
	\begin{equation*}
		R = \frac{\mathbb{Z}_2(x, y)}{\langle xy - 1\rangle}.
	\end{equation*}
	
	Так как в этом координатном кольце $xy - 1 = 0$, то оно состоит из многочленов вида $xf(x) + yg(y) + c$.
	Из этого несложно заметить, что $R\cong \mathbb{Z}_2(x, x^{-1})$.
	
	Рассмотрим идеал $(x+1)$.
	Его норма $2$, так как фактор кольцо $R/(x+1)$ состоит из элементов $0$ и $1$.
	
	Рассмотрим идеал $(x^2 + x + 1)$.
	Кольцо $R/(x^2 + x + 1)$ состоит из $x+1$, $x$, $1$, $0$.
	Следовательно, норма этого идеала $4$.

	Теперь рассмотрим произведение этих идеалов $(x+1)(x^2 + x + 1) = x^3 + 1$.
	Несложно заметить, что $R/(x^3 + 1)$ состоит из многочленов $0$, $1$, $x$, $x+1$, $x^2$, $x^2+1$, $x^2+x$, $x^2+x+1$.
	Значит норма $(x^3 + 1)$ равна $8$.

	Заметим, что
	
	\begin{equation*}
		\begin{array}{l}
			x^{3i} = 1\\
			x^{3i+1} = x\\
			x^{3i+2} = x^2
		\end{array}
	\end{equation*}
	
	Вычисляем
	\begin{equation*}
		\varphi_K((x+1)(x^2 + x + 1)) = 3
	\end{equation*}
	
	Выбираем случайное $e = 2$.
	Тогда $d=2$.
	Возьмем элемент $x^2 + 1$.
	Зашифруем его
	
	\begin{equation*}
		(x^2 + 1)^2 = x^4 + 1 = x + 1
	\end{equation*}
	
	Теперь расшифруем
	
	\begin{equation*}
		(x + 1)^2 = x^2 + 1
	\end{equation*}
\end{example}

\onlyinsubfile{
    \subfile{_10_bibliography}
    \subfile{_11_pub}
}

\end{document}
