
\usepackage{lastpage}
\usepackage{caption2}

\usepackage{lscape}

\usepackage{makeidx}
\usepackage{soul}
\usepackage{mathrsfs}
\usepackage[english, russian]{babel}
\usepackage{longtable}
\usepackage{comment}

\babelfont{rm}{Times New Roman}
\babelfont{sf}{Times New Roman}

\renewcommand{\rmdefault}{tli}

\usepackage[pdftex]{graphicx}
\usepackage{wrapfig}
\usepackage{euscript}

\usepackage{amsfonts}
\usepackage{amsmath}
\usepackage{amssymb}

\renewcommand\labelitemi{{\textbullet}}

\usepackage{totcount}
\newtotcounter{citnum}
\def\oldbibitem{} \let\oldbibitem=\bibitem
\def\bibitem{\stepcounter{citnum}\oldbibitem}

\sloppy\clubpenalty4000\widowpenalty4000

% Размеры страницы мои для формата А4
\setlength{\hoffset}{0mm} \setlength{\voffset}{-8mm}
\setlength{\textheight}{245mm}% размеры текста по высоте без колонтитула
\setlength{\textwidth}{170mm}% размеры текста по ширине
\setlength{\topmargin}{2mm}% высота над верхним колонтитулом
\setlength{\oddsidemargin}{1.9mm}% отступ слева
\setlength{\evensidemargin}{-4.0mm}
\addtolength{\evensidemargin}{8.0mm}% отступ  слева на четных страницах при двусторонней печати
\addtolength{\oddsidemargin}{1.9mm}%отступ слева на (нечетной при двусторонней)всех страницах при односторонней печати
\setlength{\headsep}{4mm}% высота между верхним колонтитулом и текстом
\setlength{\footskip}{10mm}% высота между текстом и нижним колонтитулом
\setlength{\marginparwidth}{0mm}% ширина заметок на полях
\setlength{\marginparpush}{0mm}
\setlength{\marginparsep}{0mm}% ширина между заметками на полях и текстом

\setlength{\parindent}{9mm} \setlength{\linespread}{1}

\renewcommand{\headrulewidth}{0.5pt}
\renewcommand{\footrulewidth}{0mm}

% Из Никифорова для оборотных страниц
\newlength{\lb}
\setlength{\lb}{24pt}
\newcommand{\tb}{ \hspace*{\lb}}
\newlength{\pb}
\setlength{\pb}{\textwidth} \addtolength{\pb}{-\lb}
\newlength{\wlb}
\settowidth{\wlb}{H62}
\newlength{\wpb}
\setlength{\wpb}{\lb} \addtolength{\wpb}{-\wlb}
%


% Нумерация формул
\makeatletter \@addtoreset {equation}{chapter}
%\renewcommand\theequation{\@arabic\c@section.\@arabic\c@equation}%первая section в пределах section
%\renewcommand{\theequation}{\thechapter.\arabic{equation}}%первая chapter в пределах section
\renewcommand\theequation{\thechapter.\@arabic\c@equation}%первые chapter и section

%После названия точка
\makeatletter
\def\@seccntformat#1{\csname the#1\endcsname\quad}
\makeatother

%Hacks fo less count of overfulls
\pretolerance=200
\tolerance=500

%definitions of some symbols

\def\proof{\par\noindent{Д\,о\,к\,а\,з\,а\,т\,е\,л\,ь\,с\,т\,в\,о}.\ }
\def\proofo{\par\noindent{Д\,о\,к\,а\,з\,а\,т\,е\,л\,ь\,с\,т\,в\,о}\ }
\def\endproof{\hfill{$\otimes$}}

\makeatletter

%%%%%%%%%%%%%%%%%%%% Кусок для теоремоподобных окружений
\gdef\theoremstyle#1{%
   \@ifundefined{th@#1}{\@warning
          {Unknown theoremstyle `#1'. Using `plain'}%
          \theorem@style{plain}}%
      {\theorem@style{#1}}%
      \begingroup
        \csname th@\the\theorem@style \endcsname
      \endgroup}
\global\let\@begintheorem\relax
\global\let\@opargbegintheorem\relax
\newtoks\theorem@style
\global\theorem@style{plain}
\gdef\theorembodyfont#1{%
   \def\@tempa{#1}%
   \ifx\@tempa\@empty
    \theorem@bodyfont{}%
   \else
    \theorem@bodyfont{\reset@font#1}%
   \fi
   }
\newtoks\theorem@bodyfont
\global\theorem@bodyfont{}
\gdef\theoremheaderfont#1{\gdef\theorem@headerfont{#1}%
       \gdef\theoremheaderfont##1{%
        \typeout{\string\theoremheaderfont\space should be used
                 only once.}}}
\ifx\upshape\undefined
\gdef\theorem@headerfont{\bfseries}
\else \gdef\theorem@headerfont{\normalfont\bfseries}\fi
\begingroup
\gdef\th@plain{\normalfont\itshape
  \def\@begintheorem##1##2{%
        \item[\indent\hskip\labelsep \theorem@headerfont ##1\ ##2.]}%
\def\@opargbegintheorem##1##2##3{%
   \item[\indent\hskip\labelsep \theorem@headerfont ##1\ ##2\ ##3.]}}
%\endgroup
%%%%%%%%%%%% стиль как в Определениях
%\begingroup
\gdef\th@defn{\normalfont\itshape
  \def\@begintheorem##1##2{%
        \item[\indent\hskip\labelsep ##1\ ##2.]}%
\def\@opargbegintheorem##1##2##3{%
   \item[\indent\hskip\labelsep ##1\ ##2\ ##3.]}}
%\endgroup
%%%%%%%%%%%% стиль для упражнений
%\begingroup
\gdef\th@exes{\small
  \def\@begintheorem##1##2{%
        \item[\indent\hskip\labelsep ##1\ ##2.]}%
\def\@opargbegintheorem##1##2##3{%
   \item[\indent\hskip\labelsep ##1\ ##2\ ##3.]}}
\endgroup
\gdef\@xnthm#1#2[#3]{\expandafter\@ifdefinable\csname #1\endcsname
   {%
    \@definecounter{#1}\@newctr{#1}[#3]%
    \expandafter\xdef\csname the#1\endcsname
      {\expandafter \noexpand \csname the#3\endcsname
       \@thmcountersep \@thmcounter{#1}}%
    \def\@tempa{\global\@namedef{#1}}%
    \expandafter \@tempa \expandafter{%
      \csname th@\the \theorem@style
            \expandafter \endcsname \the \theorem@bodyfont
     \@thm{#1}{#2}}%
    \global \expandafter \let \csname end#1\endcsname \@endtheorem
   }}
\gdef\@ynthm#1#2{\expandafter\@ifdefinable\csname #1\endcsname
   {\@definecounter{#1}%
    \expandafter\xdef\csname the#1\endcsname{\@thmcounter{#1}}%
    \def\@tempa{\global\@namedef{#1}}\expandafter \@tempa
     \expandafter{\csname th@\the \theorem@style \expandafter
     \endcsname \the\theorem@bodyfont \@thm{#1}{#2}}%
    \global \expandafter \let \csname end#1\endcsname \@endtheorem}}
\gdef\@othm#1[#2]#3{%
  \expandafter\ifx\csname c@#2\endcsname\relax
   \@nocounterr{#2}%
  \else
   \expandafter\@ifdefinable\csname #1\endcsname
   {\expandafter \xdef \csname the#1\endcsname
     {\expandafter \noexpand \csname the#2\endcsname}%
    \def\@tempa{\global\@namedef{#1}}\expandafter \@tempa
     \expandafter{\csname th@\the \theorem@style \expandafter
     \endcsname \the\theorem@bodyfont \@thm{#2}{#3}}%
    \global \expandafter \let \csname end#1\endcsname \@endtheorem}%
  \fi}
\gdef\@thm#1#2{\refstepcounter{#1}%
   \trivlist
   \@topsep \theorempreskipamount               % used by first \item
   \@topsepadd \theorempostskipamount           % used by \@endparenv
   \@ifnextchar [%
   {\@ythm{#1}{#2}}%
   {\@begintheorem{#2}{\csname the#1\endcsname}\ignorespaces}}
\global\let\@xthm\relax
\newskip\theorempreskipamount
\newskip\theorempostskipamount
\global\setlength\theorempreskipamount{12pt plus 5pt minus 3pt}
\global\setlength\theorempostskipamount{8pt plus 3pt minus 1.5pt}
\global\let\@endtheorem=\endtrivlist
\@onlypreamble\@xnthm
\@onlypreamble\@ynthm
\@onlypreamble\@othm
\@onlypreamble\newtheorem
\@onlypreamble\theoremstyle
\@onlypreamble\theorembodyfont
\@onlypreamble\theoremheaderfont
\theoremstyle{plain}

\makeatother
\newtheorem{lemma}{Лемма}[chapter]
\newtheorem{statement}{Утверждение}[chapter]
\newtheorem{proposition}{Предложение}[chapter]
\newtheorem{theorem}{Теорема}[chapter]

{\theoremstyle{exes}\newtheorem{exes}{У\,п\,р\,а\,ж\,н\,е\,н\,и\,е}}

{\theoremstyle{defn}\theorembodyfont{\rmfamily}
\newtheorem{definition}{О\,п\,р\,е\,д\,е\,л\,е\,н\,и\,е}[chapter]
\newtheorem{example}{\bfseries{Пример}}[chapter]
\newtheorem{algorithm}{А\,л\,г\,о\,р\,и\,т\,м}[chapter]
\newtheorem{remark}{\itshape {Замечание}}[chapter]
\newtheorem{property}{\bfseries{С\,в\,о\,й\,с\,т\,в\,о}}[chapter]
\newtheorem{corollary}{\itshape{Следствие}}[chapter]}

\newlength\eqnparsize \setlength\eqnparsize{12cm}

\PicInChaper

\linespread{1.13}

%\usepackage{titlesec}
%\titlelabel{\thetitle.\quad}
\usepackage{secdot}
\sectiondot{section}

\DeclareMathAlphabet{\mathpzc}{OT1}{pzc}{m}{it}

% page number at the top of the page
\usepackage{fancyhdr}

\fancyhf{}
\fancyheadoffset{0cm}
\renewcommand{\headrulewidth}{0pt}
\renewcommand{\footrulewidth}{0pt}
\fancypagestyle{plain}{%
\fancyhf{}%
\fancyhead[C]{\fontsize{13pt}{13pt}\selectfont\thepage}%
}
\pagestyle{plain}

\usepackage{subfiles}
\newcommand{\onlyinsubfile}[1]{#1}
\newcommand{\notinsubfile}[1]{}

\usepackage{listings}
\usepackage[usenames,dvipsnames]{color}

\definecolor{dkgreen}{rgb}{0,0.6,0}
\definecolor{gray}{rgb}{0.5,0.5,0.5}
\definecolor{mauve}{rgb}{0.58,0,0.82}

\lstset{ 
  language=R,                     % the language of the code
  basicstyle=\small\ttfamily, % the size of the fonts that are used for the code
  numbers=left,                   % where to put the line-numbers
  numberstyle=\tiny\color{Blue},  % the style that is used for the line-numbers
  stepnumber=1,                   % the step between two line-numbers. If it is 1, each line
                                  % will be numbered
  numbersep=5pt,                  % how far the line-numbers are from the code
  backgroundcolor=\color{white},  % choose the background color. You must add \usepackage{color}
  showspaces=false,               % show spaces adding particular underscores
  showstringspaces=false,         % underline spaces within strings
  showtabs=false,                 % show tabs within strings adding particular underscores
  % frame=single,                   % adds a frame around the code
  rulecolor=\color{black},        % if not set, the frame-color may be changed on line-breaks within not-black text (e.g. commens (green here))
  tabsize=2,                      % sets default tabsize to 2 spaces
  captionpos=b,                   % sets the caption-position to bottom
  breaklines=true,                % sets automatic line breaking
  breakatwhitespace=false,        % sets if automatic breaks should only happen at whitespace
  keywordstyle=\color{RoyalBlue},      % keyword style
  commentstyle=\color{YellowGreen},   % comment style
  stringstyle=\color{ForestGreen}      % string literal style
}

% чтобы номера частей были в номерах секций
% \renewcommand{\thesection}{\thechapter.\arabic{section}}
% \makeatletter
% \def\@seccntformat#1{%
%   \protect\textup{\protect\@secnumfont
%     \csname the#1\endcsname\enspace
%   }%
% }
% \renewcommand{\tocsection}[3]{%
%   \indentlabel{\@ifnotempty{#2}{\ignorespaces#1 #2\quad}}#3}
% \makeatother

\usepackage{pdfpages}

% remove all underfull box notes
\vbadness=1000000
\hbadness=1000000
