\newpage
\begin{center}
    \section*{СПИСОК ИСПОЛЬЗОВАННОЙ ЛИТЕРАТУРЫ}
    \addcontentsline{toc}{chapter}{СПИСОК ИСПОЛЬЗОВАННОЙ ЛИТЕРАТУРЫ}
\end{center}

\begin{thebibliography}{99}
\vspace{-12pt}

    \bibitem{source:Adleman}
    \textit{Adleman M. L.} On distinguishing prime numbers from composite numbers / Leonard M. Adleman, Carl Pomerance and Robert S. Rumely. // Annals of Mathematics --- 1983. --- P. 7-25.
    
    \bibitem{source:AKS}
    \textit{Agrawal M.} Primes in P. / M. Agrawal, N. Kayal, N. Saxena // Annals Of Mathematics --- 2004. --- Vol. 160, Iss. 2. --- P. 781-793.
    
    \bibitem{source:Ankeny}
    \textit{Ankeny N.C.} The least quadratic non-residue / N.C. Ankeny // Ann. of Math. --- 1952. --- P.~65-72.

    \bibitem{source:Bach}
    \textit{Bach E.} Explicit bounds for primality testing and related problems / E. Bach. // Mathematics of Computation. --- 1990. --- P.~355-380.
    
    \bibitem{source:Bach_Algorithmic_number_theory}
    \textit{Bach E.} Algorithmic number theory / E. Bach., J. Shallit // MIT. --- 1997. --- P.~80-82.
    
    \bibitem{source:Buhler}
    \textit{Buhler J.P.} Factoring integers with the number field sieve / J.P. Buhler, H.W. Lenstra, Carl Pomerance // Lecture Notes in Mathematics. --- 1993. --- V.~1554. --- P.~50-94.

    \bibitem{source:Cerri}
    \textit{Cerri Jean-Paul.} Euclidean minima of totally real number fields: algorithmic determination // Mathematics of computation. --- 2007. --- V.~76. --- {P.}~1547-1575.

    \bibitem{source:Cooke}
    \textit{Cooke G.E.} On the construction of division chains in algebraic number rings, with applications to sl2 / G.E.Cooke, P.L.Weinberger // Comm. Algebra. --- 1975. --- P.481-524.

    \bibitem{source:Coppersmith}
    \textit{Coppersmith D.} Small solutions to polynomial equations, and low exponent rsa vulnerabilities / D. Coppersmith // J. Cryptology. --- 1997. --- V.~10. --- P.~233-260.

    \bibitem{source:Coron}
    \textit{Coron J.S.} Deterministic polynomial-time equivalence of computing the rsa secret key and factoring / J.S. Coron, A. May. // J. Cryptology. --- 2007. --- V.~20. --- P.~3950.

    \bibitem{source:Couveignes}
    \textit{Couveignes J.} Computing A Square Root For The Number Field Sieve / J. Couveignes // Lecture Notes in Mathematics. --- 1993. --- V.~1554. --- P.~95-102.

    \bibitem{source:Darkey-Mensah}
    \textit{Darkey-Mensah M.K.} Intrinsic factorization of ideals in dedekind domains / M.K. Darkey-Mensah, P. Koprowski // ACM Communications in Computer Algebra --- 2019. --- V.~53. --- P.~107-109

    \bibitem{source:Dedekind}
    \textit{Dedekind R.} Lectures on Number Theory / R. Dedekind // American Mathematical Soc. --- 1999.

    \bibitem{source:El_Kassar}
    \textit{El-Kassar A.N.} Modified RSA in the Domains of Gaussian Integers and Polynomials over Finite Fields / A.N. El-Kassar, R.A. Haraty, Y.A. Awad. // Proceedings of the ISCA 18th International conference on computer applications in industry and engineering. --- 2005. --- P.~298-303.

    \bibitem{source:Elkamchouchi}
    \textit{Elkamchouchi H.} Extended RSA Cryptosystem and Digital Signature Schemes in the Domain of Gaussian Integers / H. Elkamchouchi, K. Elshenawy, H. Shaban. // Proceedings of the 8th International conference on communication systems. --- 2002. --- P.~91-95.

    \bibitem{source:Gekke}
    \textit{Gekke E.} Lectures on the Theory of Algebraic Numbers / E. Gekke. // GITTL. --- 1940.
    
    \bibitem{source:Kaltofen}
    \textit{Kaltofen E.} Arithmetic in quadratic fields with unique factorization / E. Kaltofen, H. Rolletschek. // EUROCAL '85 --- 1985. --- V.~204 --- P.~279-288.

    \bibitem{source:Koval}
    \textit{Koval A.} Analysis of RSA over Gaussian Integers Algorithm / A. Koval, B. Verkhovsky. // 5th international conference on information technology: new generations. --- 2008. --- {P.}~101-105.
    
    \bibitem{source:Lazard}
    \textit{Lazard D.} Le meilleur algorithme d'{E}uclide pour {$K[X]$} et {$Z$} / D. Lazard. // C. R. Acad. Sci. Paris S\'er. A-B. --- 1977. --- V.284. --- \textnumero~1. --- P.~A1-A4.
    
    \bibitem{source:Lenstra}
    \textit{Lenstra A.K.} The number field sieve / A.K. Lenstra, H.W. Lenstra, Jr., M.S. Manasse, J.M. Pollard // Springer, Berlin, Heidelberg. --- 1990.

    \bibitem{source:Lezowski}
    \textit{Lezowski P.} Computation of the euclidean minimum of algebraic number fields / P. Lezowski. // Mathematics of Computation, American Mathematical Society 83. --- 2014. --- P.1397-1426.

    \bibitem{source:Li}
    \textit{Li B.} Generalizations of RSA Public Key Cryptosystem. / B. Li. // IACR, Cryptology ePrint Arc. --- 2005.

    \bibitem{source:Miller}
    \textit{Miller G.} Riemann's Hypothesis and Tests for Primality / G. Miller. // Journal of Computer and System Sciences. --- 1976. --- V.~13. --- ~3. --- {P.}~300-317.
    
    \bibitem{source:Petukhova}
    \textit{Petukhova K.A.} RSA Cryptosystem for Dedekind Rings / K.A. Petukhova, S.N. Tronin // Lobachevskii Journal of Mathematics. --- 2016. --- V.~37. --- P.~284-287.
    
    \bibitem{source:Pollard}
    \textit{Pollard J.M.} Factoring with cubic integers / J.M. Pollard // Springer, Berlin, Heidelberg. --- 1993. --- V.~1554.

    \bibitem{source:Rabin}
    \textit{Rabin M.O.} Probabilistic Algorithm for Testing Primality / M.O.Rabin. // Journal of number theory. --- 1980. --- V. 12. --- P. 128-138.
    
    \bibitem{source:Rivest}
    \textit{Rivest R.L.} A method for obtaining digital signatures and public-key cryptosystems / R.L. Rivest, A. Shamir, L. Adleman. // Communications of the ACM. --- 1978. --- V.~21. --- P.~120-126.
    
    \bibitem{source:Rolletschek_1986}
    \textit{Rolletschek H.} On the number of divisions of the Euclidean algorithm applied to Gaussian integers / H. Rolletschek // J. Symbolic Comput. --- 1986. --- V.~2. --- \textnumero~3. --- P.~261-291.
    
    \bibitem{source:Rolletschek_1990}
    \textit{Rolletschek H.} Shortest division chains in imaginary quadratic number fields / H. Rolletschek // J. Symbolic Comput. --- 1990. --- V.~9. --- {\textnumero}~3. --- P.~321-354.

    \bibitem{source:Selfridge}
    \textit{Selfridge J.L.} Euclidean Quadratic Fields / J.L. Selfridge, C.B. Lacampagne, R.B. Eggleton. // Amer. Math. Monthly. --- 1992. --- V.~99. --- {\textnumero}~9. --- P.~829--837.

    \bibitem{source:Stillwell}
    \textit{Stillwell J.} Mathematics and its history / J. Stillwell // Springer, New York, NY. --- 2010.

    \bibitem{source:Solovay}
    \textit{Solovay R.} A fast Monte-Carlo test for primality / R. Solovay , S. Folker // SIAM Journal on Computing. --- 1977. --- P.84-85
    
    \bibitem{source:Wiener}
    \textit{Wiener M.J.} Cryptanalysis of short RSA secret exponents / M.J. Wiener. // IEEE Trans. Inform. Theory. --- 1990. --- V.~36. --- P.~553-558.

    \bibitem{source:Wikstrom}
    \textit{Wikstrom D.} On the l-Ary GCD-Algorithm in Rings of Integers / D. Wikstrom, L. Caires, G.F. Italiano, L. Monteiro, C. Palamidessi, M. Yung // Automata, Languages and Programming. Springer Berlin Heidelberg. --- 2005. --- P.~1189-1201.

    \bibitem{source:Barash}
    \textit{Бараш Л.Ю.} Алгоритм AKS проверки чисел на простоту и поиск констант генераторов псевдослучайных чисел / Л.Ю. Бараш // Безопасность информационных технологий. --- 2005. --- \textnumero~2. --- С.~27-38.

    \bibitem{source:Benyash-Krivets_1}
    \textit{Беняш-Кривец В.В.} Непрерывные дроби и $s$-единицы в гиперэллиптических полях. / В.В. Беняш-Кривец, В.П. Платонов // Успехи матем. наук. --- 2008. --- Т.~63. --- С.~95-96.
    
    \bibitem{source:Benyash-Krivets_2}
    \textit{Беняш-Кривец В.В.} Непрерывные дроби и $s$-единицы в функциональных полях. / В.В. Беняш-Кривец, В.П. Платонов // Доклады РАН. --- 2008. --- Т.~423. --- \textnumero~2. --- С.~155-160.

    \bibitem{source:Vinogradov}
    \textit{Виноградов И.М.} Основы теории чисел / И.М. Виноградов // Москва-Ленинград: ГИТТЛ. --- 1952.

    \bibitem{source:Glukhov}
    \textit{Глухов М.М.} Введение в теоретико-числовые методы криптографии / М.М. Глухов, И.А. Круглов, А.Б. Пичкур, А.В. Черемушкин // Санкт-Петербург: Лань. --- 2011.
    
    \bibitem{source:Matveev_2019}
    \textit{Матвеев Г.В.} Разделение секрета в кольцах многочленов от нескольких переменных с использованием китайской теоремы об остатках / Г.В. Матвеев // Журнал Белорусского государственного университета. Математика. Информатика. --- 2019. --- \textnumero~3. --- С.~129-133.
    
    \bibitem{source:Matveev_2018}
    \textit{Матвеев Г.В.} Совершенная верификация модулярной схемы / Г.В. Матвеев, В.В. Матулис // Журнал Белорусского государственного университета. Математика. Информатика. --- 2018. --- \textnumero~2. --- С.~4-9.

    \bibitem{source:Prochorov}
    \textit{Прохоров Н.П.} Вероятностный и детерминированный аналоги алгоритма Миллера-Рабина для идеалов колец целых алгебраических элементов конечных расширений поля $\mathbb{Q}$ / Н.П. Прохоров // Известия Национальной академии наук Беларуси. Серия физико-математических наук. --- 2020. --- Т. 56, \textnumero~2. --- С.144-156.
    
    \bibitem{source:Kharin}
    \textit{Харин Ю.С.} Криптология / Ю.С. Харин, С.В. Агиевич, Д.В. Васьльев, Г.В. Матвеев // Минск: БГУ. --- 2013.

    \bibitem{source:Yaschenko}
    \textit{Ященко В.В.} Введение в криптографию / В.В. Ященко // Москва: МЦНМО. --- 1999. --- 272 с.

\end{thebibliography}

% \begin{thebibliography}{99}
% \vspace{-12pt}

% \bibitem{Vaskouski_CSIST}
% \textit{Васьковский М.М.} Полиномиальная эквивалентность вычисления функции эйлера RSA-модуля и поиска секретного ключа в квадратичных евклидовых кольцах / М.М. Васьковский // Материалы международного научного конгресса по информатике: Информационные системы и технологии. --- 2016. --- С.~427-430.

% \bibitem{Agarwal}
% \textit{Agarwal S.} A New GCD Algorithm for Quadratic Number Rings with Unique Factorization / S. Agarwal, G.S. Frandsen. // LATIN 2006: Theoretical Informatics. --- 2006. --- P.~30--42.

% \bibitem{Brunyate}
% \textit{Brunyate A., Clark P.L.} Extending the Zolotarev-Frobenius approach to quadratic reciprocity / A. Brunyate, P.L. Clark // The Ramanujan Journal. --- 2015. --- V.~37. --- P.~25-50.

% \bibitem{Cohen}
% \textit{Cohen H.} A Course in Computational Algebraic Number Theory // Springer. --- 1996.

% \bibitem{Cohen_Advanced}
% \textit{Cohen H.} Advanced topics in computational number theory // Springer. --- 1999.

% \bibitem{Glukhov}
% \textit{Glukhov M.M.} Introduction to Number Theoretical Methods in Cryptography / M.M. Glukhov, I.A. Kruglov, A.B. Pichkur, A.V. Cheremushkin // Saint-Petersburg: Lan'. --- 2011.

% \bibitem{Kannan}
% \textit{Kannan R.} Polynomial Algorithms for Computing the Smith and Hermite Normal Forms of an Integer Matrix / R. Kannan, A. Bachem // SIAM Journal on Computing. --- 1979. --- P.~499--507.

% \bibitem{Koblitz}
% \textit{Koblitz N.} Course in Number Theory and Cryptography. / N. Koblitz // Springer-Verlag New York. --- 1994.

% \bibitem{Derksen}
% \textit{Derksen H., Kemper G.} Computational Invariant Theory / H. Derksen, G. Kemper // Encyclopaedia of Mathmatical Sciences. --- 2010. --- V. 130.

% \bibitem{Kranakis}
% \textit{Kranakis E.} Primality and Cryptography / E. Kranakis. // Teubner. --- 1986.

% \bibitem{Rodosskii}
% \textit{Rodosskii K.A.} Euclidean Algorithm. / K.A. Rodosskii. // Moscow: Nauka. --- 1988.

% \bibitem{Weinberger}
% \textit{Weinberger P.J.} On euclidean rings of algebraic integers / P.J. Weinberger // Proc. Sympos. Pure Math. --- 1973. --- V.~24. --- P.~321-332.

% \bibitem{Narkiewicz}
% \textit{Narkiewicz W.} Elementary and Analytic Theory of Algebraic Numbers / W.~Narkiewicz // Springer. --- 2004.
% \end{thebibliography}
