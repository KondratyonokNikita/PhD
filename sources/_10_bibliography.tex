\documentclass[_00_dissertation.tex]{subfiles}
\begin{document}

% 30. Сведения   об источниках   в подразделе   «Библиографический   список»   раздела
% «СПИСОК  ИСПОЛЬЗОВАННЫХ ИСТОЧНИКОВ»  располагаются   с абзацного   отступа
% в порядке появления ссылок на источники в тексте диссертации и нумеруются арабскими
% цифрами.
% Оформление   списка   использованных  источников  и библиографического   описания
% различных   источников,   использованных   в диссертации,   осуществляется   в соответствии
% с примерами, приведенными в приложениях Д и Е ГОСТ 7.32-2017.

\renewcommand{\bibname}{Список использованных источников}

\begin{thebibliography}{99}
\labelsep=5pt
\section*{Библиографический список}
\vspace{-12pt}

    \bibitem{source:Miller}
    Miller G. Riemann's Hypothesis and Tests for Primality // Journal of Computer and System Sciences.~--- 1976.~--- V.~13.~--- ~3.~--- P.~300-317.

    \bibitem{source:Rabin}
    Rabin M.O. Probabilistic Algorithm for Testing Primality // Journal of number theory.~--- 1980.~--- V.~12.~--- P.~128-138.

    \bibitem{source:Solovay}
    Solovay R., Folker S. A fast Monte-Carlo test for primality // SIAM Journal on Computing.~--- 1977.~--- P.~84-85.

    \bibitem{source:Ankeny}
    Ankeny N.C. The least quadratic non-residue // Annals of Mathematics.~--- 1952.~--- P.~65-72.

    \bibitem{source:AKS}
    Agrawal M., Kayal N., Saxena N. Primes in P // Annals of Mathematics.~--- 2004.~--- V.~160, Iss.~2.~--- P.~781-793.

    \bibitem{source:Huang_Prime_in_P}
    Huang D., Deng Y. Algorithm for computing the factor ring of an ideal in Dedekind domain with finite rank // Science China Mathematics.~--- 2017.~--- V.~61.~--- P.~783-796.

    \bibitem{source:Bach_Algorithmic_number_theory}
    Bach E., Shallit J. Algorithmic number theory // MIT.~--- 1997.~--- P.~80-82.

    \bibitem{source:Kaltofen}
    Kaltofen E., Rolletschek H. Arithmetic in quadratic fields with unique factorization // EUROCAL '85.~--- 1985.~--- V.~204~--- P.~279-288.

    \bibitem{source:Lazard}
    Lazard D. Le meilleur algorithme d'{E}uclide pour {$K[X]$} et {$Z$} // C. R. Acad. Sci. Paris S\'er. A-B.~--- 1977.~--- V.~284.~--- \textnumero~1.~--- P.~A1-A4.

    \bibitem{source:Rolletschek_1986}
    Rolletschek H. On the number of divisions of the Euclidean algorithm applied to Gaussian integers // Journal of Symbolic Computation.~--- 1986.~--- V.~2.~--- \textnumero~3.~--- P.~261-291.
    
    \bibitem{source:Rolletschek_1990}
    Rolletschek H. Shortest division chains in imaginary quadratic number fields // Journal of Symbolic Computation.~--- 1990.~--- V.~9.~--- \textnumero~3.~--- P.~321-354.

    \bibitem{source:Cooke}
    Cooke G.E., Weinberger P.L. On the construction of division chains in algebraic number rings, with applications to sl2 // Comm. Algebra.~--- 1975.~--- P.~481-524.

    \bibitem{source:Benyash-Krivets_1}
    Беняш-Кривец В.В., Платонов В.П. Непрерывные дроби и $s$-единицы в гиперэллиптических полях // Успехи матем. наук.~--- 2008.~--- Т.~63.~--- С.~95-96.
    
    \bibitem{source:Benyash-Krivets_2}
    Беняш-Кривец В.В., Платонов В.П. Непрерывные дроби и $s$-единицы в функциональных полях // Доклады РАН.~--- 2008.~--- Т.~423.~--- \textnumero~2.~--- С.~155-160.

    \bibitem{source:Vinogradov}
    Виноградов И.М. Основы теории чисел.~--- Москва-Ленинград: ГИТТЛ, 1952.~--- 180~с.

    \bibitem{source:Rivest}
    Rivest R., Shamir A., Adleman L. A method for obtaining digital signatures and public-key cryptosystems // Communications of the ACM.~--- 1978.~--- V.~21.~--- P.~120-126.

    \bibitem{source:El_Kassar}
    El-Kassar A.N., Haraty R.A., Awad Y.A. Modified RSA in the Domains of Gaussian Integers and Polynomials over Finite Fields // Proceedings of the ISCA 18th International conference on computer applications in industry and engineering.~--- 2005.~--- P.~298-303.

    \bibitem{source:Elkamchouchi}
    Elkamchouchi H., Elshenawy K., Shaban H. Extended RSA Cryptosystem and Digital Signature Schemes in the Domain of Gaussian Integers // Proceedings of the 8th International conference on communication systems.~--- 2002.~--- P.~91-95.

    \bibitem{source:Koval}
    Koval A., Verkhovsky B. Analysis of RSA over Gaussian Integers Algorithm // 5th international conference on information technology: new generations.~--- 2008.~--- P.~101-105.

    \bibitem{source:Li}
    Li B. Generalizations of RSA Public Key Cryptosystem // IACR, Cryptology ePrint Archive.~--- 2005.~--- P.~1-8.

    \bibitem{source:Petukhova}
    Petukhova K.A., Tronin S.N. RSA Cryptosystem for Dedekind Rings // Lobachevskii Journal of Mathematics.~--- 2016.~--- V.~37.~--- P.~284-287.

    \bibitem{source:Matveev_2019}
    Матвеев Г.В. Разделение секрета в кольцах многочленов от нескольких переменных с использованием китайской теоремы об остатках // Журнал Белорусского государственного университета. Математика. Информатика.~--- 2019.~--- \textnumero~3.~--- С.~129-133.

    \bibitem{source:Matveev_2018}
    Матвеев Г.В., Матулис В.В. Совершенная верификация модулярной схемы // Журнал Белорусского государственного университета. Математика. Информатика.~--- 2018.~--- \textnumero~2.~--- С.~4-9.

    \bibitem{source:Kharin}
    Харин Ю.С., Агиевич С.В., Васьльев Д.В., Матвеев Г.В. Криптология.~--- Минск: БГУ, 2013.~--- 516~с.

    \bibitem{source:Coppersmith}
    Coppersmith D. Small solutions to polynomial equations, and low exponent rsa vulnerabilities // Journal of Cryptology.~--- 1997.~--- V.~10.~--- P.~233-260.

    \bibitem{source:Coron}
    Coron J.S., May A. Deterministic polynomial-time equivalence of computing the rsa secret key and factoring // Journal of Cryptology.~--- 2007.~--- V.~20.~--- P.~3950.

    \bibitem{source:Wiener}
    Wiener M.J. Cryptanalysis of short RSA secret exponents // IEEE Trans. Inform. Theory.~--- 1990.~--- V.~36.~--- P.~553-558.

    \bibitem{source:Bach}
    Bach E. Explicit bounds for primality testing and related problems // Mathematics of Computation.~--- 1990.~--- P.~355-380.

    \bibitem{source:Barash}
    Бараш Л.Ю. Алгоритм AKS проверки чисел на простоту и поиск констант генераторов псевдослучайных чисел // Безопасность информационных технологий.~--- 2005.~--- \textnumero~2.~--- С.~27-38.

    \bibitem{source:Stillwell}
    Stillwell J. Mathematics and its history.~--- New York: Springer, 2010.~--- 660~p.

    \bibitem{source:Dedekind}
    Dirichlet P.G.L., Dedekind R. Lectures on Number Theory.~--- American Mathematical Society, 1999.~--- 297~p.

    \bibitem{source:Prochorov}
    Прохоров Н.П. Вероятностный и детерминированный аналоги алгоритма Миллера-Рабина для идеалов колец целых алгебраических элементов конечных расширений поля $\mathbb{Q}$ // Известия Национальной академии наук Беларуси. Серия физико-математических наук.~--- 2020.~--- Т.~56, \textnumero~2.~--- С.~144-156.

    \bibitem{source:Cerri}
    Cerri J. Euclidean minima of totally real number fields: algorithmic determination // Mathematics of computation.~--- 2007.~--- V.~76.~--- P.~1547-1575.

    \bibitem{source:Lezowski}
    Lezowski P. Computation of the euclidean minimum of algebraic number fields // Mathematics of Computation, American Mathematical Society 83.~--- 2014.~--- P.~1397-1426.

    \bibitem{source:Yaschenko}
    Ященко В.В. Введение в криптографию.~--- Москва: МЦНМО, 1999.~--- 272~с.
    
    \bibitem{source:Pollard}
    Pollard J.M. Factoring with cubic integers // The Development of the Number Field Sieve, Springer.~--- 1993.~--- V.~1554.~--- P.~4-10.

    \bibitem{source:Buhler}
    Buhler J.P., Lenstra H.W., Pomerance C. Factoring integers with the number field sieve // The Development of the Number Field Sieve, Springer.~--- 1993.~--- V.~1554.~--- P.~50-94.

    \bibitem{source:Couveignes}
    Couveignes J. Computing A Square Root For The Number Field Sieve // The Development of the Number Field Sieve, Springer.~--- 1993.~--- V.~1554.~--- P.~95-102.

    \bibitem{source:Darkey-Mensah}
    Darkey-Mensah M.K., Koprowski P. Intrinsic factorization of ideals in dedekind domains // ACM Communications in Computer Algebra.~--- 2019.~--- V.~53.~--- P.~107-109

    \bibitem{source:Lang}
    Lang S. Algebraic number theory.~--- Springer, 1986.~--- 372~p.

    \bibitem{source:Pohst}
    Pohst E.M. Computational Algebraic Number Theory.~--- Springer, 1993.~--- 88~p.

    \bibitem{source:Cohen}
    Cohen H. A Course in Computational Algebraic Number Theory.~--- Springer, 1996.~--- 536~p.

    \bibitem{source:Kannan}
    Kannan R., Bachem A. Polynomial Algorithms for Computing the Smith and Hermite Normal Forms of an Integer Matrix // SIAM Journal on Computing.~--- 1979.~--- P.~499-507.

    \bibitem{source:Schonhage}
    Schonhage A., Strassen V. Schnelle Multiplication grober Zahlen // Computing.~--- 1971.~--- \textnumero~7.~--- P.~281-292.

    \bibitem{source:Wikstrom}
    Wikstrom D., Caires L., Italiano G.F., Monteiro L., Palamidessi C., Yung M. On the l-Ary GCD-Algorithm in Rings of Integers // Automata, Languages and Programming. Springer Berlin Heidelberg.~--- 2005.~--- P.~1189-1201.

    \bibitem{source:Lenstra_Jacobi}
    Lenstra H.W. Computing Jacobi Symbols in Algebraic Number Fields // Neieuw Archief voor Wiskunde.~--- 1995.~--- P.~421-426.

    \bibitem{source:Selfridge}
    Eggleton R.B., Lacampagne C.B., Selfridge J.L. Euclidean Quadratic Fields // Amer. Math. Monthly.~--- 1992.~--- V.~99.~--- \textnumero~9.~--- P.~829--837.

    \bibitem{source:Vaskouski_CSIST}
    Васьковский М.М. Полиномиальная эквивалентность вычисления функции Эйлера RSA-модуля и поиска секретного ключа в квадратичных евклидовых кольцах // Материалы международного научного конгресса по информатике: Информационные системы и технологии.~--- 2016.~--- С.~427-430.

\end{thebibliography}

\section*{Список публикаций соискателя ученой степени}

\renewcommand{\labelenumi}{\arabic{enumi}--A}
\renewcommand{\theenumi}{\arabic{enumi}--A}

\vspace{-4ex}
\section*{\fontsize{14}{15}\selectfont Статьи в научных изданиях в соответствии с Положением о присуждении ученых степеней и присвоении ученых званий в Республике Беларусь}
\vspace{-4ex}

\begin{enumerate}

    \item \label{source:Vestnik_BSU_2013}
    Васьковский М.М., Кондратёнок Н.В. Конечные обобщенные цепные дроби в евклидовых кольцах // Вестник БГУ Серия 1 ''Физика, Математика, Информатика''.~--- 2013.~--- \textnumero~3.~--- С.~117-123.

    \item \label{source:NANB_2015}
    Vaskouski M., Kondratyonok N. Analogue of the RSA-cryprosystem in quadratic unique factorization domains // Reports of the National Academy of Sciences of Belarus.~--- 2015.~--- V.~59~--- \textnumero~5.~--- P.~18-23.

    \item \label{source:JNT_2016}
    Vaskouski M., Kondratyonok N., Prochorov N. Primes in quadratic unique factorization domains // Journal of Number Theory.~--- 2016.~--- V.~168.~--- P.~101-116.

    \item \label{source:JSC_2016}
    Vaskouski M., Kondratyonok N. Shortest division chains in unique factorization domains // Journal of Symbolic Computation.~--- 2016.~--- V.~77.~--- P.~175-188.

    \item \label{source:NANB_2017}
    Васьковский М.М., Кондратёнок Н.В., Прохоров Н.П. Аналог теста Соловея-Штрассена в квадратичных евклидовых кольцах // Доклады Национальной Академии Наук Беларуси.~--- 2017.~--- Т.~61.~--- \textnumero~5.~--- С.~28-32.

    \item \label{source:BSU_Journal_2020}
    Кондратёнок Н.В. Анализ RSA-криптосистемы в абстрактных числовых кольцах // Журнал Белорусского государственного университета. Математика. Информатика.~--- 2020.~--- \textnumero~1.~--- С.~13-21.

    \item \label{source:JSC_2021}
    Vaskouski M., Kondratyonok N. The Kronecker-Vahlen theorem fails in real quadratic norm-Euclidean fields // Journal of Symbolic Computation.~--- 2021.~--- V.~104.~--- P.~134-141.
\end{enumerate}

\vspace{-4ex}
\section*{\fontsize{14}{15}\selectfont Статьи в сборниках материалов научных конференций}
\vspace{-4ex}

\begin{enumerate}
\setcounter{enumi}{7}

    \item \label{source:Republican_Scientific_Conference_of_Students_and_Postgraduates_2013}
    Кондратёнок Н.В., Васьковский М.М. Цепные дроби в евклидовых кольцах // Материалы XVI Республиканской научной конференции студентов и аспирантов.~--- 2013.~--- Ч.~1.~--- С.~63-64.

    \item \label{source:XII_Belarussian_math_conference_2016}
    Васьковский М.М., Кондратёнок Н.В., Прохоров Н.П. Тест Соловея-Штрассена в квадратичных евклидовых кольцах // Материалы международной научной конференции ''XII Белорусская математическая конференция''.~--- 2016.~--- Ч.~5.~--- С.~15-16.

    \item \label{source:Collection_of_articles_by_laureates_2017}
    Кондратёнок Н.В., Прохоров Н.П. Критерии простоты в квадратичных кольцах с единственной факторизацией // Сборник статей лауреатов и авторов научных работ, получивших первую категорию XXIV Республиканского конкурса научных работ студентов.~--- 2017.~--- С.~19-20.

    \item \label{source:Collection_of_articles_by_laureates_2018}
    Кондратёнок Н.В., Прохоров Н.П. Аналог теоремы Кронекера-Валена и полиномиальные алгоритмы тестирования на простоту в числовых полях // Сборник статей лауреатов и авторов научных работ, получивших первую категорию XXV Республиканского конкурса научных работ студентов.~--- 2018.~--- С.~25.

    \item \label{source:Algebra_and_theory_of_algorithms}
    Васьковский М.М., Кондратёнок Н.В. Построение и анализ RSA-криптосистемы в числовых полях // Сборник материалов ''Всероссийская конференция ''Алгебра и теория алгоритмов'', посвященная 100-летию факультета математики и компьютерных наук Ивановского государственного университета''.~--- 2018.~--- С.~160-162

    \item \label{source:XIII_Belarussian_math_conference_2021}
    Кондратёнок Н.В. Свойства RSA-криптосистемы в абстрактных числовых кольцах // Материалы международной научной конференции ''XIII Белорусская математическая конференция''.~--- 2021.~--- Ч.~2.~--- С.~63-64.

    \item \label{source:CSIST_2022}
    Васьковский М.М., Кондратёнок Н.В. Аналог критерия Миллера в дедекиндовых кольцах с конечной нормой // Материалы международного научного конгресса по математике ''CSIST-2022''.~--- 2022.~--- Ч.~1.~--- С.~21-27.

\end{enumerate}

% \vspace{-4ex}
% \section*{\fontsize{14}{15}\selectfont
% Тезисы докладов научных конференций}
% \vspace{-4ex}

\end{document}
