\documentclass[_00_dissertation.tex]{subfiles}
\begin{document}

\onlyinsubfile{
    \renewcommand{\contentsname}{ОГЛАВЛЕНИЕ}
    \setcounter{tocdepth}{3}
    \tableofcontents
}

\begin{center}
    \refstepcounter{chapter}
    \chapter*{ПРИЛОЖЕНИЯ}\label{section:Appendix_code}
    \addcontentsline{toc}{chapter}{ПРИЛОЖЕНИЯ}
\end{center}

\section*{Исходный код программы нахождения контрпримеров к теореме Кронекера-Валена}\label{section:Appendix_code}

% \lstinputlisting{../additional/Find_Counterexample.r}

\section*{Список контрпримеров к теореме Кронекера-Валена}\label{section:Appendix_counterexample}

\begin{itemize}
    \item $\mathcal{O}_{\mathbb{Q}[\sqrt{2}]}$.
    Цепочка делений с выбором минимального по норме остатка:
    \begin{equation*}
        \{50, 29, -8, -11-8\sqrt{2}, -7-5\sqrt{2}, 0\}
    \end{equation*}
    Более короткая цепочка делений:
    \begin{equation*}
        \{50, 29, 21, 19601-13860\sqrt{2}, 0\}
    \end{equation*}

    \item $\mathcal{O}_{\mathbb{Q}[\sqrt{3}]}$.
    Цепочка делений с выбором минимального по норме остатка:
    \begin{equation*}
        \{52, 38, 14, 24+14\sqrt{3}, 14+8\sqrt{3}, 0\}
    \end{equation*}
    Более короткая цепочка делений:
    \begin{equation*}
        \{52, 38, -24, 2702-1560\sqrt{3}, 0\}
    \end{equation*}

    \item $\mathcal{O}_{\mathbb{Q}[\sqrt{5}]}$.
    Цепочка делений с выбором минимального по норме остатка:
    \begin{multline*}
        \{58, 39, 17101-10569\frac{1+\sqrt{5}}{2},\\
        -1974+1220\frac{1+\sqrt{5}}{2}, -377+233\frac{1+\sqrt{5}}{2}, 0\}
    \end{multline*}
    Более короткая цепочка делений:
    \begin{equation*}
        \{58, 39, 19, 1, 0\}
    \end{equation*}

    \item $\mathcal{O}_{\mathbb{Q}[\sqrt{6}]}$.
    Цепочка делений с выбором минимального по норме остатка:
    \begin{equation*}
        \{50, 33, -2425+990\sqrt{6}, -9602+3920\sqrt{6}, -485+198\sqrt{6}, 0\}
    \end{equation*}
    Более короткая цепочка делений:
    \begin{equation*}
        \{50, 33, 17, -1, 0\}
    \end{equation*}

    \item $\mathcal{O}_{\mathbb{Q}[\sqrt{7}]}$.
    Цепочка делений с выбором минимального по норме остатка:
    \begin{equation*}
        \{50, 23, -5355+2024\sqrt{7}, -20830+7873\sqrt{7}, 2024-765\sqrt{7}, 0\}
    \end{equation*}
    Более короткая цепочка делений:
    \begin{equation*}
        \{50, 23, 4, -1, 0\}
    \end{equation*}

    \item $\mathcal{O}_{\mathbb{Q}[\sqrt{11}]}$.
    Цепочка делений с выбором минимального по норме остатка:
    \begin{equation*}
        \{50, 34, -902+272\sqrt{11}, -252+76\sqrt{11}, -398+120\sqrt{11}, 0\}
    \end{equation*}
    Более короткая цепочка делений:
    \begin{equation*}
        \{50, 34, 16, 2, 0\}
    \end{equation*}

    \item $\mathcal{O}_{\mathbb{Q}[\sqrt{13}]}$.
    Цепочка делений с выбором минимального по норме остатка:
    \begin{multline*}
        \{52, 19, -355666+154451\frac{1+\sqrt{13}}{2},\\
        -128511+55807\frac{1+\sqrt{13}}{2}, -98644+42837\frac{1+\sqrt{13}}{2}, 0\}
    \end{multline*}
    Более короткая цепочка делений:
    \begin{equation*}
        \{52, 19, -5, -1, 0\}
    \end{equation*}

    \item $\mathcal{O}_{\mathbb{Q}[\sqrt{17}]}$.
    Цепочка делений с выбором минимального по норме остатка:
    \begin{multline*}
        \{51, 14, 11153-4354\frac{1+\sqrt{17}}{2},\\
        3038-1186\frac{1+\sqrt{17}}{2}, 333-130\frac{1+\sqrt{17}}{2}, 0\}
    \end{multline*}
    Более короткая цепочка делений:
    \begin{equation*}
        \{51, 14, -5, -1, 0\}
    \end{equation*}

    \item $\mathcal{O}_{\mathbb{Q}[\sqrt{19}]}$.
    Цепочка делений с выбором минимального по норме остатка:
    \begin{equation*}
        \{70, 29, -6194+1421\sqrt{19}, 3012-691\sqrt{19}, -170+39\sqrt{19}, 0\}
    \end{equation*}
    Более короткая цепочка делений:
    \begin{equation*}
        \{70, 29, 12, -57799+13260\sqrt{19}, 0\}
    \end{equation*}

    \item $\mathcal{O}_{\mathbb{Q}[\sqrt{21}]}$.
    Цепочка делений с выбором минимального по норме остатка:
    \begin{multline*}
      \{50, 33, 36845-13200\frac{1+\sqrt{21}}{2},\\
      14738-5280\frac{1+\sqrt{21}}{2}, 7369-2640\frac{1+\sqrt{21}}{2}, 0\}
    \end{multline*}
    Более короткая цепочка делений:
    \begin{equation*}
        \{50, 33, 17, -1, 0\}
    \end{equation*}

    \item $\mathcal{O}_{\mathbb{Q}[\sqrt{29}]}$.
    Цепочка делений с выбором минимального по норме остатка:
    \begin{equation*}
        \{54, 21, 201-63\frac{1+\sqrt{29}}{2}, 96-30\frac{1+\sqrt{29}}{2}, -9+3\frac{1+\sqrt{29}}{2}, 0\}
    \end{equation*}
    Более короткая цепочка делений:
    \begin{equation*}
        \{54, 21, 12, -3, 0\}
    \end{equation*}

    \item $\mathcal{O}_{\mathbb{Q}[\sqrt{33}]}$.
    Цепочка делений с выбором минимального по норме остатка:
    \begin{multline*}
        \{50, 27, 58999613-17495460\frac{1+\sqrt{33}}{2},\\
        -30811543+9136706\frac{1+\sqrt{33}}{2}, -2623473+777952\frac{1+\sqrt{33}}{2}, 0\}
    \end{multline*}
    Более короткая цепочка делений:
    \begin{equation*}
        \{50, 27, -4, -1, 0\}
    \end{equation*}

    \item $\mathcal{O}_{\mathbb{Q}[\sqrt{37}]}$.
    Цепочка делений с выбором минимального по норме остатка:
    \begin{multline*}
        \{51, 23, -2525+713\frac{1+\sqrt{37}}{2},\\
        471-133\frac{1+\sqrt{37}}{2}, 1027-290\frac{1+\sqrt{37}}{2}, 0\}
    \end{multline*}
    Более короткая цепочка делений:
    \begin{equation*}
        \{51, 23, 5, -1027+290\frac{1+\sqrt{37}}{2}, 0\}
    \end{equation*}

    \item $\mathcal{O}_{\mathbb{Q}[\sqrt{41}]}$.
    Цепочка делений с выбором минимального по норме остатка:
    \begin{multline*}
        \{51, 33, -12093+3267\frac{1+\sqrt{41}}{2},\\
        -47628+12867\frac{1+\sqrt{41}}{2}, 454959-122910\frac{1+\sqrt{41}}{2}, 0\}
    \end{multline*}
    Более короткая цепочка делений:
    \begin{equation*}
        \{51, 33, 18, -3, 0\}
    \end{equation*}

    \item $\mathcal{O}_{\mathbb{Q}[\sqrt{57}]}$.
    Цепочка делений с выбором минимального по норме остатка:
    \begin{equation*}
        \{51, 31, -11, -2, -1, 0\}
    \end{equation*}
    Более короткая цепочка делений:
    \begin{equation*}
        \{51, 31, 20, 131+40\frac{1+\sqrt{57}}{2}, 0\}
    \end{equation*}

    \item $\mathcal{O}_{\mathbb{Q}[\sqrt{73}]}$.
    Цепочка делений с выбором минимального по норме остатка:
    \begin{multline*}
        \{57, 32, -7, -580996+121751\frac{1+\sqrt{73}}{2},\\
        2548249-534000\frac{1+\sqrt{73}}{2}, 0\}
    \end{multline*}
    Более короткая цепочка делений:
    \begin{equation*}
        \{57, 32, 25, -943-250\frac{1+\sqrt{73}}{2}, 0\}
    \end{equation*}
\end{itemize}

\section*{Документы, подтверждающие практическое применение результатов диссертации}\label{section:Appendix_counterexample}

\onlyinsubfile{
    \subfile{_10_bibliography}
}

\end{document}
