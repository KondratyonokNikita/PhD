\documentclass[a4paper,12pt]{article} % Задаем класс документа и формат бумаги; в данном классе можно использовать кегель 8-12, 14, 17 и 20 пунктов.

\usepackage{polyglossia} % Конфигурируем работу с правилами типографики для разных языков.
\setdefaultlanguage{russian} % Устанавливаем русский основным языком документа.

\setmainfont{CMU Serif}
\newfontfamily{\cyrillicfont}{CMU Serif}
\setsansfont{CMU Sans Serif}
\newfontfamily{\cyrillicfontsf}{CMU Sans Serif}
\setmonofont{CMU Typewriter Text}
\newfontfamily{\cyrillicfonttt}{CMU Typewriter Text}

\usepackage{geometry}
\geometry{left=3cm}
\geometry{right= 1.5cm}
\geometry{top=2cm}
\geometry{bottom=2cm}

\usepackage{amsfonts}
\usepackage{amsmath}
\usepackage{amssymb}

\begin{document}

\begin{center}
    \textbf{Доклад Н.В. Кондратёнка на защите диссертации ''Свойства теоретико-числовых и криптографических алгоритмов в дедекиндовых кольцах''}
\end{center}

Добрый день, уважаемые участники семинара.
Вашему вниманию представляется диссертационная работа на тему ''Свойства теоретико-числовых и криптографических алгоритмов в дедекиндовых кольцах''.

Алгоритмическая теория чисел является активно развивающейся областью математики.
Она включает в себя изучение вычислительных методов для исследования и решения задач в теории чисел, включая алгоритмы для проверки на простоту и целочисленной факторизации, нахождения решений диофантовых уравнений.
В данной области активно работали такие математики как Дедекинд, Пост, Бах, Коэн, Миллер и другие.
Однако остались еще много нерешенных задач, решению некоторых из которых посвящена эта работа.

Диссертация состоит из 4 глав.

В главе 1 дается обзор литературы по теме исследования и формулируются решаемые в диссертации задачи.

Глава 2 работы посвящена доказательству новых критериев простоты в дедекиндовых кольцах, которые являются аналогами критериев Эйлера и Миллера.
Показано, что полученные критерии позволяют построить эффективные алгоритмы тестирования на простоту.

Глава 3 работы посвящена исследованию аналога алгоритма Евклида в факториальных кольцах.
В работе Кронекера и Валена доказано, что в кольце целых чисел наименьшее количество делений получается при выборке минимального по абсолютному значению остатка.
В диссертации был найден класс колец, для которого аналог теоремы Кронекера-Валена выполняется.
Так же разработан метод позволяющий доказать, что эта теорема не выполняется во всех действительных квадратичных норменно-евклидовых кольцах.
    
Главе 4 работы посвящена исследованию аналога RSA-криптосистемы в дедекиндовых кольцах с конечной нормой.
Доказан аналог теоремы Винера о малой секретной экспоненте, теоремы об эквивалентности поиска секретного ключа и факторизации модуля и другие теоремы об ограничениях на параметры криптосистемы.

\section{Проверка на простоту}

Задача проверки на простоту является одной из основных в алгоритмической теории чисел.
Для использования на практике очень важно чтобы получаемые критерии можно было использовать для построения эффективных алгоритмов тестирования на простоту.
Для кольца целых чисел такими критериями являются критерий Миллера и Эйлера.
На эти критерии опираются полиномиальные вероятностные тесты Миллера-Рабина и Соловея-Штрассена.
А вв предположении обобщенной гипотезы Римана алгоритмы можно модифицировать и сделать полиномиальными детерминированными.

В диссертации предлагается следующее условие, аналогичное условию выполнения обобщенной гипотезы Римана, которое позволит получить результат аналогичный кольцу целых чисел.
Это условие, в часности, выполнено в кольце целых чисел в предположении обобщенной гипотезы Римана.

Критерий Миллера является ключевым утверждением, на котором базируются современные методы тестирования на простоту и алгоритмы генерации больших простых чисел, в том числе алгоритм Миллера-Рабина и Гордона.
Замечу, что доказанное утверждение состоит из двух частей.
Первая не предполагает выполнения условия A и, используя такой критерий, получается вероятностный полиномиальный алгоритм.
Вторая предполагает выполнение условия А и, используя такой критерий, получается детерминированный полиномиальный алгоритм проверки на простоту.

На слайде вы можете видеть полученй в работе обобщенный алгоритм Миллера-Рабина, опирающийся на обобщенный критерий Миллера.

Следующий критерий является обобщением критерия простоты Эйлера в дедекиндовых кольцах.
Он так же состоит из двух частей.
В работе был построен аналог алгоритма Соловея-Штрассена, опирающийся на обобщенный критерий Эйлера.

Отметим, что данные алгоритмы являются полиномиальными в частности в кольцах целых алгебраических элементов числового поля.

\section{Теорема Кронекера-Валена}

Вторая глава работы посвящена разработке методов проверки выполнимости теоремы Кронекера-Валена в факториальных кольцах.
Эта теорема гласит, что для любых двух целых чисел алгоритм Евклида с выбором минимального по модулю остатка будет требовать наименьшее число делений.

Аналогичная теорема была доказана в работе Лазара для кольца многочленов над полем.
При рассмотрении аналогичной задачи в более общем случае возникают определенные трудности.
В работе Ролетчека было доказано, что аналог теоремы Кронекера-Валена выполняется во всех мнимых квадратичных кольцах кроме $\mathcal{O}_{\mathbb{Q}(\sqrt{-11})}$.
Для действительных квадратичных норменно-евклидовых колец задача существенно усложняется.
Например известно, что для кольца с бесконечной группой единиц кратчайшая цепочка делений для любых двух элементов имеет длину не более $5$.
Однако доказательство этого факта не конструктивное.

В работе выделен класс колец T и доказано, что аналог теоремы Кронекера-Валена выполнен в кольцах из этого класса.

Для упрощения проверки принадлежности кольца классу T был выделен подкласс S, проверка принадлежности которому существенно проще.

Было показано, что кольца целых чисел, многочленов над полем и другие принадлежал классу S.
Однако кольцо гауссовых чисел не принадлежит этому классу, но принадлежит классу T.

Доказательство того, что для колец из класса T выполнен аналог теоремы Кронекера-Валена состоит в том, чтобы рассмотреть уравнение в виде цепной дроби и найти критерии его разрешимости.
Это было сделано с помощью индукции.
Таким образом была доказана следующая теорема.
Так же в работе доказан аналог теоремы Ламе о длине кратчайшей цепочки делений.

В работе был разработан метод доказательства невыполнимости теоремы Кронекера-Валена для колец целых алгебраических чисел.

Применив этот метод для действительных квадратичных колец, было доказано, что теорема Кронекера-Валена не выполняется ни в одном из них.
Таким образом, этот результат закрыл вопрос о выполнимости теоремы Кронекера-Валена в квадратичных норменно-евклидовых кольцах.

\section{RSA-криптосистема}

Криптосистема RSA является простой, но устойчивой к взлому так как основана на вчислительно сложной задаче факторизации.
Можно выделить два подхода к улучшению этой криптосистемы.
Первый заключается в нахождении новых требований к параметрам криптосистемы, а второй в построении аналога криптосистемы в кольцах все более общего вида.
Известныы аналоги криптосистемы RSA в кольце многочленов, кольце гауссовых числел, квадратичных кольцах.
Так же известен аналог криптосистемы RSA в дедекиндовых кольцах с конечной нормой.
Однако исследований каким требованиям должны удовлетворять ее параметры для обеспечения криптостойкости нет.

В этой работе был доказан аналог теоремы Винера о малой секретной экспоненте, аналог теоремы об эквивалентности задачи нахождения секретного ключа криптосистемы и факторизации модуля.
А так же аналог теоремы Винера о малой секретной экспоненте.
Доказана теорема об ограничениях на параметры криптосистемы для обеспечения безопасности против взлом с помощью метода повторного шифрования.

\section{Заключение}

Таким образом на защиту выносятся следующие результаты.
Основные результаты работы были опубликованы 7 статьях зарубежных научных журналах и изданиях из списка ВАК РБ.
Всего имеется 14 опубликованных работ по теме диссертации.
Так же докладывались на 13 научных конференциях.
Результаты внедрены в учебный процесс БГУ.

На этом я закончу свой доклад.
Спасибо за внимание.

\end{document}