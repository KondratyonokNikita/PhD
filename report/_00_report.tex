\documentclass[12pt]{article}


\usepackage{lastpage}
\usepackage{caption2}

\usepackage{lscape}

\usepackage{makeidx}
\usepackage{soul}
\usepackage{mathrsfs}
\usepackage[russian=nohyphenation,english=nohyphenation]{hyphsubst}
\usepackage[russian,english]{babel}
\usepackage{longtable}
\usepackage{comment}

\babelfont{rm}{Times New Roman}
\babelfont{sf}{Times New Roman}

\renewcommand{\rmdefault}{tli}

\usepackage[pdftex]{graphicx}
\usepackage{wrapfig}
\usepackage{euscript}

\usepackage{amsfonts}
\usepackage{amsmath}
\usepackage{amssymb}

\renewcommand\labelitemi{{\textbullet}}

\usepackage{totcount}
\newtotcounter{citnum}
\def\oldbibitem{} \let\oldbibitem=\bibitem
\def\bibitem{\stepcounter{citnum}\oldbibitem}

\sloppy\clubpenalty4000\widowpenalty4000

% Размеры страницы мои для формата А4
\setlength{\hoffset}{0mm} \setlength{\voffset}{-8mm}
\setlength{\textheight}{245mm}% размеры текста по высоте без колонтитула
\setlength{\textwidth}{170mm}% размеры текста по ширине
\setlength{\topmargin}{2mm}% высота над верхним колонтитулом
\setlength{\oddsidemargin}{1.9mm}% отступ слева
\setlength{\evensidemargin}{-4.0mm}
\addtolength{\evensidemargin}{8.0mm}% отступ  слева на четных страницах при двусторонней печати
\addtolength{\oddsidemargin}{1.9mm}%отступ слева на (нечетной при двусторонней)всех страницах при односторонней печати
\setlength{\headsep}{4mm}% высота между верхним колонтитулом и текстом
\setlength{\footskip}{10mm}% высота между текстом и нижним колонтитулом
\setlength{\marginparwidth}{0mm}% ширина заметок на полях
\setlength{\marginparpush}{0mm}
\setlength{\marginparsep}{0mm}% ширина между заметками на полях и текстом

\setlength{\parindent}{9mm} \setlength{\linespread}{1}

\renewcommand{\headrulewidth}{0.5pt}
\renewcommand{\footrulewidth}{0mm}

% Из Никифорова для оборотных страниц
\newlength{\lb}
\setlength{\lb}{24pt}
\newcommand{\tb}{ \hspace*{\lb}}
\newlength{\pb}
\setlength{\pb}{\textwidth} \addtolength{\pb}{-\lb}
\newlength{\wlb}
\settowidth{\wlb}{H62}
\newlength{\wpb}
\setlength{\wpb}{\lb} \addtolength{\wpb}{-\wlb}
%


% Нумерация формул
\makeatletter \@addtoreset {equation}{chapter}
%\renewcommand\theequation{\@arabic\c@section.\@arabic\c@equation}%первая section в пределах section
%\renewcommand{\theequation}{\thechapter.\arabic{equation}}%первая chapter в пределах section
\renewcommand\theequation{\thechapter.\@arabic\c@equation}%первые chapter и section

%После названия точка
\makeatletter
\def\@seccntformat#1{\csname the#1\endcsname\quad}
\makeatother

%Hacks fo less count of overfulls
\pretolerance=200
\tolerance=500

%definitions of some symbols

\def\proof{\par\noindent{Д\,о\,к\,а\,з\,а\,т\,е\,л\,ь\,с\,т\,в\,о}.\ }
\def\proofo{\par\noindent{Д\,о\,к\,а\,з\,а\,т\,е\,л\,ь\,с\,т\,в\,о}\ }
\def\endproof{\hfill{$\otimes$}}

\makeatletter

%%%%%%%%%%%%%%%%%%%% Кусок для теоремоподобных окружений
\gdef\theoremstyle#1{%
   \@ifundefined{th@#1}{\@warning
          {Unknown theoremstyle `#1'. Using `plain'}%
          \theorem@style{plain}}%
      {\theorem@style{#1}}%
      \begingroup
        \csname th@\the\theorem@style \endcsname
      \endgroup}
\global\let\@begintheorem\relax
\global\let\@opargbegintheorem\relax
\newtoks\theorem@style
\global\theorem@style{plain}
\gdef\theorembodyfont#1{%
   \def\@tempa{#1}%
   \ifx\@tempa\@empty
    \theorem@bodyfont{}%
   \else
    \theorem@bodyfont{\reset@font#1}%
   \fi
   }
\newtoks\theorem@bodyfont
\global\theorem@bodyfont{}
\gdef\theoremheaderfont#1{\gdef\theorem@headerfont{#1}%
       \gdef\theoremheaderfont##1{%
        \typeout{\string\theoremheaderfont\space should be used
                 only once.}}}
\ifx\upshape\undefined
\gdef\theorem@headerfont{\bfseries}
\else \gdef\theorem@headerfont{\normalfont\bfseries}\fi
\begingroup
\gdef\th@plain{\normalfont\itshape
  \def\@begintheorem##1##2{%
        \item[\indent\hskip\labelsep \theorem@headerfont ##1\ ##2.]}%
\def\@opargbegintheorem##1##2##3{%
   \item[\indent\hskip\labelsep \theorem@headerfont ##1\ ##2\ ##3.]}}
%\endgroup
%%%%%%%%%%%% стиль как в Определениях
%\begingroup
\gdef\th@defn{\normalfont\itshape
  \def\@begintheorem##1##2{%
        \item[\indent\hskip\labelsep ##1\ ##2.]}%
\def\@opargbegintheorem##1##2##3{%
   \item[\indent\hskip\labelsep ##1\ ##2\ ##3.]}}
%\endgroup
%%%%%%%%%%%% стиль для упражнений
%\begingroup
\gdef\th@exes{\small
  \def\@begintheorem##1##2{%
        \item[\indent\hskip\labelsep ##1\ ##2.]}%
\def\@opargbegintheorem##1##2##3{%
   \item[\indent\hskip\labelsep ##1\ ##2\ ##3.]}}
\endgroup
\gdef\@xnthm#1#2[#3]{\expandafter\@ifdefinable\csname #1\endcsname
   {%
    \@definecounter{#1}\@newctr{#1}[#3]%
    \expandafter\xdef\csname the#1\endcsname
      {\expandafter \noexpand \csname the#3\endcsname
       \@thmcountersep \@thmcounter{#1}}%
    \def\@tempa{\global\@namedef{#1}}%
    \expandafter \@tempa \expandafter{%
      \csname th@\the \theorem@style
            \expandafter \endcsname \the \theorem@bodyfont
     \@thm{#1}{#2}}%
    \global \expandafter \let \csname end#1\endcsname \@endtheorem
   }}
\gdef\@ynthm#1#2{\expandafter\@ifdefinable\csname #1\endcsname
   {\@definecounter{#1}%
    \expandafter\xdef\csname the#1\endcsname{\@thmcounter{#1}}%
    \def\@tempa{\global\@namedef{#1}}\expandafter \@tempa
     \expandafter{\csname th@\the \theorem@style \expandafter
     \endcsname \the\theorem@bodyfont \@thm{#1}{#2}}%
    \global \expandafter \let \csname end#1\endcsname \@endtheorem}}
\gdef\@othm#1[#2]#3{%
  \expandafter\ifx\csname c@#2\endcsname\relax
   \@nocounterr{#2}%
  \else
   \expandafter\@ifdefinable\csname #1\endcsname
   {\expandafter \xdef \csname the#1\endcsname
     {\expandafter \noexpand \csname the#2\endcsname}%
    \def\@tempa{\global\@namedef{#1}}\expandafter \@tempa
     \expandafter{\csname th@\the \theorem@style \expandafter
     \endcsname \the\theorem@bodyfont \@thm{#2}{#3}}%
    \global \expandafter \let \csname end#1\endcsname \@endtheorem}%
  \fi}
\gdef\@thm#1#2{\refstepcounter{#1}%
   \trivlist
   \@topsep \theorempreskipamount               % used by first \item
   \@topsepadd \theorempostskipamount           % used by \@endparenv
   \@ifnextchar [%
   {\@ythm{#1}{#2}}%
   {\@begintheorem{#2}{\csname the#1\endcsname}\ignorespaces}}
\global\let\@xthm\relax
\newskip\theorempreskipamount
\newskip\theorempostskipamount
\global\setlength\theorempreskipamount{12pt plus 5pt minus 3pt}
\global\setlength\theorempostskipamount{8pt plus 3pt minus 1.5pt}
\global\let\@endtheorem=\endtrivlist
\@onlypreamble\@xnthm
\@onlypreamble\@ynthm
\@onlypreamble\@othm
\@onlypreamble\newtheorem
\@onlypreamble\theoremstyle
\@onlypreamble\theorembodyfont
\@onlypreamble\theoremheaderfont
\theoremstyle{plain}

\makeatother
\newtheorem{lemma}{Лемма}[chapter]
\newtheorem{statement}{Утверждение}[chapter]
\newtheorem{proposition}{Предложение}[chapter]
\newtheorem{theorem}{Теорема}[chapter]

{\theoremstyle{exes}\newtheorem{exes}{У\,п\,р\,а\,ж\,н\,е\,н\,и\,е}}

{\theoremstyle{defn}\theorembodyfont{\rmfamily}
\newtheorem{definition}{О\,п\,р\,е\,д\,е\,л\,е\,н\,и\,е}[chapter]
\newtheorem{example}{\bfseries{Пример}}[chapter]
\newtheorem{algorithm}{А\,л\,г\,о\,р\,и\,т\,м}[chapter]
\newtheorem{remark}{\itshape {Замечание}}[chapter]
\newtheorem{property}{\bfseries{С\,в\,о\,й\,с\,т\,в\,о}}[chapter]
\newtheorem{corollary}{\itshape{Следствие}}[chapter]}

\newlength\eqnparsize \setlength\eqnparsize{12cm}

\PicInChaper

\linespread{1.13}

%\usepackage{titlesec}
%\titlelabel{\thetitle.\quad}
\usepackage{secdot}
\sectiondot{section}

\DeclareMathAlphabet{\mathpzc}{OT1}{pzc}{m}{it}

% page number at the top of the page
\usepackage{fancyhdr}

\fancyhf{}
\fancyheadoffset{0cm}
\renewcommand{\headrulewidth}{0pt}
\renewcommand{\footrulewidth}{0pt}
\fancypagestyle{plain}{%
\fancyhf{}%
\fancyhead[C]{\fontsize{13pt}{13pt}\selectfont\thepage}%
}
\pagestyle{plain}

\usepackage{subfiles}
\newcommand{\onlyinsubfile}[1]{#1}
\newcommand{\notinsubfile}[1]{}

\usepackage{listings}
\usepackage[usenames,dvipsnames]{color}

\definecolor{dkgreen}{rgb}{0,0.6,0}
\definecolor{gray}{rgb}{0.5,0.5,0.5}
\definecolor{mauve}{rgb}{0.58,0,0.82}

\lstset{ 
  language=R,                     % the language of the code
  basicstyle=\small\ttfamily, % the size of the fonts that are used for the code
  numbers=left,                   % where to put the line-numbers
  numberstyle=\tiny\color{Blue},  % the style that is used for the line-numbers
  stepnumber=1,                   % the step between two line-numbers. If it is 1, each line
                                  % will be numbered
  numbersep=5pt,                  % how far the line-numbers are from the code
  backgroundcolor=\color{white},  % choose the background color. You must add \usepackage{color}
  showspaces=false,               % show spaces adding particular underscores
  showstringspaces=false,         % underline spaces within strings
  showtabs=false,                 % show tabs within strings adding particular underscores
  % frame=single,                   % adds a frame around the code
  rulecolor=\color{black},        % if not set, the frame-color may be changed on line-breaks within not-black text (e.g. commens (green here))
  tabsize=2,                      % sets default tabsize to 2 spaces
  captionpos=b,                   % sets the caption-position to bottom
  breaklines=true,                % sets automatic line breaking
  breakatwhitespace=false,        % sets if automatic breaks should only happen at whitespace
  keywordstyle=\color{RoyalBlue},      % keyword style
  commentstyle=\color{YellowGreen},   % comment style
  stringstyle=\color{ForestGreen}      % string literal style
}

% чтобы номера частей были в номерах секций
% \renewcommand{\thesection}{\thechapter.\arabic{section}}
% \makeatletter
% \def\@seccntformat#1{%
%   \protect\textup{\protect\@secnumfont
%     \csname the#1\endcsname\enspace
%   }%
% }
% \renewcommand{\tocsection}[3]{%
%   \indentlabel{\@ifnotempty{#2}{\ignorespaces#1 #2\quad}}#3}
% \makeatother

\usepackage{pdfpages}

% remove all underfull box notes
\vbadness=1000000
\hbadness=1000000

% \pagestyle{empty}

\begin{document}

\begin{center}
\textbf{Доклад Н.В. Кондратёнок на защите диссертации ''Свойства теоретико-числовых и криптографических алгоритмов в дедекиндовых кольцах''}
\end{center}

Добрый день, уважаемые участники семинара.
Алгоритмическая теория чисел является активно развивающейся областью математики.
% посмотреть диссертацию ММ и указать что известные математики относительно надевно много и продукитвно работали над этой областтью, но все еще осталось много нерешенных проблем
% плюс актуально для теории защиты информации
Особо полезными для диссертации оказались работы Дедекинда, Поста, Баха, Коэна, Миллера и других математиков.

Представленная диссертация посвящена доказательству новых критериев простоты идеалов в дедекиндовых кольцах и исследованию свойств теоретико-числовых и криптографических алгоритмов в дедекиндовых кольцах.
Основная часть диссертации состоит из 4 глав.

В главе 1 дается обзор литературы по теме исследования и формулируются решаемые в диссертации задачи.

Глава 2 работы посвящена доказательству новых критериев простоты в дедекиндовых кольцах, которые являются аналогами критериев Эйлера и Миллера.
Показано, что полученные критерии позволяют построить эффективные алгоритмы тестирования на простоту.

Глава 3 работы посвящена исследованию аналога алгоритма Евклида в факториальных кольцах.
В работе Кронекера и Валена доказано, что в кольце целых чисел наименьшее количество делений получается при выборке минимального по абсолютному значению остатка.
В диссертации был найден класс колец, для которого аналог теоремы Кронекера-Валена выполняется.
Так же разработан метод позволяющий доказать, что эта теорема не выполняется во всех действительных квадратичных норменно-евклидовых кольцах.
    
Главе 4 работы посвящена исследованию аналога RSA-криптосистемы в дедекиндовых кольцах с конечной нормой.
Доказан аналог теоремы Винера о малой секретной экспоненте, теоремы об эквивалентности поиска секретного ключа и факторизации модуля и другие теоремы об ограничениях на параметры криптосистемы.

\section{Введение}

Основными алгебраическими структурами, используемыми в работе являются дедекиндовы кольца с конечной нормой, которые в частности охватывают кольца целых алгебраических элементов числовых полей.

\section{Проверка на простоту}

Перейдем к более подробному обзору результатов.
Первая глава работы посвящена доказательству аналогов критериев простоты Эйлера и Миллера в дедекиндовых кольцах.

В кольце целых чисел в предположении обобщенной гипотезы Римана в критериях Эйлера и Миллера можно существенно сократить количество элементов на которые критерий накладывает условия.
Для доказательства аналогичных критериев в дедекиндовых кольцах нам понадобится следующее условие A.
Отметим, что в предположении обобщенной гипотезы Римана условие выполняется в кольце целых чисел.
А в предположении расширенной гипотезы Римана оно выполняется во всех кольцах целых алгебраических чисел.

Критерий Миллера является ключевым утверждением, на котором базируются современные методы тестирования на простоту и алгоритмы генерации больших простых чисел, в том числе алгоритм Миллера-Рабина и Гордона.

Доказательство усложняется относительно доказательства для целых чисел из-за того, что в дедекиндовых кольцах норма простого идеала примарная, а не простая как в кольце целых чисел.

Теорема состоит из двух частей.
Первая не предполагает выполнения условия A и из нее получается вероятностный полиномиальный алгоритм, а вторая предполагает и из нее получается детерминированный полиномиальный алгоритм проверки на просоту.

В работе был построен обобщенный алгоритм Миллера-Рабина, опирающийся на обобщенный критерий Миллера.

Следующий критерий является обобщением критерия простоты Эйлера в дедекиндовых кольцах.
Он так же состоит из двух частей.
В работе был построен аналог алгоритма Соловея-Штрассена, опирающийся на обобщенный критерий Эйлера.

Отметим, что данные алгоритмы являются полиномиальными в частности в кольцах целых алгебраических элементов числового поля.

\section{Теорема Кронекера-Валена}

Вторая глава работы посвящена разработке методов проверки выполнимости теоремы Кронекера-Валена в факториальных кольцах.
Эта теорема гласит, что для любых двух целых чисел алгоритм Евклида с выбором минимального по модулю остатка будет требовать наименьшее число делений.
Аналогичная теорема была доказана в работе Лазара для кольца многочленов над полем.
В работе Ролетчека было доказано, что аналог теоремы Кронекера-Валена выполняется во всех мнимых квадратичных кольцах кроме $\mathcal{O}_{\mathbb{Q}(\sqrt{-11})}$.
    
Так как рассматриваемые в работе кольца не евклидовы, а факториальные, то не для каждой пары элементов кольца будет существовать конечный алгоритм Евклида.
Поэтому вместо понятия алгоритм Евклида будем использовать цепочки делений.
Под цепочкой делений будем понимать последовательность элементов кольца каждый следующий из которых получается при делении с остатком двух предыдущих, но норма остатка не зависит от норм делимых элементов.
Будем рассматривать только конечные цепочки делений, у которых только последний элемент равен $0$.
Если для пары элементов кольца таких цепочек нет, то будем говорить, что для них цепочка делений имеет бесконечную длину.
    
Под цепочкой делений с выбором минимального по норме остатка будем понимать такую цепочку делений, где каждый следующий элемент получается из двух предыдущих с помощью деления с выбором минимального по норме остатка.

В работе выделен класс колец T и доказано, что аналог теоремы Кронекера-Валена выполнен в кольцах из этого класса.
Для упрощения проверки принадлежности кольца классу T был выделен подкласс S, проверка принадлежности которому существенно проще.
Было показано, что кольца целых чисел, многочленов над полем и другие принадлежал классу S.
Однако кольцо гауссовых чисел не принадлежит этому классу, но принадлежит классу T.

Доказательство того, что для колец из класса T выполнен аналог теоремы Кронекера-Валена состоит в том, чтобы рассмотреть уравнение в виде цепной дроби и найти критерии его разрешимости.
Это было сделано с помощью индукции.

Таким образом была доказана следующая теорема.
Так же в работе доказан аналог теоремы Ламе о длине кратчайшей цепочки делений.

В работе был разработан метод доказательства невыполнимости теоремы Кронекера-Валена для колец целых алгебраических чисел.

Применив этот метод для действительных квадратичных колец, было доказано, что теорема Кронекера-Валена не выполняется ни в одном из них.

\section{RSA-криптосистема}

В работах других математиков был предложен аналог RSA-криптосистемы в дедекиндовых кольцах с конечной нормой.
Там было показано, что алгоритмы шифрования и расшифрования является корректными.
Однако исследований свойств этой криптосистемы нет.

В этой работе был доказан аналог теоремы Винера о малой секретной экспоненте, аналог теоремы об эквивалентности задачи нахождения секретного ключа криптосистемы и факторизации модуля.
А так же аналог теоремы Винера о малой секретной экспоненте.
Доказана теорема об ограничениях на параметры криптосистемы для обеспечения безопасности против взлом с помощью метода повторного шифрования.

\section{Заключение}

Таким образом на защиту выносятся следующие результаты.
Основные результаты работы были опубликованы 7 статьях зарубежных научных журналах и изданиях из списка ВАК РБ.
Всего имеется 14 опубликованных работ по теме диссертации.
Так же докладывались на 13 научных конференциях.
Результаты внедрены в учебный процесс БГУ.

На этом я закончу свой доклад.
Спасибо за внимание.

\end{document}